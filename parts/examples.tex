%======================================================================
\chapter{Examples}
%======================================================================

We will start this chapter by presenting an example policy statement expressed using \ac{odrl}'s XML-based syntax. The statement ``the asset TheReport may be printed a total of 2 times by Alice only'' is illustrated in listing~\ref{lst:agreementinxml}. 

\lstset{language=XML}
%\begin{minipage}[c]{0.95\textwidth}
\begin{lstlisting}[caption={First Agreement for Alice and Bob in XML}, label={lst:agreementinxml}]
<agreement>
 <asset> <context> <uid> The Report </uid> </context> </asset>
 <permission>
   <print>
    <constraint> <name> Alice </name> </constraint>
    <constraint> <count> 2 </count> </constraint>
   </print>
 </permission>
 <party> <context> <name> Alice </name> </context> </party>
 <party> <context> <name> Bob </name> </context> </party>
</agreement>
\end{lstlisting}
%\end{minipage} 

The statement in listing~\ref{lst:agreementinxml} is shown in listing~\ref{lst:agreementinAST} using the abstract syntax from~\cite{pucella2006}.

\lstset{language=Pucella2006}
%\begin{minipage}[c]{0.95\textwidth}
\begin{lstlisting}[frame=single, caption={First Agreement for Alice and Bob}, label={lst:agreementinAST}, mathescape]
agreement
 for Alice and Bob 
 about The Report 
 with True -> and[Alice, count[2]] =>$_{id1}$ print.
\end{lstlisting}
%\end{minipage} 


A similar but slightly more complex (than the one in~\ref{lst:agreementinxml}) example is discussed in~\cite{pucella2006} and we list it here (see listing~\ref{lst:agreementinxmlfrompucella}).  This particular policy statement is not expressible in \ac{ACCPL} due to lack of support for two kinds of conditions. One kind is the ``prePay[n]'' condition supported by Pucella and Weissman~\cite{pucella2006}'s fragment of \ac{odrl} (as a prerequisite). The second kind mentioned in this statement, ``Mary's computer'', is some kind of location based condition that the authors of~\cite{pucella2006} refer to, but we don't find it defined anywhere. As we discuss later in this thesis (see sections~\ref{sec:missingreqs} and~\ref{sec:extendwithgeneralconditions}) both of these prerequisites (and others) could potentially be implemented to extend expressibility of \ac{ACCPL}.

\lstset{language=XML}
\begin{minipage}[c]{0.95\textwidth}
\begin{lstlisting}[caption={Agreement for Alice and Bob in XML~\cite{pucella2006}}, label={lst:agreementinxmlfrompucella}]
<agreement>
 <asset> <context> <uid> Treasure Island </uid> </context> </asset>
 <permission>
   <display>
    <constraint> <name> Mary's computer </name> </constraint>
   </display>
   <print>
    <constraint> <count> 2 </count> </constraint>
   </print>
 </permission>
 <party> <context> <name> Mary </name> </context> </party>
</agreement>
\end{lstlisting}
\end{minipage} 

The statement in listing~\ref{lst:agreementinxmlfrompucella} is shown in listing~\ref{lst:agreementinASTfrompucella} using the abstract syntax form.

\lstset{language=Pucella2006}
\begin{minipage}[c]{0.95\textwidth}
\begin{lstlisting}[frame=single, caption={Agreement for Alice and Bob~\cite{pucella2006}}, label={lst:agreementinASTfrompucella}, mathescape]
agreement
 for Mary 
 about Treasure Island 
 with prePay[5.00] -> and[[Mary's computer] =>$_{id1}$ display, 
                                   count[2] =>$_{id2}$ print.
\end{lstlisting}
\end{minipage} 



%While the statement in listing~\ref{lst:exampleone} is expressible in \ac{ACCPL}, the statement ``the asset TheReport may be printed a total of 5 times by either Alice or Bob and twice more by Alice only''~\cite{pucella2006} (shown in listing~\ref{lst:morethanonePS}) is not expressible in \ac{ACCPL}. This statement is fully expressible in \ac{odrl} and also in Pucella and Weissman's fragment of \ac{odrl}~\cite{pucella2006}. We discuss the design choices that leads to this limitation in expressive power for \ac{ACCPL} in section~\ref{sec:policysetcombo} but at a high-level we have traded off the possibility of combining policy sets (see section~\ref{lst:policySetast}) in exchange for a conflict-free policy language in \ac{ACCPL}.

%\lstset{language=Pucella2006}
%\begin{minipage}[c]{0.95\textwidth}
%\begin{lstlisting}[frame=single, caption={More than One Policy Set}, label={lst:morethanonePS}]
%agreement
% for Alice and Bob 
% about The Report 
% with and[True -> count[5] => print, True -> and[Alice, count[2]] => print].
%\end{lstlisting}
%\end{minipage}
%% From here on, move to an Examples Chapter where 13-examples used to be. The below this part should be added to end of introduction
