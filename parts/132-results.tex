%======================================================================
\chapter{Formal Proofs and Decidability}\label{chap:results}
%======================================================================

                  
%----------------------------------------------------------------------
\section{Introduction}\label{sec:introresults}


We had set out to prove formally that the \ac{ACCPL} language was decidable. \ac{ACCPL} started out as a reduced version of Pucella and Weissman`s subset of \ac{odrl}~\cite{pucella2006} but quickly morphed into something different based on the direction of the proof we were developing. An important lesson learned was that the process of designing a certifiable policy language cannot be done in isolation since while certifiability of the language derives the design one way, expressiveness requirements of the language usually drive it in the opposite direction. Certifiability cannot be a post-design activity as the changes it will impose are often fundamental ones.

In the following section we will list the main theorems that we have proved and the supporting theorems that needed to be proved to lead up to the main results. In general however we will not show the actual proofs in the dissertation (with some exceptions) and instead refer the reader to the GitHub repository where the source code for \ac{ACCPL}, including the proofs, lives~\cite{BahmanSistany2015}.

\section{Decidability of ACCPL}\label{sec:maintheorems}

The theorem $trans\_agreement\_dec$ is the declaration of the main decidability result for \ac{ACCPL}(see listing~\ref{lst:agreementdecidablecoq}). Together with other theorems we will describe below, we have ``certified'' the decidability of \ac{ACCPL}. In words, the nonempty list that the agreement translation function $trans\_agreement$ returns will return a set of results per each $primPolicy$. Most if not all of the results will be of type Unregulated. In the case where all the results are Unregulated we answer the access query with a result of Unregulated. However in the case where the whole set is not comprised of Unregulated results, we have two mutually exclusive cases. The first case is when the set has a at least one Permitted result; we answer the access query in this case with a result of Permitted. The second case is when the set has at least one NotPermitted; we answer the access query in this case with a result of NotPermitted (see answer and result types in~\ref{sec:answerandresulttypes}).


\lstset{language=Coq, frame=single, caption={Agreement Translation is Decidable},label={lst:agreementdecidablecoq}}
%\begin{minipage}[c]{0.95\textwidth}
\begin{lstlisting}
Theorem trans_agreement_dec:
  forall
  (e:environment)(ag:agreement)(action_from_query:act)
   (subject_from_query:subject)(asset_from_query:asset),

 (isResultInQueryResult 
    (Result Permitted subject_from_query action_from_query asset_from_query)
    (trans_agreement e ag action_from_query subject_from_query asset_from_query)) 
\/
 (isResultInQueryResult 
    (Result NotPermitted subject_from_query action_from_query asset_from_query)
    (trans_agreement e ag action_from_query subject_from_query asset_from_query))
\/
 (isResultInQueryResult 
    (Result Unregulated subject_from_query action_from_query asset_from_query)
    (trans_agreement e ag action_from_query subject_from_query asset_from_query)).
\end{lstlisting}
%\end{minipage}


The next three theorems establish the mutual exclusivity of Permitted and NotPermitted results. The $trans\_agreement\_perm\_implies\_not\_notPerm\_dec$ (see listing~\ref{lst:permimpliesnotnotperm}) says that the existence of a Permitted result in the set of results returned by $trans\_agreement$, excludes a NotPermitted result while  $trans\_agreement\_NotPerm\_implies\_not\_Perm\_dec$ (see listing~\ref{lst:notpermimpliesnotperm}) shows the inverse. Finally the proof for $trans\_agreement\_not\_Perm\_and\_ \newline NotPerm\_at\_once$ establishes that both Permitted and NotPermitted results cannot exist in the same set returned by $trans\_agreement$ (see listing~\ref{lst:permandnotpermmutualexclusive}). This result also establishes the fact that in \ac{ACCPL} conflicts are not possible.


\lstset{language=Coq, frame=single, caption={Permitted result Implies no NotPermitted result},label={lst:permimpliesnotnotperm}}
%\begin{minipage}[c]{0.95\textwidth}
\begin{lstlisting}
Theorem trans_agreement_perm_implies_not_notPerm_dec:
  forall
    (e:environment)(ag:agreement)(action_from_query:act)
    (subject_from_query:subject)(asset_from_query:asset),

    (isResultInQueryResult 
      (Result Permitted subject_from_query action_from_query asset_from_query)
        (trans_agreement e ag action_from_query subject_from_query asset_from_query)) ->

   ~(isResultInQueryResult 
      (Result NotPermitted subject_from_query action_from_query asset_from_query)
       (trans_agreement e ag action_from_query subject_from_query asset_from_query)).
\end{lstlisting}
%\end{minipage}


\lstset{language=Coq, frame=single, caption={NotPermitted result Implies no Permitted result},label={lst:notpermimpliesnotperm}}
%\begin{minipage}[c]{0.95\textwidth}
\begin{lstlisting}
Theorem trans_agreement_NotPerm_implies_not_Perm_dec:
  forall
    (e:environment)(ag:agreement)(action_from_query:act)
    (subject_from_query:subject)(asset_from_query:asset),

    (isResultInQueryResult 
      (Result NotPermitted subject_from_query action_from_query asset_from_query)
        (trans_agreement e ag action_from_query subject_from_query asset_from_query)) ->

   ~(isResultInQueryResult 
      (Result Permitted subject_from_query action_from_query asset_from_query)
       (trans_agreement e ag action_from_query subject_from_query asset_from_query)).
\end{lstlisting}
%\end{minipage}


\lstset{language=Coq, frame=single, caption={Permitted and NotPermitted results mutually exclusive},label={lst:permandnotpermmutualexclusive}}
%\begin{minipage}[c]{0.95\textwidth}
\begin{lstlisting}
Theorem trans_agreement_not_Perm_and_NotPerm_at_once:
  forall
  (e:environment)(ag:agreement)(action_from_query:act)
   (subject_from_query:subject)(asset_from_query:asset),


 ~((isResultInQueryResult 
    (Result Permitted subject_from_query action_from_query asset_from_query)
    (trans_agreement e ag action_from_query subject_from_query asset_from_query)) 
/\

 (isResultInQueryResult 
    (Result NotPermitted subject_from_query action_from_query asset_from_query)
    (trans_agreement e ag action_from_query subject_from_query asset_from_query))).

\end{lstlisting}
%\end{minipage}

The proof for the next theorem $trans\_agreement\_not\_NotPerm\_and\_not\_Perm \newline \_implies\_Unregulated\_dec$ shows that in the case where neither a Permitted nor NotPermitted result exists in the set returned by $trans\_agreement$, there does exist at least one Unregulated result (see listing~\ref{lst:notpermandnotpermimpliesunregulated}).

\lstset{language=Coq, frame=single, caption={Not (Permitted and NotPermitted) Implies Unregulated},label={lst:notpermandnotpermimpliesunregulated}}
%\begin{minipage}[c]{0.95\textwidth}
\begin{lstlisting}

Theorem trans_agreement_not_NotPerm_and_not_Perm_implies_Unregulated_dec:
  forall
    (e:environment)(ag:agreement)(action_from_query:act)
    (subject_from_query:subject)(asset_from_query:asset),

    (~(isResultInQueryResult 
      (Result Permitted subject_from_query action_from_query asset_from_query)
        (trans_agreement e ag action_from_query subject_from_query asset_from_query)) /\

    ~(isResultInQueryResult 
      (Result NotPermitted subject_from_query action_from_query asset_from_query)
        (trans_agreement e ag action_from_query subject_from_query asset_from_query))) ->

   (isResultInQueryResult 
      (Result Unregulated subject_from_query action_from_query asset_from_query)
       (trans_agreement e ag action_from_query subject_from_query asset_from_query)).
\end{lstlisting}
%\end{minipage}


\section{Proofs Hierarchy}

All the theorems shown in the listings of section~\ref{sec:maintheorems}, specify $trans\_agreement$ as the top translation function that returns a set of results one per the $primPolicy$ in the agreement. We have developed the proofs in a modular fashion such that the hierarchy of the proofs follow the structure of \ac{ACCPL} language constructs very closely. For example, we have a version of the main theorem~\ref{lst:agreementdecidablecoq} for the translation functions for a $policySet$, a $primInclusivePolicySet$, a $primExclusivePolicySet$, etc. See listings~\ref{lst:policysetdecidablecoq} for the version for the translation function for a $policySet$.

\lstset{language=Coq, frame=single, caption={PolicySet Translation is Decidable},label={lst:policysetdecidablecoq}}
%\begin{minipage}[c]{0.95\textwidth}
\begin{lstlisting}

Theorem trans_ps_dec:
  forall
  (e:environment)(action_from_query:act)(subject_from_query:subject)(asset_from_query:asset)
  (ps:policySet)
  (prin_u:prin)(a:asset),

 (isResultInQueryResult 
    (Result Permitted subject_from_query action_from_query asset_from_query)
    (trans_ps e action_from_query subject_from_query asset_from_query ps prin_u a)) 
\/

 (isResultInQueryResult 
    (Result NotPermitted subject_from_query action_from_query asset_from_query)
    (trans_ps e action_from_query subject_from_query asset_from_query ps prin_u a)) 
\/

 (isResultInQueryResult 
    (Result Unregulated subject_from_query action_from_query asset_from_query)
    (trans_ps e action_from_query subject_from_query asset_from_query ps prin_u a)).

\end{lstlisting}
%\end{minipage}


While each proof builds on top of a previously proven decidability theorem based on the structure of the language construct, the pattern of the propositions to prove doesn`t always stay the same. For example, the proof for $trans\_ps\_dec$ uses the proofs for the following two theorems: $trans\_policy\_PIPS\_dec$ and $trans\_policy\_PEPS\_dec$. However the statement of decidability is different for $trans\_policy\_PIPS\_dec$ as the set returned by this translation function will not contain a NotPermitted result; either a Permitted or an Unregulated result will be returned in this case (see listing~\ref{lst:policysetPIPSdecidablecoq}).

\lstset{language=Coq, frame=single, caption={Primitive Inclusive PolicySet Translation is Decidable},label={lst:policysetPIPSdecidablecoq}}
%\begin{minipage}[c]{0.95\textwidth}
\begin{lstlisting}
Theorem trans_policy_PIPS_dec:
  forall
  (e:environment)(prq: preRequisite)(p:policy)(subject_from_query:subject)
  (prin_u:prin)(a:asset)(action_from_query:act),

 (isResultInQueryResult
    (Result Permitted subject_from_query action_from_query a)
    (trans_policy_PIPS e prq p subject_from_query prin_u a action_from_query))
\/

  (isResultInQueryResult
    (Result Unregulated subject_from_query action_from_query a)
    (trans_policy_PIPS e prq p subject_from_query prin_u a action_from_query)).
Proof.

\end{lstlisting}
%\end{minipage}

\section{ACCPL in the Landscape of Policy-based Policy Languages}

Tschantz and Krishnamurthi~\cite{Tschantz} argue for the need for formal means to compare and contrast policy-based access-control languages and they present a set of properties to analyze the behaviour of policies in light of additional and/or explicit environmental facts, policy growth and policy decomposition. Tschantz and Krishnamurthi apply their properties to two core policy languages and compare the results. One of the core policy languages they present is a simplified \ac{xacml} and the other is Lithium~\cite{Halpern2008}. In the following sections we will describe Tschantz and Krishnamurthi's properties and evaluate \ac{ACCPL} with respect to these properties.

\section{Reasonability Properties}

The properties Tschantz and Krishnamurthi~\cite{Tschantz} present revolve around three questions:

\begin{enumerate}
\item How decisions change when the environment has more facts
\item Impact of policy growth on how decisions may change
\item How amenable policies are to compositional reasoning
\end{enumerate}

Tschantz and Krishnamurthi~\cite{Tschantz} use a simple policy as a motivating example (which they attribute to~\cite{Halpern2008}) that we will adapt and reuse here. The example will help to review the range of interpretations possible and ultimately the range of policy language design choices and where \ac{ACCPL} is on that range.
The policy in English is as follows:
\subsection{Basic Policy and Query}\label{sec:basicpolicyenglish}
\begin{itemize}
\item If the subject has a subscription to the Gold bundle, then permit that subject to print TheReport.

\item If the subject has a subscription to the Basic bundle, then do not permit that subject to print TheReport.

\item If the subject has no subscription to the Gold bundle, then permit that subject to play Terminator.
\end{itemize}

Consider a query and a fact in English as: A subscriber to the Basic bundle requests to play the movie Terminator. Let`s extract the fact that the subject is a subscriber to the Basic bundle into a separate term and make it part of the environment. The basic policy, the query in question and the environment are all expressed as a (pseudo) logic formula and listed in the listing~\ref{lst:basicpolicyAST}. Should the access query be granted?

\lstset{mathescape, language=AST}  
\begin{lstlisting}[frame=single, caption={Basic Policy},label={lst:basicpolicyAST}]

p1 = GoldBundle(s) -> Permitted(s, print, TheReport) 
p2 = BasicBundle(s) -> NotPermitted(s, print, TheReport) 
p3 = ~GoldBundle(s) -> Permitted(s, play, Terminator)
q1 = (s, play, Terminator) 
e1 = BasicBundle(s)
\end{lstlisting}


According to Tschantz and Krishnamurthi~\cite{Tschantz} at least three different interpretations are possible. We will list the possibilities and then evaluate how \ac{ACCPL} would answer the query.

\subsection{Possible Interpretations}\label{sec:threeinterpretations}
\begin{itemize}
\item Implicit: Grant access because of $p3$. This interpretation assumes that if a subject has a subscription to the Basic bundle, the subject does not have a subscription to the Basic bundle. This is an example of ``implicit'' knowledge and ``closed world assumptions''.

\item Explicit: The policy does not apply to the query. In other words, the query is ``Unregulated''. $p1$ and $p2$ don`t apply since the asset and the action in the request don`t match the ones in $p1$ and $p2$. $p3$ does not apply either since no fact exists to show that $~GoldBundle(s)$. This interpretation is similar to how \ac{ACCPL} would answer the query.

\item Implicit with automatic proof capability: grants the query by automatically proving that $~GoldBundle(s)$. Under this interpretation, proof by contradiction is used to establish $~GoldBundle(s)$ since if $GoldBundle(s)$, then $p1$ and $p2$ would contradict.
\end{itemize}

\section{Policy Based Access-Control Languages}

Tschantz and Krishnamurthi~\cite{Tschantz} define an access-control policy language as a tuple $L = (P, Q, G, N, \left\langle\left\langle . \right\rangle\right\rangle)$ where $P$ is a set of policies, $Q$ is a set of requests or queries, $G$ is the granting decisions, $N$ is the non-granting decisions, $\left\langle\left\langle . \right\rangle\right\rangle$ is a function taking a policy $p \in P$ to a relation between Q and $G \cup N$ where $G \cap N = \emptyset$. Let $D$ represent $G \cup N$. Policy $p$ assigns a decision of $d \in D$ to the query $q \in Q$. $L$ also defines a partial order on decisions such that $d <= d'$ if either $d, d' \in N$ or $d, d' \in G$ or $d \in N$ and $d' \in G$, so for example $NotPermitted <= Permitted$ in \ac{ACCPL}.

\section{Types of Decisions}

Policy based access-control languages typically support at least the Permit and Deny decisions. When a decision for a query is not granted, one design choice for a language is to return a Deny. However in this case Deny stands for ``not Permit''. It is possible to have cases when the policy truly doesn't specify either Permit or Deny. In such cases arbitrarily returning the decision of Deny makes it difficult to compose policies and an explicit decision of NonApplicable or Unregulated is warranted. Some languages may decide to only have Permit decisions. In such languages lack of a Permit decision for a query signifies a Deny so Deny decisions are not explicit. Although the policies of these languages may be more readable than those with more explicit decisions but they result in ambiguity on whether a Deny decision was really intended or not. Finally some languages define an explicit decision of ``Error'' for cases such as when both Permit and Deny decisions are reached for the same query. An explicit Error decision is preferable to undefined behaviour because it can lead to improvements to policies and/or how the queries are built. 

\section{Policy Combinators}

Primitive policies are often developed by independent entities within an organization therefore it is natural to want a way of combining them into a policy for the whole organization using policy combinators. Some languages such \ac{odrl} have nested sub-policies inside other policies. In such cases one design decision could be to allow for different combinators for different layers to make the language more expressive. Some examples of policy combinators are: conjunction, disjunction, exclusive disjunction, Permit override (if any of the primitive policies returns a Permit, return only Permit), Deny override (if any of the primitive policies returns a Deny, return only Deny).

\section{ACCPL}

When designing \ac{ACCPL} and to make the decidability proofs possible, we sought out all the different combinations and cases possible (through interaction with the proof process) and explicitly enumerated them in translation functions. Once we made explicit all the different cases where a decision was possible, we assigned Permitted and NotPermitted decisions (equivalent to Permit and Deny decisions in this chapter) to two specific cases while every other enumerated case was rendered with an Unregulated (not applicable) decision. Explicit enumeration of all the different cases has been certified to be complete by the fact that we have the decidability results. For future work, one can modify the decision assignments to different cases, deciding for the semantics to give more Permit decisions for example or more Deny decisions or other changes in the same vein. One can also remove the Unregulated decisions from the translation functions and assign instead Permit or Deny depending on whether we want to make the language more permissive or more prohibitive. In this case however the current decidability theorem will no longer hold and the Unregulated case would have to be removed from the statement of the theorem and the proof to be modified (see listing~\ref{lst:agreementdecidablecoqSecond}).

\lstset{language=Coq, frame=single, caption={Agreement Translation is Decidable (for Permitted or NotPermitted only)},label={lst:agreementdecidablecoqSecond}}
%\begin{minipage}[c]{0.95\textwidth}
\begin{lstlisting}
Theorem trans_agreement_dec:
  forall
  (e:environment)(ag:agreement)(action_from_query:act)
   (subject_from_query:subject)(asset_from_query:asset),

 (isResultInQueryResult 
    (Result Permitted subject_from_query action_from_query asset_from_query)
    (trans_agreement e ag action_from_query subject_from_query asset_from_query)) 
\/
 (isResultInQueryResult 
    (Result NotPermitted subject_from_query action_from_query asset_from_query)
    (trans_agreement e ag action_from_query subject_from_query asset_from_query)).

\end{lstlisting}
%\end{minipage}
























