%======================================================================
\chapter{Formal Proofs and Correctness}\label{chap:results}
%======================================================================

                  
%----------------------------------------------------------------------
The tension in designing a policy language is usually between how to make the language expressive enough, such that the design goals for the policy language may be expressed, and how to make the policies verifiable with respect to the stated goals. A central question or goal in policy language design is whether a permission is implied by a set of policy statements. This is referred to as the decidability problem in this thesis. For example, Halpern and Weissman~\cite{HalpernW08} show that in general determining whether a permission is implied by a set of policy statements in \ac{xrml} is undecidable whereas Pucella and Weissman show that answering such questions is decidable but NP-hard for \ac{odrl}~\cite{pucella2006}.
 

We had set out to prove formally that the \ac{ACCPL} language had certifiably correct semantics. Recall that we specified the semantics of \ac{ACCPL} using a translation from policy statements together with an access request and an environment containing all relevant facts, to decisions. The goal was to use the translation algorithm to determine when a request for accessing a resource is implied by a given policy, when the request is not implied by the policy and finally when the request is not regulated by the given policy. More explicitly we wanted to show that answering access requests always produced a decision and that the translation algorithm terminated on all input policies with either a decision of \syn{Permitted}, or a decision of\syn{NotPermitted} or a decision of neither \syn{Permitted} nor \syn{NotPermitted}. In this chapter we discuss the correctness proofs and other main results we had set out to accomplish. In doing so we will show that the translation algorithm meets its specified goal (as stated earlier), given a policy. 


\ac{ACCPL} started out as a reduced version of Pucella and Weissman's subset of \ac{odrl}\\~\cite{pucella2006} but based on the direction of the proofs we were developing significant changes had to be made to Pucella and Weissman's language to make the proofs possible. An important lesson learned was that the process of designing a certifiably correct policy language cannot be done in isolation since while certifying correctness for the language derives the design one way, expressiveness requirements of the language usually drive it in the opposite direction. Certifying a policy language correct cannot be a post-design activity as the changes it will impose are often fundamental ones.

In the following section we will list the main theorems that we have proved and the supporting theorems that needed to be proved to lead up to the main results. In general however we will not show the actual proofs in the dissertation (with some exceptions) and instead refer the reader to the source code for \ac{ACCPL}, including the proofs, which lives at~\cite{BahmanSistany2015}.

\section{Correctness of ACCPL}\label{sec:maintheorems}

The theorem \syn{trans_agreement_dec2} is the declaration of the main correctness result for \ac{ACCPL} (see listing~\ref{lst:agreementdecidablecoq}). Together with other theorems we will describe below, we have ``certified'' \ac{ACCPL} correct by proving this theorem. The nonempty list that the agreement translation function \syn{trans_agreement} returns will contain results one per each primitive policy (\syn{primPolicy}) found in the agreement. Specifically the predicate \syn{isResultInQueryResult} takes a \syn{result} and a nonempty list of \syn{result}'s which \syn{trans_agreement} produces, and calls the \syn{In} predicate. The \syn{In} predicate is adapted from the \syn{Coq.Lists.List} standard library and checks for the existence of the input \syn{result} in the nonempty list of results. Definitions for predicates \syn{isResultInQueryResult} and \syn{In} are listed in listing~\ref{lst:agreementdecidablecoq}. The definitions for \syn{answer} and \syn{result} are repeated in the same listing. 


Note that by mentioning the agreement translation function directly in the statement of the theorem~\ref{lst:agreementdecidablecoq}, we tie the correcness property to how the translation functions work. To prove the theorem and with each successive subgoal during the interactive proof process, the definition of the translation function in scope gets unfolded and used so the translation functions have to be defined such that each subgoal is discharged and the proof is completed. This is quite different from trying to show a list has some property. 


As an example and also a visual aid to understanding how queries are answered, see listing~\ref{lst:permittedcoq}. The \syn{isResultInQueryResult} predicate looks for a result with an answer of Permitted in the list that \syn{trans_agreement} has produced, for an agreement for three primitive policies (since the set contains three results). In words, we are asking whether Alice is allowed to print the asset ebook, given a policy. 

\lstset{language=Coq, frame=single, caption={Access Request Resulting in Decision of \syn{Permitted}},label={lst:permittedcoq}}
%\begin{minipage}[c]{0.95\textwidth}
\begin{lstlisting}
isResultInQueryResult
(Result Permitted Alice Print ebook) 
[ (Result Unregulated Alice Print ebook) ; (Result Unregulated Alice Print ebook) ; (Result Permitted Alice Print ebook) ]
\end{lstlisting}
%\end{minipage}
    
In the case where the whole set is not comprised of \syn{Unregulated} results, we have two mutually exclusive cases. The first case is when the set has a at least one \syn{Permitted} result; we answer the access query in this case with a result of \syn{Permitted} (this would be the case in the listing~\ref{lst:permittedcoq}). The second case is when the set has at least one \syn{NotPermitted}; we answer the access query in this case with a result of \syn{NotPermitted}. 


\lstset{language=Coq, frame=single, caption={Agreement Translation's Correctness Property},label={lst:agreementdecidablecoq}}
\begin{minipage}[c]{0.95\textwidth}
\begin{lstlisting}
Definition isResultInQueryResult (res:result)(results: nonemptylist result) : Prop :=
  In res results.

Fixpoint In (a:X) (l:nonemptylist) : Prop :=
    match l with
      | Single s => s=a
      | NewList s rest => s = a \/ In a rest
    end.

Inductive answer : Set :=
  | Permitted : answer
  | Unregulated : answer
  | NotPermitted : answer.

Inductive result : Set :=
  | Result : answer -> subject -> act -> asset -> result.

Theorem trans_agreement_dec2:
  forall
  (e:environment)(ag:agreement)(action_from_query:act)
   (subject_from_query:subject)(asset_from_query:asset),


 (isResultInQueryResult 
    (Result Permitted subject_from_query action_from_query asset_from_query)
    (trans_agreement e ag action_from_query subject_from_query asset_from_query)) 
\/

 (isResultInQueryResult 
    (Result NotPermitted subject_from_query action_from_query asset_from_query)
    (trans_agreement e ag action_from_query subject_from_query asset_from_query))
\/

 (~(isResultInQueryResult 
    (Result Permitted subject_from_query action_from_query asset_from_query)
    (trans_agreement e ag action_from_query subject_from_query asset_from_query)) /\
  ~(isResultInQueryResult 
    (Result NotPermitted subject_from_query action_from_query asset_from_query)
    (trans_agreement e ag action_from_query subject_from_query asset_from_query))).

\end{lstlisting}
\end{minipage}

Typically most, if not all of the results will be of type \syn{Unregulated}. In the case where all the results are \syn{Unregulated} we answer the access query with a result of \syn{Unregulated}. We show this case indirectly in the theorem in listing~\ref{lst:agreementdecidablecoq} by stating the set does not contain a \syn{Permitted} result nor a \syn{NotPermitted} result. 

\section{Mutual Exclusivity of \syn{Permitted} and \syn{NotPermitted}}\label{sec:mutualexclusive}

The \syn{trans_agreement_perm_implies_not_notPerm_dec} (see listing~\ref{lst:permimpliesnotnotperm}) says that the existence of a \syn{Permitted} result in the set of results returned by \syn{trans_agreement}, excludes a \syn{NotPermitted} result.


\lstset{language=Coq, frame=single, caption={\syn{Permitted} result Implies no \syn{NotPermitted} result},label={lst:permimpliesnotnotperm}}
\begin{minipage}[c]{0.95\textwidth}
\begin{lstlisting}
Theorem trans_agreement_perm_implies_not_notPerm_dec:
  forall
    (e:environment)(ag:agreement)(action_from_query:act)
    (subject_from_query:subject)(asset_from_query:asset),

    (isResultInQueryResult 
      (Result Permitted subject_from_query action_from_query asset_from_query)
        (trans_agreement e ag action_from_query subject_from_query asset_from_query)) ->

   ~(isResultInQueryResult 
      (Result NotPermitted subject_from_query action_from_query asset_from_query)
       (trans_agreement e ag action_from_query subject_from_query asset_from_query)).
\end{lstlisting}
\end{minipage}


The  \syn{trans_agreement_NotPerm_implies_not_Perm_dec} (see listing~\ref{lst:notpermimpliesnotperm}) shows that the existence of a \syn{NotPermitted} result in the set of results returned by \syn{trans_agreement}, excludes a \syn{Permitted} result.

\lstset{language=Coq, frame=single, caption={\syn{NotPermitted} result Implies no \syn{Permitted} result},label={lst:notpermimpliesnotperm}}
%\begin{minipage}[c]{0.95\textwidth}
\begin{lstlisting}
Theorem trans_agreement_NotPerm_implies_not_Perm_dec:
  forall
    (e:environment)(ag:agreement)(action_from_query:act)
    (subject_from_query:subject)(asset_from_query:asset),

    (isResultInQueryResult 
      (Result NotPermitted subject_from_query action_from_query asset_from_query)
        (trans_agreement e ag action_from_query subject_from_query asset_from_query)) ->

   ~(isResultInQueryResult 
      (Result Permitted subject_from_query action_from_query asset_from_query)
       (trans_agreement e ag action_from_query subject_from_query asset_from_query)).
\end{lstlisting}
%\end{minipage}

Finally the proof for \syn{trans_agreement_not_Perm_and_ \newline NotPerm_at_once} establishes that both \syn{Permitted} and \syn{NotPermitted} results cannot exist in the same set returned by \syn{trans_agreement} (see listing~\ref{lst:permandnotpermmutualexclusive}). This result also establishes the fact that in \ac{ACCPL} rendering conflicting decisions is not possible given an agreement. 

Note that the ``no conflict'' result does not apply across agreements. In particular, we don't handle the case of multiple agreements with the same asset and set of users which may be an interesting sub-case. The result applies to a single agreement and specifically to the \syn{<policySet>} field of an agreement. Recall that the \syn{<policySet>} element expands into the \syn{<policy>} element which in turn contains multiple \syn{<primPolicy>}s each with multiple \syn{<preRequisite>}s. In this way, we cover the majority of cases that conflict detection algorithms typically cover, including the most relevant ones.

\lstset{language=Coq, frame=single, caption={\syn{Permitted} and \syn{NotPermitted}: Mutually Exclusive},label={lst:permandnotpermmutualexclusive}}
%\begin{minipage}[c]{0.95\textwidth}
\begin{lstlisting}
Theorem trans_agreement_not_Perm_and_NotPerm_at_once:
  forall
  (e:environment)(ag:agreement)(action_from_query:act)
   (subject_from_query:subject)(asset_from_query:asset),


 ~((isResultInQueryResult 
    (Result Permitted subject_from_query action_from_query asset_from_query)
    (trans_agreement e ag action_from_query subject_from_query asset_from_query)) 
/\

 (isResultInQueryResult 
    (Result NotPermitted subject_from_query action_from_query asset_from_query)
    (trans_agreement e ag action_from_query subject_from_query asset_from_query))).

\end{lstlisting}
%\end{minipage}

The proof for the next theorem \syn{trans_agreement_not_NotPerm_and_not_Perm \newline _implies_Unregulated_dec} shows that in the case where neither a \syn{Permitted} nor a \syn{NotPermitted} result exists in the set returned by \syn{trans_agreement}, there does exist at least one \syn{Unregulated} result (see listing~\ref{lst:notpermandnotpermimpliesunregulated}).

\lstset{language=Coq, frame=single, caption={Not (\syn{Permitted} and \syn{NotPermitted}) Implies \syn{Unregulated}},label={lst:notpermandnotpermimpliesunregulated}}
\begin{minipage}[c]{0.95\textwidth}
\begin{lstlisting}

Theorem trans_agreement_not_NotPerm_and_not_Perm_implies_Unregulated_dec:
  forall
    (e:environment)(ag:agreement)(action_from_query:act)
    (subject_from_query:subject)(asset_from_query:asset),

    (~(isResultInQueryResult 
      (Result Permitted subject_from_query action_from_query asset_from_query)
        (trans_agreement e ag action_from_query subject_from_query asset_from_query)) /\

    ~(isResultInQueryResult 
      (Result NotPermitted subject_from_query action_from_query asset_from_query)
        (trans_agreement e ag action_from_query subject_from_query asset_from_query))) ->

   (isResultInQueryResult 
      (Result Unregulated subject_from_query action_from_query asset_from_query)
       (trans_agreement e ag action_from_query subject_from_query asset_from_query)).
\end{lstlisting}
\end{minipage}


\section{Decidable as an Inductive Predicate}

We encode the cases discussed in Section~\ref{sec:maintheorems} in an inductively defined predicate called \syn{decidable} in listing~\ref{lst:inductivepredicatedecidablecoq}. Recall that the nonempty list that the agreement translation function \syn{trans_agreement} returns will return a set of results one per each \syn{primPolicy}. There are three cases. First when there is at least a \syn{NotPermitted} result in the set and there is no \syn{Permitted} result, second when there is at least a \syn{Permitted} result in the set and no \syn{NotPermitted} result, and third when there are no \syn{Permitted} and no \syn{NotPermitted} results in the set. We encode these three cases in three constructors for the type \syn{decidable}. The constructors are: \syn{Denied}, \syn{Granted} and \syn{NotApplicable}. Notice that we use \syn{Denied}, \syn{Granted} and \syn{NotApplicable} synonymously with \syn{NotPermitted}, \syn{Permitted} and \syn{Unregulated}. The reason is a practical one and is due to Coq; we needed to use new names for the \syn{decidable} constructors since we already used \syn{NotPermitted}, \syn{Permitted} and \syn{Unregulated} as constructors for \syn{answer}. 

\lstset{language=Coq, frame=single, caption={Inductive Predicate decidable},label={lst:inductivepredicatedecidablecoq}}
\begin{minipage}[c]{0.95\textwidth}
\begin{lstlisting}

Inductive decidable : environment -> agreement -> act -> subject -> asset -> Prop :=

     | Denied : forall
  (e:environment)(ag:agreement)(action_from_query:act)
   (subject_from_query:subject)(asset_from_query:asset), 

 (isResultInQueryResult 
    (Result NotPermitted subject_from_query action_from_query asset_from_query)
       (trans_agreement e ag action_from_query subject_from_query asset_from_query)) 
  ->
 ~(isResultInQueryResult 
    (Result Permitted subject_from_query action_from_query asset_from_query)
       (trans_agreement e ag action_from_query subject_from_query asset_from_query)) 

  -> decidable e ag action_from_query subject_from_query asset_from_query

     | Granted : forall
  (e:environment)(ag:agreement)(action_from_query:act)
   (subject_from_query:subject)(asset_from_query:asset), 
 (isResultInQueryResult 
    (Result Permitted subject_from_query action_from_query asset_from_query)
       (trans_agreement e ag action_from_query subject_from_query asset_from_query)) 
 ->
 ~(isResultInQueryResult 
    (Result NotPermitted subject_from_query action_from_query asset_from_query)
       (trans_agreement e ag action_from_query subject_from_query asset_from_query)) 
 
 -> decidable e ag action_from_query subject_from_query asset_from_query

     | NonApplicable : forall
  (e:environment)(ag:agreement)(action_from_query:act)
   (subject_from_query:subject)(asset_from_query:asset), 

  ~(isResultInQueryResult 
    (Result Permitted subject_from_query action_from_query asset_from_query)
    (trans_agreement e ag action_from_query subject_from_query asset_from_query)) 
     ->
  ~(isResultInQueryResult 
    (Result NotPermitted subject_from_query action_from_query asset_from_query)
    (trans_agreement e ag action_from_query subject_from_query asset_from_query)) 
  -> decidable e ag action_from_query subject_from_query asset_from_query.
\end{lstlisting}
\end{minipage}


We declare and prove the correctness of the semantics for \ac{ACCPL} in terms of the \syn{decidable} predicate as seen in listing~\ref{lst:decidabletypecoq}. The proof is made simple since we use the previously proven theorem \syn{trans_agreement_dec2}. The proof also makes use of two theorems we saw earlier in Section~\ref{sec:mutualexclusive}. Recall that the first theorem, \syn{trans_agreement_perm_implies_not_notPerm_dec}, says that the existence of a \syn{Permitted} result in the set of results returned by \syn{trans_agreement}, will exclude any \syn{NotPermitted} result and that the second theorem, \syn{trans_agreement_NotPerm_implies_not_Perm_dec}, shows that the existence of a \syn{NotPermitted} result in the set of results returned by \syn{trans_agreement}, will exclude a \syn{Permitted} result. The proof shows the series of tactics that were applied to 
complete the proof interactively. We do not explain the tactics here, but just show the proof to illustrate by example what proofs look like.

We will list the intermediate theorems that we have declared and proved which are used in the proof of theorem \syn{trans_agreement_dec2} and subsequent proofs in Section~\ref{sec:intermediatetheorems}.

\lstset{language=Coq, frame=single, caption={Correctness Of Agreement Translation},label={lst:decidabletypecoq}}
%\begin{minipage}[c]{0.95\textwidth}
\begin{lstlisting}
Theorem trans_agreement_decidable:
  forall
  (e:environment)(ag:agreement)(action_from_query:act)
   (subject_from_query:subject)(asset_from_query:asset),
     decidable e ag action_from_query subject_from_query asset_from_query.

Proof.
intros e ag action_from_query subject_from_query asset_from_query.
specialize trans_agreement_dec2 with 
   e ag action_from_query subject_from_query asset_from_query.
intros H. destruct H as [H1 | H2].

apply Granted. assumption.
simpl. intuition.
apply trans_agreement_perm_implies_not_notPerm_dec in H1. contradiction. 

destruct H2 as [H21 | H22].
apply Denied. assumption.
simpl. intuition.
apply trans_agreement_NotPerm_implies_not_Perm_dec in H21. contradiction. 

apply NonApplicable. 
destruct H22 as [H221 H222].
exact H221.
destruct H22 as [H221 H222].
exact H222.
Defined.
\end{lstlisting}
%\end{minipage}

 

\section{Intermediate Theorems}\label{sec:intermediatetheorems}
The theorem \syn{trans_agreement_dec_sb_Permitted} states as a \syn{sumbool} that the nonempty list that \syn{trans_agreement} produces, contains a \syn{Permitted} result or not and this is decidable (see listing~\ref{lst:decsbpermitted}).


\lstset{language=Coq, frame=single, caption={Translation Function for Agreement Returns a \syn{Permitted} result or Not},label={lst:decsbpermitted}}
%\begin{minipage}[c]{0.95\textwidth}
\begin{lstlisting}
Theorem trans_agreement_dec_sb_Permitted:
  forall
  (e:environment)(ag:agreement)(action_from_query:act)
   (subject_from_query:subject)(asset_from_query:asset),


 {(isResultInQueryResult 
    (Result Permitted subject_from_query action_from_query asset_from_query)
    (trans_agreement e ag action_from_query subject_from_query asset_from_query))} +
 {~(isResultInQueryResult 
    (Result Permitted subject_from_query action_from_query asset_from_query)
    (trans_agreement e ag action_from_query subject_from_query asset_from_query))}.

\end{lstlisting}
%\end{minipage}

The theorem \syn{trans_agreement_dec_sb_NotPermitted} states as a \syn{sumbool} that the nonempty list that \syn{trans_agreement} produces, contains a \syn{NotPermitted} result or not and this is decidable (see listing~\ref{lst:decsbnotpermitted}).

\lstset{language=Coq, frame=single, caption={Translation Function for Agreement Returns a \syn{NotPermitted} result or Not},label={lst:decsbnotpermitted}}
%\begin{minipage}[c]{0.95\textwidth}
\begin{lstlisting}
Theorem trans_agreement_dec_sb_NotPermitted:
  forall
  (e:environment)(ag:agreement)(action_from_query:act)
   (subject_from_query:subject)(asset_from_query:asset),

 {(isResultInQueryResult 
    (Result NotPermitted subject_from_query action_from_query asset_from_query)
    (trans_agreement e ag action_from_query subject_from_query asset_from_query))} +
 {~(isResultInQueryResult 
    (Result NotPermitted subject_from_query action_from_query asset_from_query)
    (trans_agreement e ag action_from_query subject_from_query asset_from_query))}.

\end{lstlisting}
%\end{minipage}

The theorem \syn{resultInQueryResult_dec} states as a \syn{sumbool} that a given result is either in the nonempty list of \syn{results} or not and this is decidable (see listing~\ref{lst:decresultInResults}).

\lstset{language=Coq, frame=single, caption={Result is in Results or Not},label={lst:decresultInResults}}
%\begin{minipage}[c]{0.95\textwidth}
\begin{lstlisting}
Theorem resultInQueryResult_dec :
    forall (res:result)(results: nonemptylist result), 
 {isResultInQueryResult res results} + {~isResultInQueryResult res results}.
\end{lstlisting}
%\end{minipage}

The theorem \syn{trans_agreement_not_Perm_and_NotPerm_at_once} states that you may not have both a \syn{Permitted} and \syn{NotPermitted} result in the nonempty list that \syn{trans_agreement} produces (see listing~\ref{lst:permandnotpermatonce}).

\lstset{language=Coq, frame=single, caption={Translation Function for Agreement will not have both \syn{Permitted} and \syn{NotPermitted}},label={lst:permandnotpermatonce}}
%\begin{minipage}[c]{0.95\textwidth}
\begin{lstlisting}
Theorem trans_agreement_not_Perm_and_NotPerm_at_once:
  forall
  (e:environment)(ag:agreement)(action_from_query:act)
   (subject_from_query:subject)(asset_from_query:asset),


 ~((isResultInQueryResult 
    (Result Permitted subject_from_query action_from_query asset_from_query)
    (trans_agreement e ag action_from_query subject_from_query asset_from_query)) 
/\

 (isResultInQueryResult 
    (Result NotPermitted subject_from_query action_from_query asset_from_query)
    (trans_agreement e ag action_from_query subject_from_query asset_from_query))).

\end{lstlisting}
%\end{minipage}

The theorem \syn{trans_policy_PIPS_dec_not} states that the translation function \syn{trans_policy_PIPS} produces a nonempty list of results that will not have a \syn{NotPermitted} result. This makes intuitive sense as \syn{trans_policy_PIPS} is called for inclusive policy sets only (see listing~\ref{lst:pipsdecnot}).
\lstset{language=Coq, frame=single, caption={Translation Function for PIPS will not return a \syn{NotPermitted} result},label={lst:pipsdecnot}}
%\begin{minipage}[c]{0.95\textwidth}
\begin{lstlisting}
Theorem trans_policy_PIPS_dec_not:
  forall
  (e:environment)(prq: preRequisite)(p:policy)(subject_from_query:subject)
  (prin_u:prin)(a:asset)(action_from_query:act),

 ~(isResultInQueryResult 
    (Result NotPermitted subject_from_query action_from_query a)
    (trans_policy_PIPS e prq p subject_from_query prin_u a action_from_query)).

\end{lstlisting}
%\end{minipage}

The theorem \syn{trans_policy_PEPS_perm_implies_not_notPerm_dec} states that having a \syn{Permitted} result in the nonempty list of results that the translation function \syn{trans_policy_PEPS} produces implies there exists no \syn{NotPermitted} result in the nonempty list (see listing~\ref{lst:pepspermimpliesnotperm}).

\lstset{language=Coq, frame=single, caption={\syn{Permitted} implies no \syn{NotPermitted}},label={lst:pepspermimpliesnotperm}}
\begin{minipage}[c]{0.95\textwidth}
\begin{lstlisting}
Theorem trans_policy_PEPS_perm_implies_not_notPerm_dec:
  forall
  (e:environment)(prq: preRequisite)(p:policy)(subject_from_query:subject)
  (prin_u:prin)(a:asset)(action_from_query:act),

    (isResultInQueryResult 
      (Result Permitted subject_from_query action_from_query a)
        (trans_policy_PEPS e prq p subject_from_query prin_u a action_from_query)) ->

   ~(isResultInQueryResult 
      (Result NotPermitted subject_from_query action_from_query a)
       (trans_policy_PEPS e prq p subject_from_query prin_u a action_from_query)).

\end{lstlisting}
\end{minipage}

The theorem \syn{AnswersNotEqual} states that two answers not being equal implies that results built from those answers are not equal as well (see listing~\ref{lst:answersnotequal}).

\lstset{language=Coq, frame=single, caption={Non Equal Answers Give Non Equal Results},label={lst:answersnotequal}}
%\begin{minipage}[c]{0.95\textwidth}
\begin{lstlisting}
Theorem AnswersNotEqual: forall (ans1:answer)(ans2:answer)(s:subject)(ac:act)(ass:asset),
  (ans1<>ans2) -> ((Result ans1 s ac ass) <> (Result ans2 s ac ass)).

\end{lstlisting}
%\end{minipage}

The theorem \syn{trans_policy_positive_dec_not} states that the translation function \syn{trans_policy_positive} produces a nonempty list of results that will not have a \syn{NotPermitted} result (see listing~\ref{lst:positivedecnot}).


\lstset{language=Coq, frame=single, caption={Translation Function for Permitting Policies will not return a \syn{NotPermitted} result},label={lst:positivedecnot}}
\begin{minipage}[c]{0.95\textwidth}
\begin{lstlisting}

Theorem trans_policy_positive_dec_not:
  forall 
(e:environment)(s:subject)(p:policy)(prin_u:prin)(a:asset)
  (action_from_query: act),
 
  ~(isResultInQueryResult 
    (Result NotPermitted s action_from_query a)
    (trans_policy_positive e s p prin_u a action_from_query)).
\end{lstlisting}
\end{minipage}

The theorem \syn{trans_policy_negative_dec_not} states that the translation function \syn{trans_policy_negative} produces a nonempty list of results that will not have a \syn{Permitted} result (see listing~\ref{lst:negativedecnot}).

\lstset{language=Coq, frame=single, caption={Translation Function for NotPermitting Policies will not return a \syn{Permitted} result},label={lst:negativedecnot}}
\begin{minipage}[c]{0.95\textwidth}
\begin{lstlisting}
Theorem trans_policy_negative_dec_not:
  forall 
(e:environment)(s:subject)(p:policy)(a:asset)
  (action_from_query: act),
 
  ~(isResultInQueryResult 
    (Result Permitted s action_from_query a)
    (trans_policy_negative e s p a action_from_query)).

\end{lstlisting}
\end{minipage}

The theorem \syn{trans_policy_unregulated_dec_not} states that the translation function \syn{trans_policy_unregulated} produces a nonempty list of results that will not have a \syn{Permitted} nor a \syn{NotPermitted} result (see listing~\ref{lst:unregulateddecnot}).

\lstset{language=Coq, frame=single, caption={Translation Function for \syn{Unregulated} Requests will not return \syn{Permitted} nor \syn{NotPermitted}},label={lst:unregulateddecnot}}

%\begin{minipage}[c]{0.95\textwidth}
\begin{lstlisting}
Theorem trans_policy_unregulated_dec_not:
  forall 
(e:environment)(s:subject)(p:policy)(a:asset)
  (action_from_query: act),
 
  ~(isResultInQueryResult 
    (Result Permitted s action_from_query a)
    (trans_policy_unregulated e s p a action_from_query)) /\
  
  ~(isResultInQueryResult 
    (Result NotPermitted s action_from_query a)
    (trans_policy_unregulated e s p a action_from_query)).
\end{lstlisting}
%\end{minipage}















