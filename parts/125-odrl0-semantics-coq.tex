%======================================================================
\chapter{ACCPL Semantics In Coq}
%======================================================================

% ---------------------------------- COQ -----------------------

\section{Introduction}
\label{sec:envConsistentP}  

% COQ
The translation functions plus the auxiliary types and infrastructure which implement the semantics have been encoded in Coq. Translation functions build Coq terms of type $Prop$. Well-formed propositions (or $Prop$s) are assertions one can express about values such as mathematical objects or even programs (e.g. 3 < 8) in Coq. 

Whether a permission is granted or denied depends on the agreements in question but also on the facts recorded in the environment. For ODRL0 those facts revolve around the number of times a policy has been used to justify an action (see section~\ref{sec:odrl0} for more details on odrl0). We encode this information in an \emph{environment} which is a conjunction of equalities of the form $count(s, policyId) = n$. 

% COQ
The Coq version of the count equality is a new inductive type called \emph{count\_equality}. An environment is defined to be a non-empty list of $count\_equality$ objects (see listing~\ref{lst:environmentcoq}). Function $make\_count\_equality$ in listing~\ref{lst:environmentcoq} is simply a convenience function that builds $count\_equality$s. For an example of how environments are created see listing~\ref{lst:environmentusagecoq}.

\lstset{language=Coq, frame=single, caption={Environments and Counts},label={lst:environmentcoq}}
%\begin{minipage}[c]{0.95\textwidth}
\begin{lstlisting}
Inductive count_equality : Set := 
   | CountEquality : subject -> policyId -> nat -> count_equality.

Definition make_count_equality
  (s:subject)(id:policyId)(n:nat): count_equality :=
  CountEquality s id n.
  
Inductive environment : Set := 
  | SingleEnv : count_equality -> environment
  | ConsEnv :  count_equality ->  environment -> environment.

\end{lstlisting}
%\end{minipage}

\lstset{language=Coq, frame=single, caption={Defining Environments},label={lst:environmentusagecoq}}
\begin{minipage}[c]{0.95\textwidth}
\begin{lstlisting}

Definition e1 : environment := 
  (SingleEnv (make_count_equality Alice id1 8)).

\end{lstlisting}
\end{minipage}
  

We also define a \emph{getCount} function (see listing~\ref{lst:getCountcoq}) that given a pair consisting of a subject and policy id, looks for a corresponding count in the environment. Note the use of the keyword \emph{Fixpoint}. Fixpoint is the Coq keyword for recursive functions and we use it instead of 'Definition' when the function being defined is recursive such as is the case for function $getCount$. $getCount$ assumes the given environment is consistent, so it returns the first matched $count$ it sees for a $(subject, id)$ pair. If a $count$ for a $(subject, id)$ pair is not found it returns 0. 


\lstset{language=Coq, frame=single, caption={getCount Function},label={lst:getCountcoq}}
\begin{minipage}[c]{0.95\textwidth}
\begin{lstlisting}
Fixpoint getCount 
  (e:environment)(s:subject)(id: policyId): nat :=
  match e with
  | SingleEnv f  => 
      match f with 
	  | CountEquality s1 id1 n1 => 
          if (beq_nat s s1) 
          then if (beq_nat id id1) then n1 else 0 
          else 0  
      end			
  | ConsEnv f rest =>
      match f with 
	  | CountEquality s1 id1 n1 => 
          if (beq_nat s s1)
          then if (beq_nat id id1) then n1 else (getCount rest s id)  
          else (getCount rest s id)
      end
  end.
\end{lstlisting}
\end{minipage}

\section{Translations}

Translation of the top level $agreement$ element proceeds by case analysis on the structure of the agreement. However an agreement can only be built one way; by calling the constructor $Agreement$. The translation proceeds by calling the translation function for the corresponding $policySet$ namely the parameter to $Agreement$ called $ps$. See listing~\ref{lst:transagreement} which is the Coq implementation of~\ref{lst:transAgreementast}. The formal argument $e$ of type $environment$ is passed along many translation function but will only eventually be used to get the count information from the $getCount$ function.


\lstset{language=Coq, frame=single, caption={Translation of Agreement},label={lst:transagreement}}
\begin{minipage}[c]{0.95\textwidth}
\begin{lstlisting}
Definition trans_agreement
   (e:environment)(ag:agreement)(action_from_query:act)
   (subject_from_query:subject)(asset_from_query:asset) : nonemptylist result :=

   match ag with
   | Agreement prin_u a ps => 
       (trans_ps e action_from_query subject_from_query asset_from_query ps prin_u a)
   end.
\end{lstlisting}
\end{minipage}

Translation of a $policySet$ (called $trans\_ps$ in listing~\ref{lst:transpsCoq}), takes as input $e$, the environment, $ps$, the policy set, $prin_{u}$, the agreement's user, and $a$, the asset, and proceeds by case analysis of different policySet constructors and recursing into translation functions for the composing elements. A policySet is either a $PrimitivePolicySet$, $PrimitiveExclusivePolicySet$ or a $AndPolicySet$. 

Note that to implement the translation for an $AndPolicySet$ a local function $trans\_ps\_list$ has been defined where for a single $policySet$, $trans\_ps$ is called, and for a list of $policySet$s, the conjunction of $trans_ps$s are returned.

\lstset{language=Coq, frame=single, caption={Query Answers and Results},
    label={lst:transAnswersCoq}}
%\begin{minipage}{\linewidth}
\begin{lstlisting}


Inductive answer : Set :=
  | Permitted : answer
  | Unregulated : answer
  | NotPermitted : answer.

Inductive result : Set :=
  | Result : answer -> subject -> act -> asset -> result.

Definition makeResult
  (ans:answer)(s:subject)(ac:act)(ass:asset) : result := 
 (Result ans s ac ass).

\end{lstlisting}
%\end{minipage}



\lstset{language=Coq, frame=single, caption={Translation of unregulated policies},
    label={lst:transUnregulatedCoq}}
%\begin{minipage}{\linewidth}
\begin{lstlisting}
Definition process_single_pp_trans_policy_unregulated
   (pp: primPolicy)(x:subject)
   (a:asset)(action_from_query: act)  : nonemptylist result :=
  match pp with
    | PrimitivePolicy prq' policyId action =>
        (Single (makeResult Unregulated x action_from_query a))
  end.

Fixpoint trans_policy_unregulated
  (e:environment)(x:subject)(p:policy)(a:asset)
  (action_from_query: act){struct p} : nonemptylist result :=

  match p with
       | Policy pp_list => trans_pp_list_trans_policy_unregulated pp_list x a action_from_query
  end.

\end{lstlisting}
%\end{minipage}

\lstset{language=Coq, frame=single, caption={Translation of negative policies},
    label={lst:transNegativeCoq}}
%\begin{minipage}{\linewidth}
\begin{lstlisting}
Definition process_single_pp_trans_policy_negative
   (pp: primPolicy)(x:subject)
   (a:asset)(action_from_query: act)  : nonemptylist result :=
  match pp with
    | PrimitivePolicy prq' policyId action =>
 
          if (eq_nat_dec action_from_query action)
          then
            (Single 
              (makeResult NotPermitted x action_from_query a))
          else
            (Single 
              (makeResult Unregulated x action_from_query a))

  end.

Fixpoint trans_policy_negative
  (e:environment)(x:subject)(p:policy)(a:asset)
  (action_from_query: act){struct p} : nonemptylist result :=
  
  match p with  
       | Policy pp_list => trans_pp_list_trans_policy_negative pp_list x a action_from_query
  end.
\end{lstlisting}
%\end{minipage}

\lstset{language=Coq, frame=single, caption={Translation of list of policies: positive, negative and unregualted use the same pattern},
    label={lst:transListOfPositiveCoq}}
\begin{lstlisting}
Fixpoint trans_pp_list_trans_policy_<positive | negative | unregulated> 
  (pp_list:nonemptylist primPolicy)(e:environment)(x:subject)
  (prin_u:prin)(a:asset)(action_from_query: act){struct pp_list}:=
   match pp_list with
     | Single pp1 => process_single_pp_trans_policy_<positive | negative | unregulated> pp1 e x prin_u a action_from_query
     | NewList pp pp_list' => app_nonempty
	 (process_single_pp_trans_policy_<positive | negative | unregulated> pp e x prin_u a action_from_query) 
	 (trans_pp_list_trans_policy_<positive | negative | unregulated> pp_list' e x prin_u a action_from_query)
   end.
\end{lstlisting}
%\end{minipage}


\lstset{language=Coq, frame=single, caption={Translation of positive policies},
    label={lst:transPositiveCoq}}
%\begin{minipage}{\linewidth}
\begin{lstlisting}
Definition process_single_pp_trans_policy_positive 
   (pp: primPolicy)(e:environment)(x:subject)(prin_u:prin)
   (a:asset)(action_from_query: act)  : nonemptylist result :=
  match pp with
    | PrimitivePolicy prq' policyId action =>
        if (trans_preRequisite_dec e x prq' (Single policyId) prin_u)
        then (* prin /\ prq /\ prq' *)
          if (eq_nat_dec action_from_query action)
          then
            (Single 
              (makeResult Permitted x action_from_query a))
          else
            (Single 
              (makeResult Unregulated x action_from_query a))
               else (* prin /\ prq /\ ~prq' *)
          (Single 
              (makeResult Unregulated x action_from_query a))
      end.

Definition trans_policy_positive
  (e:environment)(x:subject)(p:policy)(prin_u:prin)(a:asset)
  (action_from_query: act) : nonemptylist result :=
  match p with
       | Policy pp_list => trans_pp_list_trans_policy_positive pp_list e x prin_u a action_from_query
  end.

\end{lstlisting}
%\end{minipage}

\lstset{language=Coq, frame=single, caption={Translation of PIPS and PEPS},
    label={lst:transpipsAndpepsCoq}}
%\begin{minipage}{\linewidth}
\begin{lstlisting}
Definition trans_policy_PIPS
  (e:environment)(prq: preRequisite)(p:policy)(x:subject)
  (prin_u:prin)(a:asset)(action_from_query:act) : nonemptylist result :=
  
    if (trans_prin_dec x prin_u)
    then (* prin *)
      if (trans_preRequisite_dec e x prq (getId p) prin_u)
      then (* prin /\ prq *)
        (trans_policy_positive e x p prin_u a action_from_query)                           
      else (* prin /\ ~prq *)
        (trans_policy_unregulated e x p a action_from_query)
    else (* ~prin *)
      (trans_policy_unregulated e x p a action_from_query).

Definition trans_policy_PEPS
  (e:environment)(prq: preRequisite)(p:policy)(x:subject)
  (prin_u:prin)(a:asset)(action_from_query:act) : nonemptylist result :=
  
  if (trans_prin_dec x prin_u)
  then (* prin *)
    if (trans_preRequisite_dec e x prq (getId p) prin_u)
    then (* prin /\ prq *)
      (trans_policy_positive e x p prin_u a action_from_query)
    else (* prin /\ ~prq *)
      (trans_policy_unregulated e x p a action_from_query)
  else (* ~prin *)
    (trans_policy_negative e x p a action_from_query).

\end{lstlisting}
%\end{minipage}

The listing~\ref{lst:transpsCoq} is the Coq implementation of~\ref{lst:transpolicysetdefinitionAST}.

\lstset{language=Coq, frame=single, caption={Translation of Policy Set},label={lst:transpsCoq}}
%\begin{minipage}{\linewidth}
\begin{lstlisting}

Fixpoint trans_ps
  (e:environment)(action_from_query:act)(subject_from_query:subject)(asset_from_query:asset)
  (ps:policySet)
  (prin_u:prin)(a:asset){struct ps} : nonemptylist result :=

let process_single_ps := (fix process_single_ps (pps: primPolicySet):=
  
  match pps with 
    | PIPS pips => 
        match pips with 
          | PrimitiveInclusivePolicySet prq p => 
            (trans_policy_PIPS e prq p subject_from_query prin_u a action_from_query)                
        end
    | PEPS peps => 
        match peps with 
          | PrimitiveExclusivePolicySet prq p => 
            (trans_policy_PEPS e prq p subject_from_query prin_u a action_from_query)
        end  
   end) in

if (eq_nat_dec asset_from_query a)
then (* asset_from_query = a *)  
    match ps with
      | PPS pps => process_single_ps pps
    end
else (* asset_from_query <> a *)
       (Single 
          (makeResult 
             Unregulated subject_from_query action_from_query asset_from_query)).
\end{lstlisting}
%\end{minipage}



% COQ
Translation of a \emph{prin} (called $trans\_prin$ in listing~\ref{lst:transprinCoq}) takes as input $x$, the $subject$ in question, $p$, the principal or the $prin$,  and proceeds based on whether $p$ is a single subject or a list of subjects. If $p$ is a single subject, $s$, the $Prop$ $x=s$ is returned. Otherwise the disjunction of the translation of the first subject in $p$ ($s$) and the $rest$ of the subjects is returned.
The listing~\ref{lst:transprinCoq} is the Coq implementation of~\ref{lst:transprindefinitionAST}.


\lstset{language=Coq, frame=single, caption={Translation of a Prin},label={lst:transprinCoq}}
\begin{lstlisting}

Fixpoint trans_prin
  (x:subject)(p: prin): Prop :=

  match p with
    | Single s => (x=s)
    | NewList s rest => ((x=s) \/ trans_prin x rest)
  end.
\end{lstlisting}


A positive translation for a policy (called $trans\_policy\_positive$ in listing~\ref{lst:transpolicypositiveCoq}) takes as input $e$, the $environment$, $x$, the $subject$, $p$, the $policy$ to translate, $prin_{u}$, the agreement's user, and $a$, the asset and proceeds based on whether we have a $PrimitivePolicy$ or a $AndPolicy$. If the policy is a $PrimitivePolicy$ an implication is returned which indicates $x$ is \emph{permitted} to do $action$ to $a$, if the $preRequisite$ holds. 

$Permitted$ is a predicate specified as $Parameter Permitted : subject -> act -> asset -> Prop.$ So $Permitted$ predicate takes a $subject$, an $act$ (an action) and an $asset$ and builds a term of type $Prop$. 

Note that to implement the translation for an $AndPolicy$ a local function $trans\_p\_list$ has been defined where for a single $policy$, $trans\_policy\_positive$ is returned, and for a list of $policy$s, the conjunction of $trans\_policy\_positive$s are returned.

The listing~\ref{lst:transpolicypositiveCoq} is the Coq implementation of~\ref{lst:transprimitivepolicyAST} and~\ref{lst:transAndpolicyAST} when $polarity=positive$.

\lstset{language=Coq, frame=single, caption={Translation of a Positive Policy},label={lst:transpolicypositiveCoq}}
\begin{lstlisting}

Fixpoint trans_policy_positive
  (e:environment)(x:subject)(p:policy)(prin_u:prin)(a:asset){struct p} : Prop :=

let trans_p_list := (fix trans_p_list (p_list:nonemptylist policy)(prin_u:prin)(a:asset){struct p_list}:=
                  match p_list with
                    | Single p1 => trans_policy_positive e x p1 prin_u a
                    | NewList p p_list' => 
                        ((trans_policy_positive e x p prin_u a) /\ 
                         (trans_p_list p_list' prin_u a))
                  end) in


  match p with
    | PrimitivePolicy prq policyId action => ((trans_preRequisite e x prq (Single policyId) prin_u) ->
                                              (Permitted x action a))
    | AndPolicy p_list => trans_p_list p_list prin_u a
  end.
\end{lstlisting}

A negative translation for a policy (called $trans\_policy\_negative$ in listing~\ref{lst:transpolicynegativeCoq}) takes as input $e$, the $environment$, $x$, the $subject$, $p$, the $policy$ to translate, and $a$ the asset and proceeds based on whether we have a $PrimitivePolicy$ or a $AndPolicy$. If the policy is a $PrimitivePolicy$ an implication is returned which indicates $x$ is forbidden to do $action$ to $a$ regardless of whether $preRequisite$ holds. As is the case for the positive translation, to implement the translation for an $AndPolicy$ a local function $trans\_p\_list$ has been defined where for a single $policy$, $trans\_policy\_negative$ is returned, and for a list of $policy$s, the conjunction of $trans\_policy\_negative$s are returned.

The listing~\ref{lst:transpolicynegativeCoq} is the Coq implementation of~\ref{lst:transprimitivepolicyAST} and~\ref{lst:transAndpolicyAST} when $polarity=negative$.


\lstset{language=Coq, frame=single, caption={Translation of a Negative Policy},label={lst:transpolicynegativeCoq}}
\begin{minipage}[c]{0.95\textwidth}
\begin{lstlisting}

Fixpoint trans_policy_negative
  (e:environment)(x:subject)(p:policy)(a:asset){struct p} : Prop :=
let trans_p_list := (fix trans_p_list (p_list:nonemptylist policy)(a:asset){struct p_list}:=
                  match p_list with
                    | Single p1 => trans_policy_negative e x p1 a
                    | NewList p p_list' => ((trans_policy_negative e x p a) /\ 
                                            (trans_p_list p_list' a))
                  end) in


  match p with
    | PrimitivePolicy prq policyId action => not (Permitted x action a)
    | AndPolicy p_list => trans_p_list p_list a
  end.
\end{lstlisting}
\end{minipage}

The translation of a $prerequisite$ (called $trans\_preRequisite$ in listing~\ref{lst:transpreRequisiteCoq}) takes as input $e$, the $environment$, $x$, the $subject$, $prq$, the $preRequisite$ to translate, $IDs$, the set of identifiers (of policies implied by the $prq$), $prin_{u}$, the agreement's user, and proceeds by case analysis on the structure of the $prerequisite$. A $prerequisite$ is either a $TruePrq$, a $Constraint$, a $ForEachMember$, a $NotCons$, a $AndPrqs$, a $OrPrqs$ or a $XorPrqs$. 

In listing~\ref{lst:transpreRequisiteCoq} the translation for $TruePrq$ is the Prop $True$, the translations for $Constraint$, $ForEachMember$ and $NotCons$ simply call respective translation functions for corresponding types $constraint$ and $forEachMember$ (namely $trans\_constraint$, $trans\_forEachMember$ and $trans\_notCons$). Note that the translation for $AndPrqs$, $OrPrqs$ and $XorPrqs$ have not yet been implemented but based on the their many-sorted-logic formulas' specifications (listing~\ref{lst:preRequisiteast}) they will be conjunctions, disjunctions and exclusive disjunctions of translations for each $prerequisite$.

The listing~\ref{lst:transpreRequisiteCoq} is the Coq implementation of~\ref{lst:transprerequisitedefinitionAST}.


\lstset{language=Coq, frame=single, caption={Translation of a PreRequisite},label={lst:transpreRequisiteCoq}}
\begin{minipage}[c]{0.95\textwidth}
\begin{lstlisting}

Definition trans_preRequisite
  (e:environment)(x:subject)(prq:preRequisite)(IDs:nonemptylist policyId)(prin_u:prin) : Prop := 

  match prq with
    | TruePrq => True
    | Constraint const => trans_constraint e x const IDs prin_u 
    | ForEachMember prn const_list => trans_forEachMember e x prn const_list IDs 
    | NotCons const => trans_notCons e x const IDs prin_u 
    | AndPrqs prqs => True 
    | OrPrqs prqs => True 
    | XorPrqs prqs => True 
  end.
\end{lstlisting}
\end{minipage}

The translation of a $constraint$ (called $trans\_constraint$ in listing~\ref{lst:transconstraintCoq}) takes as input $e$ the $environment$, $x$ the $subject$, $const$, the $constraint$ to translate, $IDs$, the set of identifiers (of policies implied by the parent $preRequisite$) and $prin_{u}$, the agreement's user and proceeds by case analysis on the structure of the $constraint$. A $constraint$ is either a $Principal$, a $Count$ or a $CountByPrin$. The translation for $Principal$ returns the translation function (namely $trans\_prin$) for the $prn$ (the $prin$ that accompanies the $const$ constraint). The translation for $Count$ and $CountByPrin$ return the translation function $trans\_count$. For $Count$ the $prin$ used is the agreement's user, whereas the $prin$ used is the one passed to $CountByPrin$ namely $prn$.

The listing~\ref{lst:transconstraintCoq} is the Coq implementation of~\ref{lst:transpreRequisiteConstraint}.


\lstset{language=Coq, frame=single, caption={Translation of a Constraint},label={lst:transconstraintCoq}}
\begin{lstlisting}

Fixpoint trans_constraint 
  (e:environment)(x:subject)(const:constraint)(IDs:nonemptylist policyId)
  (prin_u:prin){struct const} : Prop := 
  match const with
    | Principal prn => trans_prin x prn
  
    | Count n => trans_count e n IDs prin_u 

    | CountByPrin prn n => trans_count e n IDs prn 

  end.
  
\end{lstlisting}

The translation of a $forEachMember$ (called $trans\_forEachMember$ in listing~\ref{lst:transforEachMemberCoq}) takes as input $e$ the $environment$, $x$ the $subject$, $principals$, the set of subjects that override the agreement's user(s), $const\_list$ the set of constraints and $IDs$, the set of identifiers (of policies implied by the parent $preRequisite$).

To implement the translation for a $forEachMember$ we start by calling an auxiliary function $process\_two\_lists$ that effectively returns a new list composed of pairs of members of the first list and the second list (the cross-product of the two input lists). In the case of a $forEachMember$ translation, the call is ``$process\_two\_lists$ $principals$ $const\_list$'' which returns a list of pairs of subject and constraint namely $prins\_and\_constraints$. $prins\_and\_constraints$ is then passed to a locally defined function \emph{ trans_forEachMember_Aux} where for a single pair of subject and constraint $trans\_constraint$ is called and for a list of pairs of subject and constraints, the conjunction of $trans\_constraint$s (for the first pair) and $trans\_forEachMember\_Aux$s (for the rest of the pairs) are returned.

The listing~\ref{lst:transforEachMemberCoq} is the Coq implementation of~\ref{lst:transpreRequisiteForEachMember}.

\lstset{language=Coq, frame=single, caption={Translation of forEachMember},label={lst:transforEachMemberCoq}}
\begin{minipage}[c]{0.95\textwidth}
\begin{lstlisting}

Fixpoint trans_forEachMember
         (e:environment)(x:subject)(principals: nonemptylist subject)(const_list:nonemptylist constraint)
         (IDs:nonemptylist policyId){struct const_list} : Prop := 

let trans_forEachMember_Aux   
  := (fix trans_forEachMember_Aux
         (prins_and_constraints : nonemptylist (Twos subject constraint))
         (IDs:nonemptylist policyId){struct prins_and_constraints} : Prop :=

      match prins_and_constraints with
        | Single pair1 => trans_constraint e x (right pair1) IDs (Single (left pair1)) 
        | NewList pair1 rest_pairs =>
            (trans_constraint e x (right pair1) IDs (Single (left pair1))) /\
            (trans_forEachMember_Aux rest_pairs IDs)
      end) in

      let prins_and_constraints := process_two_lists principals const_list in
      trans_forEachMember_Aux prins_and_constraints IDs.

\end{lstlisting}
\end{minipage}

The translation of a $NotCons$ (called $trans\_notCons$ in listing~\ref{lst:transnotConsCoq}) takes as input $e$ the $environment$, $x$ the $subject$, $const$, the $constraint$ to translate, $IDs$, the set of identifiers (of policies implied by the parent $preRequisite$) and $prin_{u}$, the agreement's user and proceeds to return the negation of $trans\_constraint$ (see listing~\ref{lst:transconstraintCoq}).

The listing~\ref{lst:transnotConsCoq} is the Coq implementation of~\ref{lst:transpreRequisiteNotConstraint}.

\lstset{language=Coq, frame=single, caption={Translation of NotCons},label={lst:transnotConsCoq}}
\begin{lstlisting}

Definition trans_notCons
  (e:environment)(x:subject)(const:constraint)(IDs:nonemptylist policyId)(prin_u:prin) : Prop :=
  ~ (trans_constraint e x const IDs prin_u).
\end{lstlisting}




The translation of a $Count$ or a $CountByPrin$ (called $trans\_count$ in listing~\ref{lst:transcountCoq}) takes as input $e$ the $environment$, $n$ the total number of times the subjects mentioned in $prin_{u}$ (last parameter) may invoke the policies identified by $IDs$ (third parameter).

To implement the translation for a $Count$ or a $CountByPrin$ we start by calling an auxiliary function $process\_two\_lists$ that effectively returns a new list composed of pairs of members of the first list and the second list (the cross-product of the two input lists). In the case of $trans\_count$, the call is ``$process\_two\_lists$ $IDs$ $prin_u$'' which returns a list of pairs of $policyId$ and $subject$ namely $ids\_and\_subjects$. $ids\_and\_subjects$ is then passed to a locally defined function \emph{trans_count_aux}.

$trans\_count\_aux$ returns the current count for a single pair of $policyId$ and $subject$ (the call to $getCount$ which looks up the environment $e$ and returns the current count per each $subject$ and $policyId$) and for a list of pairs of $policyId$ and $subject$s, the addition of $get\_count$ (for the first pair) and $trans\_count\_aux$s (for the rest of the pairs) is returned. 

A local variable $running_total$ has the value returned by $trans\_count\_aux$. Finally the proposition $running_total < n$ is returned as the translation for a $Count$ or a $CountByPrin$.

Note that the only difference between translations for a $Count$ and a $CountByPrin$ is the additional $prn$ parameter for $CountByPrin$ which allows for getting counts for subjects not necessarily the same as $prin_{u}$, the agreement's user(s).

The listing~\ref{lst:transcountCoq} is the Coq implementation of~\ref{lst:transconstraintCount} and of~\ref{lst:transconstraintCountbyPrin}.

\lstset{language=Coq, frame=single, caption={Translation of count},label={lst:transcountCoq}}
\begin{minipage}[c]{0.95\textwidth}
\begin{lstlisting}
Fixpoint trans_count 
  (e:environment)(n:nat)(IDs:nonemptylist policyId)
  (prin_u:prin) : Prop := 

  let trans_count_aux 
    := (fix trans_count_aux
         (ids_and_subjects : nonemptylist (Twos policyId subject)) : nat :=
     match ids_and_subjects with
        | Single pair1 => getCount e (right pair1) (left pair1)
        | NewList pair1 rest_pairs =>
            (getCount e (right pair1)(left pair1)) +
            (trans_count_aux rest_pairs)
      end) in
  
  let ids_and_subjects := process_two_lists IDs prin_u in
  let running_total := trans_count_aux ids_and_subjects in
  running_total < n.
\end{lstlisting}
\end{minipage}

\section{Decision Procedures}
\label{sec:decisionProcs}  


\lstset{language=Coq, frame=single, caption={Translation of count TEMP},label={lst:transcountCoq123}}
%\begin{minipage}[c]{0.95\textwidth}
\begin{lstlisting}

Theorem double_neg_constraint:
 forall (e:environment)(x:subject)
 (const:constraint)(IDs:nonemptylist policyId)(prin_u:prin),
  (trans_constraint e x const IDs prin_u) ->
   ~~(trans_constraint e x const IDs prin_u).
Proof.
intros e x const IDs prin_u.
intros H. unfold not. intros H'. apply H'. exact H. Qed.

Theorem trans_negation_constraint_dec :
forall (e:environment)(x:subject)
(const:constraint)(IDs:nonemptylist policyId)(prin_u:prin),
     {~trans_constraint e x const IDs prin_u} + 
     {~~trans_constraint e x const IDs prin_u}.

Proof.
intros e x const IDs prin_u.
pose (j:= trans_constraint_dec e x const IDs prin_u). 
destruct j. apply double_neg_constraint in t. right. exact t. left. exact n. Defined.

Theorem trans_notCons_dec :
forall (e:environment)(x:subject)
(const:constraint)(IDs:nonemptylist policyId)(prin_u:prin),
   {trans_notCons e x const IDs prin_u} + 
   {~ trans_notCons e x const IDs prin_u}.
Proof.
intros e x const IDs prin_u. 
unfold trans_notCons. apply trans_negation_constraint_dec. Defined.
  
Theorem trans_preRequisite_dec :
  forall (e:environment)(x:subject)
    (prq:preRequisite)(IDs:nonemptylist policyId)(prin_u:prin),
       {trans_preRequisite e x prq IDs prin_u} +
       {~ trans_preRequisite e x prq IDs prin_u}.
Proof.
intros e x prq IDs prin_u.
induction prq.
simpl. auto.
simpl. apply trans_constraint_dec.
simpl. apply trans_notCons_dec.
simpl. auto.
simpl. auto.
simpl. auto.
Qed. 


Theorem subject_in_prin_dec :
    forall (a:subject) (l:prin), {is_subject_in_prin a l} + {~ is_subject_in_prin a l}.

Proof.
induction l as [| a0 l IHl].
apply eq_nat_dec.
destruct (eq_nat_dec a0 a); simpl; auto.
destruct IHl; simpl; auto.
right; unfold not; intros [Hc1| Hc2]; auto.
Defined.


Theorem trans_prin_dec :
   forall (x:subject)(p: prin),
     {trans_prin x p} + {~trans_prin x p}.

Proof.

apply subject_in_prin_dec.
Defined.
\end{lstlisting}
%\end{minipage}

