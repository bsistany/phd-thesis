%======================================================================
\chapter{ACCPL Semantics In Coq}\label{chap:accplsemanticscoq}
%======================================================================

% ---------------------------------- COQ -----------------------

\section{Introduction}
\label{sec:envConsistentP}  

% COQ
The translation functions plus the auxiliary types and infrastructure which implement the semantics for \ac{ACCPL} have been encoded in Coq. Translation functions build Coq terms of type $Prop$ for the most part. Well-formed propositions (or $Prop$s) are assertions one can express about values such as mathematical objects or even programs (e.g. 3 < 8) in Coq. 

Whether a permission is granted or denied depends on the agreements in question but also on the facts recorded in the environment. For \ac{ACCPL} those facts revolve around the number of times a policy has been used to justify an action (see section~\ref{sec:odrl0} for more details on \ac{ACCPL}). We encode this information in an \emph{environment} which is a conjunction of equalities of the form $count(s, policyId) = n$. 

% COQ
The Coq version of the count equality is a new inductive type called \emph{count\_equality}. An environment is defined to be a non-empty list of $count\_equality$ objects (see listing~\ref{lst:environmentcoq}). Function $make\_count\_equality$ in listing~\ref{lst:environmentcoq} is simply a convenience function that builds $count\_equality$s. For an example of how environments are created see listing~\ref{lst:environmentusagecoq}.

\lstset{language=Coq, frame=single, caption={Environments and Counts},label={lst:environmentcoq}}
%\begin{minipage}[c]{0.95\textwidth}
\begin{lstlisting}
Inductive count_equality : Set := 
   | CountEquality : subject -> policyId -> nat -> count_equality.

Definition make_count_equality
  (s:subject)(id:policyId)(n:nat): count_equality :=
  CountEquality s id n.
  
Inductive environment : Set := 
  | SingleEnv : count_equality -> environment
  | ConsEnv :  count_equality ->  environment -> environment.

\end{lstlisting}
%\end{minipage}

\lstset{language=Coq, frame=single, caption={Defining Environments},label={lst:environmentusagecoq}}
\begin{minipage}[c]{0.95\textwidth}
\begin{lstlisting}

Definition e1 : environment := 
  (SingleEnv (make_count_equality Alice id1 8)).

\end{lstlisting}
\end{minipage}
  

We also define a \emph{getCount} function (see listing~\ref{lst:getCountcoq}) that given a pair consisting of a subject and policy id, looks for a corresponding count in the environment. $getCount$ assumes the given environment is consistent (meaning it won`t return two different counts for the same pair of subject and policy id), so it returns the first matched $count$ it sees for a $(subject, id)$ pair. If a $count$ for a $(subject, id)$ pair is not found it returns 0. 


\lstset{language=Coq, frame=single, caption={getCount Function},label={lst:getCountcoq}}
\begin{minipage}[c]{0.95\textwidth}
\begin{lstlisting}
Fixpoint getCount 
  (e:environment)(s:subject)(id: policyId): nat :=
  match e with
  | SingleEnv f  => 
      match f with 
	  | CountEquality s1 id1 n1 => 
          if (beq_nat s s1) 
          then if (beq_nat id id1) then n1 else 0 
          else 0  
      end			
  | ConsEnv f rest =>
      match f with 
	  | CountEquality s1 id1 n1 => 
          if (beq_nat s s1)
          then if (beq_nat id id1) then n1 else (getCount rest s id)  
          else (getCount rest s id)
      end
  end.
\end{lstlisting}
\end{minipage}

\section{Decision Procedures and the sumbool type}

sumbool is a boolean type defined in the Coq standard library module $Coq.Init.Specif$. sumbool is equipped with the justification of its value, which really helps with proofs. sumbool A B, which is written \{A\}+\{B\}, is the informative disjunction ``A or B'', where A and B are logical propositions. Its extraction is isomorphic to the type of booleans. A boolean is either true or false, and this is decidable~\cite{Coq:manual}. See listing~\ref{lst:sumbooltypeCoq} for the definition of $sumbool$ and $sumbool\_of\_bool$. With sumbools similar to bool and other 2-constructor inductive types, one can use the ``if then else'' construct to select the desired case when doing proofs which makes for much more readability of the given proofs.

\lstset{language=Coq, frame=single, caption={sumbool type},label={lst:sumbooltypeCoq}}
\begin{minipage}[c]{0.95\textwidth}
\begin{lstlisting}
Inductive sumbool (A B:Prop) : Set :=
  | left : A -> {A} + {B}
  | right : B -> {A} + {B}
 where "{ A } + { B }" := (sumbool A B) : type_scope.
 
 Definition sumbool_of_bool : forall b:bool, {b = true} + {b = false}.
\end{lstlisting}
\end{minipage}

We have used the sumbool type to declare and prove decision procedures that we have subsequently used in the semantics and also in the proofs. In the following we will list only the type declaration of each decision procedure and the accompanying functions. The individual decision procedures were used as a building block to make the final decidability proofs for \ac{ACCPL} possible. We will refer to these listings later when we discuss the translations.

\lstset{language=Coq, frame=single, caption={Decision Procedures: eq_nat_dec~\cite{Coq:manual}},label={lst:eqnatdeccoq}}
\begin{lstlisting}
Theorem eq_nat_dec : forall n m, {n = m} + {n <> m}.
\end{lstlisting}


\lstset{language=Coq, frame=single, caption={Decision Procedures: is_subject_in_prin},label={lst:issubjectinprincoq}}
\begin{lstlisting}
Fixpoint is_subject_in_prin (s:subject)(p:prin): Prop :=
  match p with
  | Single s'  => s=s'
  | NewList s' rest => s=s' \/ (is_subject_in_prin s rest)
  end.

Theorem subject_in_prin_dec :
    forall (a:subject) (l:prin), {is_subject_in_prin a l} + {~ is_subject_in_prin a l}.

\end{lstlisting}

\lstset{language=Coq, frame=single, caption={Decision Procedures: trans_prin_dec},label={lst:transprindeccoq}}
\begin{lstlisting}
Theorem trans_prin_dec :
   forall (x:subject)(p: prin), {trans_prin x p} + {~trans_prin x p}.
\end{lstlisting}

\lstset{language=Coq, frame=single, caption={Decision Procedures: trans_count_dec},label={lst:transcountdeccoq}}
\begin{lstlisting}
Theorem trans_count_dec :
    forall (n:nat), {trans_count n} + {~trans_count n}.
\end{lstlisting}

\lstset{language=Coq, frame=single, caption={Decision Procedures: trans_constraint_dec},label={lst:transconstraintdeccoq}}
\begin{lstlisting}
Theorem trans_constraint_dec :
    forall (e:environment)(x:subject)(const:constraint)(IDs:nonemptylist policyId)(prin_u:prin),
       {trans_constraint e x const IDs prin_u} + {~trans_constraint e x const IDs prin_u}.
\end{lstlisting}

\lstset{language=Coq, frame=single, caption={Decision Procedures: trans_negation_constraint_dec},label={lst:transnegationconstraintdeccoq}}
\begin{lstlisting}
Theorem double_neg_constraint:
    forall (e:environment)(x:subject)(const:constraint)(IDs:nonemptylist policyId)(prin_u:prin),
       (trans_constraint e x const IDs prin_u) -> ~~(trans_constraint e x const IDs prin_u).
  
Theorem trans_negation_constraint_dec :
    forall (e:environment)(x:subject)(const:constraint)(IDs:nonemptylist policyId)(prin_u:prin),
       {~trans_constraint e x const IDs prin_u} + {~~trans_constraint e x const IDs prin_u}.
       
\end{lstlisting}

\lstset{language=Coq, frame=single, caption={Decision Procedures: trans_notCons_dec},label={lst:transnotConsdeccoq}}
\begin{lstlisting}
Theorem trans_notCons_dec :
    forall (e:environment)(x:subject)(const:constraint)(IDs:nonemptylist policyId)(prin_u:prin),
       {trans_notCons e x const IDs prin_u} + {~ trans_notCons e x const IDs prin_u}.

\end{lstlisting}


\lstset{language=Coq, frame=single, caption={Decision Procedures: trans_preRequisite_dec},label={lst:transpreRequisitedeccoq}}
\begin{lstlisting}
Theorem trans_preRequisite_dec :
    forall (e:environment)(x:subject)(prq:preRequisite)(IDs:nonemptylist policyId)(prin_u:prin),
       {trans_preRequisite e x prq IDs prin_u} + {~ trans_preRequisite e x prq IDs prin_u}.
\end{lstlisting}


\section{answer and result types}\label{sec:answerandresulttypes}

The policy translation functions ultimately will return one of the following answers: Permitted, NotPermitted or Unregulated. Permitted answer signifies that the access request has been granted. The NotPermitted answer is for when the access request is denied. And finally the Unregulated answer is for all the times when neither Permitted or NotPermitted answers are applicable. The $answer$ type in the Coq listing~\ref{lst:answercoq} implements the answer concept. We wrap $answer$s in a $result$ type and add some context. Intuitively a result will tell us whether a subject may perform an action on an asset or not, or else that the request is unregulated.


\lstset{language=Coq, frame=single, caption={Decision Procedures: },label={lst:answercoq}}
\begin{minipage}[c]{0.95\textwidth}
\begin{lstlisting}

Inductive answer : Set :=
  | Permitted : answer
  | Unregulated : answer
  | NotPermitted : answer.
  
Inductive result : Set :=
  | Result : answer -> subject -> act -> asset -> result.
\end{lstlisting}
\end{minipage}

\section{Translations}


Translation of the top level $agreement$ element proceeds by case analysis on the structure of the agreement. However an agreement can only be built one way; by calling the constructor $Agreement$. The translation proceeds by calling the translation function for the corresponding $policySet$ namely the parameter to $Agreement$ called $ps$. See listing~\ref{lst:transagreement}. The formal argument $e$ of type $environment$ is passed along many translation functions but will only eventually be used to get the count information from the $getCount$ function. Other formal arguments that are passed are the agreement itself to be translated, plus the subject, action and asset coming from a ``query'' or request for access: $action\_from\_query$, $subject\_from\_query$ and $asset\_from\_query$. Notice that the translation function for agreements returns a nonempty list of \emph{result}s.

\subsection{Policy Combinators And Interpretation Of Results}\label{subsec:policycombinators}

The nonempty list of $result$s returned by the agreement translation function will have a result per primitive policy ($primPolicy$) in the agreement. As we will see later when we discuss the proofs, we have proven that results containing a Permitted answer and a NotPermitted answer are mutually exclusive. Therefore existence of a single (or more) Permitted result in the nonempty list makes the set ``Permitted'' whereas the existence of a single (or more) NotPermitted type result makes the whole set "NotPermitted". When no Permitted or NotPermitted is seen in the set, we only have Unregulated results which would make the whole set "Unregulated". 

\lstset{language=Coq, frame=single, caption={Translation of Agreement},label={lst:transagreement}}
%\begin{minipage}[c]{0.95\textwidth}
\begin{lstlisting}
Definition trans_agreement
   (e:environment)(ag:agreement)(action_from_query:act)
   (subject_from_query:subject)(asset_from_query:asset) : nonemptylist result :=

   match ag with
   | Agreement prin_u a ps => 
       (trans_ps e action_from_query subject_from_query asset_from_query ps prin_u a)
   end.
\end{lstlisting}
%\end{minipage}

Translation of a $policySet$ (called $trans\_ps$ in listing~\ref{lst:transpsCoq}), takes as input $e$, the environment, the subject, action and asset coming from a query: $action\_from\_query$, $subject\_from\_query$ and $asset\_from\_query$, $ps$, the policy set, $prin\_{u}$, the agreement's user, and $a$, the asset mentioned in the agreement. 

The translation starts with checking that there is a proof for the equality of asset_from_query and a (the asset from the agreement). If so, the translation function recurses on the composing $primPolicySet$. Otherwise, a result of Unregulated is immediately is returned. Intuitively we are just explicitly saying that the semantics won`t apply to queries about assets that are not mentioned in the agreement. 

A $primPolicySet$ was either constructed by a $PIPS$ or by a $PEPS$ which distinguish between $primInclusivePolicySet$ and $primExclusivePolicySet$ subtypes. The translation functions $trans\_policy\_PIPS$ and $trans\_policy\_PEPS$ are called in turn as the case may be (see listing~\ref{lst:transpipsAndpepsCoq}).

$trans\_policy\_PIPS$ translation function for a $primInclusivePolicySet$ checks whether there is a proof for the subject in question, $x$, being in $prin\_u$. If so there is a check for whether the $preRequisite$ from the policySet holds or not. If so the translation function $trans\_policy\_positive$(~\ref{lst:transPositiveCoq}) is called. In all other cases $trans\_policy\_unregulated$(~\ref{lst:transUnregulatedCoq}) is called.

$trans\_policy\_PEPS$ translation function for a $primExclusivePolicySet$ checks whether there is a proof for the subject in question, $x$, being in $prin\_u$. If not, the translation function $trans\_policy\_negative$ (~\ref{lst:transNegativeCoq}) is called since the policySet is exclusive to agreement`s users we expect that all subjects not in the $prin\_u$ are denied access. In case the check for $x$, being in $prin\_u$ succeeds, there is a check for whether the $preRequisite$ from the policySet holds or not. If so the translation function $trans\_policy\_positive$ is called, else $trans\_policy\_unregulated$ is returned.

$trans\_policy\_positive$, $trans\_policy\_negative$ and $trans\_policy\_unregulated$ translation functions start by recursing on a list of $primPolicy$s where the results of processing each $primPolicy$ is appended to the rest of the set (see high level pseudocode in listing~\ref{lst:transListOfPositiveCoq}) for the common pattern). 

$trans\_policy\_positive$ acting on a single $primPolicy$ (see process_single_pp_trans_policy_positive in listing~\ref{lst:transPositiveCoq}), first checks whether the $preRequisite$ from the policy holds or not. If so the translation function checks that there is a proof for the equality of action_from_query and action (the action from the agreement). If so, the translation function finally returns a result of Permitted. In all other cases, a result of Unregulated is returned. Intuitively we are just explicitly saying that the semantics won`t apply to queries about actions that are not mentioned in the agreement.

$trans\_policy\_negative$ acting on a single $primPolicy$ (see process_single_pp_trans_policy_negative in listing~\ref{lst:transNegativeCoq}), first checks that there is a proof for the equality of action_from_query and action (the action from the agreement). If so, the translation function returns a result of NotPermitted. Otherwise, a result of Unregulated is returned. Note that compared to $trans\_policy\_positive$ we don`t check for whether $preRequisite$ from the policy holds or not since is only called for policy sets of type $primExclusivePolicySet$.

$trans\_policy\_unregulated$ acting on a single $primPolicy$ (see process_single_pp_trans_policy_unregulated in listing~\ref{lst:transUnregulatedCoq}), returns a result of Unregulated.

\lstset{language=Coq, frame=single, caption={Translation of unregulated policies},
    label={lst:transUnregulatedCoq}}
%\begin{minipage}{\linewidth}
\begin{lstlisting}
Definition process_single_pp_trans_policy_unregulated
   (pp: primPolicy)(x:subject)
   (a:asset)(action_from_query: act)  : nonemptylist result :=
  match pp with
    | PrimitivePolicy prq' policyId action =>
        (Single (makeResult Unregulated x action_from_query a))
  end.

Fixpoint trans_policy_unregulated
  (e:environment)(x:subject)(p:policy)(a:asset)
  (action_from_query: act){struct p} : nonemptylist result :=

  match p with
       | Policy pp_list => trans_pp_list_trans_policy_unregulated pp_list x a action_from_query
  end.

\end{lstlisting}
%\end{minipage}

\lstset{language=Coq, frame=single, caption={Translation of negative policies},
    label={lst:transNegativeCoq}}
%\begin{minipage}{\linewidth}
\begin{lstlisting}
Definition process_single_pp_trans_policy_negative
   (pp: primPolicy)(x:subject)
   (a:asset)(action_from_query: act)  : nonemptylist result :=
  match pp with
    | PrimitivePolicy prq' policyId action =>
 
          if (eq_nat_dec action_from_query action)
          then
            (Single 
              (makeResult NotPermitted x action_from_query a))
          else
            (Single 
              (makeResult Unregulated x action_from_query a))

  end.

Fixpoint trans_policy_negative
  (e:environment)(x:subject)(p:policy)(a:asset)
  (action_from_query: act){struct p} : nonemptylist result :=
  
  match p with  
       | Policy pp_list => trans_pp_list_trans_policy_negative pp_list x a action_from_query
  end.
\end{lstlisting}
%\end{minipage}

\lstset{language=Coq, frame=single, caption={Translation of positive policies},
    label={lst:transPositiveCoq}}
%\begin{minipage}{\linewidth}
\begin{lstlisting}
Definition process_single_pp_trans_policy_positive 
   (pp: primPolicy)(e:environment)(x:subject)(prin_u:prin)
   (a:asset)(action_from_query: act)  : nonemptylist result :=
  match pp with
    | PrimitivePolicy prq' policyId action =>
        if (trans_preRequisite_dec e x prq' (Single policyId) prin_u)
        then (* prin /\ prq /\ prq' *)
          if (eq_nat_dec action_from_query action)
          then
            (Single 
              (makeResult Permitted x action_from_query a))
          else
            (Single 
              (makeResult Unregulated x action_from_query a))
               else (* prin /\ prq /\ ~prq' *)
          (Single 
              (makeResult Unregulated x action_from_query a))
      end.

Definition trans_policy_positive
  (e:environment)(x:subject)(p:policy)(prin_u:prin)(a:asset)
  (action_from_query: act) : nonemptylist result :=
  match p with
       | Policy pp_list => trans_pp_list_trans_policy_positive pp_list e x prin_u a action_from_query
  end.

\end{lstlisting}
%\end{minipage}

\lstset{language=Coq, frame=single, caption={Translation of list of policies: positive, negative and unregulated use the same pattern},
    label={lst:transListOfPositiveCoq}}
\begin{lstlisting}
Fixpoint trans_pp_list_trans_policy_<positive | negative | unregulated> 
  (pp_list:nonemptylist primPolicy)(e:environment)(x:subject)
  (prin_u:prin)(a:asset)(action_from_query: act){struct pp_list}:=
   match pp_list with
     | Single pp1 => process_single_pp_trans_policy_<positive | negative | unregulated> pp1 e x prin_u a action_from_query
     | NewList pp pp_list' => app_nonempty
	 (process_single_pp_trans_policy_<positive | negative | unregulated> pp e x prin_u a action_from_query) 
	 (trans_pp_list_trans_policy_<positive | negative | unregulated> pp_list' e x prin_u a action_from_query)
   end.
\end{lstlisting}
%\end{minipage}


\lstset{language=Coq, frame=single, caption={Translation of PIPS and PEPS},
    label={lst:transpipsAndpepsCoq}}
\begin{minipage}{\linewidth}
\begin{lstlisting}
Definition trans_policy_PIPS
  (e:environment)(prq: preRequisite)(p:policy)(x:subject)
  (prin_u:prin)(a:asset)(action_from_query:act) : nonemptylist result :=
  
    if (trans_prin_dec x prin_u)
    then (* prin *)
      if (trans_preRequisite_dec e x prq (getId p) prin_u)
      then (* prin /\ prq *)
        (trans_policy_positive e x p prin_u a action_from_query)                           
      else (* prin /\ ~prq *)
        (trans_policy_unregulated e x p a action_from_query)
    else (* ~prin *)
      (trans_policy_unregulated e x p a action_from_query).

Definition trans_policy_PEPS
  (e:environment)(prq: preRequisite)(p:policy)(x:subject)
  (prin_u:prin)(a:asset)(action_from_query:act) : nonemptylist result :=
  
  if (trans_prin_dec x prin_u)
  then (* prin *)
    if (trans_preRequisite_dec e x prq (getId p) prin_u)
    then (* prin /\ prq *)
      (trans_policy_positive e x p prin_u a action_from_query)
    else (* prin /\ ~prq *)
      (trans_policy_unregulated e x p a action_from_query)
  else (* ~prin *)
    (trans_policy_negative e x p a action_from_query).

\end{lstlisting}
\end{minipage}


\lstset{language=Coq, frame=single, caption={Translation of Policy Set},label={lst:transpsCoq}}
%\begin{minipage}{\linewidth}
\begin{lstlisting}

Fixpoint trans_ps
  (e:environment)(action_from_query:act)(subject_from_query:subject)(asset_from_query:asset)
  (ps:policySet)
  (prin_u:prin)(a:asset){struct ps} : nonemptylist result :=

let process_single_ps := (fix process_single_ps (pps: primPolicySet):=
  
  match pps with 
    | PIPS pips => 
        match pips with 
          | PrimitiveInclusivePolicySet prq p => 
            (trans_policy_PIPS e prq p subject_from_query prin_u a action_from_query)                
        end
    | PEPS peps => 
        match peps with 
          | PrimitiveExclusivePolicySet prq p => 
            (trans_policy_PEPS e prq p subject_from_query prin_u a action_from_query)
        end  
   end) in

if (eq_nat_dec asset_from_query a)
then (* asset_from_query = a *)  
    match ps with
      | PPS pps => process_single_ps pps
    end
else (* asset_from_query <> a *)
       (Single 
          (makeResult 
             Unregulated subject_from_query action_from_query asset_from_query)).
\end{lstlisting}
%\end{minipage}



% COQ
Translation of a \emph{prin} (called $trans\_prin$ in listing~\ref{lst:transprinCoq}) takes as input $x$, the $subject$ in question, $p$, the principal or the $prin$,  and proceeds based on whether $p$ is a single subject or a list of subjects. If $p$ is a single subject, $s$, the $Prop$ $x=s$ is returned. Otherwise the disjunction of the translation of the first subject in $p$ ($s$) and the $rest$ of the subjects is returned.

\lstset{language=Coq, frame=single, caption={Translation of a Prin},label={lst:transprinCoq}}
\begin{lstlisting}

Fixpoint trans_prin
  (x:subject)(p: prin): Prop :=

  match p with
    | Single s => (x=s)
    | NewList s rest => ((x=s) \/ trans_prin x rest)
  end.
\end{lstlisting}



The translation of a $prerequisite$ (called $trans\_preRequisite$ in listing~\ref{lst:transpreRequisiteCoq}) takes as input $e$, the $environment$, $x$, the $subject$, $prq$, the $preRequisite$ to translate, $IDs$, the set of identifiers (of policies implied by the $prq$), $prin_{u}$, the agreement's user, and proceeds by case analysis on the structure of the $prerequisite$. A $prerequisite$ is either a $TruePrq$, a $Constraint$, a $NotCons$, or a $AndPrqs$. 

In listing~\ref{lst:transpreRequisiteCoq} the translation for $TruePrq$ is the Prop $True$, the translations for $Constraint$, $NotCons$ and $AndPrqs$ simply call respective translation functions for corresponding types $constraint$ (namely $trans\_constraint$ and $trans\_notCons$ and $trans\_andPrqs$). 


\lstset{language=Coq, frame=single, caption={Translation of a PreRequisite},label={lst:transpreRequisiteCoq}}
\begin{minipage}[c]{0.95\textwidth}
\begin{lstlisting}

Definition trans_preRequisite
  (e:environment)(x:subject)(prq:preRequisite)(IDs:nonemptylist policyId)(prin_u:prin) : Prop := 

  match prq with
    | TruePrq => True
    | Constraint const => trans_constraint e x const IDs prin_u  
    | NotCons const => trans_notCons e x const IDs prin_u 
    | AndPrqs prqs =>  trans_andPrqs x prq IDs prin_u a
  end.
\end{lstlisting}
\end{minipage}

The translation of a $constraint$ (called $trans\_constraint$ in listing~\ref{lst:transconstraintCoq}) takes as input $e$ the $environment$, $x$ the $subject$, $const$, the $constraint$ to translate, $IDs$, the set of identifiers (of policies implied by the parent $preRequisite$) and $prin_{u}$, the agreement's user and proceeds by case analysis on the structure of the $constraint$. A $constraint$ is either a $Principal$, a $Count$ or a $CountByPrin$. The translation for $Principal$ returns the translation function (namely $trans\_prin$) for the $prn$ (the $prin$ that accompanies the $const$ constraint). The translation for $Count$ and $CountByPrin$ return the translation function $trans\_count$. 


\lstset{language=Coq, frame=single, caption={Translation of a Constraint},label={lst:transconstraintCoq}}
\begin{lstlisting}

Fixpoint trans_constraint 
  (e:environment)(x:subject)(const:constraint)(IDs:nonemptylist policyId)
  (prin_u:prin){struct const} : Prop := 
  match const with
    | Principal prn => trans_prin x prn
  
    | Count n => trans_count e n IDs prin_u

    | CountByPrin prn n => trans_count e n IDs prn 

  end.
  
\end{lstlisting}


The translation of a $NotCons$ (called $trans\_notCons$ in listing~\ref{lst:transnotConsCoq}) takes as input $e$ the $environment$, $x$ the $subject$, $const$, the $constraint$ to translate, $IDs$, the set of identifiers (of policies implied by the parent $preRequisite$) and $prin_{u}$, the agreement's user and proceeds to return the negation of $trans\_constraint$ (see listing~\ref{lst:transconstraintCoq}).

\lstset{language=Coq, frame=single, caption={Translation of NotCons},label={lst:transnotConsCoq}}
\begin{lstlisting}

Definition trans_notCons
  (e:environment)(x:subject)(const:constraint)(IDs:nonemptylist policyId)(prin_u:prin) : Prop :=
  ~ (trans_constraint e x const IDs prin_u).
\end{lstlisting}


The translation of a $Count$ or a $CountByPrin$ (called $trans\_count$ in listing~\ref{lst:transcountCoq}) takes as input $e$ the $environment$, $n$ the total number of times the subjects mentioned in $prin_{u}$ (last parameter) may invoke the policies identified by $IDs$ (third parameter).

To implement the translation for a $Count$ or a $CountByPrin$ we start by calling an auxiliary function $process\_two\_lists$ that effectively returns a new list composed of pairs of members of the first list and the second list (the cross-product of the two input lists). In the case of $trans\_count$, the call is ``$process\_two\_lists$ $IDs$ $prin_u$'' which returns a list of pairs of $policyId$ and $subject$ namely $ids\_and\_subjects$. $ids\_and\_subjects$ is then passed to a locally defined function \emph{trans_count_aux}.

$trans\_count\_aux$ returns the current count for a single pair of $policyId$ and $subject$ (the call to $getCount$ which looks up the environment $e$ and returns the current count per each $subject$ and $policyId$) and for a list of pairs of $policyId$ and $subject$s, the addition of $get\_count$ (for the first pair) and $trans\_count\_aux$s (for the rest of the pairs) is returned. 

A local variable $running_total$ has the value returned by $trans\_count\_aux$. Finally the proposition $running_total < n$ is returned as the translation for a $Count$ or a $CountByPrin$.

Note that the only difference between translations for a $Count$ and a $CountByPrin$ is the additional $prn$ parameter for $CountByPrin$ which allows for getting counts for subjects not necessarily the same as $prin_{u}$, the agreement's user(s).


\lstset{language=Coq, frame=single, caption={Translation of count},label={lst:transcountCoq}}
\begin{minipage}[c]{0.95\textwidth}
\begin{lstlisting}
Fixpoint trans_count 
  (n:nat): Prop := 

  let trans_count_aux 
    := (fix trans_count_aux
         (ids_and_subjects : nonemptylist (Twos policyId subject)) : nat :=
     match ids_and_subjects with
        | Single pair1 => getCount e (right pair1) (left pair1)
        | NewList pair1 rest_pairs =>
            (getCount e (right pair1)(left pair1)) +
            (trans_count_aux rest_pairs)
      end) in
  
  let ids_and_subjects := process_two_lists IDs prin_u in
  let running_total := trans_count_aux ids_and_subjects in
  running_total < n.
\end{lstlisting}
\end{minipage}

