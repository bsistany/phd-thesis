%======================================================================
\chapter{Further Future Work Directions}
%======================================================================

%----------------------------------------------------------------------
%\section{Summary}
%
%We started off by looking at \ac{drel}s and specifically at \ac{odrl} and considered its formal semantics as captured by~\cite{pucella2006}. We presented the encodings and semantics of the constructs for a significant subset of \ac{odrl} in Coq and then defined what queries looked like and what the decision problem was in this context. We have also encoded the decision algorithms as presented in~\cite{pucella2006} in Coq in order to perform formal verification of theorems of interest. We noted the common thread between \ac{rel}s and policy languages for access control systems such as those between \ac{odrl} and \ac{selinux}. We discussed the goal of generalizing the concept of a policy language from strictly representing subsets of \ac{odrl}, to representing (subsets of) both \ac{odrl} and \ac{selinux} policy languages with the goal of applying the decision algorithms to both types of policies, in a unified manner.
%%----------------------------------------------------------------------
%
%
%\section{Machine-Checked Proof of Decidability of Queries}
%
%By defining formal semantics for \ac{odrl} authors of~\cite{pucella2006} were able to show some important results. First result is that answering the question of whether a set of ODRL statements imply a permission, denial or other possibilities is decidable and also that its complexity is NP-hard.
%
%The authors of~\cite{pucella2006} then prove that by removing the construct $not[policySet]$ from \ac{odrl}'s syntax answering the same query remains decidable and efficient (polynomial time complexity). 
%
%We will prove equivalent results as above starting with the decidability result of answering a query in ODRL0 (which does not include $not[policySet]$). The theorem in listing~\ref{lst:decidabilityodrl0coq} states that for all environments, all single agreements, all subjects, all actions and all assets, either permission is granted, permission is denied, permission is unregulated or query is inconsistent. 
%
%\lstset{language=Coq, frame=single, caption={Environments and Counts},label={lst:decidabilityodrl0coq}}
%\begin{minipage}[c]{0.95\textwidth}
%\begin{lstlisting}
%Theorem queriesAreDecidable: forall (e:environment), 
%                forall (agr: agreement),
%                forall (s:subject),
%                forall (action:act),
%                forall (a:asset),
%
%(permissionGranted e [agr] s action a) \/
%(permissionDenied e [agr] s action a)  \/
%(queryInconsistent e [agr] s action a) \/
%(permissionUnregulated e [agr] s action a).
%
%\end{lstlisting}
%\end{minipage}
%
%We will then augment ODRL0 with the constructs we omitted from the full ODRL (resulting in what we have earlier called ODRL1 or ODRL2) including the troublesome construct $not[policySet]$ and attempt to prove that the decidability results remain intact. There is a chance that a proof is not possible due to particulars of the Coq encoding we have used, in in which case, we will adjust our encoding.

 
We have designed \ac{ACCPL} as a suitable core policy language for \ac{pbac} systems that can be used for expressing general access-control expressions and also as a target language for deploying
policies written in other languages. We could capture, implement and study the semantics of these other policy-based access-control using the \ac{ACCPL} translation function framework and ultimately certify the semantics of those languages with respect to their specifications the same way \ac{ACCPL} has been certified.

\section{SELinux}

While \ac{xacml} is a high-level and platform independent access control system, \ac{selinux} is platform dependent (e.g. Linux based) and low-level. \ac{selinux} enhances the \ac{DAC} that most unix based systems employ by \ac{MAC} where designed access control policies are applied throughout the system possibly overriding whatever \ac{DAC} is in place by the system users. 

\ac{selinux} uses Linux's extended file attributes to attach a \emph{security context} to passive entities (e.g. files, directories, sockets) and also to each active entity, typically a Linux user space process. Security context is a data structure that is composed of a user, a role and a domain (or type). While users can map directly to ordinary user names they can also be defined separately. Roles are meant to group users and add flexibility when assigning permissions and are the basis for \ac{rbac} in \ac{selinux}. Finally domains or types are the basis for defining common access control requirements for both passive and active entities. 


The enforcement of \ac{selinux} policies are performed by the \emph{security server}. Whenever a security operation is requested from the Linux kernel by a program running in user space, the security server is invoked to arbitrate the operation and either allow the operation or to deny it. Each operation is identified by two pieces of information: an object class (e.g. file) and a permission (e.g. read, write). When an operation is requested to be performed on an object, the class and the permissions associated with the object along with security contexts of the source (typically the source entity is a process) and the object are passed to the security server. The security server consults the loaded policy (loaded at boot time) and allows or denies the access request~\cite{Sarna}.

\section{SELinux Policy Language}

The \ac{selinux} policy has four different kinds of statements: declarations, rules, constraints and assertions~\cite{ArcherLP03}. Assertions are compile time checks that the \syn{checkpolicy} tool performs at compile time. The other three kinds of statements however are evaluated at run-time. 

Declaration statements are used to declare types, roles and users. Type declaration statements are used to introduce new types. Roles are declared and authorized for particular domains (types) through role declarations, and finally user declarations are used to define each user and to specify the set of authorized roles for each of these users (see~\ref{lst:declsselinux}). In the following listings we will present a simplified and modified version of the official \ac{selinux} syntax that accomadates better the use of the \ac{ACCPL} Coq framework.

\lstset{language=selinux}
\begin{lstlisting}[frame=single, caption={Declaration Statements},label={lst:declsselinux}]

'type' T ';' &\Comment{; type T}&

'role' R T ';' &\Comment{; role R is associated with type T }&

'user' U R ';' &\Comment{; user U is associated with role R }&

\end{lstlisting}

Rule statements define access vector rules. Access vector (AV) rules (see listing~\ref{lst:avruleselinux}) specify which operations are allowed and whether to audit (log). Any operation not covered by AV rules are denied by default and all denied operations are logged. The semantics of the AV rule with \syn{<avkind>} \syn{allow} is: processes with type \syn{T1} are allowed to perform operations in \syn{P} on objects with class \syn{C} and type \syn{T2}. Note that in our modified syntax for \ac{selinux} we will use an explicit \syn{deny} as can be seen in listing~\ref{lst:avruleselinux}.

\lstset{language=selinux}
\begin{lstlisting}[frame=single, caption={AV Rule},label={lst:avruleselinux}]

<avRule> ::= <avkind> T1 T2:C P ';'

<avkind> ::= 'allow' | 'deny'
\end{lstlisting}

When a process changes security context, the role may change, assuming a ``role transition'' rule exists, relating the old and the new roles. There is a related AV rule called the ``type transition'' rule where a process with type \syn{T1} is allowed/denied to transition to type \syn{T2} when C=process and P=transition (see~\ref{lst:typeAndroletransselinux}).

\lstset{mathescape, language=AST} 
\begin{minipage}[c]{0.95\textwidth}
\begin{lstlisting}[frame=single, caption={Type Transition and Role-Allow Rules},label={lst:typeAndroletransselinux}]
<avkind> T1 T2:process transition ';'

<role_transition_rule> ::= <avkind> R1 R2 ';' &\Comment{; when a process changes security contexts this rule must hold }&
\end{lstlisting}
\end{minipage}

Constraints are additional conditions on permissions in the form of boolean expressions that must hold in order for the specified permissions to be allowed (see listing~\ref{lst:constrainselinux}). Whenever a permission is requested on an object class C, the security server checks that the constraints hold.

\lstset{language=selinux}
%\begin{minipage}[c]{0.95\textwidth}
\begin{lstlisting}[frame=single, caption={Constraint Definition},label={lst:constrainselinux}]

<constraint> ::= 'constrain' C, P, <expr> ';'

<expr> ::= 'not' <expr> | <expr> 'and' <expr> | <expr> or <expr> | U1 <op> U2 | T1 <op> T2 | R1 <op> R2

<op> ::= '==' | '!='
\end{lstlisting}
%\end{minipage}

\section{Agreements in SELinux}

We will start by limiting the \ac{selinux} policy language to only allow AV rules. As mentioned earlier an operation not covered by an allow rule is denied by default in \ac{selinux} proper. We will make up explicit deny rules, such that an agreement is defined to be a combination of allow and deny rules. Allow and deny rules as mappings are defined in listing~\ref{lst:allowmappingastselinux}.

\lstset{language=AST}
\begin{lstlisting}[frame=single, caption={'allow'/'deny' Rule as a Mapping},label={lst:allowmappingastselinux}]
AV rule : $T \times (T \times C) \rightarrow 2^{P}$
\end{lstlisting}

\lstset{language=AST}
\begin{lstlisting}[frame=single, caption={\ac{selinux} Agreement},label={lst:agreementastselinux}]
<agreement> ::= <avRule> ';' <agreement> 
\end{lstlisting}


\section{Environments}

Environments are collections of \emph{role-type} and \emph{user-role} relations. A role-type relation \syn{role(R, T)} simply associates a role with a type. A user-role relation \syn{user(U, R)} associates a user with a role. An environment is consistent with respect to a security context \syn{<T, R, U>}, if and only if \syn{role(R, T)} and \syn{user(U, R)} relations hold in the environment. 

\section{Queries in SELinux}

The decision problem in \ac{selinux} access control is whether an entity with security context \syn{<T1, R1, U1>} may perform action \syn{P1} to entity with object class \syn{C1} with security context \syn{<T2, R2, U2>}.

To answer such queries we use the authorization relation \syn{auth(C, P, T1, R1, U1, T2, R2, U2)} which is conceptually equivalent to the \syn{Permitted} answer from \ac{ACCPL} (see listing~\ref{lst:fqplussel}).

\lstset{mathescape, language=AST} 
\begin{lstlisting}[frame=single, caption={\syn{Permitted} for \ac{selinux}},label={lst:fqplussel}]
$allow (T1, T2, C, P)$ $\land$ $(E$ $consistent$ $wrt$ <T1, R1, U1> $\land$ <T2, R2, U2>) $\land$ $(((C,P)==(process, transition))$ $\implies$ $allow (R1, R2))$ $\implies auth(C, P, T1, R1, U1, T2, R2, U2)$ 
\end{lstlisting}

The negation of the authorization relation \syn{auth(C, P, T1, R1, U1, T2, R2, U2)} is conceptually equivalent to the \syn{NotPermitted} answer from \ac{ACCPL} (see listing~\ref{lst:fqminussel}).

\lstset{mathescape, language=AST} 
\begin{lstlisting}[frame=single, caption={\syn{NotPermitted} for \ac{selinux}},label={lst:fqminussel}]

$deny (T1, T2, C, P)$ $\lor$ $\lnot (E$ $consistent$ $wrt$ <T1, R1, U1> $\land$ <T2, R2, U2>) $\lor$ $(((C,P)==(process, transition))$ $\implies$ $deny (R1, R2))$ $\implies \lnot auth(C, P, T1, R1, U1, T2, R2, U2)$ 

\end{lstlisting}

\section{Decidability of Queries in SELinux}

As future work, we will be investigating the question of decidability for answering queries given an \ac{selinux} policy. We will first state a decidability theorem similar to the theorem in listing~\ref{lst:agreementdecidablecoq} (minor adjustments may be needed to allow for differences with \ac{selinux} policy language) and present a proof for it in Coq. Most policy based access-control languages typically use a two-valued decision set to indicate whether an access request is granted or denied. The literature for the \ac{selinux} implies only these two outcomes are possible so initially we will attempt to prove this conjecture in Coq. 

As a next step for future work on \ac{selinux}, we will add constraint relations to our simplified \ac{selinux} policy language and prove the decidability results stand for the augmented policy language.

\section{Related Work}

Policies in some \ac{pbac} based languages such as \ac{xacml} and \ac{odrl} are expressed in \ac{xml}. Authors of~\cite{ArdagnaDVS04} argue that \ac{xml}-based \ac{pbac} systems are well suitable for the Internet context where simplicity and easy integration with existing technology and tools must be ensured. \ac{xml}-based languages are also well suited for the easy interchange of policies~\cite{ArdagnaDVS04}. Access rules along with conditions under which access is granted or denied in such languages are expressed in a declarative manner using fragments of a natural language like English, resulting in ambiguity with regards to the intended behaviour of the system encoded in these access rules. The ambiguity in semantics of these languages may lead to implementations varying in their interpretations of the access-control language and ultimately making real the possibility of security breaches (e.g. access granted to unauthorized subjects). 

\section{MENTION MORE XML based languages HERE}

\subsection{ODRL}

\ac{odrl} is a declarative XML based \ac{rel} which has been accepted as part of the W3C community with the mandate of standardizing how rights and policies, related to the usage of digital content on the Open Web Platform, OWP~\cite{openwebplatform}, are expressed. \ac{odrl} is also used to express the conditions under which users' rights may be exercised. \ac{odrl} 2.0 supports expression of rights and also privacy rules for social media while \ac{odrl} 1.0 was only dealing with the mobile ecosystem. 

\subsection{xrml}
\ac{xrml} is another popular declarative XML based language used for writing software licenses~\cite{HalpernW08}. \ac{xrml} has been selected by the Moving Picture Experts Group (MPEG) to be the \ac{rel} for MPEG-21 which is an ISO standard for multimedia applications. First released in 2000, \ac{xrml} received the support of many technology providers, content owners, distributors, and retailers with some agreeing to build products and/or services that are \ac{xrml} compliant~\cite{HalpernW08}.

\subsection{xacml}
A widely used \ac{pbac} based XML based language is \ac{xacml}, an OASIS standard. \ac{xacml} defines a language for expression of policies and access requests, and a workflow to achieve policy enforcement~\cite{DBLP:conf/essos/MasiPT12}. \ac{xacml} also defines a standard format to be used to transport policies
between disparate access control systems~\cite{ArdagnaDVS04}.

\subsection{WS-Policy}
WS-Policy is a XML based specification which has been accepted as part of the W3C community since September 2007. WS-Policy is used to describe policies for Web services on security, trust, quality, etc and for Web service clients to advertise their policy requirements~\cite{wspolicy}.
According to~\cite{wspolicy} WS-Policy's scope is limited to allowing endpoints to specify requirements and capabilities needed for establishing a connection, therefore WS-Policy is not a language for expressing complex and application specific policies that take effect after the connection is established. WS-Policy uses a simple and extensible grammar for expressing policies which are composed of assertions and alternatives~\cite{wspolicy}. Assertions are requirements to be met by the policy's consumers (client Web services) and alternatives are mutually exclusive assertions. 

\section{MENTION SOME languages that have FORMAL semantics}

Whether a \ac{pbac} based language is based on \ac{xml} or not, formal specification of how an access-control mechanism should behave, is typically given in some sort of logic, often a subset of first order logic. 



odrl has formal semantics that are based on logic
xrml has formal semantics that are based on logic

xacml: 
Formalisation and Implementation of the XACML Access Control Mechanism
Core Xacml is "Towards Reasonability Properties for Access-Control Policy Languages"

To show that an access-control system actually behaves correctly with respect to its specification, proofs are needed, however the proofs that are often presented in the literature are hard or impossible to formally verify. The verification difficulty is partly due to the fact that the language used to do the proofs while mathematical in nature, utilizes intuitive justifications to derive the proofs. Intuitive language in proofs means that the proofs could be incomplete and/or contain subtle errors. SHOW SOME RELATED WORK (pucella) that have proofs but the proofs cannot be formally verified.


NOW bring in AMY's WORKS...

NOW CLOSE WITH YOUR WORK


GO OVER MY CONCLUSIONS/CONTRIBUTIONS AGAIN





























