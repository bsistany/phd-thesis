%======================================================================
\chapter{Conclusions and Future Work}
%======================================================================

%----------------------------------------------------------------------
\section{Conclusions}

In this thesis, we presented the design and implementation of \ac{ACCPL} as a minimal and certified policy language. \ac{ACCPL} is a \ac{pbac} system that can be used to express general access-control rules and policies. In addition we have defined formal semantics for \ac{ACCPL} where we have discovered and added all possible cases when answering a query on whether to allow or deny an action to be performed on an asset. We have subsequently used the Coq Proof Assistant to state theorems about the expected behaviour of \ac{ACCPL} when evaluating a request with respect to a given policy, to develop proofs for those theorems and to machine-check the proofs ensuring correctness guarantees are provided. We have in particular stated, developed and proved a decidability result for \ac{ACCPL}. 

We have also studied \ac{ACCPL} in terms of Tschantz and Krishnamurthi's~\cite{Tschantz} reasonability properties which make policy languages more amenable to reasoning and concluded that, since \ac{ACCPL} supported these properties, \ac{ACCPL} was significantly more amenable to analysis and reasoning than other access-control policy languages. 

We additionally described why certain design choices were made and how they contributed to the ease of reasoning for \ac{ACCPL}. Admittedly some expressive power present in other access-control policy languages, were omitted from \ac{ACCPL} in order to achieve the reported correctness proofs. For example, in \ac{ACCPL} we only support base policy sets (policy sets that are not composed of other policy sets) i.e. no combining of base policy sets using conjunctions or other combining operators are supported in \ac{ACCPL}. Such omitted features and/or other extensions could be added to \ac{ACCPL} provided the proofs for the stated theorems could still be derived. 
The translation function framework we have developed for \ac{ACCPL} is meant to be used by policy language designers to keep the delicate balance between adding in expressiveness while keeping the theorems provable. This is an instance of an important lesson we learned while developing \ac{ACCPL}: the process of designing a certifiable policy language cannot be done in isolation since while certifiability of the language derives the design one way, expressiveness requirements of the language usually drive it in the opposite direction. Certifiability cannot be a post-design activity as the changes it will impose are often fundamental ones.


\section{Future Work}

\ac{ACCPL} was also designed as a target language for deploying policies written in other languages. We could capture, implement and study the semantics of these other policy-based access-control using the \ac{ACCPL} translation function framework and ultimately certify the semantics of those languages with respect to their specifications the same way \ac{ACCPL} has been certified.

While \ac{xacml} is a high-level and platform independent access control system, \ac{selinux} is platform dependent (e.g. Linux based) and low-level. \ac{selinux} enhances the \ac{DAC} that most unix based systems employ by \ac{MAC} where designed access control policies are applied throughout the system possibly overriding whatever \ac{DAC} is in place by the system users. 

\ac{selinux} uses Linux's extended file attributes to attach a \emph{security context} to passive entities (e.g. files, directories, sockets) and also to each active entity, typically a Linux user space process. Security context is a data structure that is composed of a user, a role and a domain (or type). While users can map directly to ordinary user names they can also be defined separately. Roles are meant to group users and add flexibility when assigning permissions and are the basis for \ac{rbac} in \ac{selinux}. Finally domains or types are the basis for defining common access control requirements for both passive and active entities. 


The enforcement of \ac{selinux} policies are performed by the \emph{security server}. Whenever a security operation is requested from the Linux kernel by a program running in user space, the security server is invoked to arbitrate the operation and either allow the operation or to deny it. Each operation is identified by two pieces of information: an object class (e.g. file) and a permission (e.g. read, write). When an operation is requested to be performed on an object, the class and the permissions associated with the object along with security contexts of the source (typically the source entity is a process) and the object are passed to the security server. The security server consults the loaded policy (loaded at boot time) and allows or denies the access request~\cite{Sarna}.

\section{SELinux Policy Language}

The \ac{selinux} policy has four different kinds of statements: declarations, rules, constraints and assertions~\cite{ArcherLP03}. Assertions are compile time checks that the \syn{checkpolicy} tool performs at compile time. The other three kinds of statements however are evaluated at run-time. 

Declaration statements are used to declare types, roles and users. Type declaration statements are used to introduce new types. Roles are declared and authorized for particular domains (types) through role declarations, and finally user declarations are used to define each user and to specify the set of authorized roles for each of these users (see~\ref{lst:declsselinux}). In the following listings we will present a simplified and modified version of the official \ac{selinux} syntax that accommodates better the use of the \ac{ACCPL} Coq framework.

\lstset{language=selinux}
\begin{lstlisting}[frame=single, caption={Declaration Statements},label={lst:declsselinux}]

'type' T ';' &\Comment{; type T}&

'role' R T ';' &\Comment{; role R is associated with type T }&

'user' U R ';' &\Comment{; user U is associated with role R }&

\end{lstlisting}

Rule statements define access vector rules. Access vector (AV) rules (see listing~\ref{lst:avruleselinux}) specify which operations are allowed and whether to audit (log). Any operation not covered by AV rules are denied by default and all denied operations are logged. The semantics of the AV rule with \syn{<avkind>} \syn{allow} is: processes with type \syn{T1} are allowed to perform operations in \syn{P} on objects with class \syn{C} and type \syn{T2}. Note that in our modified syntax for \ac{selinux} we will use an explicit \syn{deny} as can be seen in listing~\ref{lst:avruleselinux}.

\lstset{language=selinux}
\begin{lstlisting}[frame=single, caption={AV Rule},label={lst:avruleselinux}]

<avRule> ::= <avkind> T1 T2:C P ';'

<avkind> ::= 'allow' | 'deny'
\end{lstlisting}

When a process changes security context, the role may change, assuming a ``role transition'' rule exists, relating the old and the new roles. There is a related AV rule called the ``type transition'' rule where a process with type \syn{T1} is allowed/denied to transition to type \syn{T2} when C=process and P=transition (see~\ref{lst:typeAndroletransselinux}).

\lstset{mathescape, language=AST} 
\begin{minipage}[c]{0.95\textwidth}
\begin{lstlisting}[frame=single, caption={Type Transition and Role-Allow Rules},label={lst:typeAndroletransselinux}]
<avkind> T1 T2:process transition ';'

<role_transition_rule> ::= <avkind> R1 R2 ';' &\Comment{; when a process changes security contexts this rule must hold }&
\end{lstlisting}
\end{minipage}

Constraints are additional conditions on permissions in the form of boolean expressions that must hold in order for the specified permissions to be allowed (see listing~\ref{lst:constrainselinux}). Whenever a permission is requested on an object class C, the security server checks that the constraints hold.

\lstset{language=selinux}
%\begin{minipage}[c]{0.95\textwidth}
\begin{lstlisting}[frame=single, caption={Constraint Definition},label={lst:constrainselinux}]

<constraint> ::= 'constrain' C, P, <expr> ';'

<expr> ::= 'not' <expr> | <expr> 'and' <expr> | <expr> or <expr> | U1 <op> U2 | T1 <op> T2 | R1 <op> R2

<op> ::= '==' | '!='
\end{lstlisting}
%\end{minipage}

\section{Agreements in SELinux}

We will start by limiting the \ac{selinux} policy language to only allow AV rules. As mentioned earlier an operation not covered by an allow rule is denied by default in \ac{selinux} proper. We will make up explicit deny rules, such that an agreement is defined to be a combination of allow and deny rules. Allow and deny rules as mappings are defined in listing~\ref{lst:allowmappingastselinux}.

\lstset{language=AST}
\begin{lstlisting}[frame=single, caption={'allow'/'deny' Rule as a Mapping},label={lst:allowmappingastselinux}]
AV rule : $T \times (T \times C) \rightarrow 2^{P}$
\end{lstlisting}

\lstset{language=AST}
\begin{lstlisting}[frame=single, caption={\ac{selinux} Agreement},label={lst:agreementastselinux}]
<agreement> ::= <avRule> ';' <agreement> 
\end{lstlisting}


\section{Environments}

Environments are collections of \emph{role-type} and \emph{user-role} relations. A role-type relation \syn{role(R, T)} simply associates a role with a type. A user-role relation \syn{user(U, R)} associates a user with a role. An environment is consistent with respect to a security context \syn{<T, R, U>}, if and only if \syn{role(R, T)} and \syn{user(U, R)} relations hold in the environment. 

\section{Queries in SELinux}

The decision problem in \ac{selinux} access control is whether an entity with security context \syn{<T1, R1, U1>} may perform action \syn{P1} to entity with object class \syn{C1} with security context \syn{<T2, R2, U2>}.

To answer such queries we use the authorization relation \syn{auth(C, P, T1, R1, U1, T2, R2, U2)} which is conceptually equivalent to the \syn{Permitted} answer from \ac{ACCPL} (see listing~\ref{lst:fqplussel}).

\lstset{mathescape, language=AST} 
\begin{lstlisting}[frame=single, caption={\syn{Permitted} for \ac{selinux}},label={lst:fqplussel}]
$allow (T1, T2, C, P)$ $\land$ $(E$ $consistent$ $wrt$ <T1, R1, U1> $\land$ <T2, R2, U2>) $\land$ $(((C,P)==(process, transition))$ $\implies$ $allow (R1, R2))$ $\implies auth(C, P, T1, R1, U1, T2, R2, U2)$ 
\end{lstlisting}

The negation of the authorization relation \syn{auth(C, P, T1, R1, U1, T2, R2, U2)} is conceptually equivalent to the \syn{NotPermitted} answer from \ac{ACCPL} (see listing~\ref{lst:fqminussel}).

\lstset{mathescape, language=AST} 
\begin{lstlisting}[frame=single, caption={\syn{NotPermitted} for \ac{selinux}},label={lst:fqminussel}]

$deny (T1, T2, C, P)$ $\lor$ $\lnot (E$ $consistent$ $wrt$ <T1, R1, U1> $\land$ <T2, R2, U2>) $\lor$ $(((C,P)==(process, transition))$ $\implies$ $deny (R1, R2))$ $\implies \lnot auth(C, P, T1, R1, U1, T2, R2, U2)$ 

\end{lstlisting}

\section{Decidability of Queries in SELinux}

As future work, we will be investigating the question of decidability for answering queries given an \ac{selinux} policy. We will first state a decidability theorem similar to the theorem in listing~\ref{lst:agreementdecidablecoq} (minor adjustments may be needed to allow for differences with \ac{selinux} policy language) and present a proof for it in Coq. Most policy based access-control languages typically use a two-valued decision set to indicate whether an access request is granted or denied. The literature for the \ac{selinux} implies only these two outcomes are possible so initially we will attempt to prove this conjecture in Coq. 

As a next step for future work on \ac{selinux}, we will add constraint relations to our simplified \ac{selinux} policy language and prove the decidability results stand for the augmented policy language.



%----------------------------------------------------------------------

\section{Reasonability Properties}

% determinism and total
In this chapter we have conjectured that \ac{ACCPL} is deterministic, total, safe, has independent composition property and supports a monotonic policy combinator. However, we have not represented these definitions in Coq for \ac{ACCPL}, nor have we proved \ac{ACCPL} having these properties, as we claim. We defer proving these properties for \ac{ACCPL} as future work.  































