% A B S T R A C T
% ---------------

\begin{center}\textbf{Abstract}\end{center}

In this thesis we present the design and implementation of a \ac{ACCPL} that can be used to express access-control rules and polices. Although full-blown access-control policy languages such as \ac{xacml}~\cite{xacml} already exist, because access rules in such languages are often expressed in a declarative manner using fragments of a natural language like English, it isn't always clear what the intended behaviour of the system encoded in these access rules should be. To remedy this ambiguity, formal specification of how an access-control mechanism should behave, is typically given in some sort of logic, often a subset of first order logic. To show that an access-control system actually behaves correctly with respect to its specification, proofs are needed, however the proofs that are often presented in the literature are hard or impossible to formally verify. The verification difficulty is partly due to the fact that the language used to do the proofs while mathematical in nature, utilizes intuitive justifications to derive the proofs. 

\ac{ACCPL} is minimal by design. Minimality refers to the size of the language; the syntax, auxiliary definitions and the semantics of \ac{ACCPL} only take a few pages to describe. This compactness allows us to concentrate on the main goal of this thesis which is the ability to reason about the policies written in \ac{ACCPL} with respect to specific questions. By making the language compact, we have stayed away from completeness in several directions. For example, \ac{ACCPL} uses only a single policy combinator, the conjunction policy combinator. 

We also consider \ac{ACCPL} a core policy access-control language since we have retained the core features of many access-control policy languages. For instance \ac{ACCPL} employs a single condition of type ``requirement'' where other languages may have very expressive and rich set of requirements. \ac{ACCPL} is minimal also in the sense that, the proofs of the theorems we have completed are minimal and could be extended to mirror the future extensions to \ac{ACCPL}.

The Coq proof assistant is used to encode the syntax and model the behaviour of \ac{ACCPL}. We use certain properties described in~\cite{Tschantz} to show how amenable to analysis and reasoning \ac{ACCPL} policies are with respect to other access-control policy languages and how the design choices that were made contributed to this ease of reasoning. In particular decidability criteria for \ac{ACCPL} are expressed and proofs are developed in the Coq proof assistant.


\cleardoublepage
%\newpage

