% A B S T R A C T
% ---------------

\begin{center}\textbf{Abstract}\end{center}

We present a \ac{ACCPL} that can be used to express access-control rules in a minimal way. Full-blown access-control policy languages such as \ac{xacml}~\cite{xacml} already exist however because access rules in such languages are often expressed in a declarative manner using fragments of a natural language like English, it isn't always clear what the intended behaviour of the system encoded in the access rules should be. To remedy this ambiguity, formal specification of how an access-control mechanism should behave is typically given in some sort of logic, often a subset of first order logic. To show that an access-control system actually behaves correctly with respect to its specification, proofs are needed, however the proofs that are often presented in the literature are hard or impossible to formally verify. The verification difficulty is partly due to the fact that the language used to do the proofs while mathematical in nature, utilizes intuitive justifications to derive the proofs. 

In this thesis, we describe the design and implementation of \ac{ACCPL} in Coq. The Coq proof assistant is used to encode and model the behaviour of \ac{ACCPL}. We will use certain properties described in~\cite{Tschantz} to study how amenable to analysis and reasoning \ac{ACCPL} policies are with respect to other access-control policy languages and how the design choices that were made contributed to this ease of reasoning. In particular decidability criteria for \ac{ACCPL} are expressed and proofs are developed in the Coq proof assistant.

\cleardoublepage
%\newpage

