% A B S T R A C T
% ---------------

\begin{center}\textbf{Abstract}\end{center}

We present the design and implementation of a Certified Core Policy Language (ACCPL) that can be used to express access-control rules and policies. Although full-blown access-control policy languages such as eXtensible Access Control Markup Language \\(XACML)~\cite{xacml3pdf} already exist, because access rules in such languages are often expressed in a declarative manner using fragments of a natural language like English, it isn't always clear what the intended behaviour of the system encoded in these access rules should be. To remedy this ambiguity, formal specification of how an access-control mechanism should behave, is typically given in some sort of logic, often a subset of first order logic. To show that an access-control system actually behaves correctly with respect to its specification, proofs are needed, however the proofs that are often presented in the literature are hard or impossible to formally verify. The verification difficulty is partly due to the fact that the language used to do the proofs while mathematical in nature, utilizes intuitive justifications to derive the proofs. Intuitive language in proofs means that the proofs could be incomplete and/or contain subtle errors.

ACCPL is small by design. By small we refer to the size of the language; the syntax, auxiliary definitions and the semantics of ACCPL only take a few pages to describe. This compactness allows us to concentrate on the main goal of this thesis which is the ability to reason about the policies written in ACCPL with respect to specific questions. By making the language compact, we have stayed away from completeness and expressive power in several directions. For example, ACCPL uses only a single policy combinator, the conjunction policy combinator. The design of ACCPL is therefore a trade-off between ease of formal proof of correctness and expressive power.

We also consider ACCPL a core policy access-control language since we have retained the core features of many access-control policy languages. For instance ACCPL employs a single condition type called a ``prerequisite'' where other languages may have very expressive and rich sets of conditions. 


\cleardoublepage
%\newpage

