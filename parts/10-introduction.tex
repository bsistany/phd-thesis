%======================================================================
\chapter{Introduction}
%======================================================================

\section{Access-Control Models}


Authorization refers to the process of rendering a decision to allow access to a resource or asset of interest. By the same token all unauthorized access requests to resources must be controlled and ultimately denied, hence the term ``access-control''. 

Various access-control models exist and below we give a short summary of some of the most important ones leading up to our own core language, \ac{ACCPL}.

\subsection{Access Control Lists}

\ac{acl} is perhaps the oldest and the most basic access-control model. A list of subjects along with their rights are kept per resource or object of interest. Every time a subject makes an access request on an object, the \ac{acl} of the object is consulted and access is either granted or denied to the requester based on whether the requester is listed in the \ac{acl} and has the correct set of rights.

\subsection{Capabilities}
Capabilities based access-control works based on a list of objects and associated rights. The list of objects and the associated rights comprise an ``unforgeable'' ticket that a reference monitor checks to allow access (or not). Capabilities based systems don't need to authenticate users as \ac{acl} based systems do.

\subsection{RBAC}

In \ac{rbac} systems a requester`s role determines whether access is granted or denied. In this model users belong to roles and rights are associated with roles so no direct association between users and rights exist. Roles are meant to group users and add flexibility when assigning rights. \ac{rbac} systems naturally solve the problem of assigning \ac{acl}s for a large group of users and manage the administration cost of changing users' rights in \ac{acl} based systems.

\subsection{ABAC}

Despite many advantages of \ac{rbac} systems, some disadvantages also exist. Often a role needs to be decomposed into sub-roles based on the type of resource to be administered, and perhaps also based on the location that the resource serves~\cite{nist}. Basically \ac{rbac} suffers from a lack of a sub-typing mechanism whereby individual members of a group/role may be differentiated and access is granted or denied based on a more granular set of attributes. \ac{abac} model was proposed to fulfill this granularity requirement. In \ac{abac} access control decisions are made based on a set of attributes, associated with the subject making the request, the environment, and/or the resource itself~\cite{nist}. 

\subsection{PBAC}
In order to harmonize access control in large environments with many subjects and objects and disparate attributes, \ac{pbac} model has been proposed~\cite{nist}. \ac{pbac} allows for a more uniform access-control model across the system. \ac{pbac} systems help create and enforce policies
that define who should have access to what resources, and under what circumstances~\cite{nist}.
There is also a need for large organizations to put in place mechanisms such that access-control rules can be easily audited. This calls for a data-driven approach to access-control where the data, in this case the access-control rules, are available to read and analyze. A data-driven approach like \ac{pbac} additionally helps with modularity of the system as changes in access control rules will have almost no impact to the underlying system the rules are meant to protect.

\section{A Core Policy Language for PBAC Systems} 

Does there already exist a suitable policy language that can be used for expressing general access-control expressions? How about \ac{xacml}, \ac{odrl}~\cite{odrloneone} or \ac{xrml}~\cite{Wang}? 

Currently the most popular \ac{rel}s are the \ac{xrml}, and the \ac{odrl}. Both of these languages are XML based and are considered declarative languages. \ac{xrml} has been selected to be the REL for \emph{MPEG-21} which is an ISO standard for multimedia applications. \ac{odrl} is also a standards based \ac{rel} which has been accepted as part of the W3C community with the mandate of standardizing how rights and policies, related to the usage of digital content on the Open Web Platform, \emph{OWP}~\cite{openwebplatform}, are expressed. \ac{odrl} 2.0 supports expression of rights and also privacy rules for social media while \ac{odrl} 1.0 was only dealing with the mobile ecosystem -- \ac{odrl} 1.0 was adopted by the Open Mobile Alliance, \emph{OMA} in 2000.

\ac{rel}s, or more precisely \ac{drel}s when dealing with digital assets deal with the ``rights definition'' aspect of the \ac{drm} ecosystem. A \ac{drel}, allows the expression and definition of digital asset usage rights such that other areas of the \ac{drm} ecosystem, namely the enforcement mechanism and the usage tracking components can function correctly.

As popular as both \ac{xrml} and \ac{odrl} are, their adoption and usage is still somewhat limited in practice. Both Apple and Microsoft for example have defined their own lightweight \ac{rel}s~\cite{problemwithrels} in \emph{Fair Play}~\cite{fairplay} and in \emph{PlayReady}~\cite{playready}. The authors of~\cite{problemwithrels} argue that both these \ac{rel}s (\ac{xrml} and \ac{odrl}) and others are simply too complex to be used effectively (for expressing rights) since they also try to cover much of the the enforcement and tracking aspects of \ac{drm}s.

\ac{drm} refers to the digital management of rights associated with the access or usage of digital assets. There are various aspects of rights management however. According to the authors of the white paper ``A digital rights management ecosystem model for the education community,''(~\cite{collier})
 digital rights management systems cover the following four areas: 1) defining rights 2) distributing/acquiring rights 3) enforcing rights and 4) tracking usage.

\ac{xacml} is a high-level and platform independent access control system that is also XML based.  \ac{xacml} is an OASIS standard which defines a language for the definition of policies and access requests, and a workflow to achieve policy enforcement~\cite{DBLP:conf/essos/MasiPT12}. According to Masi et al~\cite{DBLP:conf/essos/MasiPT12} designing XACML access control policies is difficult and error prone. Furthermore \ac{xacml} comes without a formal semantics as do both \ac{xrml} and \ac{odrl}. The \ac{xacml} standard is written in prose and contains quite a number of loose points that may give rise to different interpretations and lead to different implementation choices~\cite{DBLP:conf/essos/MasiPT12}.

\ac{xacml}, \ac{odrl} and \ac{xrml} are all \ac{pbac} based languages where \ac{odrl} and \ac{xrml} differ from \ac{xacml} by their focus on digital assets protection and in general \ac{drm}, hence the term \ac{rel}. All three are full-blown and custom languages that have one thing in common; they suffer from a lack of formal semantics. Additionally all of these languages cover much more than policy expressions leading to access decisions; they also address enforcement of the policies (\ac{odrl} and \ac{xrml} specifically and \ac{drm} in general distinguish themselves from general access-control languages by additionally addressing enforcement of policies beyond where the policies were generated). A third reason that made these custom languages unsuitable as a core policy language was the fact that they are limited in terms of what can be built on top of them; for example expressing hierarchical role-based access-control  in \ac{xacml} requires a fairly complex encoding~\cite{Tschantz}.

Rights expressions in \ac{drel}s and specifically \ac{odrl} are used to arbitrate access to assets under conditions. The main construct in \ac{odrl} is the \emph{agreement} which specifies users, asset(s) and policies whereby controls on users' access to the assets are described. This is very similar to how access control conditions are expressed in access control policy languages such as \ac{xacml}~\cite{xacml} and \ac{selinux}~\cite{selinux}. In fact several authors have worked on interoperability between \ac{rel}s and access control policy languages, specifically between \ac{odrl} and \ac{xacml},~\cite{prados2005interoperability, maronas2009architecture} and also on translation from high level policies of \ac{xacml} to low-level and fine grained policies of \ac{selinux}~\cite{alam2008usage}. 

A policy language that was based on some sort of logic with formal semantics and also one that was minimal and extendible was clearly needed. We started by looking at Lithium~\cite{Halpern2008} and subsequently Pucella and Weissman`s subset of \ac{odrl}~\cite{pucella2006} as potential core languages. Lithium uses \ac{drm} as the main application of its access-control system whereas Pucella and Weissman`s primary goal is to define formal semantics for \ac{odrl} whose main application in turn is \ac{drm}. We will use Pucella and Weissman`s subset of \ac{odrl} as the basis for \ac{ACCPL} and in doing so treat digital rights as our main access-control application without loss of generality with respect to other applications, with the final goal of performing formal verification on policies written in \ac{ACCPL}.



%----------------------------------------------------------------------
\section{Formal Semantics for PBAC Languages}
%----------------------------------------------------------------------


Formal methods help ensure that a system behaves correctly with respect to a specification of its desired behavior~\cite{TAPL}. This specification of the desired behavior is what's referred to as \emph{semantics} of the system. Using formal methods requires defining precise and formal semantics, without which analysis and reasoning about properties of the system in question would become impossible. For example, an issue with the current batch of \ac{rel}s are due to their semantics being expressed in a natural language (e.g. English) which by necessity results in ambiguous and open to interpretation behavior. 

To formalize the semantics of \ac{pbac} languages several approaches have been attempted by various authors. Most are logic based~\cite{Halpern2008, pucella2006} while others are based on finite-automata~\cite{Holzer}, operational semantics based interpreters~\cite{Safavi-naini} and web ontology (from the Knowledge Representation Field)~\cite{Kasten2010MTS}. 


\section{Logic Based Semantics}

%See equation \ref{eqn_pi} on page \pageref{eqn_pi}.\footnote{A famous %equation.}

Formal logic can represent the statements and facts we express in a natural language like English. Propositional logic is expressive enough to express simple facts as propositions and uses connectives to allow for the negation, conjunction and disjunction of the facts. In addition simple facts can be expressed conditionally using the implication connective. Propositional logic however is not expressive enough to express policies of the kind used in languages like \ac{odrl} and \ac{xrml}. For example, a simple policy expressed in English like ``All who pay 5 dollars can watch the movie Toy Story'' cannot be expressed in propositional logic because the concept of  variables doesn't exist in propositional logic. 

A richer logic such as ``Predicate Logic'', also called ``First Order Logic'' (\emph{FOL}), is more suitable and has the expressive power to represent policies written in English. Moreover, FOL can be used to capture the meaning of policies in an unambiguous way.

Halpern and Weissman~\cite{Halpern2008} propose a fragment of FOL to represent and reason about policies. The fragment of FOL they arrive at is called \emph{Lithium} which is decidable and allows for efficiently answering interesting queries. Lithium restricts policies to be written based on the concept of ``bipolarity'' which disallows by construction policies that both permit and deny an action on an object. Pucella and Weissman~\cite{pucella2006} specify a predicate logic based language that represents a subset of ODRL.


\section{Specific Problem}

Policy languages and the agreements written in those languages are meant to implement specific goals such as limiting access to specific assets. The tension in designing a policy language is usually between how to make the language expressive enough, such that the design goals for the policy language may be expressed, and how to make the policies verifiable with respect to the stated goals.

As stated earlier, an important part of fulfilling the verifiability goal is to have formal semantics defined for policy languages. For \ac{odrl}, authors of~\cite{pucella2006} have defined a formal semantics based on which they declare and prove a number of important theorems (their main focus is on stating and proving algorithm complexity results). However as with many paper-proofs, the language used to do the proofs while mathematical in nature, uses a lot of intuitive justifications to show the proofs. As such these proofs are difficult to verify or more importantly to ``derive''. Furthermore the proofs can not be used directly to render a decision on a sample policy (e.g. whether to allow or deny access to an asset). Of course one may (carefully) construct a program based on these proofs for practical purposes but certifying such programs correct presents additional verification challenges, even assuming the original proofs were in fact correct.

\section{Contributions}

In this thesis we have built a language called \ac{ACCPL} based on \ac{odrl} and starting with definitions in~\cite{pucella2006}. The \ac{ACCPL} framework has been encoded in \emph{Coq}~\cite{BC04} which is both a programming language and a proof-assistant. We have declared and proved decidability results for \ac{ACCPL} in Coq which will allow us to extract programs from the proofs; the executable programs can be used on specific policies to render a specific decision such as ``a permission has been granted''. 

We originally started with a specific subset of~\cite{pucella2006} so that we could concentrate on what we believed to be the essence or core of the language. For example, we started with only one of three different kinds of ``facts'' (that may affect the permit/deny type decisions). We also had to change some of the language productions to allow for Coq`s requirement for clearly terminating recursion. However we maintained the central semantic definitions including ``Closed World Assumptions''~\cite{pucella2006} where the semantics only specify explicitly Permitted and notPermitted answers so the semantics as stated by Pucella and Weissman~\cite{pucella2006} are not complete and therefore not decidable. We have made major modifications to the semantics of Pucella and Weissman`s language to make the new \ac{ACCPL} language decidable.
    
Our decidability results subsume an important sub-category, namely inconsistency or conflict-detection in policy expressions or rules. Authors of~\cite{st2012verified}~\cite{felty13} describe and implement in Coq a conflict detection algorithm for detecting conflicts in \ac{xacml} access control rules. \ac{xacml} is an expressive and at the same time complex policy language which makes conflict detection a difficult task. Authors of~\cite{st2012verified} then prove the conflict detection algorithm correct (or certified) by developing a formal proof in Coq. The proof is rather complex and involves a large number of cases, including many corner cases that were difficult to get right~\cite{st2012verified}. For \ac{ACCPL} we have formally proven that conflicts are not possible.

\ac{ACCPL} being a core policy language with certified semantics could be used to implement various policy languages and reason about interoperability between those languages~\cite{prados2005interoperability, maronas2009architecture}. In this manner our Coq based language \ac{ACCPL} can be viewed as \emph{abstract syntax}, complete with defined semantics, that can be used for implementing various policy languages with more concrete syntax (e.g. W3C's \ac{odrl} and \ac{selinux}). 

For access to the Coq source code for \ac{ACCPL}, please refer to the GitHub repository at \url{https://github.com/bsistany/yacpl.git}.













