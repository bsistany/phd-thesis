\chapter{ACCPL in the Landscape of Policy-based Policy Languages}\label{chap:landscapechap}

Tschantz and Krishnamurthi~\cite{Tschantz} discuss the question of how different policy-based access-control languages may be distinguished from each other. The authors note that although existing studies of complexity and expressive power for access-control policy languages, may ensure tractable verification and the ability to capture certain policies, they do not address classification based on ease of reasoning about policies in a language. Tschantz and Krishnamurthi argue for the need for formal means to compare and contrast policy-based access-control languages with respect to a set of properties common to policy-based access-control languages. They analyze the behaviour of policies in light of additional and/or explicit environmental facts, policy growth and policy decomposition. Tschantz and Krishnamurthi apply their properties to two core policy languages and compare the two languages given the results. One of the core policy languages they present is a simplified \ac{xacml} and the other is Lithium~\cite{Halpern2008}. In the following sections we will describe Tschantz and Krishnamurthi's properties and evaluate \ac{ACCPL} with respect to these properties.

\section{Reasonability Properties}\label{sec:threeinterpretations}

The properties Tschantz and Krishnamurthi~\cite{Tschantz} present revolve around three questions:

\begin{enumerate}
\item How decisions change when the environment has more facts
\item Impact of policy growth on how decisions may change
\item How amenable policies are to compositional reasoning
\end{enumerate}

Tschantz and Krishnamurthi~\cite{Tschantz} use a simple policy as a motivating example (which they attribute to~\cite{Halpern2008}) that we will adapt to match our \ac{drm} specific application throughout this thesis and reuse here. The example will help to review the range of interpretations possible and ultimately the range of policy language design choices and where \ac{ACCPL} is on that range.
The policy in English is as follows:

\begin{itemize}
\item If the subject has a subscription to the Gold bundle, then permit that subject to print \syn{TheReport}.

\item If the subject has a subscription to the Basic bundle, then do not permit that subject to print \syn{TheReport}.

\item If the subject has no subscription to the Gold bundle, then permit that subject to play Terminator.
\end{itemize}

Consider a query and a fact in English as: A subscriber to the Basic bundle requests to play the movie Terminator. Let's extract the fact that the subject is a subscriber to the Basic bundle into a separate term and make it part of the environment. The basic policy, the query in question and the environment are all expressed as a (pseudo) logic formula and listed in the listing~\ref{lst:basicpolicyAST}. The question is, should the access query be granted?

\lstset{mathescape, language=AST}  
\begin{lstlisting}[frame=single, caption={Basic Policy},label={lst:basicpolicyAST}]

p1 = GoldBundle(s) -> Permitted(s, print, TheReport) 
p2 = BasicBundle(s) -> NotPermitted(s, print, TheReport) 
p3 = ~GoldBundle(s) -> Permitted(s, play, Terminator)
q1 = (s, play, Terminator) 
e1 = BasicBundle(s)
\end{lstlisting}


According to Tschantz and Krishnamurthi~\cite{Tschantz} at least three different interpretations are possible. We will list the possibilities and then evaluate how \ac{ACCPL} would answer the query.

\begin{itemize}
\item Implicit: grant access because of \syn{p3}. This interpretation assumes that if a subject has a subscription to the Basic bundle, the subject does not have a subscription to the Gold bundle so the assumption is that there is a proof for \syn{$\neg$GoldBundle(s)}. This is an example of ``implicit'' knowledge and ``closed world assumptions''.

\item Explicit: the policy does not apply to the query. In other words, the query is ``NotApplicable''. \syn{p1} and \syn{p2} don't apply since the asset and the action in the request don't match the ones in \syn{p1} and \syn{p2}. \syn{p3} does not apply either since no proof exists to show that \syn{$\neg$GoldBundle(s)}. 

\item Implicit with automatic proof capability: grants access to the query by automatically proving that \syn{$\neg$GoldBundle(s)}. Under this interpretation, proof by contradiction is used to establish \syn{$\neg$GoldBundle(s)}. The assumption \syn{$\neg$GoldBundle(s)} is added to \syn{e1} which already contains \syn{BasicBundle(s)}. Now \syn{p1} permits \syn{TheReport} to be printed by the subject while \syn{p2} would not permit \syn{TheReport} to be printed by the subject, leading to a contradiction and the proof of \syn{$\neg$GoldBundle(s)}.
\end{itemize}

A policy language based on the first and third interpretations are not certifiable, meaning a machine-checked proof of the correctness of their semantics with respect to different properties (such as decidability) may not be possible. The main problem with such languages is the implicitness of their semantics. This was the case for the Pucella and Weissman~\cite{pucella2006} fragment of \ac{odrl}; although the interpretation found in Pucella and Weissman's fragment doesn't necessarily match the first and the third interpretations above, they are certainly in the camp of ``implicit'' interpretations. \ac{ACCPL} matches the second interpretation in the sense of having explicit semantics which leads to certifiable results we discuss elsewhere. 



%\section{ACCPL}


When designing \ac{ACCPL} and to make the decidability proofs possible, we sought out all the different combinations and cases possible (through interaction with the proof process) and explicitly enumerated them in the translation functions (see the high-level algorithms in listing~\ref{inclusivePS} and in listing~\ref{exclusivePS}). Once we made explicit all the different cases where a decision was possible, we assigned \syn{Permitted} and \syn{NotPermitted} decisions to two specific cases while every other enumerated case was rendered with an \syn{Unregulated} decision. Explicit enumeration of all the different cases has been certified to be complete by the fact that we have the decidability results. Note that we could have made different design choices in assigning the decisions. 

For example, we could replace all or some of the current \syn{Unregulated} decision assignments to either \syn{Permitted} or \syn{NotPermitted} depending on whether we want to make the language more permissive or more prohibitive. If we want to only use \syn{Permitted} or \syn{NotPermitted}, the current decidability results would no longer hold as is. The \syn{Unregulated} case would have to be removed from the statement of the theorem \syn{trans_agreement_dec2} (in listing~\ref{lst:agreementdecidablecoq}) and the theorem statement to be modified to look like the theorem listed in listing~\ref{lst:agreementdecidablecoqSecond}. A few other intermediate results would also have to be restated and proved, however we fully expect the theorem (in listing~\ref{lst:agreementdecidablecoqSecond}) (and similar) ones to be provable in \ac{ACCPL}.

\lstset{language=Coq, frame=single, caption={Agreement Translation is Decidable (for \syn{Permitted} or \syn{NotPermitted} only)},label={lst:agreementdecidablecoqSecond}}
\begin{minipage}[c]{0.95\textwidth}
\begin{lstlisting}
Theorem trans_agreement_dec:
  forall
  (e:environment)(ag:agreement)(action_from_query:act)
   (subject_from_query:subject)(asset_from_query:asset),

 (isResultInQueryResult 
    (Result Permitted subject_from_query action_from_query asset_from_query)
    (trans_agreement e ag action_from_query subject_from_query asset_from_query)) 
\/
 (isResultInQueryResult 
    (Result NotPermitted subject_from_query action_from_query asset_from_query)
    (trans_agreement e ag action_from_query subject_from_query asset_from_query)).

\end{lstlisting}
\end{minipage}

\section{Policy Combinators}\label{sec:policycombinators}

Given that basic policies are often developed by independent entities within an organization, it is natural to want a way of combining them into a higher level policy for the whole organization using policy combinators. Indeed most policy-based access-control languages provide a way to combine policies although in practice, there is a range of possibilities for combining policies. 

According to Li et al.~\cite{LiWQBRLL09} \ac{xacml} defines three policy elements: rules, policies, and policy-sets. A rule is the most basic policy element in \ac{xacml} and is equivalent in concept to \ac{ACCPL}'s primitive policies. In \ac{xacml} a policy contains rules and a policy set contains policies or other policy sets. Both policies and policy sets provide a way to combine lower level policy elements. A \ac{xacml} policy specifies a rule-combining algorithm (RCA) while a policy set specifies a policy-combining algorithm (PCA). The RCA specifies
how the decisions from the rules are combined to yield one decision while the PCA specifies how the results of evaluating the sub-policies are combined to
yield one decision. Some examples of PCAs are: permit-override (if any of the sub policies returns a permit, return only permit), deny-override (if any of the sub policies returns a deny, return only deny)~\cite{Tschantz}. 

Some languages such as \ac{odrl} have nested sub-policies inside other policies where different combinators for different layers may be used to make the language more expressive. According to Pucella and Weissman~\cite{pucella2006} \ac{odrl} uses a conjunction operator for both basic policies and policy sets (which contain basic policies).







\section{Policy Based Access-Control Languages}

Tschantz and Krishnamurthi~\cite{Tschantz} formalize the differences between the three interpretations listed above but they also enumerate some common features to all access-control languages. For example, the notion of access requests and decisions are universal in access-control languages. Some languages treat the context or the environment where decisions are rendered separately while other languages may treat environments as part of the access request -- regardless, the notion of an environment is a common feature to access-control languages.


Tschantz and Krishnamurthi formally define an access-control policy language as a tuple $L = (P, Q, G, N, \left\langle\left\langle . \right\rangle\right\rangle)$ where $P$ is a set of policies, $Q$ is a set of requests or queries, $G$ is the granting decisions, $N$ is the non-granting decisions, $\left\langle\left\langle . \right\rangle\right\rangle$ is a function taking a policy $p \in P$ to a relation between $Q$ and $G \cup N$ where $G \cap N = \emptyset$. Let $D$ represent $G \cup N$. Policy $p$ assigns a decision of $d \in D$ to the query $q \in Q$. $L$ also defines a partial order on decisions such that $d \leq d'$ if either $d, d' \in N$ or $d, d' \in G$ or $d \in N$ and $d' \in G$. Note that for \ac{ACCPL}, $D = \left\{ {\text{Unregulated}, \text{NotPermitted}, \text{Permitted}}\right\}$, $G = \left\{ {\text{Permitted}}\right\}$ and $N = \left\{ {\text{Unregulated}, \text{NotPermitted}}\right\}$.





\section{Deterministic and Total}

According to Tschantz and Krishnamurthi~\cite{Tschantz} a language $L$ is defined to be deterministic if $forall$ $p \in P$, $forall$ $q \in Q$, $forall$ $d, d' \in D$, $q \left\langle\left\langle p  \right\rangle\right\rangle d$ $\land$ $q \left\langle\left\langle p  \right\rangle\right\rangle d'$ $\implies$ $d = d'$. In words, the definition states that for a deterministic language, given a query and a policy, if the policy translation renders a decision of $d$ and a decision of $d'$, then the two decisions must be the same.

A language $L$ is total if $forall$ $p \in P$, $forall$ $q \in Q$, $exists$ $d \in D$, s.t. $q \left\langle\left\langle p  \right\rangle\right\rangle d$. In words, for a total language, the policy translation always renders a decision. As for \ac{ACCPL} being a total language we believe the fact that \ac{ACCPL} is decidable (see listing~\ref{lst:decidabletypecoq}) implies \ac{ACCPL} is a total language.


\section{Safety}

As discussed in Section~\ref{sec:threeinterpretations} policy interpretations could be implicit or explicit. Explicit interpretations distinguish between unknown information and information known to be absent. The explicit approach promotes verbose policies and requests and usually leads to too many \syn{Unregulated} decisions (in the absence of the right environmental facts). However the verbosity could be used to direct what additional facts are needed to make a granting or denying decision. Under the second interpretation in Section~\ref{sec:threeinterpretations}, while the decision is \syn{Unregulated}, the system can direct the entity making the request to provide a proof of \syn{$\neg$GoldBundle(s)} in order for a granting decision to be rendered. 

Implicit interpretations may result in granting unintended access to protected resources because no explicit facts need be mentioned in the environment. Recall that the implicit approach works by making assumptions when facts are not explicitly mentioned. Such unintended assumptions lead to unintended permissions being granted.

Tschantz and Krishnamurthi~\cite{Tschantz} define safety as $forall$ $p \in P$, $forall$ $q, q' \in Q$, \\$q$ $\leq$ $q'$ $\implies$ $q \left\langle\left\langle p  \right\rangle\right\rangle d$ $\leq$ $q \left\langle\left\langle p  \right\rangle\right\rangle d'$ which states that under a safe language $L$, a request with less (environmental) information, $q$, will lead to a decision ($d$) that is less than the decision ($d'$) reached for a request with more (environmental) information such as $q'$ in this definition. Recall that in general, granting decisions are considered greater than non-granting decisions. That is for $d \in N$ and $d' \in G$ we have $d \leq d'$. In particular, in \ac{ACCPL} we have,  $N = \left\{ {Unregulated, NotPermitted}\right\}$ and $G = \left\{ {Permitted}\right\}$. Policies written in a safe language $L$, will not result in the leakage of permissions to unintended subjects. If and when incomplete information unintentionally causes an access permission to be granted to an intruder, safety has been violated.

Could we conclude that an access-control policy language with the explicit approach such as \ac{ACCPL} is also safe? in the design of \ac{ACCPL} and the translation functions in particular we only grant or deny access in specific cases. We grant access if the asset and the action in query are found in the agreement and \syn{trans_prin} has a proof and \syn{trans_preRequisite} for both policy set and policy have proofs. We deny an access request if the policy set is an exclusive policy set (\syn{primExclusivePolicySet}) and \syn{trans_prin} does not hold. In all other cases we render the decision of \syn{Unregulated}. In other words when all the facts hold, corresponding to the greatest query in the definition of safety, we render a granting decision which is also the greatest decision in the lattice of decisions. In all other cases, where all the facts (proofs) do not hold, we render either a decision of \syn{NotPermitted} or a decision of \syn{Unregulated} which are non-granting decisions, satisfying the definition of safety.

Notice that, technically in \ac{ACCPL} we have a partially ordered set, a \syn{poset} of decisions and not a lattice. This is because for \ac{ACCPL} there is no decision acting as the greatest lower bound but we do have a least upper bound in \syn{Permitted}, so we have: \syn{Unregulated} $\leq$ \syn{Permitted} and \syn{NotPermitted} $\leq$ \syn{Permitted}.


\section{Independent Composition}

As discussed in Section~\ref{sec:threeinterpretations}, the third interpretation works by taking into account not only the third policy, $p3$, but also $p1$ and $p2$ to render a decision of granted. Under this interpretation, proof by contradiction was used to establish \syn{$\neg$GoldBundle(s)}. The assumption \syn{$\neg$GoldBundle(s)} was added to the assumption \syn{BasicBundle(s)} to make \syn{p1} permit \syn{TheReport} to be printed by the subject while making \syn{p2} not permit \syn{TheReport} to be printed by the subject, leading to a contradiction and hence the proof of \syn{$\neg$GoldBundle(s)}. 

Notice that each of the policies in isolation would result in a \syn{Unregulated} decision under the third interpretation, whereas taking into account all three policies results in a \syn{Permitted} decision. Taking into account all policies and rendering a decision that is not different from combining the decisions reached by each primitive policy in isolation, is a property called ``Independent Composition'' by Tschantz and Krishnamurthi~\cite{Tschantz}. The third interpretation in Section~\ref{sec:threeinterpretations} clearly does not have this property. Formally, a language $L$ has the independent composition property, if $forall$ $p \in P$, $forall$ $q \in Q$, $forall$ $d, d^\ast \in D$, $\boxplus$ ($q \left\langle\left\langle p_{1}  \right\rangle\right\rangle d_{1}$, $q \left\langle\left\langle p_{2}  \right\rangle\right\rangle d_{2}$, ..., $q \left\langle\left\langle p_{n}  \right\rangle\right\rangle d_{n}$) $=$ $q \left\langle\left\langle (\oplus p_{1}, p_{2}, ..., p_{n}) \right\rangle\right\rangle d^\ast$, \\where $\oplus$ is a composition operator defined in the language and $\boxplus$ is the decision composition operator. For this definition to be well-defined, there is a requirement that complete environmental facts be available for each individual policy so that a decision can be rendered for each primitive policy with respect to the request, in isolation.

Does \ac{ACCPL} have the independent composition property? As covered in the chapter on semantics of \ac{ACCPL} (chapter~\ref{chap:accplsemanticscoq}) the translation functions for \ac{ACCPL}, starting with \syn{trans_agreement} all return sets of \syn{result}s. The set of \syn{result}s returned by the agreement translation function will have a result per primitive policy in the agreement. Recall that results containing a \syn{Permitted} answer and a \syn{NotPermitted} answer are mutually exclusive and the existence of a single (or more) \syn{Permitted} result in the set makes the whole set ``Permitted'' whereas the existence of a single (or more) \syn{NotPermitted} result makes the whole set ``NotPermitted''. When no \syn{Permitted} or \syn{NotPermitted} is seen in the set, we only have \syn{Unregulated} results which would make the whole set ``Unregulated". This is the decision combination strategy we have designed for \ac{ACCPL}; the policy combination strategy \ac{ACCPL} employs simply preserves all individual decisions and returns the whole set for rendering a global decision. Note that the policy combination strategy employed by \ac{ACCPL} is what's referred to as ``Weak-consensus'' strategy according to Li et al.~\cite{LiWQBRLL09}. 

%``Weak-consensus'' policy combining strategy is defined as ``Sub-policies should not conflict with each
%other: Permit a request if some sub-policies permit a request, and
%no sub-policy denies it. Deny a request if some sub-policies deny
%a request, and no sub-policy permits it. Yield a value indicating
%conflict if some permit and some deny.'' by~\cite{LiWQBRLL09}. 


Below we consider the three different possible decisions that the decision combination strategy may reach and review whether the result of any individual policy may change as a result:

\begin{enumerate}
  \item Global decision is \syn{Permitted}. Only possible results are \syn{Permitted} or \syn{Unregulated}. In this case, only the \syn{Unregulated} results have been modified by the decision combination strategy. \syn{Permitted} results coming from any individual primitive policy will have been preserved.
  \item Global decision is \syn{NotPermitted}. Only possible results are \syn{NotPermitted} or \newline \syn{Unregulated}. In this case, similar to the first case, only the \syn{Unregulated} results have been modified by the decision combination strategy. \syn{NotPermitted} results coming from any individual primitive policy will have been preserved.
   \item Global decision is \syn{Unregulated}. Only possible results are \syn{Unregulated}. In this case, results coming from any individual primitive policy will have been preserved.
\end{enumerate}

We argue that \ac{ACCPL} has the independent composition property with respect to granting and denying policies only as detailed in the previous paragraphs. 

\section{Monotonicity of Policy Combinators}

Tschantz and Krishnamurthi~\cite{Tschantz} discuss the notion of monotonicity of policy combinators. Intuitively, in a language with a policy combinator that is monotonic, adding another primitive policy, cannot change the combined decision to go from granting to non-granting. In \ac{ACCPL}, \syn{NotPermitted} results are only possible if \syn{trans_prin} does not hold. Assume the combined (global) decision is already \syn{Permitted} for a certain agreement. In \ac{ACCPL}, \syn{Permitted} results are only possible if \syn{trans_prin} and the \syn{preRequisite} of the policy set both hold. Adding a primitive policy to the agreement will not change the global \syn{Permitted} result to a \syn{NotPermitted} or an \syn{Unregulated} result, since a primitive policy is composed of a \syn{preRequisite} and an \syn{act} only. There are two cases to consider. Either the \syn{preRequisite} of the added policy holds and the local decision for this policy is another \syn{Permitted} result, or the \syn{preRequisite} of the added policy does not hold, and the local decision for this policy is an \syn{Unregulated} result, which won't affect the combined/global decision of \syn{Permitted}.






