\documentclass{beamer}
\usepackage{coqdoc}
\usepackage{color}
\usepackage{listings,lstautogobble}
\usepackage{etoolbox}
\usepackage{fixltx2e}
\usepackage{syntax}
\usepackage{lstcoq}
\usepackage{url}
\usepackage{acro}

\definecolor{maroon}{rgb}{0.5,0,0}
\definecolor{darkgreen}{rgb}{0,0.5,0}

\usepackage[utf8]{inputenc}
\usepackage[T1]{fontenc}
\usepackage[frenchb]{babel}
\usepackage{setspace}
\usepackage{color}
\usepackage{listings}

\usetheme{Copenhagen}
\title{The title of the paper}
\author{Bahman Sistany}
\date{\today}
\setbeamertemplate{frametitle}[default][center]
\newcommand\FontForteen{\fontsize{14}\selectfont}
\newcommand\FontTen{\fontsize{10}\selectfont}
\begin{document}
%% title frame
\begin{frame}
\titlepage
\end{frame}

%% normal frame 1
\begin{frame}{First Slide}
\begin{itemize}
   \item Digital Rights Management (DRM):
   \begin{itemize}
      \item Digital management of rights associated with the access or usage of digital assets.
   \end{itemize}

   \item DRM is Concerned with Four Areas:
     \begin{itemize}
	\item Defining Rights
	\item Distributing/Acquiring Rights
	\item Enforcing Rights
	\item Tracking Usage.
     \end{itemize}
\end{itemize}
\end{frame}
%% normal frame 2
\begin{frame}[fragile]{Second Slide}
\lstset{language=Coq}
\begin{lstlisting}
Inductive agreement : Set :=
  | Agreement : prin -> asset -> policySet -> agreement.

Definition prin := nonemptylist subject.

Definition asset := nat.

Definition subject := nat.

Definition act := nat.

Definition policyId := nat.

Inductive policySet : Set :=
  | PrimitivePolicySet : preRequisite -> policy -> policySet 
  | PrimitiveExclusivePolicySet : preRequisite -> policy  -> policySet 
  | AndPolicySet : nonemptylist policySet -> policySet.

Inductive policy : Set :=
  | PrimitivePolicy : preRequisite -> policyId -> act -> policy 
  | AndPolicy : nonemptylist policy -> policy.

\end{lstlisting}
\end{frame}
%% slide 3
\begin{frame}[fragile]{Defining Rights}
%% \tiny \small \normalsize \large \Large \LARGE \huge \Huge
\LARGE
\begin{itemize}
\item Rights Expressions
   \begin{itemize}
   \item Anyone who pays 5\$ may watch the movie ``Terminator''
   \end{itemize}
\item Rights are Expressed Using Rights Expression Languages (REL)
\item Examples are ODRL and XrML
\end{itemize}
\end{frame}
%% slide 4
\begin{frame}[fragile]{RELs}
\LARGE
\begin{itemize}
\item Both XrML and ODRL are XML based and declarative
\item Adoption and Usage is Still Limited
   \begin{itemize}
   \item Due to Complexity 
   \item And the fact that they try to cover Enforcement and Tracking Aspects of DRM
   \item Instead of focusing on Expressing Rights
   \end{itemize}
\end{itemize}
\end{frame}
%% slide 5
\begin{frame}[fragile]{ODRL and Rights Expressions: Syntax}
\end{frame}
%% slide 6
\begin{frame}[fragile]{ODRL and Rights Expressions: Semantics}
\end{frame}
%% slide 7
\begin{frame}[fragile]{ODRL and Rights Expressions: Formal Semantics}
\end{frame}
%% slide 8
\begin{frame}[fragile]{A Formal Foundation for ODRL: \newline Riccardo Pucella and Vicky Weissman}
\end{frame}
%% slide 9
\begin{frame}[fragile]{Paper-Proofs}
\end{frame}
%% slide 10
\begin{frame}[fragile]{ODRL0}
\end{frame}
%% slide 11
\begin{frame}[fragile]{Coq: Why Coq?}
\end{frame}
%% slide 12
\begin{frame}[fragile]{Queries}
\end{frame}
%% slide 13
\begin{frame}[fragile]{Closed World Assumption: CWA}
\end{frame}
%% slide 14
\begin{frame}[fragile]{ACPL: A Core Policy Language}
\end{frame}
%% slide 15
\begin{frame}[fragile]{Defining Rights}
\end{frame}
%% slide 16
\begin{frame}[fragile]{Defining Rights}
\end{frame}
%% slide 17
\begin{frame}[fragile]{Defining Rights}
\end{frame}
%% slide 18
\begin{frame}[fragile]{Defining Rights}
\end{frame}
%% slide 19
\begin{frame}[fragile]{Defining Rights}
\end{frame}
%% slide 20
\begin{frame}[fragile]{Defining Rights}
\end{frame}
%% slide 21
\begin{frame}[fragile]{Conclusion}
\end{frame}
%% slide 22
\begin{frame}[fragile]{Questions?}
\end{frame}






























\end{document}
