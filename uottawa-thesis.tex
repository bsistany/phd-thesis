% uOttawa (unofficial) Thesis Template for LaTeX 
% Edited by Wail Gueaieb based on Stephen Carr's uWaterloo Template

% DON'T USE THIS TEMPLATE IF YOU DON'T KNOW WHAT YOU'RE DOING!
% Remember, it comes WITH NO WARRANTY!

% Please read the "00readme.txt" file first.
% Here is how to use this template:
%
% DON'T FORGET TO ADD YOUR OWN NAME AND TITLE in the "hyperref" package
% configuration in the "thesis-preample.tex" file. THIS INFORMATION GETS 
% EMBEDDED IN THE PDF FINAL PDF DOCUMENT.
% You can view the information if you view Properties of the PDF document.

% The template is based on the standard "book" document class which provides 
% all necessary sectioning structures and allows multi-part theses.

% DISCLAIMER
% To the best of our knowledge, this template satisfies the current 
% uOttawa thesis requirements.
% However, it is your responsibility to assure that you have met all 
% requirements of the university and your particular department.
% Many thanks to the feedback from many graduates that assisted the 
% development of this template.

% -----------------------------------------------------------------------

% When using pdflatex, by default the output is geared toward generating a PDF 
% version optimized for viewing on an electronic display, including 
% hyperlinks within the PDF.
 
% E.g. to process a thesis based on this template, run:

% (pdf)latex uottawa-thesis	-- first pass of the (pdf)latex processor
% bibtex uottawa-thesis 	-- generates bibliography from .bib data file(s) 
% (pdf)latex uottawa-thesis	-- fixes cross-references, bibliographic references, etc
% (pdf)latex uottawa-thesis	-- fixes cross-references, bibliographic references, etc
% makeindex -s nomentbl.ist -o uottawa-thesis.nls uottawa-thesis.nlo
% (pdf)latex uottawa-thesis	-- fixes cross-references, bibliographic references, etc
% (pdf)latex uottawa-thesis	-- fixes cross-references, bibliographic references, etc



% N.B. The "pdftex" program allows graphics in the following formats to be
% included with the "\includegraphics" command: PNG, PDF, JPEG, TIFF
% Tip 1: Generate your figures and photos in the size you want them to appear
% in your thesis, rather than scaling them with \includegraphics options.
% Tip 2: Any drawings you do should be in scalable vector graphic formats:
% SVG, PNG, WMF, EPS and then converted to PNG or PDF, so they are scalable in
% the final PDF as well.
% Tip 3: Photographs should be cropped and compressed so as not to be too large.

% To create a PDF output that is optimized for double-sided printing: 
%
% 1) comment-out the \documentclass statement in the preamble below, and
% un-comment the second \documentclass line.
%
% 2) change the value assigned below to the boolean variable
% "PrintVersion" from "false" to "true".
\newcommand{\Title}{A Certified Core Policy Language}
\newcommand{\Author}{Bahman Sistany}
% Acknowledgements (small sentence)
\newcommand{\SmallAcknowledgements}[1]{
    %\vspace*{5.4cm}                             % Skip some vertical space
    \begin{flushright}{#1}\end{flushright}      % Text right justified

    \thispagestyle{empty}                       % Disable the page number for this page only
    \newpage                                    % Add a new page
}

% Title page
\newcommand{\CoverPage}{

    \begin{titlepage}
    \begin{center}

        \vfill
        \begin{LARGE}\textbf{{\Title}}\end{LARGE}

        \vfill
        \begin{large}{\Author}\end{large}

        \vfill
        Thesis submitted to the Faculty of Graduate and Postdoctoral Studies \\
        In partial fulfillment of the requirements for the degree of Doctor of Philosophy in \\
        Computer Science

        \vfill
        Ottawa-Carleton Institute for Computer Science \\
        School of Information Technology and Engineering \\
        University of Ottawa, \\
        Ottawa, Canada

        \vfill
        \textcopyright~\Author, Ottawa, Canada, \the\year
    \end{center}
    \end{titlepage}


    \newpage                                    % Add a new page

    %\SectionStylePre
}

% --------------------- Start of Document Preamble -----------------------

% Specify the document class, default style attributes, and page dimensions
% For hyperlinked PDF, suitable for viewing on a computer, use this:
\documentclass[letterpaper,12pt,titlepage,oneside,final]{book}

\usepackage{acro}
%\usepackage{sectsty}% http://ctan.org/pkg/sectsty
%\usepackage{titlecaps}% http://ctan.org/pkg/titlecaps

% For PDF, suitable for double-sided printing, change the PrintVersion variable below
% to "true" and use this \documentclass line instead of the one above:
% \documentclass[letterpaper,12pt,titlepage,openright,twoside,final]{book}


% This package allows if-then-else control structures.
\usepackage{ifthen}
\newboolean{PrintVersion}
%\setboolean{PrintVersion}{false} 
% \setboolean{PrintVersion}{true} 
% CHANGE THIS VALUE TO "true" as necessary, to improve printed results 
% for hard copies by overriding some options of the hyperref package.


% Load your needed packages and other commands of yours.

% Load your needed packages and other commands of yours here:
%\usepackage{} % ... note that old .sty files can be included here
\usepackage[]{inputenc}
\usepackage[T1]{fontenc}
\usepackage{fullpage}
\usepackage{breakcites}
\usepackage{cite}

\usepackage{coqdoc}
\usepackage{color}
\usepackage{listings,lstautogobble}
\usepackage{etoolbox}
\usepackage{fixltx2e}
\usepackage{syntax}
\usepackage{lstcoq}
\usepackage{url}


\definecolor{maroon}{rgb}{0.5,0,0}
\definecolor{darkgreen}{rgb}{0,0.5,0}
\lstdefinelanguage{XML}
{
  basicstyle=\ttfamily,
  frame=single,
  breaklines=true,
  morestring=[s]{"}{"},
  morecomment=[s]{?}{?},
  morecomment=[s]{!--}{--},
  commentstyle=\color{black},
  moredelim=[s][\color{black}]{>}{<},
  moredelim=[s][\color{black}]{\ }{=},
  stringstyle=\color{black},
  identifierstyle=\color{black},
  autogobble=true
}
\lstdefinelanguage{Pucella2006}
{
  morekeywords={
    agreement, prin, asset, with, prePay, and, display, print, count
  }
}
\lstdefinelanguage{selinux}
{
  morekeywords={
    type, types, attrib, role, allow, user, constrain, avkind, 
	sourcetype, targettype, object-class, perm, allow, auditallow, dontaudit, neverallow,
	constrain, classes, perms, sourcetype, sourcerole, sourceuser, targettype, targetrole, targetuser
  }
}

\lstdefinelanguage{AST}
{
  basicstyle=\ttfamily,
  breaklines=true,
  morekeywords={
    agreement, prin, asset, subject, policySet, policy, act, policyId, preRequisite 
  }
  tabsize=1,
}

\newtheorem{innercustomthm}{Theorem}
\newenvironment{customthm}[1]
  {\renewcommand\theinnercustomthm{#1}\innercustomthm}
  {\endinnercustomthm}












%--------------------------------------------------------------------------
% Do NOT edit the rest of the preample UNLESS YOU KNOW WHAT YOU'RE DOING!
%--------------------------------------------------------------------------

\ifthenelse{\boolean{PrintVersion}}{
\usepackage[top=1in,bottom=1in,left=0.75in,right=1.25in]{geometry}   % For twoside document
}{
\usepackage[top=1in,bottom=1in,left=0.75in,right=1.25in]{geometry}   % For oneside document
}

\usepackage{amsmath,amssymb,amstext} % Lots of math symbols and environments
\usepackage{graphicx} % For including graphics 

\usepackage{nomentbl} 
\makenomenclature 

\usepackage{ifpdf}

\setcounter{secnumdepth}{4}
\setcounter{tocdepth}{4}

%\newcommand{\href}[1]{#1} % does nothing, but defines the command so the
    % print-optimized version will ignore \href tags (redefined by hyperref pkg).
%\newcommand{\texorpdfstring}[2]{#1} % does nothing, but defines the command
% Anything defined here may be redefined by packages added below...


% Hyperlinks make it very easy to navigate an electronic document.
% In addition, this is where you should specify the thesis title
% and author as they appear in the properties of the PDF document.
% Use the "hyperref" package 
% N.B. HYPERREF MUST BE THE LAST PACKAGE LOADED; ADD ADDITIONAL PKGS ABOVE
%\usepackage[\ifpdf pdftex,\fi letterpaper=true,pagebackref=false]{hyperref} % with basic options
		% N.B. pagebackref=true provides links back from the References to the body text. This can cause trouble for printing.
%\hypersetup{
%    plainpages=false,       % needed if Roman numbers in frontpages
%    pdfpagelabels=true,     % adds page number as label in Acrobat's page count
%    bookmarks=true,         % show bookmarks bar?
%    unicode=false,          % non-Latin characters in Acrobat's bookmarks
%    pdftoolbar=true,        % show Acrobat's toolbar?
%    pdfmenubar=true,        % show Acrobat's menu?
%    pdffitwindow=false,     % window fit to page when opened
%    pdfstartview={FitH},    % fits the width of the page to the window
%%    pdftitle={uOttawa\ LaTeX\ Thesis\ Template},    % title: CHANGE THIS TEXT!
%%    pdfauthor={Author},    % author: CHANGE THIS TEXT! and uncomment this line
%%    pdfsubject={Subject},  % subject: CHANGE THIS TEXT! and uncomment this line
%%    pdfkeywords={keyword1} {key2} {key3}, % list of keywords, and uncomment this line if desired
%    pdfnewwindow=true,      % links in new window
%    colorlinks=true,        % false: boxed links; true: colored links
%    linkcolor=blue,         % color of internal links
%    citecolor=green,        % color of links to bibliography
%    filecolor=magenta,      % color of file links
%    urlcolor=cyan           % color of external links
%}

%\ifthenelse{\boolean{PrintVersion}}{   % for improved print quality, change some hyperref options
%\hypersetup{	% override some previously defined hyperref options
%%    colorlinks,%
%    citecolor=black,%
%    filecolor=black,%
%    linkcolor=black,%
%    urlcolor=black}
%}{} % end of ifthenelse (no else)


\newcommand{\notimplies}{%
  \mathrel{{\ooalign{\hidewidth$\not\phantom{=}$\hidewidth\cr$\implies$}}}}

\DeclareAcronym{drm}
{
  short = DRM ,
  long  = Digital Rights Management ,
  class = abbrev
}

\DeclareAcronym{odrl}
{
  short = ODRL ,
  long  = Open Digital Rights Language ,
  class = abbrev
}

\DeclareAcronym{xrml}
{
  short = XrML ,
  long  = eXtensible rights Markup Language ,
  class = abbrev
}
\DeclareAcronym{rel}
{
  short = REL ,
  long  = Rights Expression Languages ,
  class = abbrev
}
\DeclareAcronym{drel}
{
  short = DREL ,
  long  = Digital Rights Expression Languages ,
  class = abbrev
}
\DeclareAcronym{cic}
{
  short = CIC ,
  long  = Calculus of (Co)Inductive Constructions ,
  class = abbrev
}
\DeclareAcronym{coc}
{
  short = CoC ,
  long  = Calculus of Constructions ,
  class = abbrev
}
\DeclareAcronym{xacml}
{
  short = XACML ,
  long  = eXtensible Access Control Markup Language ,
  class = abbrev
}
\DeclareAcronym{selinux}
{
  short = SELinux ,
  long  = Security Enhanced Linux ,
  class = abbrev
}
\DeclareAcronym{DAC}
{
  short = DAC ,
  long  = Discretionary access control ,
  class = abbrev
}
\DeclareAcronym{MAC}
{
  short = MAC ,
  long  = Mandatory access control ,
  class = abbrev
}
\DeclareAcronym{RBAC}
{
  short = RBAC ,
  long  = role-based access control ,
  class = abbrev
}
\DeclareAcronym{rhs}
{
  short = rhs ,
  long  = right hand side ,
  class = abbrev
}
\DeclareAcronym{lhs}
{
  short = lhs ,
  long  = left hand side ,
  class = abbrev
}




% This is where thesis margins and spaces are set.
% Setting up the page margins...
% A minimum of 1 inch (72pt) margin at the
% top, bottom, and outside page edges and a 1.125 in. (81pt) gutter
% margin (on binding side). While this is not an issue for electronic
% viewing, a PDF may be printed, and so we have the same page layout for
% both printed and electronic versions, we leave the gutter margin in.
% Set margins:
\setlength{\marginparwidth}{0pt} % width of margin notes
% N.B. If margin notes are used, you must adjust \textwidth, \marginparwidth
% and \marginparsep so that the space left between the margin notes and page
% edge is less than 15 mm (0.6 in.)
\setlength{\marginparsep}{0pt} % width of space between body text and margin notes
\setlength{\evensidemargin}{0.125in} % Adds 1/8 in. to binding side of all 
% even-numbered pages when the "twoside" printing option is selected
\setlength{\oddsidemargin}{0.125in} % Adds 1/8 in. to the left of all pages
% when "oneside" printing is selected, and to the left of all odd-numbered
% pages when "twoside" printing is selected
\setlength{\textwidth}{6.375in} % assuming US letter paper (8.5 in. x 11 in.) and 
% side margins as above
\raggedbottom

% The following statement specifies the amount of space between
% paragraphs. Other reasonable specifications are \bigskipamount and \smallskipamount.
\setlength{\parskip}{\medskipamount}

% The following statement controls the line spacing.  The default
% spacing corresponds to good typographic conventions and only slight
% changes (e.g., perhaps "1.2"), if any, should be made.
\renewcommand{\baselinestretch}{1} % this is the default line space setting

% By default, each chapter will start on a recto (right-hand side)
% page.  We also force each section of the front pages to start on 
% a recto page by inserting \cleardoublepage commands.
% In many cases, this will require that the verso page be
% blank and, while it should be counted, a page number should not be
% printed.  The following statements ensure a page number is not
% printed on an otherwise blank verso page.
\let\origdoublepage\cleardoublepage
\newcommand{\clearemptydoublepage}{%
  \clearpage{\pagestyle{empty}\origdoublepage}}
\let\cleardoublepage\clearemptydoublepage




%======================================================================
%   L O G I C A L    D O C U M E N T -- the content of your thesis
%======================================================================
\begin{document}

% For a large document, it is a good idea to divide your thesis
% into several files, each one containing one chapter.
% To illustrate this idea, the "front pages" (i.e., title page,
% declaration, borrowers' page, abstract, acknowledgements,
% dedication, table of contents, list of tables, list of figures,
% nomenclature).
%----------------------------------------------------------------------
% FRONT MATERIAL
%----------------------------------------------------------------------
%
% C O V E R  P A G E
\CoverPage

\SmallAcknowledgements{
Acknowledgements that\\
appear on the right.
}

% ------------------
%\newcommand{\thesisauthor}{Bahman Sistany}
%\newcommand{\thesisadvisor}{Amy Felty}
%\newcommand{\thesistitlecoverpage}{%
%  A Certified Core Policy Language
%}
%\newcommand{\degree}{Ph.D.} % possible values are:
%                            
%\newcommand{\nameofprogram}{Computer Science}
%\newcommand{\academicunit}{School of Electrical Engineering and Computer Science}
%\newcommand{\faculty}{Faculty of Engineering}
%\newcommand{\graduationyear}{2015}
%

%% T I T L E   P A G E
% -------------------
% Last updated May 24, 2011, by Stephen Carr, IST-Client Services
% The title page is counted as page `i' but we need to suppress the
% page number.  We also don't want any headers or footers.
\pagestyle{empty}
\pagenumbering{roman}

% The contents of the title page are specified in the "titlepage"
% environment.
\begin{titlepage}
        \begin{center}
        \vspace*{1.0cm}

        \Huge
        {\bf \thesistitlecoverpage }

        \vspace*{1.0cm}

        \normalsize
        by \\

        \vspace*{1.0cm}

        \Large
        \thesisauthor \\

        \vspace*{3.0cm}

        \normalsize
        \degree Thesis Proposal 
        For the \degree~degree in\\
        \nameofprogram\\

        \vspace*{2.0cm}

        \academicunit\\
        \faculty\\
        University of Ottawa\\

        \vspace*{1.0cm}
        %\copyright~\thesisauthor, Ottawa, Canada, \graduationyear\\
        \degree Thesis Advisor: ~\thesisadvisor
        \end{center}
\end{titlepage}

% The rest of the front pages should contain no headers and be numbered using Roman numerals starting with `ii'
\pagestyle{plain}
\setcounter{page}{2}

\cleardoublepage % Ends the current page and causes all figures and tables that have so far appeared in the input to be printed.
% In a two-sided printing style, it also makes the next page a right-hand (odd-numbered) page, producing a blank page if necessary.

%
%
% R E S T  O F  F R O N T  P A G E S
% ----------------------------------
% % D E C L A R A T I O N   P A G E
% -------------------------------
  % This page is not needed for a uOttawa thesis. Don't include it.
  % It is designed for an electronic thesis.
  \noindent
I hereby declare that I am the sole author of this thesis. This is a true copy of the thesis, including any required final revisions, as accepted by my examiners.

  \bigskip
  
  \noindent
I understand that my thesis may be made electronically available to the public.

\cleardoublepage
%\newpage
 %This is not needed in a uOttawa thesis.
%
% Edit the following 3 files with your abstract, acknowledgements, 
% and dedication.
% A B S T R A C T
% ---------------

\begin{center}\textbf{Abstract}\end{center}

We present the design and implementation of a \ac{ACCPL} that can be used to express access-control rules and policies. Although full-blown access-control policy languages such as eXtensible Access Control Markup Language \\(XACML)~\cite{xacml} already exist, because access rules in such languages are often expressed in a declarative manner using fragments of a natural language like English, it isn't always clear what the intended behaviour of the system encoded in these access rules should be. To remedy this ambiguity, formal specification of how an access-control mechanism should behave, is typically given in some sort of logic, often a subset of first order logic. To show that an access-control system actually behaves correctly with respect to its specification, proofs are needed, however the proofs that are often presented in the literature are hard or impossible to formally verify. The verification difficulty is partly due to the fact that the language used to do the proofs while mathematical in nature, utilizes intuitive justifications to derive the proofs. Intuitive language in proofs means that the proofs could be incomplete and/or contain subtle errors.

\ac{ACCPL} is small by design. By small we refer to the size of the language; the syntax, auxiliary definitions and the semantics of \ac{ACCPL} only take a few pages to describe. This compactness allows us to concentrate on the main goal of this thesis which is the ability to reason about the policies written in \ac{ACCPL} with respect to specific questions. By making the language compact, we have stayed away from completeness and expressive power in several directions. For example, \ac{ACCPL} uses only a single policy combinator, the conjunction policy combinator. The design of \ac{ACCPL} is therefore a trade-off between ease of formal proof of correctness and expressive power.

We also consider \ac{ACCPL} a core policy access-control language since we have retained the core features of many access-control policy languages. For instance \ac{ACCPL} employs a single condition of type ``requirement'' where other languages may have very expressive and rich sets of requirements. 

In this thesis, we describe the design of \ac{ACCPL} and its implementation in the Coq Proof Assistant. Using Coq to implement \ac{ACCPL} as it was being defined was an important factor in its design, allowing us to address the trade-off between expressive power and ease of formal proof of correctness. We argue that \ac{ACCPL} is significantly more amenable to analysis and reasoning than other access-control policy languages, while still retaining sufficient expressive power, and we describe how the design choices that were made contributed to this ease of reasoning. 

The semantics of \ac{ACCPL} are specified by translation from policy statements together with an access request and an environment containing all relevant facts, to decisions. We present decidability results for the decision problem that asks whether a request to access a resource may be granted or denied, given a policy. The translation functions also cover the case where a given policy does not apply to a request in which case a decision of ``request is unregulated'' is rendered. We show that the translation algorithm terminates on all input policies with a decision of granted, denied or unregulated. Furthermore, we show that the translation algorithm meets its specification and computes the correct decision, with respect to its specification, given a policy. 

\cleardoublepage
%\newpage


%% A C K N O W L E D G E M E N T S
% -------------------------------

\begin{center}\textbf{Acknowledgements}\end{center}




\cleardoublepage
%\newpage
%% D E D I C A T I O N
% -------------------

\begin{center}\textbf{Dedication}\end{center}

This thesis is dedicated to my wife Sanam who stood by me and kept me going when life threatened to get in the way, and to my wonderful sons Roitan and Rod Barzin for their sacrifice and patience during the course of my PhD studies. 

I also dedicate this work to my parents, Shamseddin and Robab, who first taught me the value of education and for their continued encouragement and support throughout my studies.


\cleardoublepage
%\newpage

%
%
% No need to edit this file.
% T A B L E   O F   C O N T E N T S
% ---------------------------------
\renewcommand\contentsname{Table of Contents}
\tableofcontents
\cleardoublepage
\phantomsection
%\newpage

% L I S T   O F   T A B L E S
% ---------------------------
%\addcontentsline{toc}{chapter}{List of Tables}
%\listoftables
%\cleardoublepage
%\phantomsection		% allows hyperref to link to the correct page
%\newpage

% L I S T   O F   F I G U R E S
% -----------------------------
%\addcontentsline{toc}{chapter}{List of Figures}
%\listoffigures
%\cleardoublepage
%\phantomsection		% allows hyperref to link to the correct page
%\newpage

% L I S T   O F   LISTINGS
% -----------------------------
\addcontentsline{toc}{chapter}{List of Listings}
\lstlistoflistings
\cleardoublepage
\phantomsection		% allows hyperref to link to the correct page
\newpage

%
% No need to edit this file. But you may want to comment the whole line if you
% don't have or want a Nomenclature section.
%% L I S T   O F   S Y M B O L S
% -----------------------------
% To include a Nomenclature section
\addcontentsline{toc}{chapter}{\textbf{Nomenclature}}

\renewcommand{\nomname}{Nomenclature}
\renewcommand{\nomAname}{\textbf{\large Abbreviations}}
\renewcommand{\nomGname}{\textbf{\large Mathematical Symbols}}
\renewcommand{\nomXname}{\textbf{\large Superscripts}}
\renewcommand{\nomZname}{\textbf{\large Subscripts}}

\printnomenclature
\cleardoublepage
\phantomsection % allows hyperref to link to the correct page
% \newpage






%%% Local Variables: 
%%% mode: latex
%%% TeX-master: "../uottawa-thesis"
%%% End:   


% Change page numbering back to Arabic numerals
%\pagenumbering{arabic}
%\let\svsection\section
%\def\section#1{\svsection{\titlecap{#1}}}
%\sectionfont{\MakeUppercase}
%----------------------------------------------------------------------
% MAIN BODY
%---------------------------------------------------------------------- 
% Chapters 
% Include your "sub" source files here (must have extension .tex)
%======================================================================
\chapter{Introduction}
%======================================================================

\section{A Core Policy Language for Access-Control} 

Does there exist a suitable policy language that can be used both for expressing general access-control and digital rights expressions? How about \ac{xacml}, \ac{odrl}~\cite{odrloneone} or \ac{xrml}~\cite{Wang}? All are well known, full-blown and custom languages that suffer from a lack of formal semantics. Furthermore all of these languages cover much more than policy expressions leading to access decisions; they also address enforcement of the policies (\ac{odrl} and \ac{xrml} specifically and \ac{drm} in general distinguish themselves from general access-control languages by additionally addressing enforcement of policies beyond where the policies were generated. The term usage-control (UCON) is used to describe this aspect of \ac{rel}s). A third reason that made these custom languages unsuitable as a core policy language was the fact that they are limited in terms of what can be built on top of them; for example expressing hierarchical role-based access-control  in \ac{xacml} requires a fairly complex encoding~\cite{Tschantz}.

A policy language that was based on some sort of logic with formal semantics and also one that was minimal and extendible was clearly needed. We started by looking at Lithium~\cite{Halpern2008} and subsequently Pucella et al`s subset of \ac{odrl}~\cite{pucella2006} as potential core languages. 

\section{Access-Control Models~\cite{nist}}
Authorization refers to the process of rendering a decision to allow access to a resource or asset of interest. By the same token all unauthorized access requests to resources must be controlled and ultimately denied, hence the term ``access-control''. 

Various access-control models exist and below we give a short summary of some of the most important ones leading up to our own core language, \ac{ACCPL}.

\subsection{~\ac{acl}}

Access Control Lists are perhaps the oldest and the most basic access-control model. A list of subjects along with their rights are kept per resource or object of interest. Every time a subject makes an access request on an object, the \ac{acl} of the object is consulted and access is either granted or denied to the requester based on whether the use is listed in the \ac{acl} and has the correct set of rights.

\subsection{Capabilities}
Capabilities based access-control works based on a list of objects and some associated rights. The list of objects and the associated rights comprise an ``unforgeable'' ticket that a reference monitor checks to allow access (or not). Capabilities based systems don't need to authenticate users as \ac{acl} based systems do.

\subsection{\ac{rbac}}

In \ac{rbac} systems a requester`s role determines whether access is granted to denied. In this model users belong to roles and rights are associated with roles so no direct association between users and rights exist. Roles are meant to group users and add flexibility when assigning rights. \ac{rbac} systems naturally solve the problem of assigning \ac{acl}`s for a large group of users and manage the administration cost of changing users' rights in \ac{acl} based systems.

\subsection{\ac{abac}}

Despite many advantages of \ac{rbac} systems, some disadvantages also exist. Often a role needs to be decomposed into sub-roles based on the type of resource to be administered, and perhaps also based on the location that the resource serves~\cite{nist}. Basically \ac{rbac} suffers from a lack of a sub-typing mechanism whereby individual members of a group/role may be differentiated and access is granted or denied based on a more granular set of attributes. \ac{abac} model was proposed to fulfill this granularity requirement. In \ac{abac} access control decisions are made based on a set of attributes, associated with the subject making the request, the environment, and/or the resource itself~\cite{nist}. 

\subsection{\ac{pbac}}
In order to harmonize access control in large environments with many subjects and objects and disparate attributes, \ac{pbac} model has been proposed~\cite{nist}. \ac{pbac} allows for a more uniform access-control model across the system. \ac{pbac} systems help create and enforce policies
that define who should have access to what resources, and under what circumstances~\cite{nist}.
There is also a need for large organizations to put in place mechanisms such that access-control rules can be easily audited. This calls for a data-driven approach to access-control where the data, in this case the access-control rules, is available to read and analyze.

\section{Digital Rights} 

\ac{drm} refers to the digital management of rights associated with the access or usage of digital assets. There are various aspects of rights management however. According to the authors of the white paper ``A digital rights management ecosystem model for the education community,''(~\cite{collier})
 digital rights management systems cover the following four areas: 1) defining rights 2) distributing/acquiring rights 3) enforcing rights and 4) tracking usage.

\section{Policy Expression Languages}

\ac{rel}s, or more precisely \ac{drel}s when dealing with digital assets deal with the ``rights definition'' aspect of the \ac{drm} ecosystem. A \ac{drel}, allows the expression and definition of digital asset usage rights such that other areas of the \ac{drm} ecosystem, namely the enforcement mechanism and the usage tracking components can function correctly.

Currently the most popular \ac{rel}s are the \ac{xrml}, and the \ac{odrl}. Both of these languages are XML based and are considered declarative languages. \ac{xrml} has been selected to be the REL for \emph{MPEG-21} which is an ISO standard for multimedia applications. \ac{odrl} is also a standards based \ac{rel} which has been accepted as part of the W3C community with the mandate of standardizing how rights and policies, related to the usage of digital content on the Open Web Platform, \emph{OWP}~\cite{openwebplatform}, are expressed. \ac{odrl} 2.0 supports expression of rights and also privacy rules for social media while \ac{odrl} 1.0 was only dealing with the mobile ecosystem -- \ac{odrl} 1.0 was adopted by the Open Mobile Alliance, \emph{OMA} in 2000.

As popular as both \ac{xrml} and \ac{odrl} are, their adoption and usage is still somewhat limited in practice. Both Apple and Microsoft for example have defined their own lightweight \ac{rel}s~\cite{problemwithrels} in \emph{Fair Play}~\cite{fairplay} and in \emph{PlayReady}~\cite{playready}. The authors of~\cite{problemwithrels} argue that both these \ac{rel}s (\ac{xrml} and \ac{odrl}) and others are simply too complex to be used effectively (for expressing rights) since they also try to cover much of the the enforcement and tracking aspects of \ac{drm}s.

Rights expressions in \ac{drel}s and specifically \ac{odrl} are used to arbitrate access to assets under conditions. The main construct in \ac{odrl} is the \emph{agreement} which specifies users, asset(s) and policies whereby controls on users' access to the assets are described. This is very similar to how access control conditions are expressed in access control policy languages such as \ac{xacml}~\cite{xacml} and \ac{selinux}~\cite{selinux}. In fact several authors have worked on interoperability between \ac{rel}s and access control policy languages, specifically between \ac{odrl} and \ac{xacml},~\cite{prados2005interoperability, maronas2009architecture} and also on translation from high level policies of \ac{xacml} to low-level and fine grained policies of \ac{selinux}~\cite{alam2008usage}. 

In this thesis we will be generalizing the concept of a policy language from strictly representing subsets of \ac{odrl}, to representing (subsets of) both \ac{odrl} and \ac{selinux} policy languages. We will accomplish this by adapting the \ac{odrl} policy language syntax and semantics in Coq to be used for \ac{selinux}. The final goal is still performing formal verification on both types of policies, in a unified manner.



%----------------------------------------------------------------------
\section{Semantics Of ODRL Policies}
%----------------------------------------------------------------------


Formal methods help ensure that a system behaves correctly with respect to a specification of its desired behavior~\cite{TAPL}. This specification of the desired behavior is what's referred to as \emph{semantics} of the system. Using formal methods requires defining precise and formal semantics, without which analysis and reasoning about properties of the system in question would become impossible. For example, an issue with the current batch of \ac{rel}s are due to their semantics being expressed in a natural language (e.g. English) which by necessity results in ambiguous and open to interpretation behavior. 

To formalize the semantics of \ac{odrl} several approaches have been attempted by various authors. Most are logic based~\cite{Halpern2008, pucella2006} while others are based on finite-automata~\cite{Holzer}, operational semantics based interpreters~\cite{Safavi-naini} and web ontology (from the Knowledge Representation Field)~\cite{Kasten2010MTS}. In this thesis we will focus on the logic based approach to formalizing semantics and will study a specific logic based language that is essentially a subset of \ac{odrl}.



\section{Logic Based Semantics}

%See equation \ref{eqn_pi} on page \pageref{eqn_pi}.\footnote{A famous %equation.}

Formal logic can represent the statements and facts we express in a natural language like English. Propositional logic is expressive enough to express simple facts as propositions and uses connectives to allow for the negation, conjunction and disjunction of the facts. In addition simple facts can be expressed conditionally using the implication connective. Propositional logic however is not expressive enough to express policies of the kind used in languages like \ac{odrl} and \ac{xrml}. For example, a simple policy expressed in English like ``All who pay 5 dollars can watch the movie Toy Story'' cannot be expressed in propositional logic because the concept of  variables doesn't exist in propositional logic. 

A richer logic such as ``Predicate Logic'', also called ``First Order Logic'' (\emph{FOL}), is more suitable and has the expressive power to represent policies written in English. Moreover, FOL can be used to capture the meaning of policies in an unambiguous way.

Halpern and Weissman~\cite{Halpern2008} propose a fragment of FOL to represent and reason about policies. The fragment of FOL they arrive at is called \emph{Lithium} which is decidable and allows for efficiently answering interesting queries. Lithium restricts policies to be written based on the concept of ``bipolarity'' which disallows by construction policies that both permit and deny an action on an object. Pucella and Weissman~\cite{pucella2006} specify a predicate logic based language that represents a subset of ODRL.


\section{Specific Problem}

Policy languages and the agreements written in those languages are meant to implement specific goals such as limiting access to specific assets. The tension in designing a policy language is usually between how to make the language expressive enough, such that the design goals for the policy language may be expressed, and how to make the policies verifiable with respect to the stated goals.

As stated earlier, an important part of fulfilling the verifiability goal is to have formal semantics defined for policy languages. For \ac{odrl}, authors of~\cite{pucella2006} have defined a formal semantics based on which they declare and prove a number of important theorems (their main focus is on stating and proving algorithm complexity results). However as with many paper-proofs, the language used to do the proofs while mathematical in nature, uses a lot of intuitive justifications to show the proofs. As such these proofs are difficult to verify or more importantly to ``derive''. Furthermore the proofs can not be used directly to render a decision on a sample policy (e.g. whether to allow or deny access to an asset). Of course one may (carefully) construct a program based on these proofs for practical purposes but we will have no way of certifying those programs correct, even assuming the original proofs were in fact correct.

While there are paper-proofs for various properties of \ac{odrl} (e.g. the ones in~\cite{pucella2006}), proofs for similar properties do not exist for an important (mandatory) access-control policy system, namely \ac{selinux}. In particular, as far as we know, no formal proofs (paper based or otherwise) of decidablity of answering access control queries given an \ac{selinux} policy exists in the literature. 


\section{Contributions}

In this thesis we will build a language representation framework based on \ac{odrl} and definitions in~\cite{pucella2006}. The framework will be formalized in \emph{Coq}~\cite{BC04} which is both a programming language and a proof-assistant. We will declare and prove decidablity results of subsets of \ac{odrl} all the way up to the complete \ac{odrl} fragment defined in~\cite{pucella2006}. We will extract programs from the proofs and demonstrate how they can be used on specific policies to render a specific decision such as ``a conflict has been detected''. 

The reason we start with a specific subset of \ac{odrl} (see ~\ref{sec:odrl0}) is so that we can concentrate on what believe to be the essence of the language. For example, we will only consider one of three different kinds of ``facts'' (that may affect the permit/deny type decisions) conjecturing that adding the remaining kinds of facts later will not change the proofs significantly.

Beside ``certified decidablity results'' for \ac{odrl}, we will investigate decidablity for \ac{selinux} policies, proving decidablity or show why a proof is not possible (if that is the case) and provide proposals to make the policy language decidable.

Decidablity results will subsume an important sub-category, namely inconsistency or conflict-detection in policy expressions or rules. Authors of~\cite{st2012verified}~\cite{felty13} describe and implement in Coq a conflict detection algorithm for detecting conflicts in \ac{xacml} access control rules. \ac{xacml} is an expressive and at the same time complex policy language which makes conflict detection a difficult task. Authors of~\cite{st2012verified} then prove the conflict detection algorithm correct (or certified) by developing a formal proof in Coq. The proof is rather complex and involves a large number of cases, including many
corner cases that were difficult to get right~\cite{st2012verified}. 

By using the Coq framework that we are building for \ac{odrl} to encode and verify agreements written in a second policy language (and different class of policy language: \ac{rel} vs access-control) we will demonstrate the suitability of this Coq based framework for other policy languages such as \ac{xacml}. 

This Coq based language framework could also be used to implement and reason about interoperability~\cite{prados2005interoperability, maronas2009architecture} between various policy languages. In this manner our Coq based language (including possible modifications and enhancements due to supporting \ac{selinux}) can be viewed as \emph{abstract syntax}, complete with defined semantics, that can be used for implementing various policy languages with more concrete syntax (e.g. W3C's \ac{odrl} and \ac{selinux}). 

\section{Work Accomplished So Far}

The encodings for a subset of \ac{odrl} which we call ODRL0 (see ~\ref{sec:odrl0}) plus some important functions implementing some of the algorithms in ~\cite{pucella2006} have been implemented in Coq. Some of the intermediate theorems have been also been defined and proved.

\section{Work Remaining}
The main decidablity result and its proof for ODRL0 will be completed first. We will add the remaining \ac{odrl} constructs incrementally while maintaining decidablity for the main decision algorithm. The remaining constructs include a trouble-some construct (~\cite{pucella2006}), namely $not[policySet]$. We will show this construct does not change the decidablity result already established. 

ODRL0 is enough to be used as a basis for \ac{selinux} policies without \emph{constrain}s. \ac{selinux} constrains are extra  conditions that need to be satisfied (in addition to policies) in order for a permission to be granted. We will investigate decidablity for this subset first. We will then add constrains to the ODRL0 subset (as pre-requisites) and investigate decidablity. The study plan calls for August of 2015 for all remaining work to be completed.














%======================================================================
\chapter{ACCPL Syntax}
\label{chap:odrl0syntax}
%======================================================================

\section{Introduction}

We follow the style of~\cite{pucella2006} by using abstract syntax to express policy statements in \ac{ACCPL}. Abstract syntax is a more compact representation than XML which is what all the XML-based policy languages such \ac{odrl} use. Furthermore abstract syntax simplifies specifying the semantics as we shall see later. As an example the agreement ``If Mary Smith pays five dollars, then she is allowed to print the eBook 'Treasure Island' twice and she is allowed to display it on her computer as many times as she likes'' written in \ac{odrl}'s XML encoding is illustrated in Listing~\ref{lst:agreementxml}~\cite{pucella2006}. 


\lstset{language=XML}
%\begin{minipage}[c]{0.95\textwidth}
\begin{lstlisting}[caption={Agreement for Mary Smith in XML},label={lst:agreementxml}]
<agreement>
 <asset> <context> <uid> Treasure Island </uid> </context> </asset>
 <permission>
   <display>
    <constraint>
     <cpu> <context> <uid> Mary's computer </uid> </context> </cpu>
    </constraint>
   </display>
   <print>
    <constraint> <count> 2 </count> </constraint>
   </print>
  <requirement>
   <prepay>
    <payment> <amount currency="AUD"> 5.00</amount> </payment>
   </prepay>
  </requirement>
 </permission>
 <party> <context> <name> Mary Smith </name> </context> </party>
</agreement>
\end{lstlisting}
%\end{minipage} 

The agreement in Listing~\ref{lst:agreementxml} is shown in~\ref{lst:agreementpucella2006} using the syntax from~\cite{pucella2006}.

\lstset{language=Pucella2006}
\begin{lstlisting}[frame=single, caption={Agreement for Mary Smith as BNF (as used in~\cite{pucella2006})},label={lst:agreementpucella2006}]
agreement
 for Mary Smith 
 about Treasure Island 
 with prePay[5.00] -> and[cpu[Mary's Computer] => display,
                                      count[2] => print].
\end{lstlisting}


% \emph{prin\textsubscript{u}}
In the following we will cover the \emph{abstract syntax} of \ac{ACCPL} that we later express using Coq's constructs such as \emph{Inductive Types} and Definitions. 

\section{Environmental Facts}\label{sec:odrl0}
In \ac{ACCPL}, agreements and facts (i.e. environments) will refer to a count of how many times each policy should be and has been used to justify an action. This is the only fact that \ac{ACCPL} will cover although we conjecture adding other facts and the machinery to support those facts should not change verification goals and results so far of \ac{ACCPL}.

In \ac{ACCPL} a \emph{prerequisite} is either $true$, a $constraint$, the negative of a $constraint$ or a conjunction of $prerequisite$s. $true$ is the prerequisite that always holds. Constraints are facts that are outside of control of users. For example, there is nothing $Alice$ can do to satisfy the constraint ``user must be Bob''. 

We will describe \ac{ACCPL} in a \emph{BNF} grammar that looks more like Pucella and Weissman`s subset grammar~\cite{pucella2006}. BNF style grammars are more abstract as they only give suggestions about the surface syntax of expressions without getting into lexical analysis and parsing related aspects such as precedence order of operators~\cite{piercesf2011}. The Coq version in contrast is more formal and could be directly used for building compilers and interpreters. We will present both the BNF version and the Coq version for each construct of \ac{ACCPL}. 


\section{Productions} \label{sec:productionast}

The top level \ac{ACCPL} production is the \emph{agreement}. An agreement expresses what actions a set of subjects may perform on an object and under what conditions. Syntactically an agreement is composed of a set of subjects/users called a \emph{principal} or \emph{prin}, an \emph{asset} and a \emph{policySet}.

% agreement
\lstset{language=AST}
\begin{lstlisting}[frame=single, caption={agreement},label={lst:agreementast}]
<agreement> ::= 
     'agreement' 'for' <prin> 'about' <asset> 'with' <policySet> 
\end{lstlisting}

Principals or prins are composed of \emph{subjects} which are specified based on the application e.g. Alice, Bob, etc.

% prin
\lstset{mathescape, language=AST}  
\begin{lstlisting}[frame=single, caption={prin},label={lst:prinast}]
<prin> ::=  { <subject$_{1}$>, ..., <subject$_{m}$> }
\end{lstlisting}

% subject
\lstset{mathescape, language=AST}  
\begin{lstlisting}[frame=single, caption={subject},label={lst:subjectast}]
<subject> ::= N
\end{lstlisting}

Assets are also application specific but similar to subjects we will continue using specific ones for the \ac{drm} application (taken from~\cite{pucella2006}). \emph{ebook}, \emph{The Report} and \emph{latestJingle} are examples of specific subjects we will be using throughout. Syntactically an asset is represented as a natural number (\emph{N}). Similarly for subjects.

% asset
\lstset{mathescape, language=AST}  
\begin{lstlisting}[frame=single, caption={asset},label={lst:assetast}]
<asset> ::= N
\end{lstlisting}

Agreements express who may perform an action on an asset. They include a set of subjects (i.e. a $prin$), an asset and a policy set. A policy set is a primitive policy set implying non-nested policy sets. Note that we could define various combining operators for policy sets such as conjunctions and disjunctions but we keep \ac{ACCPL}'s policy sets limited to the primitive kind but use a policy combining operator when it comes to dealing with policies later on. Each primitive policy set specifies a \emph{prerequisite} and a \emph{policy}. In general if the prerequisite ``holds'' the policy is taken into consideration. Otherwise the policy will not be looked at. Some primitive policy sets are specified as \emph{inclusive} as opposed to others that are explicitly specified as \emph{exclusive}. The \emph{Primitive Exclusive Policy Set}s are exclusive to agreement's users in that only those users may perform the actions specified in the policy set. The implication is that all other users who are not specified in the agreement's principal (prin) are forbidden from performing the specified actions, no matter whether the $prerequisite$ holds or not. Not surprisingly we also define \emph{Primitive Inclusive Policy Set}s that don`t enforce any exclusivity to the agreement's users.


% policySet
\newcommand*{\Comment}[1]{\hfill\makebox[7.0cm][l]{#1}}%
\lstset{mathescape, language=AST, escapechar=\&}  
\begin{lstlisting}[frame=single, caption={policySet},label={lst:policySetast}]
<policySet> ::=  
   <primPolicySet>	&\Comment{; primitive policy set }&
\end{lstlisting}

% primitive policySet
\lstset{mathescape, language=AST, escapechar=\&}  
\begin{lstlisting}[frame=single, caption={primPolicySet},label={lst:primPolicySetast}]
<primPolicySet> ::=  
   <primInclusivePolicySet>	&\Comment{; primitive inclusive policy set }&
   <primExclusivePolicySet>	&\Comment{; primitive exclusive policy set }&
\end{lstlisting}

% primInclusivePolicySet
\lstset{mathescape, language=AST, escapechar=\&}  
\begin{lstlisting}[frame=single, caption={primInclusivePolicySet},label={lst:primInclusivePolicySetast}]
<primInclusivePolicySet> ::=  
   <preRequisite> $\rightarrow$ <policy>	&\Comment{; primitive inclusive policy set }&
\end{lstlisting}

% primExclusivePolicySet
\lstset{mathescape, language=AST, escapechar=\&}  
\begin{lstlisting}[frame=single, caption={primExclusivePolicySet},label={lst:primExclusivePolicySetast}]
<primExclusivePolicySet> ::=  
   <preRequisite> $\mapsto$ <policy>	&\Comment{; primitive exclusive policy set }&
\end{lstlisting}

A primitive policy specifies an action to be performed on an asset, depending of whether the policy's prerequisite holds or not. If the prerequisite holds the agreement's user is permitted to perform the action on the agreement's asset; otherwise permission is denied. Pucella and Weissman`s subset of \ac{odrl}~\cite{pucella2006} specify a unique identifier for each policy to help the translation (from agreements to formulas). \ac{ACCPL} has maintained the identifier for future work and we include it here in our definition of the policy construct however as far as the proofs are concerned the policy identifier could be removed without a loss to the obtained results. Primitive policies could also be grouped together using the conjunction combining operator.

% policy

\lstset{mathescape, language=AST, escapechar=\&}  
\begin{lstlisting}[frame=single, caption={policy},label={lst:policyast}]
<policy> ::=  
 'and'[ <primPolicy$_{1}$>, ..., 
                 <primPolicy$_{m}$> ]	&\Comment{; conjunction }&
\end{lstlisting}

% primitive policy

\lstset{mathescape, language=AST, escapechar=\&}  
\begin{lstlisting}[frame=single, caption={primPolicy},label={lst:primPolicyast}]
<primPolicy> ::=  
   <preRequisite> $\Rightarrow_{<policyId>}$ <act> 	&\Comment{; primitive policy}&
\end{lstlisting}

An \emph{Action} (\emph{act}) is represented as a natural number. Similar to assets and subjects, actions are application specific. Some example actions taken from \cite{pucella2006} are \emph{Display} and \emph{Print}.

% act
\lstset{mathescape, language=AST}  
\begin{lstlisting}[frame=single, caption={act},label={lst:actast}]
<act> ::= N
\end{lstlisting}

A \emph{Policy Id} (\emph{policyId}) is a unique identifier specified as (increasing) positive integers. 

% id
\lstset{mathescape, language=AST}  
\begin{lstlisting}[frame=single, caption={policyId},label={lst:policyIdast}]
<policyId> ::= N
\end{lstlisting}

In \ac{ACCPL} a \emph{prerequisite} is either true or it is a \emph{constraint}. The \emph{true} prerequisite always holds. A constraint is an intrinsic part of a policy and cannot be influenced by agreement's user. Minimum height requirements for popular attractions and rides are examples of what we would consider a constraint. \emph{NotCons} is a negation of a constraint. Finally the set of prerequisites are closed under conjunction operator (\emph{AndPrqs}).

% prq

\lstset{mathescape, language=AST, escapechar=\&}  
\begin{lstlisting}[frame=single, caption={preRequisite},label={lst:preRequisiteast}]
<preRequisite> ::=  
	'True' &\Comment{; always true}&
	<constraint>	 &\Comment{; constraint}&
	'not' [ <constraint> ] &\Comment{; suspending constraint}&
	'and'[ <preRequisite$_{1}$>, ..., <preRequisite$_{m}$> ] &\Comment{; conjunction }&
\end{lstlisting}

Constraints are either \emph{Principal}, \emph{Count} or \emph{CountByPrin}. Principal constraints basically require matching to specified prins. For example, the user being Alice is a Principal constraint. A count constraint refers to a set of policies \emph{P} and specifies the number of times the user of an agreement has invoked the policies in P to justify her actions. If the count constraint is part of a policy then the set P is composed of the single policy. In the case that the count constraint is part of a policy set, the set P is the set of policies specified in the policy set.

% constraint
\lstset{mathescape, language=AST, escapechar=\&}  
\begin{lstlisting}[frame=single, caption={constraint},label={lst:constraintast}]
<constraint> ::=  
	<prin> &\Comment{; principal}&
	'Count' [N] &\Comment{; number of executions}&
	<prin> ('Count' [N]) &\Comment{; number of executions by prin}&
\end{lstlisting}
 
%======================================================================
\chapter{ACCPL Syntax In Coq}\label{chap:odrl0syntaxcoq}
%======================================================================

% ---------------------------------- COQ -----------------------
\section{Introduction to Coq}

Coq is known first and foremost as a proof-assistant. The underlying formal language that Coq uses is a much more expressive version of typed lambda calculus called \ac{cic} where proofs and programs can both be represented. For example, \ac{cic} adds polymorphism (terms depending on types), type operators (types depending on types) and dependent types (types depending on terms).

Specifications of programs in Coq may be expressed using the specification language \emph{Gallina}~\cite{gallinaref}. Coq is then used to develop proofs to show that a program's run-time behaviour satisfies its specification. Such programs are called \emph{certified} because they are formally verified and confirmed to conform to their specifications~\cite{BC04}.

Assertions or propositions are statements about values in Coq such as \syn{3<8} or \syn{8<3} that may be true, false or even be only conjectures. To verify that a proposition is true a proof needs to constructed. While paper-proofs use a combination of mathematics and natural language to describe their proofs, Coq provides a formal (and therefore unambiguous) language that is based on proof-theory to develop proofs in. Verification of complex proofs is possible because one can verify the intermediate proofs or sub-goals in steps, each step being derived from the previous by following precise derivation rules. The Coq proof engine solves successive goals by using predefined \emph{tactics}. Coq tactics are commands to manipulate the local context and to decompose a goal into simpler goals or sub-goals~\cite{BC04}.

\section{ACCPL Syntax}\label{sec:agreementConstructor}

\ac{ACCPL} productions were presented as high level abstract syntax in section~\ref{sec:productionast} of chapter~\ref{chap:odrl0syntax}. Below we present the corresponding encodings in Coq. 

An \syn{agreement} is a new inductive type in Coq by the same name. The constructor \syn{Agreement} takes a \syn{prin}, an \syn{asset} and a \syn{policySet}. \syn{prin} is defined to be a nonempty list of \syn{subject}s (see listing~\ref{lst:agreementcoq}). 

\ac{ACCPL} types \syn{asset}, \syn{subject}, \syn{act} and \syn{policyId} are base types and are simply defined as \syn{nat} which is the datatype of natural numbers defined in coq's library module \syn{Coq.Init.Datatypes} (\syn{nat} is itself an inductive datatype). We use Coq constants to refer to specific objects of each type. For example, the subject `Alice' is defined as \begin{verbatim}Definition Alice:subject := 101.\end{verbatim} and the act `Play' as \begin{verbatim}Definition Play : act := 301.\end{verbatim}For each ``nat'' type in \ac{ACCPL} we have also used constants that play the role of ``Null'' objects (see ``Null Object Pattern''~\cite{martin1998pattern}), for example \syn{NullSubject} is defined as \begin{verbatim}Definition NullSubject:subject := 100.\end{verbatim} 

This means a specific natural number has been selected as a ``reserved'' number for each \ac{ACCPL} base type to play the role of a ``Null'' value. ``Null'' values are needed because all the sequence types in \ac{ACCPL} are defined to be non-empty sequences (implemented using the \syn{nonemptylist} datatype) even though at intermediate stages during the translation the concept of an empty list is needed. In such cases, a Null object is created and added to the nonempty list.

Next we define the \syn{policySet} datatype which is the direct implementation of the abstract syntax presented in listing~\ref{lst:policySetast}. A \syn{policySet} is constructed only one way: by calling the \syn{PPS} constructor which takes a \syn{primPolicySet} as input. There are two ways a \syn{primPolicySet} can be constructed (see listing~\ref{lst:policySetast} for the abstract syntax version) corresponding to two constructors: \syn{PIPS} and \syn{PEPS}. 

\syn{PIPS} takes a \syn{primInclusivePolicySet} as input while \syn{PEPS} takes a $\linebreak$ \syn{primExclusivePolicySet}. Both  \syn{primInclusivePolicySet} and \syn{primExclusivePolicySet} types are constructed by taking a \syn{preRequisite} and a \syn{policy} as parameters (see listing~\ref{lst:primInclusivePolicySetast} and ~\ref{lst:primExclusivePolicySetast} for the abstract syntax versions). 

A \syn{policy} is defined as a datatype with the constructor \syn{Policy} which takes a nonempty list of primitive policies, or \syn{primPolicy}s. A \syn{primPolicy} is constructed by calling $\linebreak$ \syn{PrimitivePolicy} which takes a prerequisite, \syn{preRequisite}, a policy identifier, \syn{policyId}, and an action, \syn{act} (see listing~\ref{lst:primPolicyast} for the abstract syntax version). Ignoring the \syn{policyId} for a moment, a primitive policy consists of a prerequisite and an action. Intuitively if the prerequisite holds, the action is allowed to be performed on the asset. 


\lstset{language=Coq}
\begin{minipage}[c]{0.95\textwidth}
\begin{lstlisting}[frame=single, caption={ACCPL: Coq Version of Agreement},label={lst:agreementcoq}]
Inductive agreement : Set :=
  | Agreement : prin -> asset -> policySet -> agreement.

Definition prin := nonemptylist subject.

Definition asset := nat.

Definition subject := nat.

Definition act := nat.

Definition policyId := nat.

Inductive policySet : Set :=
  | PPS : primPolicySet -> policySet.
  
Inductive primPolicySet : Set :=
  | PIPS : primInclusivePolicySet -> primPolicySet
  | PEPS : primExclusivePolicySet -> primPolicySet.

Inductive primInclusivePolicySet : Set :=
  | PrimitiveInclusivePolicySet : preRequisite -> policy -> primInclusivePolicySet.

Inductive primExclusivePolicySet : Set :=
  | PrimitiveExclusivePolicySet : preRequisite -> policy  -> primExclusivePolicySet.

Inductive policy : Set :=
  | Policy : nonemptylist primPolicy -> policy.

Inductive primPolicy : Set :=
  | PrimitivePolicy : preRequisite -> policyId -> act -> primPolicy.

\end{lstlisting}
\end{minipage}

The data type \syn{nonemptylist} reflects the definition of ``policy conjunction'' (see the definition of \syn{nonemptylist} in listing~\ref{lst:nonemptylistcoq}). Essentially \syn{nonemptylist} represents a list data structure that has at least one element and it is defined as a new \emph{polymorphic} inductive type in its own Coq section. See the abstract syntax version of \syn{nonemptylist} in listing~\ref{lst:policyast}. 


\lstset{language=Coq, frame=single, caption={nonemptylist},label={lst:nonemptylistcoq}}
\begin{minipage}[c]{0.95\textwidth}
\begin{lstlisting}

Section nonemptylist.

Variable X : Set.

Inductive nonemptylist : Set :=
  | Single : X -> nonemptylist 
  | NewList : X -> nonemptylist -> nonemptylist.

End nonemptylist.
\end{lstlisting}
\end{minipage}


In listing~\ref{lst:preRequisitecoq} \syn{preRequisite} is defined as a new datatype with constructors \syn{TruePrq}, \syn{Constraint}, \syn{NotCons} and \syn{AndPrqs} (see listing~\ref{lst:preRequisiteast} for the abstract syntax equivalent).

\syn{TruePrq} represents the always true prerequisite. The \syn{Constraint} constructor takes a value of type \syn{constraint} which we will describe below. Intuitively a constraint is a prerequisite to be satisfied that is outside the control of the user(s). For example, the constraint of being `Alice'. The constructor \syn{NotCons} is defined the same way the \syn{Constraint} constructor is. This constructor is defined as the type \syn{constraint} and it is meant to represent the negation of a \syn{constraint} as we shall see in the translation (see listing~\ref{lst:transnotConsCoq}). The remaining constructor \syn{AndPrqs} takes as parameters nonempty lists of prerequisites. This constructor represents the conjunction combining operator. 


% prq
\lstset{language=Coq}
%\begin{minipage}[c]{0.95\textwidth}
\begin{lstlisting}[frame=single, caption={preRequisite},label={lst:preRequisitecoq}]
Inductive preRequisite : Set :=
  | TruePrq : preRequisite
  | Constraint : constraint -> preRequisite 
  | NotCons : constraint -> preRequisite 
  | AndPrqs : nonemptylist preRequisite -> preRequisite.
  
\end{lstlisting}
%\end{minipage}

Finally a \syn{constraint} (see  listing~\ref{lst:constraintast} for the abstract syntax equivalent) is defined as a new datatype with constructors \syn{Principal}, \syn{Count} and \syn{CountByPrin}. See listing~\ref{lst:exampleconst} for the definition of \syn{constraint} and examples of how the three different kinds of constraints are constructed.

\lstset{language=Coq}
\begin{minipage}[c]{0.95\textwidth}
\begin{lstlisting}[frame=single, caption={Constraint Definition and the Three Kinds of Constraints},label={lst:exampleconst}]

Inductive constraint : Set :=
  | Principal : prin  -> constraint 
  | Count : nat -> constraint 
  | CountByPrin : prin -> nat -> constraint.


(Constraint (Principal (Single Alice))).

(Constraint (Count 10)).

(Constraint (CountByPrin (Single Alice) 10)).

\end{lstlisting}
\end{minipage}


We started with the encoding of the agreement for \syn{Alice} and \syn{Bob} in chapter~\ref{chap:odrl0syntax} but we deferred the definition of the Coq constructs used to encode that agreement. All the definitions needed to encode the agreement for \syn{Alice} and \syn{Bob} have now been defined so we now show how that agreement looks like as encodings of \ac{ACCPL} constructs in Coq in the listing~\ref{lst:agreementAliceAndBob}. Recall the listing~\ref{lst:agreementpucella2006} for \syn{<agreement>} which used the syntax taken from~\cite{pucella2006}.



\lstset{language=Coq}
%\begin{minipage}[c]{0.95\textwidth}
\begin{lstlisting}[frame=single, caption={Agreement for Alice and Bob in \ac{ACCPL}},label={lst:agreementAliceAndBob}]

Definition p1A1:primPolicySet :=
  PIPS (PrimitiveInclusivePolicySet
    TruePrq
    (Policy (Single (PrimitivePolicy (Constraint (Count  5)) id1 Print)))).

Definition p2A1prq1:preRequisite := (Constraint (Principal (Single Alice))).
Definition p2A1prq2:preRequisite := (Constraint (Count 2)).

Definition p2A1:primPolicySet :=
  PIPS (PrimitiveInclusivePolicySet
    TruePrq
    (Policy 
      (Single 
        (PrimitivePolicy (AndPrqs (NewList p2A1prq1 (Single p2A1prq2))) id2 Print)))).


Definition A1 := Agreement (NewList Alice (Single Bob)) TheReport (PPS p1A1).
\end{lstlisting}	
%\end{minipage}	


 
%%======================================================================
\chapter{ACCPL Semantics}\label{chap:semantics}
%======================================================================

                  
%----------------------------------------------------------------------
\section{Introduction}\label{sec:introsemantics}


In this section, we follow Pucella and Weissman`s subset of \ac{odrl}~\cite{pucella2006} lead on how to describe the semantics of the \ac{ACCPL}. The semantics of \ac{ACCPL} are described by a translation from agreements to a subset of many-sorted first-order logic formulas with equality. The semantics will help answer queries of the form ``may subject \emph{s} perform action \emph{act} to asset \emph{a}?''. If the answer is yes, we say permission is granted. Otherwise permission is denied. Note that in the listings in this chapter we use $[\![]\!]$ (double square brackets) notation as a mapping of \ac{ACCPL} syntactic elements to their translations as many-sorted first-order logic formulas. $\triangleq$ are used between a translation and its corresponding formula to mean the translation is ``mapped to'' the formula.

Unless specifically mentioned, we use superscripts to denote parameters to translation mappings. However we make a distinction on whether the translation notation is used on the \ac{rhs} of a $\triangleq$ or the \ac{lhs}. The right side occurrence is similar to a function call where we pass actual parameters to the function call - here to the mapping. The left side occurrence is similar to function declarations/definitions where we define formal parameters - here when we define the translation for a construct for the first time. 

\section{Agreement Translation}

At a high-level, an agreement is translated into a formula of the form $\forall x ( prerequisites(x) \rightarrow P(x))$ where $P(x)$ itself is a conjunction of formulas of the form $ prerequisites(x) \rightarrow (\lnot) Permitted (x, act, a)$, where $Permitted (x, act, a)$ means the subject $x$ is permitted to perform action $act$ on asset $a$. More concretely, given an agreement and observing that the only way an agreement can be built is by passing a $prin$ (typically $prin_{u}$ the agreement's user(s)), an $asset$, and a $policySet$ to the constructor of the type agreement (see~\ref{sec:agreementConstructor}), the given agreement is translated by invoking the agreement translation mapping with the passed in arguments plus the subject, action and asset coming from a ``query'' or request for access: $[\![agreement]\!]^{policySet, prin_{u}, a, sub_{q}, act_{q}, ass_{q}}$. 


The agreement translation mapping declares its formal arguments as a $policySet$, a set of users, $prin$, an asset, $a$ and subject, action and asset coming from a query: $sub_{q}$, $ act_{q}$, $ass_{q}$. (see~\ref{lst:transAgreementast}).


\lstset{mathescape, language=AST}  
\begin{lstlisting}[frame=single, caption={Agreement Translation},label={lst:transAgreementast}]
$[\![ agreement ]\!]^{policySet, prin, a, sub_{q}, act_{q}, ass_{q}}$ $\triangleq$ $[\![policySet]\!]^{prin, a, sub_{q}, act_{q}, ass_{q}}$
\end{lstlisting}

\section{Policy Set Translation Definition}
The translation for a $policySet$ declares the formal arguments: a set of users, $prin$ and an asset, $a$ and the subject, action and asset coming from a query. The translation is handled by the translation function for a $primPolicySet$. The translation for $primPolicySet$ is defined by cases, one for each clause of the grammar in listing~\ref{lst:primPolicySetast}. Recall that a $primPolicySet$ is either a $primInclusivePolicySet$ or a $\linebreak primExclusivePolicySet$.

\lstset{mathescape, language=AST}  
\begin{lstlisting}[frame=single, caption={Policy Set Translation Cases},label={lst:transpolicysetdefinitionAST}]

$[\![policySet]\!]^{prin, a, sub_{q}, act_{q}, ass_{q}}$ $\triangleq$ 
   $ass_{q} = a \rightarrow$
      $[\![primPolicySet]\!]^{prin, a, sub_{q}, act_{q}, ass_{q}}$&\Comment{; primitive policy set }&
   $ass_{q} <> a \rightarrow$
      $Unregulated (sub_{q}, [\![act_{q}]\!], ass_{q})$&\Comment{; this case in Unregulated }&
\end{lstlisting}

\lstset{mathescape, language=AST}  
\begin{lstlisting}[frame=single, caption={Primitive Policy Set Translation Cases},label={lst:transpolicysetdefinitionAST}]

$[\![primInclusivePolicySet]\!]^{prin, a, sub_{q}, act_{q}, ass_{q}}$ $\triangleq$ 
   $[\![policy_{PIPS}]\!]^{prin, a, sub_{q}, act_{q}, ass_{q}}$&\Comment{; primitive inclusive policy set }&
\end{lstlisting}

\lstset{mathescape, language=AST}  
\begin{lstlisting}[frame=single, caption={Primitive Policy Set Translation Cases},label={lst:transpolicysetdefinitionAST}]

$[\![primExclusivePolicySet]\!]^{prin, a, sub_{q}, act_{q}, ass_{q}}$ $\triangleq$ 
   $[\![policy_{PEPS}]\!]^{prin, a, sub_{q}, act_{q}, ass_{q}}$&\Comment{; primitive exclusive policy set }&
\end{lstlisting}

\lstset{mathescape, language=AST}  
\begin{lstlisting}[frame=single, caption={Primitive Policy Set Translation Cases},label={lst:transpolicysetdefinitionAST}]

$[\![policy_{PIPS}]\!]^{prin, a, sub_{q}, act_{q}, ass_{q}}$ $\triangleq$ 
        $[\![ preRequisite \rightarrow policy]\!]^{prin, a}$ &\Comment{; primitive policy set }&
        $[\![ preRequisite \mapsto policy]\!]^{prin, a}$ &\Comment{; primitive exclusive policy set }&
        $[\![ and [policySet_{1}, ..., policySet_{m}]]\!]^{prin, a}$ &\Comment{; conjunction }&

\end{lstlisting}


\lstset{mathescape, language=AST}  
\begin{lstlisting}[frame=single, caption={Primitive Policy Set Translation Cases},label={lst:transpolicysetdefinitionAST}]

$[\![primPolicySet]\!]^{prin, a, sub_{q}, act_{q}, ass_{q}}$ $\triangleq$ 
        $[\![ preRequisite \rightarrow policy]\!]^{prin, a}$ &\Comment{; primitive policy set }&
        $[\![ preRequisite \mapsto policy]\!]^{prin, a}$ &\Comment{; primitive exclusive policy set }&
        $[\![ and [policySet_{1}, ..., policySet_{m}]]\!]^{prin, a}$ &\Comment{; conjunction }&

\end{lstlisting}

Each of the $policySet$ kinds has its own translation function, which will be defined in the next 3 subsections. See~\ref{sec:getIdtranslation} for the translation of $getId$ and~\ref{sec:acttranslation} for the translation of $act$ mentioned below.


\subsection{PrimitivePolicySet Translation Definition}
Translation of a $PrimitivePolicySet$ ($preRequisite \rightarrow policy$) declares two formal arguments: a set of users, $prin$ and an asset, $a$ and yields a formula that includes a test on whether the subject is in the set of agreements' users (translation of $prin$), the translation of the $policy$ and the translation of the $preRequisite$. Basically if the subject in question is a user of the agreement and the policySet's prerequisites hold, then the policy holds. Translation of the policy for a $PrimitivePolicySet$ is called a \emph{positive translation}. A positive translation is one where the actions described by the policies are permitted (see~\ref{lst:transpolicyformulaPrimitivePolicySet}).   

\lstset{mathescape, language=AST}  
\begin{lstlisting}[frame=single, caption={Policy Set Translation Definition {$\colon$} PrimitivePolicySet},label={lst:transpolicyformulaPrimitivePolicySet}]
$[\![ preRequisite \rightarrow policy]\!]^{prin, a}$ $\triangleq$ $\forall x$ $(\!( [\![prin]\!]_{x}$ $\land$ $[\![preRequisite]\!]^{getId (policy), prin, a}_{x}) \rightarrow [\![policy]\!]^{positive, prin, a}_{x}\!)$
\end{lstlisting}





\subsection{PrimitiveExclusivePolicySet Translation Definition}
Translation of a $PrimitiveExclusivePolicySet$ ($preRequisite \mapsto policy$) declares two formal arguments: a set of users, $prin$ and an asset, $a$ and yields the conjunction of two implications. The first implication, is the same as one found in the translation of $PrimitivePolicySet$. The second implication however restricts access (to make the policy set exclusive) to only those subjects that are in the agreement's user(s). Translation of the policy in the second implication is called a \emph{negative translation}. A negative translation is one where the actions described by the policies are not permitted (see~\ref{lst:transpolicyformulaPrimitiveExclusivePolicySet}).


\lstset{mathescape, language=AST}  
\begin{lstlisting}[frame=single, caption={Policy Set Translation Definition {$\colon$} PrimitiveExclusivePolicySet},label={lst:transpolicyformulaPrimitiveExclusivePolicySet}]
$[\![ preRequisite \mapsto policy]\!]^{prin, a}$ $\triangleq$ $\forall x$ $(\!( [\![prin]\!]_{x}$ $\land$ $[\![preRequisite]\!]^{getId (policy), prin, a}_{x}) \rightarrow [\![policy]\!]^{positive, prin, a}_{x}\!)$ $\land$ $\forall x$ $(\neg[\![prin]\!]_{x} \rightarrow [\![policy]\!]^{negative, prin, a}_{x})$
\end{lstlisting}

\subsection{AndPolicySet Translation Definition}
Translation of $AndPolicySet$ declares two formal arguments: a set of users, $prin$ and an asset, $a$ and yields the conjunctions of the corresponding policy set translations (see~\ref{lst:transpolicyformulaAndPolicySet}). 

\lstset{mathescape, language=AST}  
\begin{lstlisting}[frame=single, caption={Policy Set Translation Definition {$\colon$} AndPolicySet},label={lst:transpolicyformulaAndPolicySet}]
$[\![ and [policySet_{1}, ..., policySet_{m}]]\!]^{prin, a}$ $\triangleq$ $[\![policySet_{1}]\!]^{prin, a}$ $\land$ $...$ $\land$ $[\![policySet_{m}]\!]^{prin, a}$

\end{lstlisting}


\subsection{getId Translation}\label{sec:getIdtranslation}

The $getId$ function when applied to a single $policy$(see~\ref{lst:policyast}) is defined to return the $policyId$ of the $policy$.

\lstset{mathescape, language=AST}  
\begin{lstlisting}[frame=single, caption={getId for a Single policy},label={lst:getIdSinglePolicyAST}]
$[\![ getId (policy) ]\!]$ $\triangleq$ $[\![ getId (preRequisite \Rightarrow_{policyId} act) ]\!]$ $\triangleq$ $policyId$
\end{lstlisting}

$getId$ function when applied to a set of policies is the set union of the translations for each individual policy(see~\ref{lst:getIdSinglePolicyAST}).

\lstset{mathescape, language=AST}  
\begin{lstlisting}[frame=single, caption={getId for Policies Definition},label={lst:getIdAndPolicyAST}]

$[\![ getId (and [policy_{1}, ..., policy_{m}) ]\!]$ $\triangleq$ $getId (policy_{1})$ $\cup$ $...$ $\cup$ $getId (policy_{m})$
\end{lstlisting}

\subsection{action Translation}\label{sec:acttranslation}

Translation of actions, such as $Play$ and $Display$ is simply the constant corresponding to each action instance.

\lstset{mathescape, language=AST}  
\begin{lstlisting}[frame=single, caption={act Translation Definition},label={lst:actiontranslationAST}]
$[\![act]\!]$ $\triangleq$ $N$
\end{lstlisting}


\section{prin Translation Cases}
Translation for a \emph{prin} ($[\![ prin ]\!]_{x}$) declares no formal arguments and is a formula that is true if and only if the subject $x$ is in the prin set. A $prin$ is either a single subject or a list of subjects so the translation covers both cases (see~\ref{lst:transprindefinitionAST}).

\lstset{mathescape, language=AST}  
\begin{lstlisting}[frame=single, caption={Prin Translation Cases},label={lst:transprindefinitionAST}]
$[\![ prin ]\!]_{x}$ $\triangleq$ 
        $[\![ subject ]\!]_{x}$
        $[\![ \{ subject_{1}, ..., subject_{m} \} ]\!]_{x}$
\end{lstlisting}

Each of the $prin$ kinds has its own translation function, which will be defined in the following 2 subsections.



\subsection{Single Subject Translation Definition}
If the $prin$ is a single subject, the translation is a formula that is true if and only if the subject $x$ is the same as the single subject $subject$ (see ~\ref{lst:transprinSingle}).

\lstset{mathescape, language=AST}  
\begin{lstlisting}[frame=single, caption={Prin Translation Definition {$\colon$} Single subject},label={lst:transprinSingle}]
$[\![ subject ]\!]_{x}$ $\triangleq$ $x=subject$
\end{lstlisting}

\subsection{List of Subjects Translation Definition}
Translation of a list of subjects is the disjunction of the translations for each subject (see ~\ref{lst:transprinListOfSubjects}).

\lstset{mathescape, language=AST}  
\begin{lstlisting}[frame=single, caption={Prin Translation Definition {$\colon$} List of subjects},label={lst:transprinListOfSubjects}]

$[\![ \{ subject_{1}, ..., subject_{m} \} ]\!]_{x}$ $\triangleq$ $[\![subject_{1}]\!]_{x}$ $\lor$ $...$ $\lor$ $[\![subject_{m}]\!]_{x}$

\end{lstlisting}

\section{Policy Translation Cases}
The translation for a $policy$ declares three formal arguments: $polarity$, which indicates whether the policy translation is positive or negative, a set of users, $prin$ and an asset, $a$.
The translation is defined by cases, one for each clause of the grammar in listing~\ref{lst:policyast}. Recall that a $policy$ is either a $PrimitivePolicy$ or a $AndPolicy$, a conjunction of policies (see~\ref{lst:transpolicydefinitionAST}).


\lstset{mathescape, language=AST}  
\begin{lstlisting}[frame=single, caption={Policy Translation Cases},label={lst:transpolicydefinitionAST}]
$[\![ policy ]\!]^{polarity, prin, a}$ $\triangleq$ 
         $[\![ preRequisite \Rightarrow_{policyId} act ]\!]^{polarity, prin, a}_{x}$ &\Comment{; primitive policy}&
         $[\![ and [policy_{1}, ..., policy_{m}]]\!]^{polarity, prin, a}_{x}$ &\Comment{; conjunction }&

\end{lstlisting}

Each of the $policy$ kinds has its own translation function, which will be defined in the next 2 subsections. 

\subsection{PrimitivePolicy Translation Definition}
Translation of a $PrimitivePolicy$ ($preRequisite \Rightarrow_{policyId} act$) declares three formal arguments: $polarity$, which indicates whether the policy translation is positive or negative, a set of users, $prin$ and an asset, $a$. In the case of positive polarity, the translation yields a formula that 'permits' $x$ to $act$ on $a$ if translation for the $preRequisite$ holds. In the case that the polarity is negative, the translation yields a formula indicating that $x$ is not permitted to $act$ on $a$ regardless of whether the translation for $preRequisite$ holds or not (see~\ref{lst:transprimitivepolicyAST}).

\lstset{mathescape, language=AST} 
\begin{lstlisting}[frame=single, caption={PrimitivePolicy Translation Definition},label={lst:transprimitivepolicyAST}]

$[\![ preRequisite \Rightarrow_{policyId} act ]\!]^{polarity, prin, a}_{x}$ $\triangleq$ 
        $([\![ preRequisite ]\!]^{\left\{{policyId}\right\}, prin, a}_{x}) \Rightarrow Permitted(x, [\![act]\!], a)$ &\Comment{; polarity=positive }&
        $\lnot (Permitted(x, [\![act]\!], a))$ &\Comment{; polarity=negative }&

\end{lstlisting}


\subsection{AndPolicy Translation Definition}
Translation of $AndPolicy$ declares three formal arguments: $polarity$, which indicates whether the policy translation is positive or negative, a set of users, $prin$ and an asset, $a$ and yields the conjunctions of the corresponding policy translations (see~\ref{lst:transAndpolicyAST}). 

\lstset{mathescape, language=AST}  
\begin{lstlisting}[frame=single, caption={Policy Translation Definition {$\colon$} AndPolicy},label={lst:transAndpolicyAST}]
$[\![ and [policy_{1}, ..., policy_{m}]]\!]^{polarity, prin, a}$ $\triangleq$ $[\![policy_{1}]\!]^{polarity, prin, a}$ $\land$ $...$ $\land$ $[\![policy_{m}]\!]^{polarity, prin, a}$

\end{lstlisting}




\section{Prerequisite Translation Cases}

Translation for a $preRequisite$ declares three formal arguments: a set of $policyId$s (identifiers for policies that are implied by the prerequisites), a $prin$ and an asset, $a$. The translation is defined by cases, one for each clause of the grammar in listing~\ref{lst:preRequisiteast}. Recall that a $preRequisite$ is either always $TruePrq$, a $Constraint$, a $ForEachMember$, a $NotCons$, a $AndPrqs$, a $OrPrqs$ or a $XorPrqs$ (see~\ref{lst:transprerequisitedefinitionAST}). 

\lstset{mathescape, language=AST}  
\begin{lstlisting}[frame=single, caption={Prerequisite Translation Cases},label={lst:transprerequisitedefinitionAST}]

$[\![ preRequisite ]\!]^{\left\{{policyId_{1}, ..., policyId_{m}}\right\}, prin, a}_{x}$ $\triangleq$
    $[\![ true ]\!]^{\left\{{policyId_{1}, ..., policyId_{m}}\right\}, prin, a}_{x}$ &\Comment{; always true}&
    $[\![ constraint ]\!]^{\left\{{policyId_{1}, ..., policyId_{m}}\right\}, prin, a}_{x}$ &\Comment{; constraint}&      
    $[\![ forEachMember [prin^{\prime};\left\{{constraint_{1}, ..., constraint_{n}}\right\}] ]\!]^{\left\{{policyId_{1}, ..., policyId_{m}}\right\}, prin, a}_{x}$ &\Comment{; constraint distribution}&
    $[\![ not$ $constraint ]\!]^{\left\{{policyId_{1}, ..., policyId_{m}}\right\}, prin, a}_{x}$ &\Comment{; suspending constraint}&
    $[\![and$ $[preRequisite_{1}, ..., preRequisite_{k}]]\!]^{\left\{{policyId_{1}, ..., policyId_{m}}\right\}, prin, a}_{x}$ &\Comment{; conjunction }&
    $[\![or$ $[preRequisite_{1}, ..., preRequisite_{k}]]\!]^{\left\{{policyId_{1}, ..., policyId_{m}}\right\}, prin, a}_{x}$ &\Comment{; disjunction}&
    $[\![Xor$ $[preRequisite_{1}, ..., preRequisite_{k}]]\!]^{\left\{{policyId_{1}, ..., policyId_{m}}\right\}, prin, a}_{x}$ &\Comment{; exclusive disjunction}&

\end{lstlisting}

Each of the $preRequisite$ kinds has its own translation function, which will be defined in the following subsections.

\subsection{True Prerequisite Translation Definition}
The translation for a $TruePrq$ yields a formula that is always \emph{true} (see~\ref{lst:transpreRequisiteTruePrq}).

\lstset{mathescape, language=AST}  
\begin{lstlisting}[frame=single, caption={Prerequisite Translation Definition {$\colon$} Always True Prerequisite},label={lst:transpreRequisiteTruePrq}]
	$[\![ true ]\!]^{\left\{{policyId_{1}, ..., policyId_{m}}\right\}, prin, a}_{x}$ $\triangleq$ True
\end{lstlisting}

\subsection{Constraint Prerequisite Translation Definition}
The translation for a $constraint$ declares three formal arguments: a set of $policyId$s, a set of users, $prin$ and an asset, $a$. The translation is defined by cases, one for each clause of the grammar in listing~\ref{lst:constraintast}. Recall that a $constraint$ is either a $Principal$, a $Count$ or a $CountByPrin$ (see~\ref{lst:transpreRequisiteConstraint}).


\lstset{mathescape, language=AST}  
\begin{lstlisting}[frame=single, caption={Prerequisite Translation Definition {$\colon$} Constraint},label={lst:transpreRequisiteConstraint}]

$[\![ constraint ]\!]^{\left\{{policyId_{1}, ..., policyId_{m}}\right\}, prin, a}_{x}$ $\triangleq$ 
   $[\![ prin ]\!]_{x}$ &\Comment{; principal}&
   $[\![ count [N] ]\!]^{\left\{{policyId_{1}, ..., policyId_{m}}\right\}, prin, a}_{x}$	 &\Comment{; number of executions}&
   $[\![ (count [N])^{prin^{\prime}} ]\!]^{\left\{{policyId_{1}, ..., policyId_{m}}\right\}, prin, a}_{x}$ &\Comment{; number of executions by $prin^{\prime}$}&
\end{lstlisting}

Each of the $constraint$ kinds has its own translation function, which will be defined in the subsections following~\ref{sec:constraintTransDefSec}.

\subsection{ForEachMember Prerequisite Translation Definition}
The translation for a $forEachMember$ declares three formal arguments: a set of $policyId$s, a set of users, $prin$ and an asset, $a$. Note that $forEachMember$ takes additional formal arguments: a $prin^{\prime}$ that overrides the $prin$ that is passed at the $preRequisite$ level and a set of $constraint$s (see ~\ref{lst:transpreRequisiteForEachMember}).


\lstset{mathescape, language=AST}  
\begin{lstlisting}[frame=single, caption={Prerequisite Translation Definition {$\colon$} ForEachMember},label={lst:transpreRequisiteForEachMember}]

$[\![ forEachMember [prin^{\prime};\left\{{constraint_{1}, ..., constraint_{n}}\right\}] ]\!]^{\left\{{policyId_{1}, ..., policyId_{m}}\right\}, prin, a}_{x}$ $\triangleq$ $\bigwedge_{(subject, i)\in prin^{\prime} \times \left\{ {1, ..., n}\right\})}\nolimits$ $[\![constraint_{i}]\!]^{\left\{ {policyId_{1}, ..., policyId_{m}}\right\}, \left\{subject\right\}, a}_{x}$
\end{lstlisting}



\subsection{NotCons Prerequisite Translation Definition}
The translation for a $NotCons$ yields a formula that is simply the negation of the translation for a constraint (see ~\ref{lst:transpreRequisiteConstraint}).

\lstset{mathescape, language=AST}  
\begin{lstlisting}[frame=single, caption={Prerequisite Translation Definition {$\colon$} Not Constraint},label={lst:transpreRequisiteNotConstraint}]

$[\![ not$ $constraint ]\!]^{\left\{{policyId_{1}, ..., policyId_{m}}\right\}, prin, a}_{x}$ $\triangleq$ $\lnot[\![ constraint ]\!]^{\left\{{policyId_{1}, ..., policyId_{m}}\right\}, prin, a}_{x}$ 
\end{lstlisting}

\subsection{AndPrqs Prerequisite Translation Definition}
The translation for a $AndPrqs$ yields a formula that is the conjunction of the translation for each $preRequisite$ (see ~\ref{lst:transpreRequisiteAndPrqs}).

\lstset{mathescape, language=AST}  
\begin{lstlisting}[frame=single, caption={Prerequisite Translation Definition {$\colon$} Conjunction },label={lst:transpreRequisiteAndPrqs}]

$[\![and$ $[preRequisite_{1}, ..., preRequisite_{k}]]\!]^{\left\{{policyId_{1}, ..., policyId_{m}}\right\}, prin, a}_{x}$ $\triangleq$ $[\![preRequisite_{1}]\!]^{\left\{{policyId_{1}, ..., policyId_{m}}\right\}, prin, a}_{x}$ $\land$ $...$ $\land$ $[\![preRequisite_{k}]\!]^{\left\{{policyId_{1}, ..., policyId_{m}}\right\}, prin, a}_{x}$

\end{lstlisting}

\subsection{OrPrqs Prerequisite Translation Definition}
The translation for a $OrPrqs$ yields a formula that is the inclusive disjunction of the translation for each $preRequisite$ (see ~\ref{lst:transpreRequisiteOrPrqs}).

\lstset{mathescape, language=AST}  
\begin{lstlisting}[frame=single, caption={Prerequisite Translation Definition {$\colon$} Inclusive Disjunction},label={lst:transpreRequisiteOrPrqs}]

$[\![or$ $[preRequisite_{1}, ..., preRequisite_{k}]]\!]^{\left\{{policyId_{1}, ..., policyId_{m}}\right\}, prin, a}_{x}$ $\triangleq$ $[\![preRequisite_{1}]\!]^{\left\{{policyId_{1}, ..., policyId_{m}}\right\}, prin, a}_{x}$ $\lor$ $...$ $\lor$ $[\![preRequisite_{k}]\!]^{\left\{{policyId_{1}, ..., policyId_{m}}\right\}, prin, a}_{x}$

\end{lstlisting}

\subsection{XorPrqs Prerequisite Translation Definition}
The translation for a $XorPrqs$ yields a formula that is the exclusive disjunction of the translation for each $preRequisite$ (see ~\ref{lst:transpreRequisiteXorPrqs}).

\lstset{mathescape, language=AST}  
\begin{lstlisting}[frame=single, caption={Prerequisite Translation Definition {$\colon$} Exclusive Disjunction},label={lst:transpreRequisiteXorPrqs}]

$[\![Xor$ $[preRequisite_{1}, ..., preRequisite_{k}]]\!]^{\left\{{policyId_{1}, ..., policyId_{m}}\right\}, prin, a}_{x}$ $\triangleq$ $[\![preRequisite_{1}]\!]^{\left\{{policyId_{1}, ..., policyId_{m}}\right\}, prin, a}_{x}$ $\oplus $ $...$ $\oplus$ $[\![preRequisite_{k}]\!]^{\left\{{policyId_{1}, ..., policyId_{m}}\right\}, prin, a}_{x}$

\end{lstlisting}


\section{Constraint Translation Cases}\label{sec:constraintTransDefSec}

Translation for a constraint is a formula $[\![constraint]\!]^{\left\{ {policyId_{1}, ..., policyId_{m}}\right\}, prin, a}_{x}$, where the set of $policyId$s are identifiers for policies that are implied by the constraint, $prin$ is the set of users (typically the agreement's user(s) denoted by $prin_{u}$) to which the constraint applies, $x$ is a variable of type $subject$. 

\subsection{Principal Constraint Translation}
The translation for a $Principal$ is defined to be the translation of the corresponding $prin$ defined earlier in listing~\ref{lst:transprindefinitionAST}.   


\subsection{Count Constraint Translation}
The translation for a $Count$ declares three formal arguments: a set of $policyId$s, a set of users, $prin$ and an asset, $a$ and it is a formula that is true when the sum of the number of times that $subject_{i}$ (taken from $prin$) has invoked a policy with policy identifier $id_{j}$ is smaller than $N$ (see listing~\ref{lst:transconstraintCount}). 

\lstset{mathescape, language=AST}  
\begin{lstlisting}[frame=single, caption={Constraint Translation {$\colon$} Count},label={lst:transconstraintCount}]

$[\![ count [N] ]\!]^{\left\{{policyId_{1}, ..., policyId_{m}}\right\}, prin, a}_{x}$ $\triangleq$ $(\sum_{(id, subject)\in (\left\{{policyId_{1}, ..., policyId_{m}}\right\} \times prin)}\nolimits$ $getCount(subject, id)) < N$ 
\end{lstlisting}


\subsection{CountByPrin Constraint Translation}
The translation for a $CountByPrin$ declares the same formal arguments as the translation for $Count$ and it is a formula that is true when the sum of the number of times that $subject_{i}$ has invoked a policy with policy identifier $id_{j}$ is smaller than $N$. The difference from the $Count$ case, is that $prin^{\prime}$ in $CountByPrin$ overrides the passed in $prin$ (typically agreement's user(s) or $prin_{u}$). See listing~\ref{lst:transconstraintCountbyPrin}.


\lstset{mathescape, language=AST}  
\begin{lstlisting}[frame=single, caption={Constraint Translation {$\colon$} Count by Principal},label={lst:transconstraintCountbyPrin}]

$[\![ (count [N])^{prin^{\prime}} ]\!]^{\left\{{policyId_{1}, ..., policyId_{m}}\right\}, prin, a}_{x}$ $\triangleq$ $(\sum_{(id, subject)\in (\left\{{policyId_{1}, ..., policyId_{m}}\right\} \times prin^{\prime})}\nolimits$ $getCount(subject, id)) < N$ 
\end{lstlisting}


 
%======================================================================
\chapter{ACCPL Semantics In Coq}\label{chap:accplsemanticscoq}
%======================================================================

% ---------------------------------- COQ -----------------------
We specify the semantics of \ac{ACCPL} as a translation function from an agreement together with an access request and an environment containing all relevant facts, to decisions (more concretely to a set of \syn{result}s--see listing~\ref{lst:answercoq}). Specifying the semantics as translations from an agreement, an access request and an environment to decisions, mostly follows the style Pucella and Weissman~\cite{pucella2006} use to define the semantics for their \ac{odrl} fragment. In Pucella and Weissman's language, the translation functions are defined from agreements to formulas in many-sorted \ac{fol} with equality. The translation functions for \ac{ACCPL} return a specific decision based on whether there are proof terms for certain conditions and/or proof terms for the negation of those conditions (see Section~\ref{sec:sumboolsection} for more details).
 

% COQ
The translation functions plus the auxiliary types and infrastructure which implement the semantics for \ac{ACCPL} have been encoded in Coq. The Coq code implementing the semantics is high-level enough to be readable as a specification of the semantics. This is one reason why we do not use abstract syntax to describe the semantics as we did for the syntax of \ac{ACCPL} (see chapter~\ref{chap:odrl0syntax}). The other reason not to use abstract syntax the way Pucella and Weissman~\cite{pucella2006} do for their semantics, is the difficulty of coming up with abstract syntax that is intuitive and equivalent to the translation functions implemented in Coq and listed in this chapter.

Whether an agreement and the accompanying access request or query is translated into a decision indicating a permission is granted or denied depends on the agreement in question and the specifics of the request, but also on the facts recorded in the environment (e.g. time of day). For \ac{ACCPL} those facts revolve around the number of times a policy has been used to justify an action. We encode this information in Coq as a new inductive type representing the environment. An \syn{environment} (see listing~\ref{lst:environmentcoq}) is a conjunction of equalities of the form \syn{count(s, policyId) = n} which are called ``count equalities'' (see Section~\ref{sec:odrl0} for background on environments).

% COQ
The Coq version of a count equality is a new inductive type called \syn{count_equality}. An environment is defined to be a non-empty list of \syn{count_equality} objects (see listing~\ref{lst:environmentcoq}). Function \syn{make_count_equality} in listing~\ref{lst:environmentcoq} is simply a convenience function that builds \syn{count_equality} objects. 

\lstset{language=Coq, frame=single, caption={Environments and Counts},label={lst:environmentcoq}}
\begin{minipage}[c]{0.95\textwidth}
\begin{lstlisting}
Inductive count_equality : Set := 
   | CountEquality : subject -> policyId -> nat -> count_equality.

Definition make_count_equality
  (s:subject)(id:policyId)(n:nat): count_equality :=
  CountEquality s id n.
  
Inductive environment : Set := 
  | SingleEnv : count_equality -> environment
  | ConsEnv :  count_equality ->  environment -> environment.

\end{lstlisting}
\end{minipage}

For an example of how environments are created see listing~\ref{lst:environmentusagecoq}.

\lstset{language=Coq, frame=single, caption={Defining Environments},label={lst:environmentusagecoq}}
%\begin{minipage}[c]{0.95\textwidth}
\begin{lstlisting}

Definition e1 : environment := 
  (SingleEnv (make_count_equality Alice id1 8)).

\end{lstlisting}
%\end{minipage}
  

We also define a \syn{getCount} function that given a pair consisting of a subject and policy id, looks for a corresponding count in the environment. The \syn{getCount} function assumes the given environment is consistent (meaning it won't return two different counts for the same pair of subject and policy id), so it returns the first matched \syn{count} it sees for a \syn{(subject, id)} pair. If a \syn{count} for a \syn{(subject, id)} pair is not found it returns 0. 

We have defined the necessary representations in Coq in the form of an inductive predicate (called \syn{env_consistent}) to verify the consistency assumption mentioned above (see listing~\ref{lst:envsConsistentcoq}). For a given environment \syn{e1}, we would require a proof of the proposition ``\syn{env_consistent e1}'' before translation begins. Note that we have defined helper lemmas to help complete such proofs. In the listing~\ref{lst:envsConsistentcoq}, we have also shown the definition of the function \syn{inconsistent} which states that two \syn{count_equality} values are inconsistent for the same pair of subject and policy id, if the counts are different. Another helper definition shown in listing~\ref{lst:envsConsistentcoq} is the function \syn{formula_inconsistent_with_env} which captures the proposition ``this count formula is inconsistent with this environment''.

%Another option would be to restrict the environment type to a subset (consistent environments only) using Coq's \syn{sig} type defined in the Coq standard library module \syn{Coq.Init.Specif}. According to~\cite{Coq:manual} \syn{(sig A P)} or the notationally convenient \syn{\{x:A | P x\}}, represents the subset of elements of the type A which satisfy the predicate P. 

\lstset{language=Coq, frame=single, caption={Consistency of Environments},label={lst:envsConsistentcoq}}
\begin{minipage}[c]{0.95\textwidth}
\begin{lstlisting}

Inductive env_consistent : environment -> Prop :=
 | consis_1 : forall f, env_consistent (SingleEnv f)
 | consis_2 : forall f g, ~(inconsistent f g) -> env_consistent (ConsEnv f (SingleEnv g))
 | consis_more : forall f e,
    env_consistent e -> ~(formula_inconsistent_with_env f e) -> env_consistent (ConsEnv f e).

Definition inconsistent (f1 f2 : count_equality) : Prop :=
   match f1 with (CountEquality s1 id1 n1) =>
     match f2 with (CountEquality s2 id2 n2) =>
       s1 = s2 -> id1 = id2 -> n1 <> n2
     end
   end.

Fixpoint formula_inconsistent_with_env (f: count_equality)
			  (e : environment) : Prop :=
  match e with
    | SingleEnv g =>  inconsistent f g
    | ConsEnv g rest => (inconsistent f g) \/ (formula_inconsistent_with_env f rest)
  end.
\end{lstlisting}
\end{minipage}

\section{The \syn{sumbool} Type}\label{sec:sumboolsection}

\syn{sumbool} is a boolean type defined in the Coq standard library module \syn{Coq.Init.Specif}. The \syn{sumbool} type captures the idea of program values that indicate which of two propositions is true~\cite{chlipalacpdt2011}. The \syn{sumbool} type is equipped with the justification of their value~\cite{Coq:manual} which help with proofs. Using a tactic like \syn{destruct}~\cite{Coq:manual} two subgoals are generated, one for each form of the \syn{sumbool} instance, however the justifications also show up as hypothesis helping with discharging of the subgoals. The definition of \syn{sumbool} from Coq library module \syn{Coq.Init.Specif} is listed in listing~\ref{lst:sumbooltypeCoq}; notice the use of the \syn{where} clause in this listing, which allows notation to be simultaneously defined (and used) when presenting a new inductive definition.

\lstset{language=Coq, frame=single, caption={\syn{sumbool} type},label={lst:sumbooltypeCoq}}
%\begin{minipage}[c]{0.95\textwidth}
\begin{lstlisting}
Inductive sumbool (A B:Prop) : Set :=
  | left : A -> {A} + {B}
  | right : B -> {A} + {B}
 where "{ A } + { B }" := (sumbool A B) : type_scope.
\end{lstlisting}
%\end{minipage}

As an example of how \syn{sumbool} type is used in the Coq standard library, see the listing~\ref{lst:sumboolofbooltypeCoq}, for the definition of the type \syn{sumbool_of_bool}. The definition in listing~\ref{lst:sumboolofbooltypeCoq} and the following description are taken from the library \syn{Coq.Bool.Sumbool}~\cite{Coq:manual}.
\begin{quote}
\syn{sumbool A B}, which is written \syn{A+B}, is the informative disjunction ``A or B'', where A and B are logical propositions. Its extraction is isomorphic to the type of booleans. A boolean is either true or false, and this is decidable. 
\end{quote}

\lstset{language=Coq, frame=single, caption={\syn{sumbool_of_bool} type},label={lst:sumboolofbooltypeCoq}}
%\begin{minipage}[c]{0.95\textwidth}
\begin{lstlisting}
Definition sumbool_of_bool : forall b:bool, {b = true} + {b = false}.
\end{lstlisting}
%\end{minipage}
With \syn{sumbool}s similar to \syn{bool} and other 2-constructor inductive types, one can use the ``if then else'' construct to select the desired case when doing proofs which makes for much more readability of the given proofs.


We have used the \syn{sumbool} type to declare and prove decision procedures that we have subsequently used in the translation functions implementing the semantics and also in the proofs. We will list the decision procedures in Section~\ref{sec:decprocs} after reviewing the translation functions in Section~\ref{sec:translationfuncs}.


\section{Types of Decisions and their Implementation in Coq}\label{sec:answerandresulttypes}

Policy based access-control languages typically use a two-valued decision set to indicate whether an access request is granted or denied. When a decision for a query is not granted, one design choice for a language is to return an explicit deny decision. However in this case deny stands for ``not permitted''. It is possible to have cases when the policy truly doesn't specify either a permit or a deny decision. In such cases arbitrarily returning the decision of deny makes it difficult to compose policies and in fact, an explicit decision of ``non applicable'' is warranted in such cases. Some languages may decide to only support permit decisions. In such languages lack of a permit decision for a query signifies a deny decision so deny decisions are not explicit. Although the policies of these languages may be more readable than those with more explicit decisions, they result in ambiguity on whether a deny decision was really intended or not. Finally some languages define an explicit decision of ``error'' for cases such as when both permit and deny decisions are reached for the same query. An explicit error decision is preferable to undefined behaviour because it can lead to improvements to policies and/or how the queries are built~\cite{Tschantz}. 

Pucella and Weissman~\cite{pucella2006} employ a two-valued decision set where the granting decision is called \syn{Permitted} and the non-granting or denying decision is called \syn{NotPermitted}. In \ac{ACCPL} we have extended the decision set to be three-valued. We have added the \syn{Unregulated} decision to \syn{Permitted} and \syn{NotPermitted} decisions. 


The policy translation functions ultimately will return one of the following answers: \\\syn{Permitted}, \syn{NotPermitted} or \syn{Unregulated}. Note that since \ac{ACCPL} was inspired by \ac{drm} systems and their policy languages (\ac{rel}s) we will use ``unregulated'' as synonymous with ``non-applicable'' for the case where a given policy does not apply to a request.


A \syn{Permitted} answer signifies that the access request has been granted. The \syn{NotPermitted} answer is for when the access request is denied. And finally the \syn{Unregulated} answer is for all the times when neither \syn{Permitted} or \syn{NotPermitted} answers are applicable. The \syn{answer} type in the Coq listing~\ref{lst:answercoq} implements the answer concept. We wrap \syn{answer}s in a \syn{result} type and add some context. Intuitively a \syn{result} will tell us whether a subject may perform an action on an asset or not, or else that the request is unregulated. In listing~\ref{lst:answercoq} we also show the definition of the helper function \syn{makeResult} that when given an \syn{answer}, a \syn{subject}, an \syn{act} and an \syn{asset} builds and returns a \syn{result}.


\lstset{language=Coq, frame=single, caption={Decision Procedures: \syn{answer} and \syn{result} types},label={lst:answercoq}}
\begin{minipage}[c]{0.95\textwidth}
\begin{lstlisting}

Inductive answer : Set :=
  | Permitted : answer
  | Unregulated : answer
  | NotPermitted : answer.
  
Inductive result : Set :=
  | Result : answer -> subject -> act -> asset -> result.
 
Definition makeResult (ans:answer)(s:subject)(ac:act)(ass:asset) : result := 
 (Result ans s ac ass).
 
\end{lstlisting}
\end{minipage}

\section{Translations}\label{sec:translationfuncs}

Intuitively a query or request asks the following question given an agreement: ``May subject \syn{s} perform an action \syn{ac} to asset \syn{a}?''. We represent a query by its components, namely the subject, action and asset that form the query question: \syn{action_from_query}, \syn{subject_from_query} and \syn{asset_from_query}. While developing the proofs for the decidability of \ac{ACCPL} we realized we needed this query specific information deep in the translation functions to be able to render unambiguous decisions and ultimately make the proofs work. We therefore pass the query components to all the translation functions starting with the \syn{trans_agreement} function (see listing~\ref{lst:transagreement}). 

In the following paragraphs, we will describe each translation function in detail the \syn{trans_agreement} function, however a high-level description of how the main algorithm (encoded in these translation functions) works is now in order. We will show the main algorithm in two separate listings based on whether the policy set in question is inclusive or exclusive. 

The first listing (Algorithm~\ref{inclusivePS}) for inclusive policy sets, shows how a positive answer to a query in the form of a \syn{Permitted} decision is reached. All cases when a decision of \syn{Unregulated} is rendered are explicitly captured and shown. The second listing (Algorithm~\ref{exclusivePS}) for exclusive policy sets, shows how a negative answer to a query in the form of a \syn{NotPermitted} decision is reached. This listing also shows that a positive decision of \syn{Permitted} is reached in exactly the same way as the case for inclusive policy sets. All cases when a decision of \syn{Unregulated} is rendered are explicitly captured and shown.

 
\begin{algorithm}      
\algCaption{Access Decision Pseudocode: Inclusive Policy Sets}       
\label{inclusivePS}   
\begin{algorithmic}        
    \IF{$asset\_from\_query = asset\_from\_agreement$}
      \IF{$subject\_from\_query$ is in $prin\_u$}
        \IF{The $preRequisite$ from the policy set holds}
          \IF{The $preRequisite$ from the policy holds}
            \IF{$action\_from\_query = action\_from\_agreement$}
              \STATE $result$ = $subject\_from\_query$ is $Permitted$ to perform $action\_from\_query$ on $asset\_from\_query$
            \ELSE
              \STATE $result$ = $Unregulated$
            \ENDIF            
          \ELSE
             \STATE $result$ = $Unregulated$
          \ENDIF
        \ELSE
          \STATE $result$ = $Unregulated$
        \ENDIF
      \ELSE
           \STATE $result$ = $Unregulated$
      \ENDIF
    \ELSE
      \STATE $result$ = $Unregulated$
    \ENDIF
\end{algorithmic}
\end{algorithm}


\begin{algorithm}      
\algCaption{Access Decision Pseudocode: Exclusive Policy Sets}            
\label{exclusivePS}   
\begin{algorithmic}        
    \IF{$asset\_from\_query = asset\_from\_agreement$}
      \IF{$subject\_from\_query$ is in $prin\_u$}
        \IF{The $preRequisite$ from the policy set holds}
          \IF{The $preRequisite$ from the policy holds}
            \IF{$action\_from\_query = action\_from\_agreement$}
              \STATE $result$ = $subject\_from\_query$ is $Permitted$ to perform $action\_from\_query$ on $asset\_from\_query$
            \ELSE
              \STATE $result$ = $Unregulated$
            \ENDIF            
          \ELSE
             \STATE $result$ = $Unregulated$
          \ENDIF
        \ELSE
          \STATE $result$ = $Unregulated$
        \ENDIF
      \ELSE
            \IF{$action\_from\_query = action\_from\_agreement$}
              \STATE $result$ = $subject\_from\_query$ is $NotPermitted$ to perform $action\_from\_query$ on $asset\_from\_query$
            \ELSE
              \STATE $result$ = $Unregulated$
            \ENDIF
      \ENDIF
    \ELSE
      \STATE $result$ = $Unregulated$
    \ENDIF
\end{algorithmic}
\end{algorithm}


Pattern matching in the body of the \syn{trans_agreement} function (in listing~\ref{lst:transagreement}), on \syn{ag}, the agreement in question, extracts the components of the agreement. These are \syn{ps}, the policy set, \syn{prin_u}, the agreement's user(s), and \syn{a}, the asset mentioned in the agreement. Notice that formal argument \syn{e} of type \syn{environment} is passed as an argument to many translation functions but will only eventually be used to get the count information from the \syn{getCount} function. As mentioned above, the components of the query or request, namely \syn{action_from_query}, \syn{subject_from_query} and \syn{asset_from_query} make up the other parameters that are passed into the next level translation function, which is the translation for a policy set called \syn{trans_ps} (in listing~\ref{lst:transpsCoq}).

%\section{Policy Combinators And Interpretation Of Results}\label{sec:policycombinators}

The nonempty list of \syn{result}s returned by the agreement translation function will have a result per primitive policy (\syn{primPolicy}) in the agreement. As we will see later when we discuss the proofs, we have proven that results containing a \syn{Permitted} answer and a \syn{NotPermitted} answer are mutually exclusive. Therefore we interpret the existence of a single (or more) \syn{Permitted} result in the nonempty list, as the decision being ``Permitted'' whereas the existence of a single (or more) \syn{NotPermitted} type result is interpreted as ``NotPermitted". We also have a proof that shows that when there exists no \syn{Permitted} or \syn{NotPermitted} results in the returned nonempty list, only \syn{Unregulated} results are possible. In this case we conclude the decision is ``Unregulated''. 

\lstset{language=Coq, frame=single, caption={Translation of Agreement},label={lst:transagreement}}
\begin{minipage}[c]{0.95\textwidth}
\begin{lstlisting}
Definition trans_agreement
   (ag:agreement)(e:environment)(action_from_query:act)
   (subject_from_query:subject)(asset_from_query:asset) : nonemptylist result :=

   match ag with
   | Agreement prin_u a ps => 
       (trans_ps e action_from_query subject_from_query asset_from_query ps prin_u a)
   end.
\end{lstlisting}
\end{minipage}

Translation of a policy set (called \syn{trans_ps} in listing~\ref{lst:transpsCoq}), takes as input \syn{e}, the environment, the subject, action and asset coming from a query: \syn{action_from_query}, \syn{subject_from_query} and \syn{asset_from_query}, \syn{ps}, the policy set, \syn{prin_u}, the agreement's user, and \syn{a}, the asset mentioned in the agreement. 

The translation starts with checking that there is a proof for the equality of \syn{asset_from_query} and \syn{a} (the asset from the agreement). This is done by using the decision procedure \syn{eq_nat_dec} (see listing~\ref{lst:eqnatdeccoq}). If so, the translation function recurses on the composing \syn{primPolicySet} and calls the local function \syn{process_single_ps}. Otherwise, the \syn{Unregulated} answer is wrapped along with the subject, action and asset coming from a query: \syn{action_from_query}, \syn{subject_from_query} and \syn{asset_from_query} by calling the \syn{makeResult} helper function. Intuitively we are just explicitly stating that policies are not applicable to queries about assets that are not mentioned in the agreement. 

The \syn{primPolicySet} which was passed to \syn{process_single_ps} was either constructed by a \syn{PIPS} constructor or by a \syn{PEPS} constructor, which distinguish between \syn{primInclusive}--\syn{PolicySet} and \syn{primExclusivePolicySet} types. The translation functions \syn{trans_policy_PIPS} (see listing~\ref{lst:transpipsCoq}) and \syn{trans_policy_PEPS} (see listing~\ref{lst:transpepsCoq}) are called in turn as the case may be.


\lstset{language=Coq, frame=single, caption={Translation of Policy Set},label={lst:transpsCoq}}
\begin{minipage}[c]{0.95\textwidth}
\begin{lstlisting}
Fixpoint trans_ps
  (e:environment)(action_from_query:act)(subject_from_query:subject)(asset_from_query:asset)
  (ps:policySet)
  (prin_u:prin)(a:asset){struct ps} : nonemptylist result :=

let process_single_ps := (fix process_single_ps (pps: primPolicySet):= 
  match pps with 
    | PIPS pips => 
        match pips with 
          | PrimitiveInclusivePolicySet prq p => 
            (trans_policy_PIPS e prq p subject_from_query prin_u a action_from_query)                
        end
    | PEPS peps => 
        match peps with 
          | PrimitiveExclusivePolicySet prq p => 
            (trans_policy_PEPS e prq p subject_from_query prin_u a action_from_query)
        end  
   end) in

if (eq_nat_dec asset_from_query a)
then (* asset_from_query = a *)  
    match ps with
      | PPS pps => process_single_ps pps
    end
else (* asset_from_query <> a *)
       (Single 
          (makeResult 
             Unregulated subject_from_query action_from_query asset_from_query)).
\end{lstlisting}
\end{minipage}


The \syn{trans_policy_PIPS} translation function for a \syn{primInclusivePolicySet} checks whether there is a proof for the subject in question, \syn{x}, being in \syn{prin_u}, by calling the decision procedure \syn{trans_prin_dec} (see listing~\ref{lst:transprindeccoq}). If so there is a check for whether the \syn{preRequisite} from the policy set holds or not by calling the decision procedure \syn{trans_preRequisite_dec} (see listing~\ref{lst:transpreRequisitedeccoq}). If so the translation function \syn{trans_policy_positive} is called. In all other cases \syn{trans_policy_unregulated} (see listing~\ref{lst:transUnregulatedCoq}) is called.

\lstset{language=Coq, frame=single, caption={Translation of Primitive Inclusive Policy Set},
    label={lst:transpipsCoq}}
\begin{minipage}[c]{0.95\textwidth}
\begin{lstlisting}
Definition trans_policy_PIPS
  (e:environment)(prq: preRequisite)(p:policy)(x:subject)
  (prin_u:prin)(a:asset)(action_from_query:act) : nonemptylist result :=
  
    if (trans_prin_dec x prin_u)
    then (* prin *)
      if (trans_preRequisite_dec e x prq (getId p) prin_u)
      then (* prin /\ prq *)
        (trans_policy_positive e x p prin_u a action_from_query)                           
      else (* prin /\ ~prq *)
        (trans_policy_unregulated e x p a action_from_query)
    else (* ~prin *)
      (trans_policy_unregulated e x p a action_from_query).
\end{lstlisting}
\end{minipage}


The \syn{trans_policy_PEPS} translation function for a \syn{primExclusivePolicySet} checks whether there is a proof for the subject in question, \syn{x}, being in \syn{prin_u}, by calling the decision procedure \syn{trans_prin_dec}. If not, the translation function \syn{trans_policy_negative} is called. This is because we are translating a \syn{primExclusivePolicySet} so the policy set is exclusive to the agreement's users implying all subjects, not in the \syn{prin_u} are denied access. In case there is a proof for the subject \syn{x} being in \syn{prin_u}, there is a check for whether the prerequisite from the policy set holds or not (by calling the decision procedure \syn{trans_preRequisite_dec}). If so, the translation function \syn{trans_policy_positive} (see listing~\ref{lst:transPositiveCoq}) is called, else \syn{trans_policy_unregulated} is called.

\lstset{language=Coq, frame=single, caption={Translation of Primitive Exclusive Policy Set},
    label={lst:transpepsCoq}}
\begin{minipage}{\linewidth}
\begin{lstlisting}

Definition trans_policy_PEPS
  (e:environment)(prq: preRequisite)(p:policy)(x:subject)
  (prin_u:prin)(a:asset)(action_from_query:act) : nonemptylist result :=
  
  if (trans_prin_dec x prin_u)
  then (* prin *)
    if (trans_preRequisite_dec e x prq (getId p) prin_u)
    then (* prin /\ prq *)
      (trans_policy_positive e x p prin_u a action_from_query)
    else (* prin /\ ~prq *)
      (trans_policy_unregulated e x p a action_from_query)
  else (* ~prin *)
    (trans_policy_negative e x p a action_from_query).

\end{lstlisting}
\end{minipage}


\newpage

The \syn{trans_policy_positive} translation function (see listing~\ref{lst:transPositiveCoq}) starts by calling the function \syn{trans_pp_list_trans_policy_positive} (see listing~\ref{lst:transListOfPositiveCoq}) on a list of \syn{primPolicy}s.
 
\lstset{language=Coq, frame=single, caption={Translation of Positive Policies},
    label={lst:transPositiveCoq}}
\begin{minipage}{\linewidth}
\begin{lstlisting}
Definition trans_policy_positive
  (e:environment)(x:subject)(p:policy)(prin_u:prin)(a:asset)
  (action_from_query: act) : nonemptylist result :=
  match p with
       | Policy pp_list => trans_pp_list_trans_policy_positive pp_list e x prin_u a action_from_query
  end.
\end{lstlisting}
\end{minipage}

The function \syn{trans_pp_list_trans_policy_positive} (see listing~\ref{lst:transListOfPositiveCoq}) handles the case of a single primitive policy by calling the handler function \syn{process_single_pp_trans_policy_positive} (see listing~\ref{lst:transSinglePositivePolicyCoq}). When there is more than a single primitive policy, the function \syn{trans_pp_list_trans_policy_positive} appends the results of processing a single primitive policy to the results for the rest of the set.

\lstset{language=Coq, frame=single, caption={Translation of Primitive Positive Policies},
    label={lst:transListOfPositiveCoq}}
\begin{lstlisting}
Fixpoint trans_pp_list_trans_policy_positive
  (pp_list:nonemptylist primPolicy)(e:environment)(x:subject)
  (prin_u:prin)(a:asset)(action_from_query: act){struct pp_list}:=
   match pp_list with
     | Single pp1 => process_single_pp_trans_policy_positive pp1 e x prin_u a action_from_query
     | NewList pp pp_list' => app_nonempty
	 (process_single_pp_trans_policy_positive pp e x prin_u a action_from_query) 
	 (trans_pp_list_trans_policy_positive pp_list' e x prin_u a action_from_query)
   end.
\end{lstlisting}


The function \syn{process_single_pp_trans_policy_positive} acting on a single \\ \syn{primPolicy} (see listing~\ref{lst:transSinglePositivePolicyCoq}), first checks whether the prerequisite from the policy holds or not, by calling the decision procedure \syn{trans_preRequisite_dec}. If so the translation function checks that there is a proof for the equality of \syn{action_from_query} and \syn{action} (the \syn{action} from the agreement) by calling the decision procedure \syn{eq_nat_dec}. If so, the translation function finally returns a result of \syn{Permitted}. In all other cases, a result of \syn{Unregulated} is returned. Intuitively we are just explicitly stating that policies are not applicable to queries about actions that are not mentioned in the agreement. 


\lstset{language=Coq, frame=single, caption={Translation of a Primitive Positive Policy},
    label={lst:transSinglePositivePolicyCoq}}
\begin{minipage}{\linewidth}
\begin{lstlisting}
Definition process_single_pp_trans_policy_positive 
   (pp: primPolicy)(e:environment)(x:subject)(prin_u:prin)
   (a:asset)(action_from_query: act)  : nonemptylist result :=
  match pp with
    | PrimitivePolicy prq' policyId action =>
        if (trans_preRequisite_dec e x prq' (Single policyId) prin_u)
        then (* prin /\ prq /\ prq' *)
          if (eq_nat_dec action_from_query action)
          then
            (Single 
              (makeResult Permitted x action_from_query a))
          else
            (Single 
              (makeResult Unregulated x action_from_query a))
        else (* prin /\ prq /\ ~prq' *)
          (Single 
              (makeResult Unregulated x action_from_query a))
  end.
\end{lstlisting}
\end{minipage}


The \syn{trans_policy_negative} translation function (see listing~\ref{lst:transNegativeCoq}) starts by calling the function \syn{trans_pp_list_trans_policy_negative} on a list of \syn{primPolicy}s. Note that \syn{trans_pp_list_trans_policy_negative} is not listed here since it follows the exact pattern as \syn{trans_pp_list_trans_policy_positive} (see listing~\ref{lst:transListOfPositiveCoq}).

\lstset{language=Coq, frame=single, caption={Translation of Negative Policies},
    label={lst:transNegativeCoq}}
\begin{minipage}[c]{0.95\textwidth}
\begin{lstlisting}
Fixpoint trans_policy_negative
  (e:environment)(x:subject)(p:policy)(a:asset)
  (action_from_query: act){struct p} : nonemptylist result :=  
  match p with  
       | Policy pp_list => trans_pp_list_trans_policy_negative pp_list x a action_from_query
  end.
\end{lstlisting}
\end{minipage}

The function \syn{process_single_pp_trans_policy_negative} acting on a single \\ \syn{primPolicy} (see listing~\ref{lst:transSingleNegativePolicyCoq}), first checks that there is a proof for the equality of \syn{action_from_query} and \syn{action} (the \syn{action} from the agreement) by calling the decision procedure \syn{eq_nat_dec}. If so, the translation function returns a result of \syn{NotPermitted}. Otherwise, a result of \syn{Unregulated} is returned. Note that \syn{trans_policy_negative} is called from a context where \syn{trans_prin_dec} is not provable, which means that subject \syn{x}, is not in \syn{prin_u}. Also note that compared to \syn{trans_policy_positive} we don't check for whether \syn{preRequisite} from the policy holds or not, since \syn{trans_policy_negative} is only called for policy sets of type \syn{primExclusivePolicySet}.


\lstset{language=Coq, frame=single, caption={Translation of a Primitive Negative Policy},
    label={lst:transSingleNegativePolicyCoq}}
\begin{minipage}[c]{0.95\textwidth}
\begin{lstlisting}
Definition process_single_pp_trans_policy_negative
   (pp: primPolicy)(x:subject)
   (a:asset)(action_from_query: act)  : nonemptylist result :=
  match pp with
    | PrimitivePolicy prq' policyId action =>
          if (eq_nat_dec action_from_query action)
          then
            (Single 
              (makeResult NotPermitted x action_from_query a))
          else
            (Single 
              (makeResult Unregulated x action_from_query a))
  end.
\end{lstlisting}
\end{minipage}

The \syn{trans_policy_unregulated} translation function (see listing~\ref{lst:transUnregulatedCoq}) starts by calling the function \syn{trans_pp_list_trans_policy_unregulated} on a list of \syn{primPolicy}s. Note that \syn{trans_pp_list_trans_policy_unregulated} is not listed here since it follows the exact pattern as \syn{trans_pp_list_trans_policy_positive} (see listing~\ref{lst:transListOfPositiveCoq}).

\lstset{language=Coq, frame=single, caption={Translation of Unregulated Policies},
    label={lst:transUnregulatedCoq}}
\begin{lstlisting}   
Fixpoint trans_policy_unregulated
  (e:environment)(x:subject)(p:policy)(a:asset)
  (action_from_query: act){struct p} : nonemptylist result :=
  match p with
       | Policy pp_list => trans_pp_list_trans_policy_unregulated pp_list x a action_from_query
  end.
\end{lstlisting}


The function \syn{process_single_pp_trans_policy_unregulated} acting on a single \\ \syn{primPolicy} (see listing~\ref{lst:transSingleUnregulatedPolicyCoq}) simply returns a result of \syn{Unregulated}.



\lstset{language=Coq, frame=single, caption={Translation of an Unregulated Policy},
    label={lst:transSingleUnregulatedPolicyCoq}}
%\begin{minipage}{\linewidth}
\begin{lstlisting}
Definition process_single_pp_trans_policy_unregulated
   (pp: primPolicy)(x:subject)
   (a:asset)(action_from_query: act)  : nonemptylist result :=
  match pp with
    | PrimitivePolicy prq' policyId action =>
        (Single (makeResult Unregulated x action_from_query a))
  end.
\end{lstlisting}




% COQ
Translation of a \syn{prin} (called \syn{trans_prin} in listing~\ref{lst:transprinCoq}) takes as input, a subject \syn{x}, and a principal \syn{p}, and return the proposition ``subject \syn{x} is in principal \syn{p}''. More specifically \syn{trans_prin} proceeds based on whether the principal \syn{p} is a single subject or a list of subjects. If \syn{p} is a single subject \syn{s}, the proposition \syn{x=s} is returned. Otherwise the disjunction of the proposition \syn{x=s} with the proposition ``subject \syn{x} is in principal \syn{rest}'' is returned.

\lstset{language=Coq, frame=single, caption={Translation of Prin},label={lst:transprinCoq}}
\begin{minipage}{\linewidth}
\begin{lstlisting}

Fixpoint trans_prin
  (x:subject)(p: prin): Prop :=

  match p with
    | Single s => (x=s)
    | NewList s rest => ((x=s) \/ trans_prin x rest)
  end.
\end{lstlisting}
\end{minipage}


The translation of a \syn{prerequisite} (called \syn{trans_preRequisite} in listing~\ref{lst:transpreRequisiteCoq}) takes as input an environment \syn{e}, a subject \syn{x}, a \syn{preRequisite} \syn{prq} to translate, a set of policy identifiers \syn{IDs} (identifiers of policies implied by the \syn{prq}), the agreement's user(s) \syn{prin_u}, and proceeds by case analysis on the structure of the \syn{prerequisite}. A \syn{prerequisite} is either \syn{TruePrq}, \syn{Constraint},  \syn{NotCons}, or \syn{AndPrqs}. 

In listing~\ref{lst:transpreRequisiteCoq} the translation for \syn{TruePrq} is the proposition \syn{True}, the translations for \syn{Constraint}, \syn{NotCons} and \syn{AndPrqs} simply call respective translation functions for corresponding types namely \syn{trans_constraint} (see listing~\ref{lst:transconstraintCoq}) and \syn{trans_notCons} (see listing~\ref{lst:transnotConsCoq} and \syn{trans_andPrqs}. 


\lstset{language=Coq, frame=single, caption={Translation of Prerequisite},label={lst:transpreRequisiteCoq}}
\begin{minipage}[c]{0.95\textwidth}
\begin{lstlisting}

Definition trans_preRequisite
  (e:environment)(x:subject)(prq:preRequisite)(IDs:nonemptylist policyId)(prin_u:prin) : Prop := 

  match prq with
    | TruePrq => True
    | Constraint const => trans_constraint e x const IDs prin_u  
    | NotCons const => trans_notCons e x const IDs prin_u 
    | AndPrqs prqs =>  trans_andPrqs x prq IDs prin_u a
  end.
\end{lstlisting}
\end{minipage}

The translation of a \syn{constraint} (called \syn{trans_constraint} in listing~\ref{lst:transconstraintCoq}) takes as input 
an environment \syn{e}, a subject \syn{x}, a \syn{constraint} \syn{const} to translate, a set of policy identifiers \syn{IDs} (identifiers of policies implied by the parent prerequisite), the agreement's user(s) \syn{prin_u} and proceeds by case analysis on the structure of the \syn{constraint}. A \syn{constraint} is either \syn{Principal}, \syn{Count} or a \syn{CountByPrin}. The translation for \syn{Principal} calls the translation function \syn{trans_prin}, for the \syn{prn} that accompanies the \syn{const} constraint. The translation for \syn{Count} and \syn{CountByPrin} both return the proposition which is the translation function \syn{trans_count} (see listing~\ref{lst:transcountCoq}).


\lstset{language=Coq, frame=single, caption={Translation of Constraint},label={lst:transconstraintCoq}}
\begin{minipage}[c]{0.95\textwidth}
\begin{lstlisting}

Fixpoint trans_constraint 
  (e:environment)(x:subject)(const:constraint)(IDs:nonemptylist policyId)
  (prin_u:prin){struct const} : Prop := 
  match const with
    | Principal prn => trans_prin x prn
  
    | Count n => trans_count e n IDs prin_u

    | CountByPrin prn n => trans_count e n IDs prn 

  end.
  
\end{lstlisting}
\end{minipage}

The translation of a \syn{NotCons} (called \syn{trans_notCons} in listing~\ref{lst:transnotConsCoq}) takes as input 
an environment \syn{e}, a subject \syn{x}, a \syn{constraint} \syn{const} to translate, a set of policy identifiers \syn{IDs} (identifiers of policies implied by the parent prerequisite), the agreement's user(s) \syn{prin_u} and proceeds to return the negation of the proposition \syn{trans_constraint} (see listing~\ref{lst:transconstraintCoq}).

\lstset{language=Coq, frame=single, caption={Translation of Negation of Constraint},label={lst:transnotConsCoq}}
\begin{lstlisting}

Definition trans_notCons
  (e:environment)(x:subject)(const:constraint)(IDs:nonemptylist policyId)(prin_u:prin) : Prop :=
  ~ (trans_constraint e x const IDs prin_u).
\end{lstlisting}


The translation of a \syn{Count} or a \syn{CountByPrin} (called \syn{trans_count} in listing~\ref{lst:transcountCoq}) takes as input an environment \syn{e}, \syn{n} the total number of times the subjects mentioned in \syn{prin_u} may invoke the 
the policies identified by \syn{IDs}.
A local variable \syn{running_total} has the current count of the number of times subjects in \syn{prin_u} have invoked the policies. Finally the proposition \syn{running_total < n} is returned as the translation for a \syn{Count} or a \syn{CountByPrin}. Functions \syn{process_two_lists} and \syn{trans_count_aux} are defined in listings~\ref{lst:transcountprocesstwolistsCoq} and~\ref{lst:transcountauxCoq} respectively.

Note that the only difference between translations for a \syn{Count} and a \syn{CountByPrin} is the additional \syn{prn} parameter for \syn{CountByPrin} which allows for getting counts for subjects not necessarily the same as \syn{prin_u}, the agreement's user(s). We therefore omit the definition for \syn{CountByPrin}.


\lstset{language=Coq, frame=single, caption={Translation of Count},label={lst:transcountCoq}}
\begin{minipage}[c]{0.95\textwidth}
\begin{lstlisting}
Fixpoint trans_count
  (e:environment)(n:nat)(IDs:nonemptylist policyId)
  (prin_u:prin) : Prop :=
  let ids_and_subjects := process_two_lists IDs prin_u in
  let running_total := trans_count_aux e ids_and_subjects in
  running_total < n.
\end{lstlisting}
\end{minipage}


To implement the translation for a \syn{Count} or a \syn{CountByPrin} we start by calling an auxiliary function \syn{process_two_lists} (see listing~\ref{lst:transcountprocesstwolistsCoq}) that effectively returns a new list composed of pairs of members of the first list and the second list (the cross-product of the two input lists). In the case of \syn{trans_count}, the call is ``\syn{process_two_lists} \syn{IDs} \syn{prin_u}'' which returns a list of pairs of \syn{policyId} and \syn{subject} namely \syn{ids_and_subjects}. \syn{ids_and_subjects} is then passed to a locally defined function called \syn{trans_count_aux}.

\lstset{language=Coq, frame=single, caption={Count Helper: Cross-Product of Two Lists, \syn{process_two_lists}},label={lst:transcountprocesstwolistsCoq}}
%\begin{minipage}[c]{0.95\textwidth}
\begin{lstlisting}

Section Process_Lists.

Variable X : Set.
Variable Y : Set.

Fixpoint process_two_lists (l1 : nonemptylist X) (l2 : nonemptylist Y) :  nonemptylist (Twos X Y) :=

let process_element_list := (fix process_element_list (e1 : X) (l2 : nonemptylist Y) :	nonemptylist (Twos X Y) :=
  match l2 with
    | Single s => Single (mkTwos e1 s)
    | NewList s rest => app_nonempty (Single (mkTwos e1 s)) (process_element_list e1 rest)
  end) in

  match l1 with
    | Single s => process_element_list s l2
    | NewList s rest => app_nonempty (process_element_list s l2) (process_two_lists rest l2)
  end.

End Process_Lists.

\end{lstlisting}
%\end{minipage}


The function \syn{trans_count_aux} (see listing~\ref{lst:transcountauxCoq}) returns the current count for a single pair of \syn{policyId} and \syn{subject} (the call to \syn{getCount} which looks up the environment \syn{e} and returns the current count per each \syn{subject} and \syn{policyId}) and for a list of pairs of \syn{policyId} and \syn{subject}s, the addition of \syn{get_count} (for the first pair) and \syn{trans_count_aux}s (for the rest of the pairs) is returned. 

\lstset{language=Coq, frame=single, caption={Count Helper: \syn{trans_count_aux}},label={lst:transcountauxCoq}}
\begin{minipage}[c]{0.95\textwidth}
\begin{lstlisting}

Fixpoint trans_count_aux (e:environment)(ids_and_subjects : nonemptylist (Twos policyId subject)) : nat :=
  match ids_and_subjects with
	| Single pair1 => getCount e (right pair1) (left pair1)
	| NewList pair1 rest_pairs =>
	    (getCount e (right pair1)(left pair1)) +
	    (trans_count_aux e rest_pairs)
  end.
\end{lstlisting}
\end{minipage}


At this point we have covered the semantics of \ac{ACCPL} in terms of translation functions starting with \syn{trans_agreement}. In the following listings we will show an example agreement from~\cite{pucella2006} expressed as abstract syntax and as \ac{ACCPL}. We will also show how the fully built agreement looks like as \ac{ACCPL} constructs and we will show how a number of queries can be asked and what the results would be and whether the results are as expected.

The statement in listing~\ref{lst:pucellatwofourexampleAST} expresses that ``the asset TheReport may be printed and displayed by Alice and Bob only and with the restriction that Alice and Bob together may print and display the asset five times".

\lstset{language=Pucella2006}
\begin{minipage}[c]{0.95\textwidth}
\begin{lstlisting}[frame=single, caption={Agreement of Example 2.4}, label={lst:pucellatwofourexampleAST}, mathescape]
agreement
 for Alice and Bob and Charlie
 about The Report 
 with and[{Alice, Bob}, {Alice, Bob}<count[5]>] -> and[True =>$_{id1}$ print, True =>$_{id2}$ display].
\end{lstlisting}
\end{minipage} 

The policy statement in~\ref{lst:pucellatwofourexampleAST} is expressed in \ac{ACCPL} using Coq constructs, in listing~\ref{lst:pucellatwofourexamplecoq}.

\lstset{language=Coq, frame=single, caption={Expressing Agreement of Example 2.4 in ACCPL}, label={lst:pucellatwofourexamplecoq}}
\begin{minipage}[c]{0.95\textwidth}
\begin{lstlisting}
Definition ps_24_p1:primPolicy := 
  (PrimitivePolicy (makePreRequisite (TruePrq)) id1 Print).

Definition ps_24_p2:primPolicy := 
  (PrimitivePolicy (makePreRequisite (TruePrq)) id2 Display).

Definition ps_24_p:policy := 
  (Policy (NewList ps_24_p1 (Single ps_24_p2))).

Definition ps_24_prq1:primPreRequisite := 
  (Constraint (Principal (NewList Alice (Single Bob)))).

Definition ps_24_prq2:primPreRequisite := 
  (Constraint (CountByPrin (NewList Alice (Single Bob)) 1)).
 
Definition ps_24_prq:preRequisite := 
  (PreRequisite (NewList ps_24_prq1 (Single ps_24_prq2))).

Definition ps_24:primPolicySet :=
  PIPS (PrimitiveInclusivePolicySet
    ps_24_prq ps_24_p).

Definition A24 := Agreement (NewList Alice (Single Bob)) TheReport (PPS ps_24).
\end{lstlisting}
\end{minipage} 

The policy statement in listing~\ref{lst:pucellatwofourexampleAST} as an \ac{ACCPL} construct is shown, in listing~\ref{lst:pucellatwofourexamplecoq2}.

\lstset{language=Coq, frame=single, caption={Fully Built Agreement of Example 2.4 in ACCPL}, label={lst:pucellatwofourexamplecoq2}}
\begin{minipage}[c]{0.95\textwidth}
\begin{lstlisting}
Agreement (Alice, [Bob]) TheReport
         (PPS
            (PIPS
               (PrimitiveInclusivePolicySet
                  (PreRequisite
                     (Constraint (Principal (Alice, [Bob])),
                      [Constraint (CountByPrin (Alice, [Bob]) 1)])
                  (Policy
                     (PrimitivePolicy (PreRequisite [TruePrq]) id1 Print,
                      [PrimitivePolicy (PreRequisite [TruePrq]) id2 Display])))))

\end{lstlisting}
\end{minipage} 

The default environment for the example agreement is shown in listing~\ref{lst:pucellatwofourexamplecoq3}. We set all the counts to zero initially.

\lstset{language=Coq, frame=single, caption={The Default Environment for Example 2.4 in ACCPL}, label={lst:pucellatwofourexamplecoq3}}
\begin{minipage}[c]{0.95\textwidth}
\begin{lstlisting}
Definition e_24 : environment :=
 (ConsEnv (make_count_equality Bob id1 0)
   (ConsEnv (make_count_equality Bob id2 0)
     (ConsEnv (make_count_equality Alice id1 0)
       (SingleEnv (make_count_equality Alice id2 0))))).
\end{lstlisting}
\end{minipage} 

The query in listing~\ref{lst:pucellatwofourexamplecoq4}, asks whether Alice may print the asset TheReport. We expect the action to be allowed based on the first primitive policy. The second primitive policy mentions a different action (e.g. Display) so we expect the request is unregulated. 

\lstset{language=Coq, frame=single, caption={Query: May Alice Print The Report for Example 2.4 in ACCPL}, label={lst:pucellatwofourexamplecoq4}}
\begin{minipage}[c]{0.95\textwidth}
\begin{lstlisting}
Eval compute in (trans_agreement e_24 A24 Print Alice TheReport). = 
   Result Permitted Alice Print TheReport, [Result Unregulated Alice Print TheReport]
\end{lstlisting}
\end{minipage} 

The query in listing~\ref{lst:pucellatwofourexamplecoq5}, asks whether Charlie may print the asset TheReport. We expect the action to be unregulated because Charlie fails to meet the prerequisites specified for the policy set. 

\lstset{language=Coq, frame=single, caption={Query: May Charlie Print The Report for Example 2.4 in ACCPL}, label={lst:pucellatwofourexamplecoq5}}
\begin{minipage}[c]{0.95\textwidth}
\begin{lstlisting}
Eval compute in (trans_agreement e_24 A24 Print Charlie TheReport). = 
   Result Unregulated Charlie Print TheReport, [Result Unregulated Charlie Print TheReport]
\end{lstlisting}
\end{minipage} 

The query in listing~\ref{lst:pucellatwofourexamplecoq6}, asks whether Alice may display the asset TheReport. We expect the action to be allowed based on the second primitive policy. The first primitive policy mentions a different action (e.g. Print) so we expect the request is unregulated. 

\lstset{language=Coq, frame=single, caption={Query: May Alice Display The Report for Example 2.4 in ACCPL}, label={lst:pucellatwofourexamplecoq6}}
\begin{minipage}[c]{0.95\textwidth}
\begin{lstlisting}
Eval compute in (trans_agreement e_24 A24 Display Alice TheReport). = 
   Result Unregulated Alice Display TheReport, [Result Permitted Alice Display TheReport]
\end{lstlisting}
\end{minipage} 






\section{Decision Procedures}\label{sec:decprocs}

In the following we will list only the declaration of each decision procedure and the accompanying functions. The individual decision procedures were used as a building block to make the final decidability proofs for \ac{ACCPL} possible. We will refer to some of these listings later when we discuss the translations and/or proofs using the decision procedures. Please refer to the source code where the usage of a decision procedure in not mentioned in the following.

The decision procedure \syn{eq_nat_dec} takes two natural numbers, and declares that either there is a proof for their equality or there exists a proof for their inequality (see listing~\ref{lst:eqnatdeccoq}).

\lstset{language=Coq, frame=single, caption={Decision Procedures: \syn{eq_nat_dec}~\cite{Coq:manual}},label={lst:eqnatdeccoq}}
\begin{lstlisting}
Theorem eq_nat_dec : forall n m, {n = m} + {n <> m}.
\end{lstlisting}

The decision procedure \syn{subject_in_prin_dec} (see listing~\ref{lst:issubjectinprincoq}) declares that either there is a proof for the proposition \syn{is_subject_in_prin} (see listing~\ref{lst:issubjectinprincoq}) or there exists a proof for its negation. 

\lstset{language=Coq, frame=single, caption={Decision Procedures: \syn{subject_in_prin_dec}},label={lst:issubjectinprincoq}}
\begin{lstlisting}
Theorem subject_in_prin_dec :
    forall (a:subject) (l:prin), {is_subject_in_prin a l} + {~ is_subject_in_prin a l}.

Fixpoint is_subject_in_prin (s:subject)(p:prin): Prop :=
  match p with
  | Single s'  => s=s'
  | NewList s' rest => s=s' \/ (is_subject_in_prin s rest)
  end.
\end{lstlisting}

The decision procedure \syn{trans_prin_dec} declares that either there is a proof for the proposition \syn{trans_prin} or there exists a proof for its negation (see listing~\ref{lst:transprindeccoq}). The translation function \syn{trans_prin} is the proposition ``subject x is in prin'' (see listing~\ref{lst:transprinCoq}).

\lstset{language=Coq, frame=single, caption={Decision Procedures: \syn{trans_prin_dec}},label={lst:transprindeccoq}}
\begin{minipage}[c]{0.95\textwidth}
\begin{lstlisting}
Theorem trans_prin_dec :
   forall (x:subject)(p: prin), {trans_prin x p} + {~trans_prin x p}.
\end{lstlisting}
\end{minipage}

The decision procedure \syn{trans_count_dec} declares that either there is a proof for the proposition \syn{trans_count} or there exists a proof for its negation (see listing~\ref{lst:transcountdeccoq}). Recall the translation function \syn{trans_count} is mostly the proposition \syn{running_total < n} with some local context (see listing~\ref{lst:transcountCoq}).

\lstset{language=Coq, frame=single, caption={Decision Procedures: \syn{trans_count_dec}},label={lst:transcountdeccoq}}
\begin{lstlisting}
Theorem trans_count_dec: 
  forall (e:environment)(n:nat)(IDs:nonemptylist policyId)(prin_u:prin), 
    {trans_count e n IDs prin_u} + {~ trans_count e n IDs prin_u}.
\end{lstlisting}

The decision procedure \syn{trans_constraint_dec} declares that either there is a proof for the proposition \syn{trans_constraint} or there exists a proof for its negation (see listing~\ref{lst:transconstraintdeccoq}). The translation function \syn{trans_constraint} is listed in listing~\ref{lst:transconstraintCoq}. 


\lstset{language=Coq, frame=single, caption={Decision Procedures: \syn{trans_constraint_dec}},label={lst:transconstraintdeccoq}}
\begin{minipage}[c]{0.95\textwidth}
\begin{lstlisting}
Theorem trans_constraint_dec :
    forall (e:environment)(x:subject)(const:constraint)(IDs:nonemptylist policyId)(prin_u:prin),
       {trans_constraint e x const IDs prin_u} + {~trans_constraint e x const IDs prin_u}.
\end{lstlisting}
\end{minipage}

The decision procedure \syn{trans_notCons_dec} declares that either there is a proof for the proposition \syn{trans_notCons} or there exists a proof for its negation (see listing~\ref{lst:transnotConsdeccoq}). The translation function \syn{trans_notCons} is listed in listing~\ref{lst:transnotConsCoq}. The proof for \syn{trans_notCons_dec} not shown here uses two helper theorems. One is another decision procedure, \\ \syn{trans_negation_constraint_dec} which declares that either there exists a proof for the negation of \syn{trans_constraint} or there is a proof for the negation of the negation of \syn{trans_constraint}. The other helper theorem is \syn{double_neg_constraint} which basically declares that \syn{trans_constraint} implies its own double negation. The helpers are listed in~\ref{lst:transnegationconstraintdeccoq}.

\lstset{language=Coq, frame=single, caption={Decision Procedures: \syn{trans_negation_constraint_dec}},label={lst:transnegationconstraintdeccoq}}
\begin{lstlisting}
Theorem double_neg_constraint:
    forall (e:environment)(x:subject)(const:constraint)(IDs:nonemptylist policyId)(prin_u:prin),
       (trans_constraint e x const IDs prin_u) -> ~~(trans_constraint e x const IDs prin_u).
  
Theorem trans_negation_constraint_dec :
    forall (e:environment)(x:subject)(const:constraint)(IDs:nonemptylist policyId)(prin_u:prin),
       {~trans_constraint e x const IDs prin_u} + {~~trans_constraint e x const IDs prin_u}.
       
\end{lstlisting}

\lstset{language=Coq, frame=single, caption={Decision Procedures: \syn{trans_notCons_dec}},label={lst:transnotConsdeccoq}}
\begin{minipage}[c]{0.95\textwidth}
\begin{lstlisting}
Theorem trans_notCons_dec :
    forall (e:environment)(x:subject)(const:constraint)(IDs:nonemptylist policyId)(prin_u:prin),
       {trans_notCons e x const IDs prin_u} + {~ trans_notCons e x const IDs prin_u}.

\end{lstlisting}
\end{minipage}

The decision procedure \syn{trans_preRequisite_dec} declares that either there is a proof for the proposition \syn{trans_preRequisite} or there exists a proof for its negation (see listing~\ref{lst:transpreRequisitedeccoq}). The translation function \syn{trans_preRequisite} is listed in listing~\ref{lst:transpreRequisiteCoq}). 


\lstset{language=Coq, frame=single, caption={Decision Procedures: \syn{trans_preRequisite_dec}},label={lst:transpreRequisitedeccoq}}
\begin{lstlisting}
Theorem trans_preRequisite_dec :
    forall (e:environment)(x:subject)(prq:preRequisite)(IDs:nonemptylist policyId)(prin_u:prin),
       {trans_preRequisite e x prq IDs prin_u} + {~ trans_preRequisite e x prq IDs prin_u}.
\end{lstlisting}


%\section{Queries}
%
%Ultimately policy statements describing an agreement will be used to enforce those agreements. To enforce policy agreements, access queries or requests are asked from the policy engine and access is granted or denied based on the answer.
%
%In this chapter we will review our encoding of queries in Coq and Coq representations of other definitions used to prove our decidability results.
%
%Queries are tuples of the form \syn{(agreement, s, action, a, e)}. The tuple corresponds to the question of determining whether an agreement implies that a subject \syn{s} may perform action \syn{action} on an asset \syn{a} given the environment \syn{e}. The Coq representation is listed in listing~\ref{lst:querycoq}. 
%
%\lstset{language=Coq, frame=single, caption={Queries},label={lst:querycoq}}
%\begin{minipage}[c]{0.95\textwidth}
%\begin{lstlisting}
%
%Inductive single_query : Set := 
%   | SingletonQuery : agreement -> subject -> act -> asset -> environment -> single_query.
%   
%\end{lstlisting}
%\end{minipage}
%
%\subsection{Answering Queries}\label{subsec:answerqueriesodrl}
%
%Access control models typically use a two-valued decision set to indicate whether an access request is granted or denied. However many access-control models have extended this two valued set with two more values: non applicable policies to certain queries and an error case accounting for all possible error conditions when evaluating an access request~\cite{DBLP:conf/sacmat/MorissetZ14}. Answering a query could therefore lead to one of four values: error(listing~\ref{lst:errordecision}), permitted(listing~\ref{lst:permitdecision}), denied(listing~\ref{lst:denydecision}) and ``not applicable'' (listing~\ref{lst:notapplicabledecision}) as defined in~\cite{Tschantz}. Note that \syn{e} denotes 'an environment being consistent in the following listings. As we will see later in chapter~\ref{chap:results} (listing~\ref{lst:permandnotpermmutualexclusive}) \ac{ACCPL} ends up using a three valued decision set with only the permit, deny and non applicable cases. 
%
%\lstset{mathescape, language=AST} 
%\begin{lstlisting}[frame=single, caption={Answerable Queries: Error},label={lst:errordecision}]
%
%$([\![ agreement]\!] \land e) \implies Permitted(s, act, a)$ and $([\![ agreement]\!] \land e) \implies \lnot Permitted(s, act, a)$
%
%\end{lstlisting}
%
%\lstset{mathescape, language=AST} 
%\begin{lstlisting}[frame=single, caption={Answerable Queries: Permit},label={lst:permitdecision}]
%
%$([\![ agreement]\!]) \land e) \implies Permitted(s, act, a)$ and $([\![ agreement]\!] \land e) \notimplies \lnot Permitted(s, act, a)$
%
%\end{lstlisting}
%
%\lstset{mathescape, language=AST} 
%\begin{lstlisting}[frame=single, caption={Answerable Queries: Deny},label={lst:denydecision}]
%
%$([\![ agreement]\!] \land e) \notimplies Permitted(s, act, a)$ and $([\![ agreement]\!] \land e) \implies \lnot Permitted(s, act, a)$
%
%\end{lstlisting}
%
%\lstset{mathescape, language=AST} 
%\begin{lstlisting}[frame=single, caption={Answerable Queries: Not Applicable},label={lst:notapplicabledecision}]
%
%$([\![ agreement]\!] \land e) \notimplies Permitted(s, act, a)$ and $([\![ agreement]\!] \land e) \notimplies \lnot Permitted(s, act, a)$
%
%\end{lstlisting}






















%%======================================================================
\chapter{Queries}\label{chap:queries}
%======================================================================

% ---------------------------------------------------------

\section{Introduction}


We first mentioned queries in chapter~\ref{chap:semantics} on page~\pageref{chap:semantics}. Ultimately policy statements describing an agreement will be used to enforce those agreements. To enforce policy agreements, access queries are asked from the policy engine and access is granted or denied based on the answer.

By defining formal semantics for \ac{odrl} the authors of~\cite{pucella2006} were able to prove that answering a query on whether access should be granted or not, is decidable and NP-hard for the full \ac{odrl}. 

In this chapter we will review our encoding of queries in Coq and Coq representations of other definitions used in~\cite{pucella2006} which we will use to prove decidability results of our own. 


\section{Queries}

Queries are tuples of the form $(A, s, action, a, e)$ in~\cite{pucella2006}. The tuple corresponds to the question of determining whether a set $A$ of agreements imply that a subject $s$ may perform action $action$ on an asset $a$ given the environment $e$. The Coq representation is listed in listing~\ref{lst:querycoq}. We distinguish single agreement queries from multiple agreement queries by defining two separate types: $single\_query$ and $general\_query$.

\lstset{language=Coq, frame=single, caption={Queries},label={lst:querycoq}}
\begin{minipage}[c]{0.95\textwidth}
\begin{lstlisting}

Inductive single_query : Set := 
   | SingletonQuery : agreement -> subject -> act -> asset -> environment -> single_query.
   

Inductive general_query : Set := 
   | GeneralQuery : nonemptylist agreement -> subject -> act -> asset -> environment -> general_query.
\end{lstlisting}
\end{minipage}

\section{Answering Queries}\label{sec:answerqueriesodrl}

Answering a query as defined earlier can lead to one of four outcomes: error(listing~\ref{lst:errordecision}), permitted(listing~\ref{lst:permitdecision}), denied(listing~\ref{lst:denydecision}) and ``not applicable'' (listing~\ref{lst:notapplicabledecision}) defined in~\cite{Tschantz}. Note that $e$ denotes 'an environment being consistent' (see section~\ref{sec:envConsistentP}) in the following listings.

\lstset{mathescape, language=AST} 
\begin{lstlisting}[frame=single, caption={Answerable Queries: Error},label={lst:errordecision}]

$(\bigwedge [\![ agreement]\!]) \land e \implies Permitted(s, act, a)$ and $(\bigwedge [\![ agreement]\!]) \land e \implies \lnot Permitted(s, act, a)$

\end{lstlisting}

\lstset{mathescape, language=AST} 
\begin{lstlisting}[frame=single, caption={Answerable Queries: Permit},label={lst:permitdecision}]

$(\bigwedge [\![ agreement]\!]) \land e \implies Permitted(s, act, a)$ and $(\bigwedge [\![ agreement]\!]) \land e \notimplies \lnot Permitted(s, act, a)$

\end{lstlisting}

\lstset{mathescape, language=AST} 
\begin{lstlisting}[frame=single, caption={Answerable Queries: Deny},label={lst:denydecision}]

$(\bigwedge [\![ agreement]\!]) \land e \notimplies Permitted(s, act, a)$ and $(\bigwedge [\![ agreement]\!]) \land e \implies \lnot Permitted(s, act, a)$

\end{lstlisting}

\lstset{mathescape, language=AST} 
\begin{lstlisting}[frame=single, caption={Answerable Queries: Not Applicable},label={lst:notapplicabledecision}]

$(\bigwedge [\![ agreement]\!]) \land e \notimplies Permitted(s, act, a)$ and $(\bigwedge [\![ agreement]\!]) \land e \notimplies \lnot Permitted(s, act, a)$

\end{lstlisting}


In~\cite{pucella2006} a slightly different formulation is used to denote the same four decision types. ``Query Inconsistent'',``Permission Granted'', ``Permission Denied'' and ``Permission Unregulated''. They define the formulas $f^{+}_q$ and $f^{-}_q$ as below (see listings~\ref{lst:fplusformula} and~\ref{lst:fminusformula}).

\lstset{mathescape, language=AST} 
\begin{lstlisting}[frame=single, caption={$f^{+}_q$},label={lst:fplusformula}]

$f^{+}_q \triangleq (\bigwedge [\![ agreement]\!]) \implies Permitted(s, act, a)$ 

\end{lstlisting}

\lstset{mathescape, language=AST} 
\begin{lstlisting}[frame=single, caption={$f^{-}_q$},label={lst:fminusformula}]

$f^{-}_q \triangleq (\bigwedge [\![ agreement]\!]) \implies \lnot Permitted(s, act, a)$ 

\end{lstlisting}

Now answering the queries will depend on the \emph{E-validity} of $f^{+}_q$ and $f^{-}_q$. E-validity or the consistency of the environment is not captured explicitly in~\cite{pucella2006} (see listings~\ref{lst:queryinconsistentdecision}, ~\ref{lst:permissiongranteddecision}, ~\ref{lst:permissiondenieddecision} and~\ref{lst:permissionunregulateddecision}) however the decision of which answer a query results in takes the consistency of the environment into account. In this thesis we will encode and use the decision algorithms in~\cite{pucella2006}.


\lstset{mathescape, language=AST} 
\begin{lstlisting}[frame=single, caption={Answerable Queries: Query Inconsistent},label={lst:queryinconsistentdecision}]

$f^{+}_q$ and $f^{-}_q$ both hold

\end{lstlisting}

\lstset{mathescape, language=AST} 
\begin{lstlisting}[frame=single, caption={Answerable Queries: Permission Granted},label={lst:permissiongranteddecision}]

$f^{+}_q$ holds and $f^{-}_q$ does not hold

\end{lstlisting}

\lstset{mathescape, language=AST} 
\begin{lstlisting}[frame=single, caption={Answerable Queries: Permission Denied},label={lst:permissiondenieddecision}]

$f^{+}_q$ does not hold and $f^{-}_q$ holds

\end{lstlisting}

\lstset{mathescape, language=AST} 
\begin{lstlisting}[frame=single, caption={Answerable Queries: Permission Unregulated},label={lst:permissionunregulateddecision}]

$f^{+}_q$ does not hold and $f^{-}_q$ does not hold

\end{lstlisting}

























 
%======================================================================
\chapter{Formal Proofs and Decidability}\label{chap:results}
%======================================================================

                  
%----------------------------------------------------------------------
The tension in designing a policy language is usually between how to make the language expressive enough, such that the design goals for the policy language may be expressed, and how to make the policies verifiable with respect to the stated goals. A central question or goal in policy language design is whether a permission is implied by a set of policy statements. This is referred to as the decidability problem in this thesis. For example, Halpern and Weissman show that in general determining whether a permission is implied by a set of policy statements in \ac{xrml} is undecidable~\cite{HalpernW08} whereas Pucella and Weissman show that answering such questions is decidable but NP-hard for \ac{odrl}~\cite{pucella2006}.
 

We had set out to prove formally that the \ac{ACCPL} language was decidable. Recall that we specified the semantics of \ac{ACCPL} using a translation from policy statements to decisions. The goal was to use the translation algorithm to determine when a request for accessing a resource is implied by a given policy, when the request is not implied by the policy and finally when the request is not regulated by the given policy. More importantly we wanted to show that answering access requests always produced a decision and that the translation algorithm terminated on all input policies with a decision of granted, denied or unregulated. In this chapter we discuss the decidability proofs and other main results we had set out to accomplish. We will also show that the translation algorithm meets its specification and computes the correct decision, with respect to its specification, given a policy. 


\ac{ACCPL} started out as a reduced version of Pucella and Weissman's subset of \ac{odrl}\\~\cite{pucella2006} but based on the direction of the proofs we were developing significant changes had to be made to Pucella and Weissman's subset to make the proofs possible. An important lesson learned was that the process of designing a certifiable policy language cannot be done in isolation since while certifiability of the language derives the design one way, expressiveness requirements of the language usually drive it in the opposite direction. Certifiability cannot be a post-design activity as the changes it will impose are often fundamental ones.

In the following section we will list the main theorems that we have proved and the supporting theorems that needed to be proved to lead up to the main results. In general however we will not show the actual proofs in the dissertation (with some exceptions) and instead refer the reader to the GitHub repository where the source code for \ac{ACCPL}, including the proofs, lives~\cite{BahmanSistany2015}.

\section{Decidability of ACCPL}\label{sec:maintheorems}

The theorem \syn{trans_agreement_dec2} is the declaration of the main decidability result for \ac{ACCPL} (see listing~\ref{lst:agreementdecidablecoq}). Together with other theorems we will describe below, we have ``certified'' \ac{ACCPL} decidable by proving this theorem. The nonempty list that the agreement translation function \syn{trans_agreement} returns will contain results one per each primitive policy (\syn{primPolicy}) found in the agreement. Specifically the predicate \syn{isResultInQueryResult} takes a result and a nonempty list of results which \syn{trans_agreement} produces, and calls the \syn{In} predicate. The \syn{In} predicate is adapted from the \syn{Coq.Lists.List} standard library and checks for the existence of the input \syn{result} in the nonempty list of results. Definitions for predicates \syn{isResultInQueryResult} and \syn{In} are listed in listing~\ref{lst:agreementdecidablecoq}. The definitions for \syn{answer} and \syn{result} are repeated in the same listing. 

As an example and also a visual aid to understanding how queries are answered, see listing~\ref{lst:permittedcoq}. The \syn{isResultInQueryResult} predicate looks for a result with an answer of Permitted in the list that \syn{trans_agreement} has produced, for an agreement for three primitive policies (since the set contains three results). Intuitively, we are asking whether Alice is \syn{Permitted} to Print the ebook given a policy.  

\lstset{language=Coq, frame=single, caption={Access Request Is \syn{Permitted}},label={lst:permittedcoq}}
%\begin{minipage}[c]{0.95\textwidth}
\begin{lstlisting}
isResultInQueryResult
(Result Permitted Alice Print ebook) 
[ (Result Unregulated Alice Print ebook) ; (Result Unregulated Alice Print ebook) ; (Result Permitted Alice Print ebook) ]
\end{lstlisting}
%\end{minipage}
    
In the case where the whole set is not comprised of \syn{Unregulated} results, we have two mutually exclusive cases. The first case is when the set has a at least one \syn{Permitted} result; we answer the access query in this case with a result of \syn{Permitted} (this would be the case in the listing~\ref{lst:permittedcoq}). The second case is when the set has at least one \syn{NotPermitted}; we answer the access query in this case with a result of \syn{NotPermitted} (see answer and result types in~\ref{sec:answerandresulttypes}).


\lstset{language=Coq, frame=single, caption={Agreement Translation is Decidable},label={lst:agreementdecidablecoq}}
%\begin{minipage}[c]{0.95\textwidth}
\begin{lstlisting}
Definition isResultInQueryResult (res:result)(results: nonemptylist result) : Prop :=
  In res results.

Fixpoint In (a:X) (l:nonemptylist) : Prop :=
    match l with
      | Single s => s=a
      | NewList s rest => s = a \/ In a rest
    end.

Inductive answer : Set :=
  | Permitted : answer
  | Unregulated : answer
  | NotPermitted : answer.

Inductive result : Set :=
  | Result : answer -> subject -> act -> asset -> result.

Theorem trans_agreement_dec2:
  forall
  (e:environment)(ag:agreement)(action_from_query:act)
   (subject_from_query:subject)(asset_from_query:asset),


 (isResultInQueryResult 
    (Result Permitted subject_from_query action_from_query asset_from_query)
    (trans_agreement e ag action_from_query subject_from_query asset_from_query)) 
\/

 (isResultInQueryResult 
    (Result NotPermitted subject_from_query action_from_query asset_from_query)
    (trans_agreement e ag action_from_query subject_from_query asset_from_query))
\/

 (~(isResultInQueryResult 
    (Result Permitted subject_from_query action_from_query asset_from_query)
    (trans_agreement e ag action_from_query subject_from_query asset_from_query)) /\
  ~(isResultInQueryResult 
    (Result NotPermitted subject_from_query action_from_query asset_from_query)
    (trans_agreement e ag action_from_query subject_from_query asset_from_query))).

\end{lstlisting}
%\end{minipage}

Typically most, if not all of the results will be of type \syn{Unregulated}. In the case where all the results are \syn{Unregulated} we answer the access query with a result of \syn{Unregulated}. We show this case indirectly in the theorem in listing~\ref{lst:agreementdecidablecoq} by stating the set does not contain a \syn{Permitted} result nor a \syn{NotPermitted} result. 

\section{Mutual Exclusivity of \syn{Permitted} and \syn{NotPermitted}}

The \syn{trans_agreement_perm_implies_not_notPerm_dec} (see listing~\ref{lst:permimpliesnotnotperm}) says that the existence of a \syn{Permitted} result in the set of results returned by \syn{trans_agreement}, excludes a \syn{NotPermitted} result.


\lstset{language=Coq, frame=single, caption={\syn{Permitted} result Implies no \syn{NotPermitted} result},label={lst:permimpliesnotnotperm}}
%\begin{minipage}[c]{0.95\textwidth}
\begin{lstlisting}
Theorem trans_agreement_perm_implies_not_notPerm_dec:
  forall
    (e:environment)(ag:agreement)(action_from_query:act)
    (subject_from_query:subject)(asset_from_query:asset),

    (isResultInQueryResult 
      (Result Permitted subject_from_query action_from_query asset_from_query)
        (trans_agreement e ag action_from_query subject_from_query asset_from_query)) ->

   ~(isResultInQueryResult 
      (Result NotPermitted subject_from_query action_from_query asset_from_query)
       (trans_agreement e ag action_from_query subject_from_query asset_from_query)).
\end{lstlisting}
%\end{minipage}


The  \syn{trans_agreement_NotPerm_implies_not_Perm_dec} (see listing~\ref{lst:notpermimpliesnotperm}) shows that the existence of a \syn{NotPermitted} result in the set of results returned by \syn{trans_agreement}, excludes a \syn{Permitted} result.

\lstset{language=Coq, frame=single, caption={\syn{NotPermitted} result Implies no \syn{Permitted} result},label={lst:notpermimpliesnotperm}}
%\begin{minipage}[c]{0.95\textwidth}
\begin{lstlisting}
Theorem trans_agreement_NotPerm_implies_not_Perm_dec:
  forall
    (e:environment)(ag:agreement)(action_from_query:act)
    (subject_from_query:subject)(asset_from_query:asset),

    (isResultInQueryResult 
      (Result NotPermitted subject_from_query action_from_query asset_from_query)
        (trans_agreement e ag action_from_query subject_from_query asset_from_query)) ->

   ~(isResultInQueryResult 
      (Result Permitted subject_from_query action_from_query asset_from_query)
       (trans_agreement e ag action_from_query subject_from_query asset_from_query)).
\end{lstlisting}
%\end{minipage}

Finally the proof for \syn{trans_agreement_not_Perm_and_ \newline NotPerm_at_once} establishes that both \syn{Permitted} and \syn{NotPermitted} results cannot exist in the same set returned by \syn{trans_agreement} (see listing~\ref{lst:permandnotpermmutualexclusive}). This result also establishes the fact that in \ac{ACCPL} rendering conflicting decisions is not possible. 

\lstset{language=Coq, frame=single, caption={\syn{Permitted} and \syn{NotPermitted}: Mutually Exclusive},label={lst:permandnotpermmutualexclusive}}
%\begin{minipage}[c]{0.95\textwidth}
\begin{lstlisting}
Theorem trans_agreement_not_Perm_and_NotPerm_at_once:
  forall
  (e:environment)(ag:agreement)(action_from_query:act)
   (subject_from_query:subject)(asset_from_query:asset),


 ~((isResultInQueryResult 
    (Result Permitted subject_from_query action_from_query asset_from_query)
    (trans_agreement e ag action_from_query subject_from_query asset_from_query)) 
/\

 (isResultInQueryResult 
    (Result NotPermitted subject_from_query action_from_query asset_from_query)
    (trans_agreement e ag action_from_query subject_from_query asset_from_query))).

\end{lstlisting}
%\end{minipage}

The proof for the next theorem \syn{trans_agreement_not_NotPerm_and_not_Perm \newline _implies_Unregulated_dec} shows that in the case where neither a \syn{Permitted} nor \syn{NotPermitted} result exists in the set returned by \syn{trans_agreement}, there does exist at least one \syn{Unregulated} result (see listing~\ref{lst:notpermandnotpermimpliesunregulated}).

\lstset{language=Coq, frame=single, caption={Not (\syn{Permitted} and \syn{NotPermitted}) Implies \syn{Unregulated}},label={lst:notpermandnotpermimpliesunregulated}}
%\begin{minipage}[c]{0.95\textwidth}
\begin{lstlisting}

Theorem trans_agreement_not_NotPerm_and_not_Perm_implies_Unregulated_dec:
  forall
    (e:environment)(ag:agreement)(action_from_query:act)
    (subject_from_query:subject)(asset_from_query:asset),

    (~(isResultInQueryResult 
      (Result Permitted subject_from_query action_from_query asset_from_query)
        (trans_agreement e ag action_from_query subject_from_query asset_from_query)) /\

    ~(isResultInQueryResult 
      (Result NotPermitted subject_from_query action_from_query asset_from_query)
        (trans_agreement e ag action_from_query subject_from_query asset_from_query))) ->

   (isResultInQueryResult 
      (Result Unregulated subject_from_query action_from_query asset_from_query)
       (trans_agreement e ag action_from_query subject_from_query asset_from_query)).
\end{lstlisting}
%\end{minipage}


\section{Decidable As An Inductive Predicate}

We encode the cases discussed in section~\ref{sec:maintheorems} in an inductively defined predicate called \syn{decidable} in listing~\ref{lst:inductivepredicatedecidablecoq}. Recall that the nonempty list that the agreement translation function \syn{trans_agreement} returns will return a set of results one per each \syn{primPolicy}.There are three cases. First when there is a \syn{NotPermitted} result in the set, second when there is a \syn{Permitted} result in the set, and third when there are no \syn{Permitted} and no \syn{NotPermitted} results in the set. We encode these three cases in three constructors for the type decidable. The constructors are: \syn{Denied}, \syn{Granted} and \syn{NotApplicable}.

\lstset{language=Coq, frame=single, caption={Inductive Predicate decidable},label={lst:inductivepredicatedecidablecoq}}
%\begin{minipage}[c]{0.95\textwidth}
\begin{lstlisting}

Inductive decidable : environment -> agreement -> act -> subject -> asset -> Prop :=

     | Denied : forall
  (e:environment)(ag:agreement)(action_from_query:act)
   (subject_from_query:subject)(asset_from_query:asset), 

 (isResultInQueryResult 
    (Result NotPermitted subject_from_query action_from_query asset_from_query)
       (trans_agreement e ag action_from_query subject_from_query asset_from_query)) 
  -> decidable e ag action_from_query subject_from_query asset_from_query

     | Granted : forall
  (e:environment)(ag:agreement)(action_from_query:act)
   (subject_from_query:subject)(asset_from_query:asset), 
 (isResultInQueryResult 
    (Result Permitted subject_from_query action_from_query asset_from_query)
       (trans_agreement e ag action_from_query subject_from_query asset_from_query)) 
  -> decidable e ag action_from_query subject_from_query asset_from_query

     | NonApplicable : forall
  (e:environment)(ag:agreement)(action_from_query:act)
   (subject_from_query:subject)(asset_from_query:asset), 

  ~(isResultInQueryResult 
    (Result Permitted subject_from_query action_from_query asset_from_query)
    (trans_agreement e ag action_from_query subject_from_query asset_from_query)) 
     ->
  ~(isResultInQueryResult 
    (Result NotPermitted subject_from_query action_from_query asset_from_query)
    (trans_agreement e ag action_from_query subject_from_query asset_from_query)) 
  -> decidable e ag action_from_query subject_from_query asset_from_query.
\end{lstlisting}
%\end{minipage}


We declare and prove the decidability of \ac{ACCPL} in terms of the ``decidable'' predicate as seen in listing~\ref{lst:decidabletypecoq}. The proof is made simple since we use the previously proven theorem \syn{trans_agreement_dec2}. We will list the intermediate theorems that we have declared and proved which are used in the proof of theorem \syn{trans_agreement_dec2} and subsequent proofs in section~\ref{sec:intermediatetheorems}.

\lstset{language=Coq, frame=single, caption={Decidability Of Agreement Translation},label={lst:decidabletypecoq}}
%\begin{minipage}[c]{0.95\textwidth}
\begin{lstlisting}
Theorem trans_agreement_decidable:
  forall
  (e:environment)(ag:agreement)(action_from_query:act)
   (subject_from_query:subject)(asset_from_query:asset),
     decidable e ag action_from_query subject_from_query asset_from_query.
Proof.

intros e ag action_from_query subject_from_query asset_from_query.
specialize trans_agreement_dec2 with 
   e ag action_from_query subject_from_query asset_from_query.
intros H. destruct H as [H1 | H2].
apply Granted. assumption.
destruct H2 as [H21 | H22].
apply Denied. assumption.
apply NonApplicable. 
destruct H22 as [H221 H222].
exact H221.
destruct H22 as [H221 H222].
exact H222.
Defined.

\end{lstlisting}
%\end{minipage}

 
\section{ACCPL in the Landscape of Policy-based Policy Languages}

Tschantz and Krishnamurthi~\cite{Tschantz} argue for the need for formal means to compare and contrast policy-based access-control languages and they present a set of properties to analyze the behaviour of policies in light of additional and/or explicit environmental facts, policy growth and policy decomposition. Tschantz and Krishnamurthi apply their properties to two core policy languages and compare the results. One of the core policy languages they present is a simplified \ac{xacml} and the other is Lithium~\cite{Halpern2008}. In the following sections we will describe Tschantz and Krishnamurthi's properties and evaluate \ac{ACCPL} with respect to these properties.

\section{Reasonability Properties}\label{sec:threeinterpretations}

The properties Tschantz and Krishnamurthi~\cite{Tschantz} present revolve around three questions:

\begin{enumerate}
\item How decisions change when the environment has more facts
\item Impact of policy growth on how decisions may change
\item How amenable policies are to compositional reasoning
\end{enumerate}

Tschantz and Krishnamurthi~\cite{Tschantz} use a simple policy as a motivating example (which they attribute to~\cite{Halpern2008}) that we will adapt and reuse here. The example will help to review the range of interpretations possible and ultimately the range of policy language design choices and where \ac{ACCPL} is on that range.
The policy in English is as follows:

\begin{itemize}
\item If the subject has a subscription to the Gold bundle, then permit that subject to print TheReport.

\item If the subject has a subscription to the Basic bundle, then do not permit that subject to print TheReport.

\item If the subject has no subscription to the Gold bundle, then permit that subject to play Terminator.
\end{itemize}

Consider a query and a fact in English as: A subscriber to the Basic bundle requests to play the movie Terminator. Let`s extract the fact that the subject is a subscriber to the Basic bundle into a separate term and make it part of the environment. The basic policy, the query in question and the environment are all expressed as a (pseudo) logic formula and listed in the listing~\ref{lst:basicpolicyAST}. Should the access query be granted?

\lstset{mathescape, language=AST}  
\begin{lstlisting}[frame=single, caption={Basic Policy},label={lst:basicpolicyAST}]

p1 = GoldBundle(s) -> Permitted(s, print, TheReport) 
p2 = BasicBundle(s) -> NotPermitted(s, print, TheReport) 
p3 = ~GoldBundle(s) -> Permitted(s, play, Terminator)
q1 = (s, play, Terminator) 
e1 = BasicBundle(s)
\end{lstlisting}


According to Tschantz and Krishnamurthi~\cite{Tschantz} at least three different interpretations are possible. We will list the possibilities and then evaluate how \ac{ACCPL} would answer the query.

\begin{itemize}
\item Implicit: Grant access because of \syn{p3}. This interpretation assumes that if a subject has a subscription to the Basic bundle, the subject does not have a subscription to the Gold bundle so the assumption is that there is a proof for \syn{~GoldBundle(s)}. This is an example of ``implicit'' knowledge and ``closed world assumptions''.

\item Explicit: The policy does not apply to the query. In other words, the query is ``NotApplicable''. \syn{p1} and \syn{p2} don't apply since the asset and the action in the request don't match the ones in \syn{p1} and \syn{p2}. \syn{p3} does not apply either since no proof exists to show that \syn{~GoldBundle(s)}. 

\item Implicit with automatic proof capability: grants access to the query by automatically proving that \syn{~GoldBundle(s)}. Under this interpretation, proof by contradiction is used to establish \syn{~GoldBundle(s)}.  The assumption \syn{GoldBundle(s)} is added to \syn{e1} which already contains \syn{BasicBundle(s)}. Now \syn{p1} permits TheReport to be printed by the subject while \syn{p2} would not permit TheReport to be printed by the subject, leading to a contradiction and the proof of \syn{~GoldBundle(s)}.
\end{itemize}

A policy language based on the first and third interpretations are not certifiable, meaning a machine-checked proof of the correctness of their semantics with respect to different properties (such as decidability) may not be possible. The main problem with such languages is the implicitness of their semantics. This was the case for the Pucella and Weissman~\cite{pucella2006} fragment of \ac{odrl}; although the interpretation found in Pucella and Weissman's fragment doesn't necessarily match the first and the third interpretations above, they are certainly in the camp of ``implicit'' interpretations. \ac{ACCPL} matches the second interpretation in the sense of having explicit semantics which leads to certifiably results we discuss elsewhere.


\section{Policy Based Access-Control Languages}

Tschantz and Krishnamurthi~\cite{Tschantz} define an access-control policy language as a tuple $L = (P, Q, G, N, \left\langle\left\langle . \right\rangle\right\rangle)$ where $P$ is a set of policies, $Q$ is a set of requests or queries, $G$ is the granting decisions, $N$ is the non-granting decisions, $\left\langle\left\langle . \right\rangle\right\rangle$ is a function taking a policy $p \in P$ to a relation between $Q$ and $G \cup N$ where $G \cap N = \emptyset$. Let $D$ represent $G \cup N$. Policy $p$ assigns a decision of $d \in D$ to the query $q \in Q$. $L$ also defines a partial order on decisions such that $d \leq d'$ if either $d, d' \in N$ or $d, d' \in G$ or $d \in N$ and $d' \in G$. Note that for \ac{ACCPL}, $D = \left\{ {Unregulated, NotPermitted, Permitted}\right\}$, $G = \left\{ {Permitted}\right\}$ and $N = \left\{ {Unregulated, NotPermitted}\right\}$.


\section{Policy Combinators}\label{sec:policycombinators}

Primitive policies are often developed by independent entities within an organization therefore it is natural to want a way of combining them into a policy for the whole organization using policy combinators. Some languages such as \ac{odrl} have nested sub-policies inside other policies. In such cases one design decision could be to allow for different combinators for different layers to make the language more expressive. Some examples of policy combinators are: conjunction, disjunction, exclusive disjunction, \syn{Permitted} override (if any of the primitive policies returns a \syn{Permitted}, return only \syn{Permitted}), \syn{NotPermitted} override (if any of the primitive policies returns a \syn{NotPermitted}, return only \syn{NotPermitted})~\cite{Tschantz}.

\section{ACCPL}

When designing \ac{ACCPL} and to make the decidability proofs possible, we sought out all the different combinations and cases possible (through interaction with the proof process) and explicitly enumerated them in the translation functions. Once we made explicit all the different cases where a decision was possible, we assigned \syn{Permitted} and \syn{NotPermitted} decisions to two specific cases while every other enumerated case was rendered with an \syn{Unregulated} (not applicable) decision. Explicit enumeration of all the different cases has been certified to be complete by the fact that we have the decidability results. Note that we could have made different design choices in assigning the decisions. 

For example, we could have replaced all or some of the current \syn{Unregulated} decision assignments to either \syn{Permitted} or \syn{NotPermitted} depending on whether we want to make the language more permissive or more prohibitive. Note if we were to implement the changes proposed here, the current decidability results would no longer hold and the \syn{Unregulated} case would have to be removed from the statement of the theorem and the theorem statement to be modified to look like the theorem listed in listing~\ref{lst:agreementdecidablecoqSecond}.

\lstset{language=Coq, frame=single, caption={Agreement Translation is Decidable (for \syn{Permitted} or \syn{NotPermitted} only)},label={lst:agreementdecidablecoqSecond}}
\begin{minipage}[c]{0.95\textwidth}
\begin{lstlisting}
Theorem trans_agreement_dec:
  forall
  (e:environment)(ag:agreement)(action_from_query:act)
   (subject_from_query:subject)(asset_from_query:asset),

 (isResultInQueryResult 
    (Result Permitted subject_from_query action_from_query asset_from_query)
    (trans_agreement e ag action_from_query subject_from_query asset_from_query)) 
\/
 (isResultInQueryResult 
    (Result NotPermitted subject_from_query action_from_query asset_from_query)
    (trans_agreement e ag action_from_query subject_from_query asset_from_query)).

\end{lstlisting}
\end{minipage}



\section{Deterministic and Total}

According to Tschantz and Krishnamurthi~\cite{Tschantz} a language $L$ is defined to be deterministic if $forall$ $p \in P$, $forall$ $q \in Q$, $forall$ $d, d' \in D$, $q \left\langle\left\langle p  \right\rangle\right\rangle d$ $\land$ $q \left\langle\left\langle p  \right\rangle\right\rangle d'$ $\implies$ $d = d'$. In words, the definition states that for a deterministic language, given a query and a policy, if the policy translation renders a decision of $d$ and a decision of $d'$, then the two decisions must be the same.

A language $L$ is total if $forall$ $p \in P$, $forall$ $q \in Q$, $exists$ $d \in D$, s.t. $q \left\langle\left\langle p  \right\rangle\right\rangle d$. In words, for a total language, the policy translation always renders a decision.

We conjecture that \ac{ACCPL} is both deterministic and total. We defer the proof of determinism of \ac{ACCPL} to future work. As for \ac{ACCPL} being a total language we believe the fact that \ac{ACCPL} is decidable (see listing~\ref{lst:decidabletypecoq}) implies \ac{ACCPL} is a total language. We defer the detailed declaration and proof of totality of \ac{ACCPL} to future work.

\section{Safety}

As discussed in section~\ref{sec:threeinterpretations} policy interpretations could be implicit or explicit. Explicit interpretations distinguish between unknown information and information known to be absent. The explicit approach promotes verbose policies and requests and usually leads to too many \syn{Unregulated} or NotApplicable decisions (in the absence of the right environmental facts). However the verbosity could be used to direct what additional facts are needed to make a granting and denying decision. Under the second interpretation in~\ref{sec:threeinterpretations} while the decision is NotApplicable, the system can direct the entity making the request provide a proof of \syn{~GoldBundle(s)} in order for a granting decision to be rendered. 

Implicit interpretations may result in granting unintended access to protected resources because no explicit facts need be mentioned in the environment. Recall that the implicit approach works by making assumptions when facts are not explicitly mentioned. Such unintended assumptions lead to unintended permissions being granted.

Tschantz and Krishnamurthi~\cite{Tschantz} define safety as $forall$ $p \in P$, $forall$ $q, q' \in Q$, \\$q$ $\leq$ $q'$ $\implies$ $q \left\langle\left\langle p  \right\rangle\right\rangle d$ $\leq$ $q \left\langle\left\langle p  \right\rangle\right\rangle d'$ which intuitively says that policies written in a safe language $L$, will not result in the leakage of permissions to unintended subjects. For example, incomplete information may unintentionally cause an access permission to be granted to an intruder. 

Could we conclude that an access-control policy language with the explicit approach such as \ac{ACCPL} is also safe? in the design of \ac{ACCPL} and the translation functions in particular we only grant or deny access in specific cases. We grant access if the asset and the action in query are found in the agreement and \syn{trans_prin} has a proof and \syn{trans_preRequisite} for both policy set and policy have proofs. We deny an access request if the policy set is an exclusive policy set (\syn{primExclusivePolicySet}) and \syn{trans_prin} does not hold. In all other cases we render the decision of \syn{Unregulated}. In other words, in the absence of necessary facts (proofs), we give the least decision (e.g. \syn{Unregulated}), when there are more facts available in the environment (such as \syn{trans_preRequisite} holding), we grant either deny or permit decisions. However when all the facts hold, corresponding to the greatest query in the definition of safety, we render a granting decision which is also the greatest decision in the lattice of decisions: \syn{Unregulated} $\leq$ \syn{NotPermitted} $\leq$ \syn{Permitted}.


\section{Independent Composition}

As discussed in section~\ref{sec:threeinterpretations}, the third interpretation works by taking into account not only the third policy, $p3$, but also $p1$ and $p2$ to render a decision of granted. However each of the policies in isolation would result in the NotApplicable decision. 

Taking into account all policies and rendering a decision that is not different from combining the decisions reached by each primitive policy in isolation, is a property called ``Independent Composition'' by Tschantz and Krishnamurthi~\cite{Tschantz}. The third interpretation in section~\ref{sec:threeinterpretations} is clearly not have this property. Formally, a language $L$ has the independent composition property, if $forall$ $p \in P$, $forall$ $q \in Q$, $forall$ $d, d^\ast \in D$, \\$\boxplus$ ($q \left\langle\left\langle p_{1}  \right\rangle\right\rangle d_{1}$, $q \left\langle\left\langle p_{2}  \right\rangle\right\rangle d_{2}$, ..., $q \left\langle\left\langle p_{n}  \right\rangle\right\rangle d_{n}$) $=$ $q \left\langle\left\langle (\oplus p_{1}, p_{2}, ..., p_{n}) \right\rangle\right\rangle d^\ast$, where $\oplus$ is a composition operator defined in the language and $\boxplus$ is the decision composition operator. For this definition to be well-defined, there is a requirement that complete environmental facts be available for each individual policy so that a decision can be rendered for each primitive policy with respect to the request, in isolation.

Does \ac{ACCPL} have the independent composition property? As covered in chapter~\ref{chap:accplsemanticscoq} and section~\ref{sec:policycombinators} the translation functions for \ac{ACCPL}, starting with \syn{trans_agreement} all return sets of \syn{result}s.  The set of \syn{result}s returned by the agreement translation function will have a result per \syn{primPolicy} (type of primitive policies) in the agreement. Recall that results containing a \syn{Permitted} answer and a \syn{NotPermitted} answer are mutually exclusive and the existence of a single (or more) \syn{Permitted} result in the set makes the whole set ``Permitted'' whereas the existence of a single (or more) \syn{NotPermitted} result makes the whole set ``NotPermitted''. When no \syn{Permitted} or \syn{NotPermitted} is seen in the set, we only have \syn{Unregulated} results which would make the whole set ``Unregulated". This is the decision combination strategy we have designed for \ac{ACCPL}; the policy combination strategy \ac{ACCPL} simply preserves all individual decisions and returns the whole set for rendering a global decision. Below we consider the three different possible decisions that the decision combination strategy may reach and review whether the result of any individual policy may change as a result.

\begin{enumerate}
  \item Global decision is \syn{Permitted}. Only possible results are \syn{Permitted} or \syn{Unregulated}. In this case, only the \syn{Unregulated} results have been modified by the decision combination strategy. \syn{Permitted} results coming from any individual primitive policy will have been preserved.
  \item Global decision is \syn{NotPermitted}. Only possible results are \syn{NotPermitted} or \syn{Unregulated}. In this case, similar to the first case, only the \syn{Unregulated} results have been modified by the decision combination strategy. \syn{NotPermitted} results coming from any individual primitive policy will have been preserved.
   \item Global decision is \syn{Unregulated}. Only possible results are \syn{Unregulated}.  In this case, results coming from any individual primitive policy will have been preserved.
\end{enumerate}

We argue that \ac{ACCPL} has the independent composition property with respect to granting and denying policies only as detailed in the previous paragraphs. 

\section{Monotonicity of Policy Combinators}

Tschantz and Krishnamurthi~\cite{Tschantz} discuss the notion of monotonicity of policy combinators. Intuitively, in a language with a policy combinator that is monotonic, adding another primitive policy, cannot change the combined decision to go from grating to non-granting. In \ac{ACCPL}, NotApplicable results are only possible if \syn{trans_prin} does not hold. Assume the combined (global) decision is already \syn{Permitted} for a certain agreement. In the case when \syn{trans_prin} does not hold, adding a primitive policy to the agreement will not change the \syn{Permitted} result to a \syn{NotPermitted} result, since a primitive policy is composed of a \syn{preRequisite} and an \syn{act} only. In the case where \syn{trans_prin} does hold, there are two cases. Either the \syn{preRequisite} of the added policy holds and the decision is another \syn{Permitted} result, or the \syn{preRequisite} of the added policy does not hold, and the decision is an \syn{Unregulated} which won't affect the combined decision.





\section{Intermediate Theorems}\label{sec:intermediatetheorems}
The theorem \syn{trans_agreement_dec_sb_Permitted} states as a sumbool that the nonempty list that \syn{trans_agreement} produces, contains a \syn{Permitted} result or not and this is decidable (see listing~\ref{lst:decsbpermitted}).


\lstset{language=Coq, frame=single, caption={Translation Function for Agreement Returns a \syn{Permitted} result or Not},label={lst:decsbpermitted}}
%\begin{minipage}[c]{0.95\textwidth}
\begin{lstlisting}
Theorem trans_agreement_dec_sb_Permitted:
  forall
  (e:environment)(ag:agreement)(action_from_query:act)
   (subject_from_query:subject)(asset_from_query:asset),


 {(isResultInQueryResult 
    (Result Permitted subject_from_query action_from_query asset_from_query)
    (trans_agreement e ag action_from_query subject_from_query asset_from_query))} +
 {~(isResultInQueryResult 
    (Result Permitted subject_from_query action_from_query asset_from_query)
    (trans_agreement e ag action_from_query subject_from_query asset_from_query))}.

\end{lstlisting}
%\end{minipage}

The theorem \syn{trans_agreement_dec_sb_NotPermitted} states as a sumbool that the nonempty list that \syn{trans_agreement} produces, contains a \syn{NotPermitted} result or not and this is decidable (see listing~\ref{lst:decsbnotpermitted}).

\lstset{language=Coq, frame=single, caption={Translation Function for Agreement Returns a \syn{NotPermitted} result or Not},label={lst:decsbnotpermitted}}
%\begin{minipage}[c]{0.95\textwidth}
\begin{lstlisting}
Theorem trans_agreement_dec_sb_NotPermitted:
  forall
  (e:environment)(ag:agreement)(action_from_query:act)
   (subject_from_query:subject)(asset_from_query:asset),

 {(isResultInQueryResult 
    (Result NotPermitted subject_from_query action_from_query asset_from_query)
    (trans_agreement e ag action_from_query subject_from_query asset_from_query))} +
 {~(isResultInQueryResult 
    (Result NotPermitted subject_from_query action_from_query asset_from_query)
    (trans_agreement e ag action_from_query subject_from_query asset_from_query))}.

\end{lstlisting}
%\end{minipage}

The theorem \syn{resultInQueryResult_dec} states as a sumbool that a given result is either in the nonempty list of \syn{results} or not and this is decidable (see listing~\ref{lst:decresultInResults}).

\lstset{language=Coq, frame=single, caption={Result is in Results or Not},label={lst:decresultInResults}}
%\begin{minipage}[c]{0.95\textwidth}
\begin{lstlisting}
Theorem resultInQueryResult_dec :
    forall (res:result)(results: nonemptylist result), 
 {isResultInQueryResult res results} + {~isResultInQueryResult res results}.
\end{lstlisting}
%\end{minipage}

The theorem \syn{trans_agreement_not_Perm_and_NotPerm_at_once} states that you may not have both a \syn{Permitted} and \syn{NotPermitted} result in the nonempty list that \syn{trans_agreement} produces (see listing~\ref{lst:permandnotpermatonce}).

\lstset{language=Coq, frame=single, caption={Translation Function for Agreement will not have both \syn{Permitted} and \syn{NotPermitted}},label={lst:permandnotpermatonce}}
%\begin{minipage}[c]{0.95\textwidth}
\begin{lstlisting}
Theorem trans_agreement_not_Perm_and_NotPerm_at_once:
  forall
  (e:environment)(ag:agreement)(action_from_query:act)
   (subject_from_query:subject)(asset_from_query:asset),


 ~((isResultInQueryResult 
    (Result Permitted subject_from_query action_from_query asset_from_query)
    (trans_agreement e ag action_from_query subject_from_query asset_from_query)) 
/\

 (isResultInQueryResult 
    (Result NotPermitted subject_from_query action_from_query asset_from_query)
    (trans_agreement e ag action_from_query subject_from_query asset_from_query))).

\end{lstlisting}
%\end{minipage}

The theorem \syn{trans_policy_PIPS_dec_not} states that the translation function trans_policy_PIPS produces a nonempty list of results that will not have a \syn{NotPermitted} result. This makes intuitive sense as \syn{trans_policy_PIPS} is called for inclusive policy sets only (see listing~\ref{lst:pipsdecnot}).
\lstset{language=Coq, frame=single, caption={Translation Function for PIPS will not return a \syn{NotPermitted} result},label={lst:pipsdecnot}}
%\begin{minipage}[c]{0.95\textwidth}
\begin{lstlisting}
Theorem trans_policy_PIPS_dec_not:
  forall
  (e:environment)(prq: preRequisite)(p:policy)(subject_from_query:subject)
  (prin_u:prin)(a:asset)(action_from_query:act),

 ~(isResultInQueryResult 
    (Result NotPermitted subject_from_query action_from_query a)
    (trans_policy_PIPS e prq p subject_from_query prin_u a action_from_query)).

\end{lstlisting}
%\end{minipage}

The theorem \syn{trans_policy_PEPS_perm_implies_not_notPerm_dec} states that having a \syn{Permitted} result in the nonempty list of results that the translation function trans_policy_PEPS produces implies there exists no \syn{NotPermitted} result in the nonempty list (see listing~\ref{lst:pepspermimpliesnotperm}).

\lstset{language=Coq, frame=single, caption={\syn{Permitted} in results by Translation Function for PEPS implies no \syn{NotPermitted}},label={lst:pepspermimpliesnotperm}}
%\begin{minipage}[c]{0.95\textwidth}
\begin{lstlisting}
Theorem trans_policy_PEPS_perm_implies_not_notPerm_dec:
  forall
  (e:environment)(prq: preRequisite)(p:policy)(subject_from_query:subject)
  (prin_u:prin)(a:asset)(action_from_query:act),

    (isResultInQueryResult 
      (Result Permitted subject_from_query action_from_query a)
        (trans_policy_PEPS e prq p subject_from_query prin_u a action_from_query)) ->

   ~(isResultInQueryResult 
      (Result NotPermitted subject_from_query action_from_query a)
       (trans_policy_PEPS e prq p subject_from_query prin_u a action_from_query)).

\end{lstlisting}
%\end{minipage}
The theorem AnswersNotEqual states that two answers not being equal implies that results built from those answers are not equal as well (see listing~\ref{lst:answersnotequal}).

\lstset{language=Coq, frame=single, caption={Non Equal Answers Give Non Equal Results},label={lst:answersnotequal}}
%\begin{minipage}[c]{0.95\textwidth}
\begin{lstlisting}
Theorem AnswersNotEqual: forall (ans1:answer)(ans2:answer)(s:subject)(ac:act)(ass:asset),
  (ans1<>ans2) -> ((Result ans1 s ac ass) <> (Result ans2 s ac ass)).

\end{lstlisting}
%\end{minipage}

The theorem \syn{trans_policy_positive_dec_not} states that the translation function trans_policy_positive produces a nonempty list of results that will not have a \syn{NotPermitted} result (see listing~\ref{lst:positivedecnot}).


\lstset{language=Coq, frame=single, caption={Translation Function for Permitting Policies will not return a \syn{NotPermitted} result},label={lst:positivedecnot}}
%\begin{minipage}[c]{0.95\textwidth}
\begin{lstlisting}

Theorem trans_policy_positive_dec_not:
  forall 
(e:environment)(s:subject)(p:policy)(prin_u:prin)(a:asset)
  (action_from_query: act),
 
  ~(isResultInQueryResult 
    (Result NotPermitted s action_from_query a)
    (trans_policy_positive e s p prin_u a action_from_query)).
\end{lstlisting}
%\end{minipage}

The theorem \syn{trans_policy_negative_dec_not} states that the translation function trans_policy_negative produces a nonempty list of results that will not have a \syn{Permitted} result (see listing~\ref{lst:negativedecnot}).

\lstset{language=Coq, frame=single, caption={Translation Function for NotPermitting Policies will not return a \syn{Permitted} result},label={lst:negativedecnot}}
%\begin{minipage}[c]{0.95\textwidth}
\begin{lstlisting}
Theorem trans_policy_negative_dec_not:
  forall 
(e:environment)(s:subject)(p:policy)(a:asset)
  (action_from_query: act),
 
  ~(isResultInQueryResult 
    (Result Permitted s action_from_query a)
    (trans_policy_negative e s p a action_from_query)).

\end{lstlisting}
%\end{minipage}

The theorem \syn{trans_policy_unregulated_dec_not} states that the translation function trans_policy_unregulated produces a nonempty list of results that will not have a \syn{Permitted} nor a \syn{NotPermitted} result (see listing~\ref{lst:unregulateddecnot}).

\lstset{language=Coq, frame=single, caption={Translation Function for \syn{Unregulated} Requests will not return \syn{Permitted} nor \syn{NotPermitted}},label={lst:unregulateddecnot}}

%\begin{minipage}[c]{0.95\textwidth}
\begin{lstlisting}
Theorem trans_policy_unregulated_dec_not:
  forall 
(e:environment)(s:subject)(p:policy)(a:asset)
  (action_from_query: act),
 
  ~(isResultInQueryResult 
    (Result Permitted s action_from_query a)
    (trans_policy_unregulated e s p a action_from_query)) /\
  
  ~(isResultInQueryResult 
    (Result NotPermitted s action_from_query a)
    (trans_policy_unregulated e s p a action_from_query)).
\end{lstlisting}
%\end{minipage}
















%======================================================================
\chapter{Examples}\label{chap:examples}

%======================================================================

%----------------------------------------------------------------------
\section{Introduction}
%----------------------------------------------------------------------

%%%%%%% Examples 

In this chapter we will take a tour of the syntax and semantics we have so far developed by examining some example agreements. In the following we will start by reviewing some of the examples used in~\cite{pucella2006}.

Ultimately the goal of specifying all the syntax and semantics is to declare some interesting theorems about policy expressions and proving them in Coq. However we will first start with some specific propositions/theorems about these examples to get a feel for how proofs are done in Coq.

\section{Agreement 2.1}

Consider example 2.1 (from~\cite{pucella2006}) where the $policySet$ is a $AndPolicySet$ with $p1$ and $p2$ as the individual $policySet$s. Let $p1$ be defined as $Count[5]$ $\rightarrow$ $print$ and $p2$ as $and[Alice, Count[2]]$ $\rightarrow$ $print$. 

The agreement is that the asset \emph{The Report} may be printed a total of five times by either \emph{Alice} or \emph{Bob}, and twice more by Alice. So if Alice and Bob have used policy $p1$ to justify their printing of the asset $m_{p1}$ and $n_{p1}$ times, respectively, then either may do so again if $m_{p1} + n_{p1} < 5$. If they have used $p2$ to justify their printing of the asset $m_{p2}$ and $n_{p2}$ times, respectively, then only Alice may do so again if $m_{p2} + n_{p2} < 2$. Note that since Bob doesn't meet the prerequisite of being Alice, $n_{p2}$ is effectively $0$, so we have $m_{p2} < 2$ as the condition for Alice being able to print again (Alice does meet the prerequisite of being Alice).

\lstset{language=Pucella2006}
\begin{lstlisting}[frame=single, caption={Agreement 2.1 (as used in~\cite{pucella2006})},label={lst:example21pucella2006}]
agreement
 for {Alice, Bob} 
 about TheReport 
 with and [p1, p2].
\end{lstlisting}

The Coq version of the agreement 2.1 (listing~\ref{lst:example21pucella2006}) and its sub-parts is listed below. It is best to start with the agreement itself called $A2.1$ in the listing and compare to the agreement 2.1 listed in ~\ref{lst:example21pucella2006}.

\lstset{language=Coq}
\begin{minipage}[c]{0.95\textwidth}
\begin{lstlisting}[frame=single, caption={Agreement 2.1 in Coq},label={lst:example21}]


Definition p1A1:policySet :=
  PrimitivePolicySet
    TruePrq
    (PrimitivePolicy (Constraint (Count  5)) id1 Print).

Definition p2A1prq1:preRequisite := (Constraint (Principal (Single Alice))).
Definition p2A1prq2:preRequisite := (Constraint (Count 2)).

Definition p2A1:policySet :=
  PrimitivePolicySet
    TruePrq
    (PrimitivePolicy (AndPrqs (NewList p2A1prq1 (Single p2A1prq2))) id2 Print).

Definition A2.1 := Agreement (NewList Alice (Single Bob)) TheReport
                  (AndPolicySet (NewList p1A1 (Single p2A1))).

\end{lstlisting}                  
\end{minipage}

\section{Agreement 2.5}

Consider example 2.5 (from~\cite{pucella2006}) where the $policySet$ is a $PrimitivePolicySet$ with a $Count$ constraint as prerequisite and a $AndPolicy$ as the policy. The $AndPolicy$ is the conjunction of two $PrimitivePolicy$s. Both policies have prerequisites of type $ForEachMember$ with actions $display$ and $print$ respectively. The $prin$ component for both $ForEachMember$s is ${Alice, Bob}$, whereas the constraint for the first $ForEachMember$ is $Count[5]$ and for the second is $Count[2]$.

\lstset{language=Pucella2006}
\begin{lstlisting}[frame=single, caption={Agreement 2.5 (as used in~\cite{pucella2006})},label={lst:example25pucella2006}]
agreement
 for {Alice, Bob} 
 about ebook 
 with Count [10] $\rightarrow$ and [forEachMember[{Alice, Bob}; Count[5]] $\Rightarrow_{id1}$ display,
                                 forEachMember[{Alice, Bob}; Count[1]] $\Rightarrow_{id2}$ print].
\end{lstlisting}

The Coq version of the agreement 2.5 (listing~\ref{lst:example25pucella2006}) and its sub-parts is listed below. See agreement 2.5 listed in ~\ref{lst:example25pucella2006} for comparison.

The agreement is that the asset \emph{ebook} may be displayed up to five times by Alice and Bob each, and printed once by each. However the total number of actions (either $display$ or $print$) justified by the two policies by either Alice and Bob is at most 10.


\lstset{language=Coq}
\begin{lstlisting}[frame=single, caption={Example 2.5},label={lst:example25}]

Definition tenCount:preRequisite := (Constraint (Count 10)).
Definition fiveCount:constraint := (Count 5).
Definition oneCount:constraint := (Count 1).

Definition prins2_5 := (NewList Alice (Single Bob)).
Definition forEach_display:preRequisite := ForEachMember prins2_5 (Single fiveCount).
Definition forEach_print:preRequisite := ForEachMember prins2_5 (Single oneCount).

Definition primPolicy1:policy := PrimitivePolicy forEach_display id1 Display.
Definition primPolicy2:policy := PrimitivePolicy forEach_print id2 Print.

Definition policySet2_5:policySet :=
  PrimitivePolicySet tenCount (AndPolicy (NewList primPolicy1 (Single primPolicy2))).
                     
Definition A2.5 := Agreement prins2_5 ebook policySet2_5.

\end{lstlisting}







          
%======================================================================
\chapter{Some Simple Theorems}
%======================================================================

%----------------------------------------------------------------------
\section{Introduction}
%----------------------------------------------------------------------

In this chapter we will declare and prove some very simple theorems about the examples from chapter~\ref{chap:examples}. This simple introduction is only meant to give us a feel for how theorems are stated in Coq and how proofs are constructed using Coq \emph{tactics}. 

As mentioned earlier, propositions are types in Coq whose type is the sort $Prop$. Any term $t$ whose type is a proposition is a proof term or, for short, a proof. A \emph{Hypothesis} is a local declaration $h : P$ where $h$ is an identifier and $P$ is a proposition. An \emph{Axiom} is similar to a hypothesis except it is declared at the global scope and so it is always available. 

A \emph{Theorem} or \emph{Lemma} is stated by giving an identifier whose type is a proposition (~\cite{BC04}).  The proposition is the statement of the theorem or lemma.  It must be followed by a proof. Keywords ``Hypothesis'', ``Axiom'' and ``Theorem'' or ``Lemma'' are used in each case respectively. 

To build a proof in Coq the user states the proposition to prove; this is called a goal to be proved or discharge, along with some hypothesis that makes up the local context. The user then uses commands called tactics to manipulate the local context and to decompose the goal into simpler goals. The goal simplification into sub-goals will continue until all the sub-goals are solved.

In listing~\ref{lst:proofexample} we have declared a theorem called $example1$ and the corresponding proposition $forall x:nat, x < x + 1.$
 
Note that the notation $P : T$ is also used to declare program $P$ has type $T$. This duality of notation is due to Curry-Howard isomorphism which relates the two worlds of type theory and structural logic together~\cite{BC04}. Once the Theorem has been declared Coq displays the proposition to be proved under a horizontal line written --------, and displays the context of local facts and hypothesis, if any, above the horizontal line. At this point one can enter proof mode by using \emph{Proof.} upon which Coq is ready to accept tactics. Entering tactics that can break the stated goal (under the horizontal line) into one or more sub-goals is how one progresses until no goals left at which point Coq responds with ``No more subgoals''~\cite{CoqHurry}.

\lstset{language=Coq}
\begin{lstlisting}[frame=single, caption={Proof Example},label={lst:proofexample}]
Theorem example1: forall x:nat, x < x + 1.
\end{lstlisting}

In the following listings, for the sake of completeness of the presentation, we will include the Coq commands that complete the proof of the respective theorem. We will not however explain the individual commands further here. Showing the commands in the listings, is meant as an indication of the size of the proof in terms of lines of Coq script.


\section{Theorem One}

In listing~\ref{lst:theoremone} we define a $policySet$ with a $constraint$ such that if $Alice$ has used the policy with $id1$ to justify her printing $a_{1}$ times, she may do so again if $a_{1} < 5$. The agreement $AgreeCan$ simply links the asset $TheReport$ with the subject $Alice$ and the $policySet$ previously defined. 

We capture the fact that $Alice$ has used the policy with $id1$ to justify her printing $2$ times in an environment called $eA1$. Recall that environments are defined to be non-empty lists of $count_equality$ objects (see listing~\ref{lst:environmentcoq}). 

We also declare a hypothesis $H$ with the proposition that results from the translation of the agreement (see definition of $trans_agreement$ in listing~\ref{lst:transagreement}) and the environment. The proposition can be shown in Coq after some clean-up (e.g. replaced 101 by Alice) and using the form $Eval$ $compute$ : 

\lstset{language=Coq}
\begin{lstlisting}[frame=single, caption={Hypothesis for Theorem One},label={lst:theoremonehypo}]
forall x : subject, x = Alice /\ True -> 2 < 5 -> Permitted x Print TheReport.
\end{lstlisting}

The theorem $One$ that we are going to prove is trivial but nonetheless in English it states that Alice is Permitted to Print TheReport. The proof comes after the command 'Proof.' and ends with 'Qed'. 

\lstset{language=Coq}
\begin{minipage}[c]{0.95\textwidth}
\begin{lstlisting}[frame=single, caption={Theorem One},label={lst:theoremone}]

Definition psA1:policySet :=
  PrimitivePolicySet
    TruePrq
    (PrimitivePolicy (Constraint (Count  5)) id1 Print).

Definition AgreeCan := Agreement (Single Alice) TheReport psA1.

Definition eA1 : environment := 
  (SingleEnv (make_count_equality Alice id1 2)).

Hypothesis H: trans_agreement eA1 AgreeCan.

Theorem One: Permitted Alice Print TheReport.

Proof. simpl in H. apply H. split. reflexivity. auto.  omega. Qed.


\end{lstlisting}
\end{minipage}

\section{Theorem Two}

In listing~\ref{lst:theoremtwo} we define an exclusive policy set $policySet$ containing a policy $pol$ that allows printing. The agreement $AgreeA5$ includes the exclusive policy set to express that Bob may print $LoveAndPeace$. However any subject that is not the agreement's user (e.g. Bob) is forbidden from printing $LoveAndPeace$. 

Notice that due to the fact that environments are defined as non-empty lists, we have added a Null count to it (see $eA5$). We continue to capture the relevant facts from the environment and the agreement through defining a hypothesis (e.g. $H$). The hypothesis is shown in listing~\ref{lst:theoremtwo}. 

\lstset{language=Coq}
\begin{lstlisting}[frame=single, caption={Hypothesis for Theorem Two}, label={lst:theoremtwohypo}]
forall x : subject, 
        (x = Bob /\ True -> True -> Permitted x Print LoveAndPeace) /\
       ((x = Bob -> False) -> Permitted x Print LoveAndPeace -> False). 
\end{lstlisting}

Theorem $T1\_A5$ states the exclusivity of the policy set, namely that any subject that is not Bob is not permitted to print the asset LoveAndPeace. Theorem $T2\_A5$ uses $T1\_A5$ to prove Alice is not permitted to print the asset.

\lstset{language=Coq}
\begin{minipage}[c]{0.95\textwidth}
\begin{lstlisting}[frame=single, caption={Theorem Two},label={lst:theoremtwo}]

Definition prin_bob := (Single Bob).
Definition pol:policy := PrimitivePolicy TruePrq id3 Print.
Definition pol_set:policySet := PrimitiveExclusivePolicySet TruePrq pol.
Definition AgreeA5 := Agreement prin_bob LoveAndPeace pol_set.
Definition eA5 : environment := (SingleEnv (make_count_equality NullSubject NullId 0)).


Hypothesis H: trans_agreement eA5 AgreeA5.


Theorem T1_A5: forall x, x<>Bob -> ~Permitted x Print LoveAndPeace.
Proof. simpl in H. apply H. Qed.

Theorem T2_A5: ~Permitted Alice Print LoveAndPeace.
Proof. simpl in H. apply T1_A5. apply not_eq_S. omega. Qed.


End A5.


\end{lstlisting}
\end{minipage}

\section{Theorem Three}

In listing~\ref{lst:environmentcoq} we defined environments as non-empty lists of $count_equality$ objects which are in turn defined as counts per each subject, policy-id pair. These count formulas represent how many times each policy has been used to justify an action by a subject wrt a policy (specified by the policy id) and semantically it makes sense that they are unique in time. When and if two $count_equality$ objects with the same subject and policy id refer to different counts, we say we have \emph{inconsistent} count formulas. The listing~\ref{lst:inconsistentcounts} defines the binary predicate $inconsistent$.

\lstset{language=Coq}
\begin{lstlisting}[frame=single, caption={Inconsistent Count Formulas}, label={lst:inconsistentcounts}]

Definition inconsistent (f1 f2 : count_equality) : Prop :=
   match f1 with (CountEquality s1 id1 n1) =>
     match f2 with (CountEquality s2 id2 n2) =>       
       s1 = s2 -> id1 = id2 -> n1 <> n2
     end 
   end.

\end{lstlisting}


Next we would like to expand the notion of inconsistency to more than two count formulas. We first define a predicate over a count formula and an environment as in listing~\ref{lst:inconsistentcountandenv}. If the environment is a singleton then we just compare the two count formulas for inconsistency, else we build the disjunction of the inconsistency between the count formula on one hand and the head of the environment and the rest of the environment, respectively. 

\lstset{language=Coq}
\begin{lstlisting}[frame=single, caption={Inconsistent Count Formula And Environment}, label={lst:inconsistentcountandenv}]
Fixpoint formula_inconsistent_with_env (f: count_equality)
                          (e : environment) : Prop :=
  match e with
    | SingleEnv g =>  inconsistent f g
    | ConsEnv g rest => (inconsistent f g) \/ (formula_inconsistent_with_env f rest)
  end.
\end{lstlisting}

Finally we define a new inductive data type that represents $consistent$ environments (see listing~\ref{lst:inconsistentenv}). An environment is consistent if it is a singleton count formula, if it consists of only two consistent count formulas and finally if the environment consists of a consistent environment and the consistent  composition of a count formula and the consistent environment (see constructor \emph{consis_more}.

\lstset{language=Coq}
\begin{lstlisting}[frame=single, caption={Inconsistent Environment}, label={lst:inconsistentenv}]

Inductive env_consistent : environment -> Prop :=
| consis_1 : forall f, env_consistent (SingleEnv f)
| consis_2 : forall f g, ~(inconsistent f g) -> env_consistent (ConsEnv f (SingleEnv g))
| consis_more : forall f e, 
   env_consistent e -> ~(formula_inconsistent_with_env f e) -> env_consistent (ConsEnv f e).

\end{lstlisting}


We will now pose several small theorems about consistency of count formulas and environments and provide proofs for them (see listing~\ref{lst:inconsistenttheorems}).

\lstset{language=Coq}
%\begin{minipage}[c]{0.95\textwidth}
\begin{lstlisting}[frame=single, caption={Inconsistent Environment}, label={lst:inconsistenttheorems}]

Theorem f1_and_f2_are_inconsistent: inconsistent f1 f2.
Proof. 
unfold inconsistent. simpl. omega. Qed.


Theorem f1_and___env_of_f2_inconsistent: formula_inconsistent_with_env f1 (SingleEnv f2).
Proof. 
unfold formula_inconsistent_with_env. apply f1_and_f2_are_inconsistent. Qed.


Theorem two_inconsistent_formulas_imply_env_inconsistent: 
  forall f g, inconsistent f g -> ~env_consistent (ConsEnv f (SingleEnv g)).
Proof. 
intros. unfold not. intros H'. 
inversion H'. intuition. intuition. Qed.


Theorem e2_is_inconsistent: ~env_consistent e2.
Proof.
apply two_inconsistent_formulas_imply_env_inconsistent. 
apply f1_and_f2_are_inconsistent. Qed.


Theorem env_consistent_implies_two_consistent_formulas: 
  forall (f g: count_equality), 
    env_consistent (ConsEnv f (SingleEnv g))-> ~inconsistent f g.
Proof. 
intros. inversion H. exact H1. intuition. Qed.


Theorem two_consistent_formulas_imply_env_consistent: 
  forall (f g: count_equality), 
    ~inconsistent f g -> env_consistent (ConsEnv f (SingleEnv g)).
Proof. 
intros. apply consis_2. exact H. Qed.

Theorem env_inconsistent_implies_two_inconsistent_formulas: 
  forall (f g: count_equality), 
    ~env_consistent (ConsEnv f (SingleEnv g))-> inconsistent f g.
Proof.
induction f.
induction g.
unfold inconsistent.
intros.
subst.
generalize (dec_eq_nat n n0).
intro h; elim h.
intro; subst.
elim H.
apply consis_2.
unfold inconsistent.
intro.
assert (s0=s0); auto.
assert (p0=p0); auto.
specialize H0 with (1:=H1) (2:=H2).
elim H0; auto.
auto.
Qed.


Theorem same_subjects_policyids_different_counts_means_inconsistent : forall (s1 s2: subject),
                forall (id1 id2: policyId),
                forall (n1 n2: nat),

  (s1 = s2 /\ id1 = id2 /\ n1 <> n2) -> 
  inconsistent (CountEquality s1 id1 n1) (CountEquality s2 id2 n2).

Proof. 
intros. unfold inconsistent. intros. intuition. Qed.



\end{lstlisting}
%\end{minipage}















          
%======================================================================
\chapter{Further Future Work Directions}
%======================================================================

%----------------------------------------------------------------------
%\section{Summary}
%
%We started off by looking at \ac{drel}s and specifically at \ac{odrl} and considered its formal semantics as captured by~\cite{pucella2006}. We presented the encodings and semantics of the constructs for a significant subset of \ac{odrl} in Coq and then defined what queries looked like and what the decision problem was in this context. We have also encoded the decision algorithms as presented in~\cite{pucella2006} in Coq in order to perform formal verification of theorems of interest. We noted the common thread between \ac{rel}s and policy languages for access control systems such as those between \ac{odrl} and \ac{selinux}. We discussed the goal of generalizing the concept of a policy language from strictly representing subsets of \ac{odrl}, to representing (subsets of) both \ac{odrl} and \ac{selinux} policy languages with the goal of applying the decision algorithms to both types of policies, in a unified manner.
%%----------------------------------------------------------------------
%
%
%\section{Machine-Checked Proof of Decidability of Queries}
%
%By defining formal semantics for \ac{odrl} authors of~\cite{pucella2006} were able to show some important results. First result is that answering the question of whether a set of ODRL statements imply a permission, denial or other possibilities is decidable and also that its complexity is NP-hard.
%
%The authors of~\cite{pucella2006} then prove that by removing the construct $not[policySet]$ from \ac{odrl}'s syntax answering the same query remains decidable and efficient (polynomial time complexity). 
%
%We will prove equivalent results as above starting with the decidability result of answering a query in ODRL0 (which does not include $not[policySet]$). The theorem in listing~\ref{lst:decidabilityodrl0coq} states that for all environments, all single agreements, all subjects, all actions and all assets, either permission is granted, permission is denied, permission is unregulated or query is inconsistent. 
%
%\lstset{language=Coq, frame=single, caption={Environments and Counts},label={lst:decidabilityodrl0coq}}
%\begin{minipage}[c]{0.95\textwidth}
%\begin{lstlisting}
%Theorem queriesAreDecidable: forall (e:environment), 
%                forall (agr: agreement),
%                forall (s:subject),
%                forall (action:act),
%                forall (a:asset),
%
%(permissionGranted e [agr] s action a) \/
%(permissionDenied e [agr] s action a)  \/
%(queryInconsistent e [agr] s action a) \/
%(permissionUnregulated e [agr] s action a).
%
%\end{lstlisting}
%\end{minipage}
%
%We will then augment ODRL0 with the constructs we omitted from the full ODRL (resulting in what we have earlier called ODRL1 or ODRL2) including the troublesome construct $not[policySet]$ and attempt to prove that the decidability results remain intact. There is a chance that a proof is not possible due to particulars of the Coq encoding we have used, in in which case, we will adjust our encoding.

 
We have designed \ac{ACCPL} as a suitable core policy language for \ac{pbac} systems that can be used for expressing general access-control expressions and also as a target language for deploying
policies written in other languages. We could capture, implement and study the semantics of these other policy-based access-control using the \ac{ACCPL} translation function framework and ultimately certify the semantics of those languages with respect to their specifications the same way \ac{ACCPL} has been certified.

\section{SELinux}

While \ac{xacml} is a high-level and platform independent access control system, \ac{selinux} is platform dependent (e.g. Linux based) and low-level. \ac{selinux} enhances the \ac{DAC} that most unix based systems employ by \ac{MAC} where designed access control policies are applied throughout the system possibly overriding whatever \ac{DAC} is in place by the system users. 

\ac{selinux} uses Linux's extended file attributes to attach a \emph{security context} to passive entities (e.g. files, directories, sockets) and also to each active entity, typically a Linux user space process. Security context is a data structure that is composed of a user, a role and a domain (or type). While users can map directly to ordinary user names they can also be defined separately. Roles are meant to group users and add flexibility when assigning permissions and are the basis for \ac{rbac} in \ac{selinux}. Finally domains or types are the basis for defining common access control requirements for both passive and active entities. 


The enforcement of \ac{selinux} policies are performed by the \emph{security server}. Whenever a security operation is requested from the Linux kernel by a program running in user space, the security server is invoked to arbitrate the operation and either allow the operation or to deny it. Each operation is identified by two pieces of information: an object class (e.g. file) and a permission (e.g. read, write). When an operation is requested to be performed on an object, the class and the permissions associated with the object along with security contexts of the source (typically the source entity is a process) and the object are passed to the security server. The security server consults the loaded policy (loaded at boot time) and allows or denies the access request~\cite{Sarna}.

\section{SELinux Policy Language}

The \ac{selinux} policy has four different kinds of statements: declarations, rules, constraints and assertions~\cite{ArcherLP03}. Assertions are compile time checks that the \syn{checkpolicy} tool performs at compile time. The other three kinds of statements however are evaluated at run-time. 

Declaration statements are used to declare types, roles and users. Type declaration statements are used to introduce new types. Roles are declared and authorized for particular domains (types) through role declarations, and finally user declarations are used to define each user and to specify the set of authorized roles for each of these users (see~\ref{lst:declsselinux}). In the following listings we will present a simplified and modified version of the official \ac{selinux} syntax that accomadates better the use of the \ac{ACCPL} Coq framework.

\lstset{language=selinux}
\begin{lstlisting}[frame=single, caption={Declaration Statements},label={lst:declsselinux}]

'type' T ';' &\Comment{; type T}&

'role' R T ';' &\Comment{; role R is associated with type T }&

'user' U R ';' &\Comment{; user U is associated with role R }&

\end{lstlisting}

Rule statements define access vector rules. Access vector (AV) rules (see listing~\ref{lst:avruleselinux}) specify which operations are allowed and whether to audit (log). Any operation not covered by AV rules are denied by default and all denied operations are logged. The semantics of the AV rule with \syn{<avkind>} \syn{allow} is: processes with type \syn{T1} are allowed to perform operations in \syn{P} on objects with class \syn{C} and type \syn{T2}. Note that in our modified syntax for \ac{selinux} we will use an explicit \syn{deny} as can be seen in listing~\ref{lst:avruleselinux}.

\lstset{language=selinux}
\begin{lstlisting}[frame=single, caption={AV Rule},label={lst:avruleselinux}]

<avRule> ::= <avkind> T1 T2:C P ';'

<avkind> ::= 'allow' | 'deny'
\end{lstlisting}

When a process changes security context, the role may change, assuming a ``role transition'' rule exists, relating the old and the new roles. There is a related AV rule called the ``type transition'' rule where a process with type \syn{T1} is allowed/denied to transition to type \syn{T2} when C=process and P=transition (see~\ref{lst:typeAndroletransselinux}).

\lstset{mathescape, language=AST} 
\begin{minipage}[c]{0.95\textwidth}
\begin{lstlisting}[frame=single, caption={Type Transition and Role-Allow Rules},label={lst:typeAndroletransselinux}]
<avkind> T1 T2:process transition ';'

<role_transition_rule> ::= <avkind> R1 R2 ';' &\Comment{; when a process changes security contexts this rule must hold }&
\end{lstlisting}
\end{minipage}

Constraints are additional conditions on permissions in the form of boolean expressions that must hold in order for the specified permissions to be allowed (see listing~\ref{lst:constrainselinux}). Whenever a permission is requested on an object class C, the security server checks that the constraints hold.

\lstset{language=selinux}
%\begin{minipage}[c]{0.95\textwidth}
\begin{lstlisting}[frame=single, caption={Constraint Definition},label={lst:constrainselinux}]

<constraint> ::= 'constrain' C, P, <expr> ';'

<expr> ::= 'not' <expr> | <expr> 'and' <expr> | <expr> or <expr> | U1 <op> U2 | T1 <op> T2 | R1 <op> R2

<op> ::= '==' | '!='
\end{lstlisting}
%\end{minipage}

\section{Agreements in SELinux}

We will start by limiting the \ac{selinux} policy language to only allow AV rules. As mentioned earlier an operation not covered by an allow rule is denied by default in \ac{selinux} proper. We will make up explicit deny rules, such that an agreement is defined to be a combination of allow and deny rules. Allow and deny rules as mappings are defined in listing~\ref{lst:allowmappingastselinux}.

\lstset{language=AST}
\begin{lstlisting}[frame=single, caption={'allow'/'deny' Rule as a Mapping},label={lst:allowmappingastselinux}]
AV rule : $T \times (T \times C) \rightarrow 2^{P}$
\end{lstlisting}

\lstset{language=AST}
\begin{lstlisting}[frame=single, caption={\ac{selinux} Agreement},label={lst:agreementastselinux}]
<agreement> ::= <avRule> ';' <agreement> 
\end{lstlisting}


\section{Environments}

Environments are collections of \emph{role-type} and \emph{user-role} relations. A role-type relation \syn{role(R, T)} simply associates a role with a type. A user-role relation \syn{user(U, R)} associates a user with a role. An environment is consistent with respect to a security context \syn{<T, R, U>}, if and only if \syn{role(R, T)} and \syn{user(U, R)} relations hold in the environment. 

\section{Queries in SELinux}

The decision problem in \ac{selinux} access control is whether an entity with security context \syn{<T1, R1, U1>} may perform action \syn{P1} to entity with object class \syn{C1} with security context \syn{<T2, R2, U2>}.

To answer such queries we use the authorization relation \syn{auth(C, P, T1, R1, U1, T2, R2, U2)} which is conceptually equivalent to the \syn{Permitted} answer from \ac{ACCPL} (see listing~\ref{lst:fqplussel}).

\lstset{mathescape, language=AST} 
\begin{lstlisting}[frame=single, caption={\syn{Permitted} for \ac{selinux}},label={lst:fqplussel}]
$allow (T1, T2, C, P)$ $\land$ $(E$ $consistent$ $wrt$ <T1, R1, U1> $\land$ <T2, R2, U2>) $\land$ $(((C,P)==(process, transition))$ $\implies$ $allow (R1, R2))$ $\implies auth(C, P, T1, R1, U1, T2, R2, U2)$ 
\end{lstlisting}

The negation of the authorization relation \syn{auth(C, P, T1, R1, U1, T2, R2, U2)} is conceptually equivalent to the \syn{NotPermitted} answer from \ac{ACCPL} (see listing~\ref{lst:fqminussel}).

\lstset{mathescape, language=AST} 
\begin{lstlisting}[frame=single, caption={\syn{NotPermitted} for \ac{selinux}},label={lst:fqminussel}]

$deny (T1, T2, C, P)$ $\lor$ $\lnot (E$ $consistent$ $wrt$ <T1, R1, U1> $\land$ <T2, R2, U2>) $\lor$ $(((C,P)==(process, transition))$ $\implies$ $deny (R1, R2))$ $\implies \lnot auth(C, P, T1, R1, U1, T2, R2, U2)$ 

\end{lstlisting}

\section{Decidability of Queries in SELinux}

As future work, we will be investigating the question of decidability for answering queries given an \ac{selinux} policy. We will first state a decidability theorem similar to the theorem in listing~\ref{lst:agreementdecidablecoq} (minor adjustments may be needed to allow for differences with \ac{selinux} policy language) and present a proof for it in Coq. Most policy based access-control languages typically use a two-valued decision set to indicate whether an access request is granted or denied. The literature for the \ac{selinux} implies only these two outcomes are possible so initially we will attempt to prove this conjecture in Coq. 

As a next step for future work on \ac{selinux}, we will add constraint relations to our simplified \ac{selinux} policy language and prove the decidability results stand for the augmented policy language.

\section{Related Work}

Policies in some \ac{pbac} languages such as \ac{xacml} and \ac{odrl} are expressed in \ac{xml}. Authors of~\cite{surveyXML} do a survey of XML-based access control languages describing and comparing the following languages and systems: Author-X, FASTER, \ac{odrl}, \ac{xrml}, \ac{xacl}, \ac{saml}, \ac{xacml} and the language designed by the authors called \ac{spl}. Other examples are WS-Policy, \ac{wsdl} and \ac{wspl} reviewed in~\cite{ArdagnaDVS04}.

Authors of~\cite{ArdagnaDVS04} argue that \ac{xml}-based \ac{pbac} systems are well suitable for the Internet context where simplicity and easy integration with existing technology and tools must be ensured. \ac{xml}-based languages are also well suited for the easy interchange of policies~\cite{ArdagnaDVS04}. 

While policies in \ac{xml}-based languages are expressed using \ac{xml}, the meaning of the policy statements in such languages are expressed using fragments of a natural language like English, resulting in ambiguity with regards to the intended behaviour of the system encoded in these policy statements. The ambiguity in semantics of these languages may lead to implementations varying in their interpretations of the access-control language and ultimately making real the possibility of security breaches (e.g. access granted to unauthorized subjects). 

In the following sections we will review related work and approaches to define semantics for \ac{pbac} based languages such that one can determine without any ambiguity whether a permission or prohibition follows from a set of policy statements.
 
\section{Lithium}
Halpern and Weissman~\cite{Halpern2008} use \ac{fol} to represent and reason about policies; policies describes the conditions under which a request to perform an action, such as reading a file, is granted or denied. They restrict \ac{fol} to get tractability for answering the query of whether a request to access a resource may be granted or denied, given a policy, and argue that despite the tractability results their language is still expressive. They contrast their approach with approaches based on Datalog~\cite{datalog} and point out that Datalog based work is made tractable by restricting function symbols and negation but at the cost of losing some expressive power. For example, Datalog based languages cannot distinguish between rendering explicitly forbidden decisions vs unregulated ones.

Halpern and Weissman~\cite{Halpern2008} focus on satisfying three requirements in the design of Lithium:
\begin{quote}
\begin{enumerate}
  \item It must be expressive enough to capture in an easy and natural way the policies that people want to discuss.
  \item It must be tractable enough to allow interesting queries about policies to be answered efficiently.
  \item It must be usable by non-experts, because we cannot expect policymakers and administrators to be well-versed in logic or programming languages.
\end{enumerate}
\end{quote}
We note although \ac{ACCPL} may be a good candidate to satisfy Halpern and Weissman's first goal, this was not a primary goal for \ac{ACCPL}. \ac{ACCPL} was designed as a core policy language with certified semantics such that it could be extended in various ways to add expressiveness as long as the semantics would remain certified with respect to various results established for \ac{ACCPL} (e.g. decidability results).  

As far as the Halpern and Weissman's second goal, we note that for \ac{ACCPL} tractability of answering an access query was not a goal; instead \ac{ACCPL} focused on achieving decidability results with respect to answering the access query. The third requirement Lithium was designed to satisfy, namely the usability of the language for non-experts was simply not a goal for \ac{ACCPL}. 

\section{Trace-based Semantics}
Gunter, et al~\cite{GunterWW01} propose an abstract model and a formal language to express access and usage rights for digital assets. A set of ``realities'' representing a sequence of \emph{payment} and \emph{render} (e.g. work is rendered by a device) actions make up a license. Semantics in the authors' model are expressed as a function mapping terms of the language to elements of the domain of licenses. The authors argue that their semantics is similar to those used for concurrency where language constructs are modelled as traces of allowed events.

Xiang, et al~\cite{xiang2008formal} use \ac{ots} modeling to describe licenses to use digital assets. In this formalism licenses and \ac{drm} systems are modeled as \ac{ots}s, described in CafeOBJ~\cite{cafeobj} which is an algebraic specification language. The authors' approach not only models static properties of licenses, dynamic evolutions or traces of licenses can also be observed and denoted by the actions in \ac{ots}s, respectively. Finally, formal verification of licenses expressed in CafeOBJ may be performed using a theorem proving facility which provides an integrated platform for the formal modeling, specification, and verification of licenses and \ac{drm} systems. This feature enables one to analyze and prove the licenses as well as \ac{drm} systems in a more effective way.

\section{Semantics Based on Linear Logic}

Barth and Mitchell~\cite{BarthM06} express semantics of digital rights using propositional linear logic. Linear logic deals with dynamic properties or finite resources, while classical logic deals with stable truths, or static properties~\cite{Girard87}. According to Girard as cited by Lincoln~\cite{Lincoln}, ``Linear logic is a resource conscious logic''. Bart and Mitchell introduce the notion of monotonicity which captures the idea that acquiring extra rights by a user should not lead to the user having fewer rights and show that the algorithm the \ac{oma} uses for assigning actions to rights is non-monotonic. Barth and Mitchell consider whether a sequence of actions complies with a license and show that answering this question for the \ac{oma} language is NP-complete. They propose an algorithm based on propositional linear logic to evaluate sequences of actions that is monotonic.

\section{Automata-based Semantics}
Holzer, et al~\cite{Holzer} give a semantics for \ac{odrl} that models the actions that are allowed according to a contract or an agreement. This model is presented in terms of automata. Each trace through the automaton represents a valid sequence of actions for each participant. The states of the automaton encode the state of the license at each point in time, meaning, which actions are allowed at what point considering the action that have taken place in the past. \ac{odrl} requirements and constraints are modeled as labels that are associated with edges in the automaton: an edge can only be taken if the related requirement is satisfied. 

Sheppard and Safavi-Naini~\cite{SheppardS09} point out that Holzer et al, don't present an algorithm in~\cite{Holzer} for building their automaton, leading to the conclusion that the examples in their paper are constructed by hand.

\section{Operational Semantics Based}
Sheppard and Safavi-Naini~\cite{SheppardS09} propose an operational model for both formalizing and enforcing digital rights using a \emph{right expression compiler}. They use their model to develop operational semantics for \ac{oma}, from which an interpreter could be derived. The authors argue their semantics provide for 
\begin{quote}
\begin{itemize}
  \item Actions whose effects are neither instantaneous nor fallible.
  \item Constraints whose satisfaction changes over time.
  \item Constraints that modify the form of an action rather than permit or prohibit it outright. The quality and watermark constraints
of \ac{odrl}, for example, modify an action by altering the resolution of the output and inserting a watermark, respectively.
\end{itemize}
\end{quote}


To show that an \ac{pbac} systems actually behaves correctly with respect to its specification, proofs are needed, however the proofs that are often presented in the literature (e.g.~\cite{Halpern2008, pucella2006, Tschantz}), are hard or impossible to formally verify. The verification difficulty is partly due to the fact that the language used to do the proofs while mathematical in nature, utilizes intuitive justifications to derive the proofs. Intuitive language in proofs means that the proofs could be incomplete and/or contain subtle errors. 

\section{Conflict Detection Algorithms}

In the following we will review some related work to ours where the authors have used the Coq Proof Assistant to develop conflict detection algorithms (for policies in particular \ac{pbac} languages), to state theorems and specifications about the behaviour of their algorithms, to develop formal proofs of those theorems and finally to machine-check those proofs. The resulting proofs are certified, meaning they provide strong correctness guarantees, due to the fact that they are machine-checked.

Capretta et al~\cite{CaprettaSFM07} present an conflict detection algorithm for the Cisco firewall specification~\cite{ciscofirewall} and formalize a correctness proof for it in the Coq proof assistant. They define two rules being in conflict as, given a access request, one rule would allow access while the another rule would deny access. The authors present their algorithm in Coq's functional programming language along with access rules and requests which are also encoded in Coq. An OCaml version of
the algorithm is extracted and successfully used on actual firewall specifications. The extracted program, according to the authors, can detect conflicts in firewalls with hundreds of thousands of rules. Capretta et al also prove in~\cite{CaprettaSFM07} that their algorithm finds all conflicts and only the correct conflicts in a set of rules. The algorithm is therefore verified formally to be both sound and complete.

St-Martin and Felty~\cite{Stmartin} represent policies for a fragment of \ac{xacml} 2.0 in the Coq proof assistant and propose an algorithm for detecting conflicts in \ac{xacml} policies. They state and prove correctness of their algorithm in the Coq proof assistant. Their \ac{xacml} subset includes some complex conditions such as time constraints. The authors compare their work with the conflict detection presented in~\cite{CaprettaSFM07} and conclude that conflict detection in \ac{xacml} is more complex and results in having to consider many cases including many subtle corner cases. 

St-Martin and Felty's definition of a conflict is the usual one and is similar to the one defined in~\cite{CaprettaSFM07} for firewall rules: when one rule in a policy permits a request and another denies the same request. Since \ac{xacml} allows conflicts by design, rule-combining is also provided. A rule-combining algorithm is meant to resolve conflicts if more than one rule applies. For example, one strategy (or algorithm) for resolving conflicts would be to use the first applicable rule that applies. Policy writers often make use of rule-combining algorithms, but unintended errors such as of conflicts are common.
St-Martin and Felty's conflict detection algorithm provides a tool for static analysis and policy debugging and it finds all conflicts, whether intended or not. Policy maintainers may then use the results of running the algorithm on their policies to decide which errors or conflicts were intended and which were not.


NOW CLOSE WITH YOUR WORK


GO OVER MY CONCLUSIONS/CONTRIBUTIONS AGAIN





























          
%% Some LaTeX commands I define for my own nomenclature.
% If you have to, it's better to change nomenclature once here than in a 
% million places throughout your thesis!
\newcommand{\package}[1]{\textbf{#1}} % package names in bold text
\newcommand{\cmmd}[1]{\textbackslash\texttt{#1}} % command name in tt font 


%======================================================================
\chapter{Observations}
%======================================================================

This would be a good place for some figures and tables.

Some notes on figures and photographs\ldots

\begin{itemize}
\item A well-prepared PDF should be 
  \begin{enumerate}
    \item Of reasonable size, {\it i.e.} photos cropped and compressed.
    \item Scalable, to allow enlargment of text and drawings. 
  \end{enumerate} 
\item Photos must be bit maps, and so are not scaleable by definition. TIFF and
BMP are uncompressed formats, while JPEG is compressed. Most photos can be
compressed without losing their illustrative value.
\item Drawings that you make should be scalable vector graphics, \emph{not} 
bit maps. Some scalable vector file formats are: EPS, SVG, PNG, WMF. These can
all be converted into PNG or PDF, that pdflatex recognizes. Your drawing 
package probably can export to one of these formats directly. Otherwise, a 
common procedure is to print-to-file through a Postscript printer driver to 
create a PS file, then convert that to EPS (encapsulated PS, which has a 
bounding box to describe its exact size rather than a whole page). 
Programs such as GSView (a Ghostscript GUI) can create both EPS and PDF from PS files.
Appendix~\ref{ch:Appendix-Matlab} shows how to generate properly sized Matlab plots and save them as PDF.
\item It's important to crop your photos and draw your figures to the size that
you want to appear in your thesis. Scaling photos with the 
includegraphics command will cause loss of resolution. And scaling down 
drawings may cause any text annotations to become too small.
\end{itemize}
 
For more information on \LaTeX\, see the uWaterloo Skills for the Academic Workplace 
course notes at \href{http://saw.uwaterloo.ca/latex}{saw.uwaterloo.ca/latex}. 
\footnote{
Note that while it is possible to include hyperlinks to external documents,
it is not wise to do so, since anything you can't control may change over time. 
It \emph{would} be appropriate and necessary to provide external links to 
additional resources for a multimedia ``enhanced'' thesis. 
But also note that if the \package{hyperref} package is not included, 
as for the print-optimized option in this thesis template, any \cmmd{href} 
commands in your logical document are no longer defined.
A work-around employed by this thesis template is to define a dummy \cmmd{href} 
command (which does nothing) in the preamble of the document, 
before the \package{hyperref} package is included. 
The dummy definition is then redifined by the
\package{hyperref} package when it is included.
}

%The classic book by Leslie Lamport~\cite{lamport.book}, author of \LaTeX , is worth a look too, and the many available add-on packages are described by 
%Goossens \textit{et~al.}~\cite{goossens.book}. Some on-line documentation is linked
%to from \href{http://saw.uwaterloo.ca/latex}{saw.uwaterloo.ca/latex}. 



Here is an example of how to include figures in \LaTeX. 
Figure~\ref{fig.beam} shows a cantilever beam of circular cross-section
subjected to a point load and a uniformly distributed load, both of which are uncertain. Note that it is better not to include the extension of the figure's source file.

\begin{figure}[!htbp]
 \begin{center}
  \includegraphics[clip=true]{figures/beam}
 \end{center}
\caption{Cantilever Beam}
\label{fig.beam}
\end{figure}



%----------------------------------------------------------------------
\section{Adding Nomenclature}
%----------------------------------------------------------------------

The following example is part of the ``nomentbl'' package. Refer to the package's documentation for more details.

\bigskip

\noindent
Let's start with equations to show how to use greek and mathematical symbols within Nomenclature.

Here is an equation
%
\begin{equation}\label{eq:heatflux}
  \dot{Q} = k \cdot A \cdot \Delta T
\end{equation}%
%
%% Greek and math symbols
\nomenclature[gQ]{$\dot{Q}$}{heat flux}{W}{}%
\nomenclature[gk]{$k$}{overall heat transfer coefficient}{$\frac{\mathrm{W}}{\mathrm{m}^2\mathrm{K}}$}{see eq.~(\ref{eq:ohtc})}%
\nomenclature[gA]{$A$}{area}{m$^2$}{$L^2$}%
\nomenclature[gL]{$L$}{length}{m}{SI base quantity}%
\nomenclature[gT]{$T$}{temperature}{K}{SI base quantity}%
\nomenclature[gT]{$\Delta T$}{temperature difference}{K}{SI base quantity}%

Here is another one
%
\begin{equation}\label{eq:ohtc}
  \frac{1}{k} = \left[\frac{1}{\alpha _{\mathrm{i}}\,r_{\mathrm{i}}} +
    \sum^n_{j=1}\frac{1}{\lambda _j}\,
    \ln \frac{r_{\mathrm{a},j}}{r_{\mathrm{i},j}} +
    \frac{1}{\alpha _{\mathrm{a}}\,
      r_{\mathrm{a}}}\right] \cdot r_{\mathrm{reference}}
\end{equation}%
%
%% Greek and math symbols
\nomenclature[ga]{$\alpha$}{convection heat transfer coefficient}{$\frac{\mathrm{W}}{\mathrm{m}^2\mathrm{K}}$}{}%
\nomenclature[gl]{$\lambda$}{thermal conductivity}{$\frac{\mathrm{W}}{\mathrm{m K}}$}{}%
%
%% Subscripts
\nomenclature[za]{a}{out}{}{}%
\nomenclature[zi]{i}{in}{}{}%
\nomenclature[zj]{$j$}{running parameter}{}{}% 
\nomenclature[zn]{$n$}{number of walls}{}{}%


\bigskip

\noindent
The following example is to show how to use abbreviations within the Nomenclature.\\
EECS is a school at the UO.
%% Abbreviations
\nomenclature[a]{EECS}{Electrical Engineering and Computer Science}{}{}%
\nomenclature[a]{UO}{University of Ottawa}{}{}%



\bigskip

\noindent
Don't forget to run:
\begin{verbatim}
makeindex -s nomentbl.ist -o uottawa-thesis.nls uottawa-thesis.nlo
\end{verbatim}



%%% Local Variables: 
%%% mode: latex
%%% TeX-master: "../uottawa-thesis"
%%% End: 



%----------------------------------------------------------------------
% APPENDICES
%---------------------------------------------------------------------- 
%\appendix
% Designate with \appendix declaration which just changes numbering style 
% from here on
% Add a title page before the appendices and a line in the Table of Contents
%\chapter*{APPENDICES}
\addcontentsline{toc}{chapter}{APPENDICES} 
%
%% An appendix
%======================================================================
\chapter{Sources of Information and Help}
\label{ch:Appendix-Sources-of-Info}
%======================================================================
%The best source of information about \LaTeX\ is the two books mentioned in this course \cite{lamport.book,goossens.book}.
%Another excellent resource is the UseNet newsgroup \verb=comp.text.tex=.
%A frequently-asked-questions (FAQ) list is also maintained by this news group.
%You might also search the World Wide Web for ``LaTeX'' for other sources of help.
 %"Sources of Information and Help"
%% An appendix
%======================================================================
\chapter[PDF Plots From Matlab]{Matlab Code for Making a PDF Plot}
\label{ch:Appendix-Matlab} 
%======================================================================
\section{Using the GUI}
Properties of Matab plots can be adjusted from the plot window via a graphical interface. Under the Desktop menu in the Figure window, select the Property Editor. You may also want to check the Plot Browser and Figure Palette for more tools. To adjust properties of the axes, look under the Edit menu and select Axes Properties.

To set the figure size and to save as PDF or other file formats, click the Export Setup button in the figure Property Editor.

\section{From the Command Line} 
All figure properties can also be manipulated from the command line. Here's an example: 
\begin{verbatim}
x=[0:0.1:pi];
hold on % Plot multiple traces on one figure
plot(x,sin(x))
plot(x,cos(x),'--r')
plot(x,tan(x),'.-g')
title('Some Trig Functions Over 0 to \pi') % Note LaTeX markup!
legend('{\it sin}(x)','{\it cos}(x)','{\it tan}(x)')
hold off
set(gca,'Ylim',[-3 3]) % Adjust Y limits of "current axes"
set(gcf,'Units','inches') % Set figure size units of "current figure"
set(gcf,'Position',[0,0,6,4]) % Set figure width (6 in.) and height (4 in.)
cd n:\thesis\plots % Select where to save
print -dpdf plot.pdf % Save as PDF
\end{verbatim}  %"Matlab Code for Making a PDF Plot"

%----------------------------------------------------------------------
% END MATERIAL
%----------------------------------------------------------------------

% B I B L I O G R A P H Y
% -----------------------
%
% The following statement selects the style to use for references.  It controls the sort order of the entries in the bibliography and also the formatting for the in-text labels.
\bibliographystyle{alpha}
% This specifies the location of the file containing the bibliographic information.  
% It assumes you're using BibTeX (if not, why not?).
\ifthenelse{\boolean{PrintVersion}}{
\cleardoublepage % This is needed if the book class is used, to place the anchor in the correct page,
                 % because the bibliography will start on its own page.
}{
\clearpage       % Use \clearpage instead if the document class uses the "oneside" argument
}
\phantomsection  % With hyperref package, enables hyperlinking from the table of contents to bibliography             
% The following statement causes the title "References" to be used for the bibliography section:
\renewcommand*{\bibname}{References}

% Add the References to the Table of Contents
\addcontentsline{toc}{chapter}{\textbf{References}}

\bibliography{bibliography/keylatex}
% Tip 5: You can create multiple .bib files to organize your references. 
% Just list them all in the \bibliogaphy command, separated by commas (no spaces).


%----------------------------------------------------------------------
\end{document}
%======================================================================



%%% Local Variables: 
%%% mode: latex
%%% TeX-master: t
%%% End: 
